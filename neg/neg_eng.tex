\input{../doc}
\input{../preamb_eng}

\begin{document}

\section{\negintro}
Earlier on we have seen that (for example) 5 on a number line is placed 5 one-lengths to the right of 0. 
\fig{neg1}
But what if we move in the other direction, that is to the left? The question is answered by introducing \textit{negative numbers}\index{numbers!negative}\index{numbers!positive}.\regv

\reg[Positive and negative numbers]{
On a number line the following applies:
\begin{itemize}
	\item Numbers placed to \textsl{the right} of 0 are positive numbers.
	\item Numbers placed to \textsl{the left og} 0 is negative numbers.
\end{itemize}
\fig{neg2}
}\vsk 
However, relying on the number line every time negative numnbers are involved would be very inconvenient, and therefore we also use a symbol to indicate negative numbers. This is \sym{$ - $}, that is the same symbol used to indicate subtraction. From this it follows that $ 5 $ is a positive number, while $ -5 $ is a negative number. On the number line we have that
\begin{itemize}
	\item 5 is placed 5 one-lengths to \textsl{the right} of 0.
	\item $ -5 $ is placed 5 one-lengths to \textsl{the left} of 0.
\end{itemize}
\fig{neg3}
Hence, the big difference between $ 5 $ and $ -5 $ is on which side of 0 the numbers are placed. Since $ 5 $ and $ -5 $ have the same distance from 0, we say that $ 5 $ and $ -5 $ have equal \textit{length}\index{length}. \regv

\reg[Length (absolute value/modulus/magnitude)\index{talverdi}\index{absolute value}]{
The length of a number is expressed by the symbol \sym{| |}.\vsk	
	
The length of a positive number equals the value of the number.\vsk
	
The length of a negative number equals the value of the positive number with the corresponding digits.
} 
\eks[1]{ \vs \vs
\[ |27|=27 \]
}
\eks[2]{ \vs \vs
\[ |-27|=27 \]
}
\info{Forteikn}{
\textit{Sign}\index{sign} is a collective name of \sym{$ + $} and \sym{$ - $}. \sym{$ + $} is the sign of $ 5 $ and \sym{$ - $} is the sign of $ -5 $.
}
\newpage
\section{\negrekn \label{rekmneg}}
The introduction of negative numbers brings new aspects to the elementary operations, aspects which we are to discuss here. When adding, subtracting, multiplying or dividing by negative numbers we'll frequently, to make it more clear, write negative numbers enclosed by parenthesis. Then we'll write for example $ -4 $ as $ (-4) $. 

\subsection*{Addisjon}
When adding in \refsec{Addisjon} \sym{$ + $} implied moving to \textsl{the right}. Negative numbers forces an alternation of the interpretation of \sym{$ + $}\,: \regv
\st{\begin{center}
		\begin{tabular}{cl}
			$ + $&''As long and in \textsl{the same} direction as''	
		\end{tabular}
\end{center}}
Let's study the calculation
\[ 7+(-4) \]
Our alternated definition of \sym{$ + $} implies that
\[ 7+(-4)=\text{''}7 \text{ and as long and in the \textsl{same} direction as } (-4)\text{''} \]
$ (-4) $ has length 4 and direction to \textsl{the left}. Hence, the calculation tells us to start at 7 and then move the length of 4 to \textsl{the left}.
\[ 7+(-4)=3 \]
\fig{neg6}

\reg[Addisjon med negative tal]{
Adding a negative numbers is the same as subtracting the number of equal magnitude.
}
\eks[1]{ \vs
	\[ 4+(-3)=4-3=1 \]
}
\eks[2]{ \vs
	\[ -8+(-3)=-8-3=-11 \]
}
\info{Notice}{
\rref{adkom} declares that addition is commutative. This also applies after introducing negative numbers, for example is	
\[ 7+(-3)=4=-3+7 \]	
}

\subsection*{Subtraksjon}
In \refsec{Subtraksjon} \sym{$ - $} implied moving to \textsl{the left}. The interpretation of \sym{$ - $} also needs an alternation when working with negative numbers:\regv

\st{\begin{center}
	\begin{tabular}{cl}
		$ - $&''As long and in \textsl{opposite} direction as''	
	\end{tabular}
\end{center}}
Let's study the calculation
\[ 2-(-6) \]
Our alternated definition of \sym{$-$} implies that
\[ 2-(-6)=\text{''}2 \text{ and as long and in the \textsl{opposite} direction as } (-6)\text{''} \]
$ -6 $ have length 6 and direction to \textsl{the left}. When moving an equal length, but in the \textsl{opposite} direction, we have to move the length of 6 to \textsl{the right}\footnote{Once again, recall that the red colored arrow indicates starting at the arrow head, then moving to the other end.}. This is equivalent to adding 6:
\[ 2-(-6)=2+6=8 \] \vs
\fig{neg7}
\reg[Subtraksjon med negative tal]{
Subtracting a negative number is the same as adding the number of equal magnitude.
}
\eks[1]{ \vs
\[ 11-(-9)=11+9=20 \]
}
\eks[2]{ \vs
	\[ -3-(-7)=-3+7=4 \]
}
\begin{comment}
	\info{Notice}{
	Subtraction can be interpreted as addition of negative numbers. For expample is $ {7-3=7+(-3)} $ and $ {2-7=2+(-7)} $.
}
\end{comment}

\subsection*{Multiplikasjon}
In \refsec{Gonging} multiplication by positive integers were introduced as repeated addition. By our alternated interpretations of addition and subtraction we can now also alternate the interpreation of multiplication: \regv

\reg[Multiplication by positive and negative integers I \label{gongneg}]{ \vs
\begin{itemize}
		\item Multiplication by a positive integer corresponds to repeated addiotion.
		\item Multiplication by a negative integer corresponds to repeated subtraction.
\end{itemize}}
\eks[1]{ \vs \vs \label{negeksempel}
\alg{
	2\cdot3 &=\text{''As long and in the \textsl{same} direction as 2, 3 times''}\\
	&=2+2+2 \\
	&=6
}
}
\eks[2]{ \vs \vs
\alg{
	(-2)\cdot3&=\text{''As long and in the \textsl{same} direction as }(-2) \text{, 3 times''} \\[-12pt]
	&=-2-2-2\\
	&=-6
}
}
\eks[3]{ \vs \vs
\alg{
	2\cdot(-3)&=\text{''As long and in the \textsl{opposite} direction as 2, 3 times''} \\[-12pt]
	&=-2-2-2 \\
	&=-6
}
}
\newpage
\eks[4]{ \vs \vs
	\alg{
		(-3)\cdot(-4)&=\text{\small ''As long and in the \textsl{opposite} direction as $ -3 $, 4 times''}\\[-12pt] 
		&=3+3+3+3\\
		&=12
	}	 	 
}
\info{Multiplikasjon er kommutativ}{
\textsl{Example 2} and \textsl{Example 3} on page \pageref{negeksempel} illustrates that \rref{gangkom} also implies after introducing negative numbers:
\[ (-2)\cdot3=3\cdot(-2) \]
}  \vsk
It would be very laborious to calculate multiplication by repeated addition/subtrction every time a negative number is involved, however, as a direct consequence of \rref{gongneg} we can make the two following rules: \regv

\reg[\gangmnegto \label{gangmnegto}]{
The product of a negative and a positive number is a negative number. \vsk

The magnitude of the factors multiplied together gives the magnitude of the product.
}
\eks[1]{
Calculate $ (-7)\cdot8 $

\sv
Since $ 7\cdot8=56 $, we have $ (-7)\cdot8=-56 $
}
\eks[2]{
	Calculate $ 3\cdot(-9) $.
	
	\sv
	Since $ 3\cdot9=27 $, we have $ 3\cdot(-9)=-27 $
}
\vsk

\reg[\gangmnegtre \label{gangmnegtre}]{
The product of two negative numbers is a positive number. \vsk

The magnitude of the factors multiplied together gives the value of the product.
} 
\eks[1]{ \vsb
\[ (-5)\cdot(-10)=5\cdot10=50 \]
}
\eks[2]{
\vsb
\[ (-2)\cdot(-8)=2\cdot8=16 \]
}
\subsection*{Division}
From the definition of division (see \refsec{Divisjon}), combined with what we now know about multiplication involving negative number, it follows that\regv

\st{\begin{center}
		\algv{
			-18:6=\text{''Talet eg må gonge 6 med for å få $ -18 $''}
		}
		$ 6\cdot(-3)=-18 $, altså er $ -18:6=-3 $
\end{center}} \vsk
\st{
\begin{center}
	\algv{
		42:(-7)=\text{''Talet eg må gonge $ -7 $ med for å få 42''}
	}
	$ (-7)\cdot(-8)=42 $, altså er $ 42:(-7)=-8 $
\end{center}
}\vsk
\st{
\begin{center}
	\algv{
		-45:(-5)=\text{''Talet eg må gonge $ -5 $ med for å få $ -45 $''}
	}
	$ (-5)\cdot9=-45 $, altså er $ -45:(-5)=9 $
\end{center}
 }\vsk
\reg[Division involving negative numbers]{
Division between a positive and a negative number results in a negative number.\vsk

Division between two negative numbers results in a positive number. \vsk

The magnitude of the dividend divided by the magnitude of the divisoren gives the magnitude of the quotient. 
}
\eks[1]{ \vsb
	\[ -24:6=-4 \]
}
\eks[2]{ \vsb
	\[ 24:(-2)=-12 \]
}
\eks[3]{ \vsb
	\[ -24:(-3)=8 \]
}
\eks[4]{ \vs
\[ \frac{2}{-3}=-\frac{2}{3} \]
}
\eks[5]{ \vs
	\[ \frac{-10}{7}=-\frac{10}{7} \]
}
\newpage

\newpage
\section{\negmeng \label{negmeng}}
\textit{Attention! This view of negative numbers will first come into use in \refsec{likloysfire}, a section which a lot of readers can skip without loss of understanding.}\vsk

So far we have studied negative number by the aid of number lines. Studying negative numbers as amounts is at first difficult because negative amounts makes no sense! To make an interpretation of negative numbers through the perspective of amounts, we'll use what we shall call the \textit{weight principle}. Then we look upon the numbers as amounts of forces. The positive numbers are amounts of forces acting downward while the negative numbers are amounts of forces working upwards\footnote{From reality one can look upon the positive and the negative numbers as balloons filled with air and helium, respectively. Balloons filled with air acts with a force downwards (they fall), while balloons filled with helium acts with a force upwards (they rise).}. In this way, the results of calculations involving positive and negative numbers can be looked upon as the result of a weighing of the amounts. Hence, a positive number and a negative number of equal magnitude will cancel each other.
\vsk

\reg[Negative tal som mengde]{
Negative tal vil vi indikere som ei lyseblå mengde:
\fig{negm1}
}
\eks[]{
\[ 1+(-1)=0 \]
\fig{negm2}
}

\end{document}


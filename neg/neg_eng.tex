\documentclass[english,hidelinks,pdftex, 11 pt, class=report,crop=false]{standalone}
\usepackage[T1]{fontenc}
\usepackage[utf8]{luainputenc}
\usepackage{lmodern} % load a font with all the characters
\usepackage{geometry}
\geometry{verbose,paperwidth=16.1 cm, paperheight=24 cm, inner=2.3cm, outer=1.8 cm, bmargin=2cm, tmargin=1.8cm}
\setlength{\parindent}{0bp}
\usepackage{import}
\usepackage[subpreambles=false]{standalone}
\usepackage{amsmath}
\usepackage{amssymb}
\usepackage{esint}
\usepackage{babel}
\usepackage{tabu}
\makeatother
\makeatletter

\usepackage{titlesec}
\usepackage{ragged2e}
\RaggedRight
\raggedbottom
\frenchspacing

% Norwegian names of figures, chapters, parts and content
\addto\captionsenglish{\renewcommand{\figurename}{Figure}}
\makeatletter
\addto\captionsenglish{\renewcommand{\chaptername}{Chapter}}
%\addto\captionsenglish{\renewcommand{\partname}{Part}}

%\addto\captionsenglish{\renewcommand{\contentsname}{Content}}

\usepackage{graphicx}
\usepackage{float}
\usepackage{subfig}
\usepackage{placeins}
\usepackage{cancel}
\usepackage{framed}
\usepackage{wrapfig}
\usepackage[subfigure]{tocloft}
\usepackage[font=footnotesize,labelfont=sl]{caption} % Figure caption
\usepackage{bm}
\usepackage[dvipsnames, table]{xcolor}
\definecolor{shadecolor}{rgb}{0.105469, 0.613281, 1}
\colorlet{shadecolor}{Emerald!15} 
\usepackage{icomma}
\makeatother
\usepackage[many]{tcolorbox}
\usepackage{multicol}
\usepackage{stackengine}

% For tabular
\usepackage{array}
\usepackage{multirow}
\usepackage{longtable} %breakable table

% Ligningsreferanser
\usepackage{mathtools}
\mathtoolsset{showonlyrefs}

% index
\usepackage{imakeidx}
\makeindex[title=Index]

%Footnote:
\usepackage[bottom, hang, flushmargin]{footmisc}
\usepackage{perpage} 
\MakePerPage{footnote}
\addtolength{\footnotesep}{2mm}
\renewcommand{\thefootnote}{\arabic{footnote}}
\renewcommand\footnoterule{\rule{\linewidth}{0.4pt}}
\renewcommand{\thempfootnote}{\arabic{mpfootnote}}

%colors
\definecolor{c1}{cmyk}{0,0.5,1,0}
\definecolor{c2}{cmyk}{1,0.25,1,0}
\definecolor{n3}{cmyk}{1,0.,1,0}
\definecolor{neg}{cmyk}{1,0.,0.,0}

% Lister med bokstavar
\usepackage{enumitem}

\newcounter{rg}
\numberwithin{rg}{chapter}
\newcommand{\reg}[2][]{\begin{tcolorbox}[boxrule=0.3 mm,arc=0mm,colback=blue!3] {\refstepcounter{rg}\phantomsection \large \textbf{\therg \;#1} \vspace{5 pt}}\newline #2  \end{tcolorbox}\vspace{-5pt}}

\newcommand\alg[1]{\begin{align} #1 \end{align}}

\newcommand\eks[2][]{\begin{tcolorbox}[boxrule=0.3 mm,arc=0mm,enhanced jigsaw,breakable,colback=green!3] {\large \textbf{Example #1} \vspace{5 pt}\\} #2 \end{tcolorbox}\vspace{-5pt} }

\newcommand{\st}[1]{\begin{tcolorbox}[boxrule=0.0 mm,arc=0mm,enhanced jigsaw,breakable,colback=yellow!12]{ #1} \end{tcolorbox}}

\newcommand{\spr}[1]{\begin{tcolorbox}[boxrule=0.3 mm,arc=0mm,enhanced jigsaw,breakable,colback=yellow!7] {\large \textbf{The language box} \vspace{5 pt}\\} #1 \end{tcolorbox}\vspace{-5pt} }

\newcommand{\sym}[1]{\colorbox{blue!15}{#1}}

\newcommand{\info}[2]{\begin{tcolorbox}[boxrule=0.3 mm,arc=0mm,enhanced jigsaw,breakable,colback=cyan!6] {\large \textbf{#1} \vspace{5 pt}\\} #2 \end{tcolorbox}\vspace{-5pt} }

\newcommand\algv[1]{\vspace{-11 pt}\begin{align*} #1 \end{align*}}

\newcommand{\regv}{\vspace{5pt}}
\newcommand{\mer}{\textsl{Notice}: }
\newcommand{\merk}{Notice}
\newcommand\vsk{\vspace{11pt}}
\newcommand\vs{\vspace{-11pt}}
\newcommand\vsb{\vspace{-16pt}}
\newcommand\sv{\vsk \textbf{Answer} \vspace{4 pt}\\}
\newcommand\br{\\[5 pt]}
\newcommand{\asym}[1]{../fig/#1}
\newcommand\algvv[1]{\vs\vs\begin{align*} #1 \end{align*}}
\newcommand{\y}[1]{$ {#1} $}
\newcommand{\os}{\\[5 pt]}
\newcommand{\prbxl}[2]{
\parbox[l][][l]{#1\linewidth}{#2
	}}
\newcommand{\prbxr}[2]{\parbox[r][][l]{#1\linewidth}{
		\setlength{\abovedisplayskip}{5pt}
		\setlength{\belowdisplayskip}{5pt}	
		\setlength{\abovedisplayshortskip}{0pt}
		\setlength{\belowdisplayshortskip}{0pt} 
		\begin{shaded}
			\footnotesize	#2 \end{shaded}}}

\renewcommand{\cfttoctitlefont}{\Large\bfseries}
\setlength{\cftaftertoctitleskip}{0 pt}
\setlength{\cftbeforetoctitleskip}{0 pt}

\newcommand{\bs}{\\[3pt]}
\newcommand{\vn}{\\[6pt]}
\newcommand{\fig}[1]{\begin{figure}
		\centering
		\includegraphics[]{\asym{#1}}
\end{figure}}

\newcommand{\sectionbreak}{\clearpage} % New page on each section

% Equation comments
\newcommand{\cm}[1]{\llap{\color{blue} #1}}

\newcommand\fork[2]{\begin{tcolorbox}[boxrule=0.3 mm,arc=0mm,enhanced jigsaw,breakable,colback=yellow!7] {\large \textbf{#1 (explanation)} \vspace{5 pt}\\} #2 \end{tcolorbox}\vspace{-5pt} }


%%% SECTION HEADLINES %%%

% Our numbers
\newcommand{\likteikn}{The equal sign}
\newcommand{\talsifverd}{Numbers, digits and values}
\newcommand{\koordsys}{Coordinate systems}

% Calculations
\newcommand{\adi}{Addition}
\newcommand{\sub}{Subtraction}
\newcommand{\gong}{Multiplication}
\newcommand{\del}{Division}

%Factorization and order of operations
\newcommand{\fak}{Factorization}
\newcommand{\rrek}{Order of operations}

%Fractions
\newcommand{\brgrpr}{Introduction}
\newcommand{\brvu}{Values, expanding and simplifying}
\newcommand{\bradsub}{Addition and subtraction}
\newcommand{\brgngheil}{Fractions multiplied by integers}
\newcommand{\brdelheil}{Fractions divided by integers}
\newcommand{\brgngbr}{Fractions multiplied by fractions}
\newcommand{\brkans}{Cancelation of fractions}
\newcommand{\brdelmbr}{Division by fractions}
\newcommand{\Rasjtal}{Rational numbers}

%Negative numbers
\newcommand{\negintro}{Introduction}
\newcommand{\negrekn}{The elementary operations}
\newcommand{\negmeng}{Negative numbers as amounts}

% Geometry 1
\newcommand{\omgr}{Terms}
\newcommand{\eignsk}{Attributes of triangles and quadrilaterals}
\newcommand{\omkr}{Perimeter}
\newcommand{\area}{Area}

%Algebra 
\newcommand{\algintro}{Introduction}
\newcommand{\pot}{Powers}
\newcommand{\irrasj}{Irrational numbers}

%Equations
\newcommand{\ligintro}{Introduction}
\newcommand{\liglos}{Solving with the elementary operations}
\newcommand{\ligloso}{Solving with elementary operations summarized}

%Functions
\newcommand{\fintro}{Introduction}
\newcommand{\lingraf}{Linear functions and graphs}

%Geometry 2
\newcommand{\geoform}{Formulas of area and perimeter}
\newcommand{\kongogsim}{Congruent and similar triangles}
\newcommand{\geofork}{Explanations}

% Names of rules
\newcommand{\adkom}{Addition is commutative}
\newcommand{\gangkom}{Multiplication is commutative}
\newcommand{\brdef}{Fractions as rewriting of division}
\newcommand{\brtbr}{Fractions multiplied by fractions}
\newcommand{\delmbr}{Fractions divided by fractions}
\newcommand{\gangpar}{Distributive law}
\newcommand{\gangparsam}{Paranthesis multiplied together}
\newcommand{\gangmnegto}{Multiplication by negative numbers I}
\newcommand{\gangmnegtre}{Multiplication by negative numbers II}
\newcommand{\konsttre}{Unique construction of triangles}
\newcommand{\kongtre}{Congruent triangles}
\newcommand{\topv}{Vertical angles}
\newcommand{\trisum}{The sum of angles in a triangle}
\newcommand{\firsum}{The sum of angles in a quadrilateral}
\newcommand{\potgang}{Multiplication by powers}
\newcommand{\potdivpot}{Division by powers}
\newcommand{\potanull}{The special case of \boldmath $a^0$}
\newcommand{\potneg}{Powers with negative exponents}
\newcommand{\potbr}{Fractions as base}
\newcommand{\faktgr}{Factors as base}
\newcommand{\potsomgrunn}{Powers as base}
\newcommand{\arsirk}{The area of a circle}
\newcommand{\artrap}{The area of a trapezoid}
\newcommand{\arpar}{The area of a parallelogram}
\newcommand{\pyt}{Pythagoras's theorem}
\newcommand{\forform}{Ratios in similar triangles}
\newcommand{\vilkform}{Terms of similar triangles}
\newcommand{\omkrsirk}{The perimeter of a circle (and the value of $ \bm \pi $)}
\newcommand{\artri}{The area of a triangle}
\newcommand{\arrekt}{The area of a rectangle}
\newcommand{\liknflyt}{Moving terms across the equal sign}
\newcommand{\funklin}{Linear functions}

%License
\newcommand{\lic}{\textit{First Principles of Math by Sindre Sogge Heggen is licensed under CC BY-NC-SA 4.0. To view a copy of this license, visit\\ 
		\net{http://creativecommons.org/licenses/by-nc-sa/4.0/}{http://creativecommons.org/licenses/by-nc-sa/4.0/}}}

%referances
\newcommand{\net}[2]{{\color{blue}\href{#1}{#2}}}
\newcommand{\hrs}[2]{\hyperref[#1]{\color{blue}\textsl{#2 \ref*{#1}}}}
\newcommand{\rref}[1]{\hrs{#1}{Rule}}
\newcommand{\refkap}[1]{\hrs{#1}{Chapter}}
\newcommand{\refsec}[1]{\hrs{#1}{Section}}

\usepackage{datetime2}
\usepackage[]{hyperref}


\begin{document}

\section{\negintro}
Earlier we have seen that e.g. 5 on a number line is placed 5 units to the right of 0. 
\fig{neg1}
But what if we move in the other direction, that is to the left? The question is answered by introducing \textit{negative numbers}\index{number!negative}\index{number!positive}.\regv

\reg[Positive and negative numbers]{
On a number line, the following applies:
\begin{itemize}
	\item Numbers placed to \textsl{the right} of 0 are positive numbers.
	\item Numbers placed to \textsl{the left} of 0 are negative numbers.
\end{itemize}
\fig{neg2_eng}
}\vsk 
However, relying on the number line every time negative numbers are involved would be very inconvenient, therefore we use a symbol to indicate negative numbers. This is \sym{$ - $}, simply the same as the symbol of subtraction. From this it follows that $ 5 $ is a positive number, while $ -5 $ is a negative number. On the number line,
\begin{itemize}
	\item 5 is placed 5 units to \textsl{the right} of 0.
	\item $ -5 $ is placed 5 units to \textsl{the left} of 0.
\end{itemize}
\fig{neg3}
Hence, the big difference between $ 5 $ and $ -5 $ is on which side of 0 the numbers are placed. Since $ 5 $ and $ -5 $ have the same distance from 0, we say that $ 5 $ and $ -5 $ have equal \textit{length}\index{length}. \regv

\reg[Length (absolute value/modulus/magnitude)\index{magnitude}\index{number!length of}\index{absolute value}]{
The length of a number is expressed by the symbol \sym{| |}.\vsk	
	
The length of a positive number equals the value of the number.\vsk
	
The length of a negative number equals the value of the positive number with corresponding digits.
} 
\eks[1]{ \vs \vs
\[ |27|=27 \]
}
\eks[2]{ \vs \vs
\[ |-27|=27 \]
}
\info{Sign}{
\textit{Sign}\index{sign} is a collective name of \sym{$ + $} and \sym{$ - $}. \sym{$ + $} is the sign of $ 5 $ and \sym{$ - $} is the sign of $ -5 $.
}
\newpage
\section{\negrekn \label{rekmneg}}
The introduction of negative numbers brings new aspects to the elementary operations. When adding, subtracting, multiplying or dividing by negative numbers, we'll frequently, for clarity, enclose negative numbers by parentheses. Then we'll write e.g. $ -4 $ as $ (-4) $. 

\subsection*{Addition}
When adding in \refsec{Addisjon} \sym{$ + $} implied moving to \textsl{the right}. Negative numbers bring an alternation of the interpretation of \sym{$ + $}\,: \regv
\st{\begin{center}
		\begin{tabular}{cl}
			$ + $&''As long and in \textsl{the same} direction as''	
		\end{tabular}
\end{center}}
Let's study the calculation
\[ 7+(-4) \]
Our alternated definition of \sym{$ + $} implies that
\[ 7+(-4)=\text{''}7 \text{ and as long and in the \textsl{same} direction as } (-4)\text{''} \]
$ (-4) $ has length 4 and direction to \textsl{the left}. Hence, the calculation tells us to start at 7 and then move the length of 4 to \textsl{the left}.
\[ 7+(-4)=3 \]
\fig{neg6}

\reg[Addition involving negative numbers]{
Adding a negative number is the same as subtracting the number of equal magnitude.
}
\eks[1]{ \vs
	\[ 4+(-3)=4-3=1 \]
}
\eks[2]{ \vs
	\[ -8+(-3)=-8-3=-11 \]
}
\info{Notice}{
\rref{adkom} declares that addition is commutative. This also applies after introducing negative numbers, for example is	
\[ 7+(-3)=4=-3+7 \]	
}

\subsection*{Subtraction}
In \refsec{Subtraksjon}, \sym{$ - $} implied moving to \textsl{the left}. The interpretation of \sym{$ - $} also needs an alternation when working with negative numbers:\regv

\st{\begin{center}
	\begin{tabular}{cl}
		$ - $&''As long and in the \textsl{opposite} direction as''	
	\end{tabular}
\end{center}}
Let's study the calculation
\[ 2-(-6) \]
Our alternated definition of \sym{$-$} implies that
\[ 2-(-6)=\text{''}2 \text{ and as long and in the \textsl{opposite} direction as } (-6)\text{''} \]
$ -6 $ have length 6 and direction to \textsl{the left}. When moving an equal length, but in the \textsl{opposite} direction, we have to move the length of 6 to \textsl{the right}\footnote{Once again, recall that the red colored arrow indicates starting at the arrowhead, then moving to the other end.}. This is equivalent to adding 6:
\[ 2-(-6)=2+6=8 \] \vs
\fig{neg7}
\reg[Subtraction involving negative numbers]{
Subtracting a negative number is the same as adding the \\positive number of equal magnitude.
}
\eks[1]{ \vs
\[ 11-(-9)=11+9=20 \]
}
\eks[2]{ \vs
	\[ -3-(-7)=-3+7=4 \]
}
\begin{comment}
	\info{Notice}{
	Subtraction can be interpreted as addition of negative numbers. For expample is $ {7-3=7+(-3)} $ and $ {2-7=2+(-7)} $.
}
\end{comment}

\subsection*{Multiplication}
In \refsec{Gonging}, multiplication by positive integers were introduced as repeated addition. By our alternated interpretations of addition and subtraction we can now also alternate the interpretation of multiplication: \regv

\reg[Multiplication by positive and negative integers\label{gongneg}]{ \vs
\begin{itemize}
		\item Multiplication by a positive integer corresponds to \\[-4pt] repeated addition.
		\item Multiplication by a negative integer corresponds to\\[-4pt] repeated subtraction.
\end{itemize}}
\eks[1]{ \vs \vs \label{negeksempel}
\alg{
	2\cdot3 &=\text{''As long and in the \textsl{same} direction as 2, 3 times''}\\
	&=2+2+2 \\
	&=6
}
}
\eks[2]{ \vs \vs
\alg{
	(-2)\cdot3&=\text{''As long and in the \textsl{same} direction as }(-2) \text{, 3 times''} \\[-12pt]
	&=-2-2-2\\
	&=-6
}
}
\eks[3]{ \vs \vs
\alg{
	2\cdot(-3)&=\text{''As long and in the \textsl{opposite} direction as 2, 3 times''} \\[-12pt]
	&=-2-2-2 \\
	&=-6
}
}
\newpage
\eks[4]{ \vs \vs
	\alg{
		(-3)\cdot(-4)&=\text{\small ''As long and in the \textsl{opposite} direction as $ -3 $, 4 times''}\\[-12pt] 
		&=3+3+3+3\\
		&=12
	}	 	 
}
\info{Multiplication is commutative}{
\textsl{Example 2} and \textsl{Example 3} on page \pageref{negeksempel} illustrates that \rref{gangkom} also implies after introducing negative numbers:
\[ (-2)\cdot3=3\cdot(-2) \]
}  \vsk
It would be laborious to calculate multiplication by repeated addition/subtraction every time a negative number were involved, however, as a direct consequence of \rref{gongneg} we can make the two following rules: \regv

\reg[\gangmnegto \label{gangmnegto}]{
The product of a negative number and a positive number is a negative number. \vsk

The magnitude of the factors multiplied together yields the magnitude of the product.
}
\eks[1]{
Calculate $ (-7)\cdot8 $

\sv
Since $ 7\cdot8=56 $, we have $ (-7)\cdot8=-56 $
}
\eks[2]{
	Calculate $ 3\cdot(-9) $.
	
	\sv
	Since $ 3\cdot9=27 $, we have $ 3\cdot(-9)=-27 $
}
\vsk

\reg[\gangmnegtre \label{gangmnegtre}]{
The product of two negative numbers is a positive number. \vsk

The magnitude of the factors multiplied together yields the value of the product.
} 
\eks[1]{ \vsb
\[ (-5)\cdot(-10)=5\cdot10=50 \]
}
\eks[2]{
\vsb
\[ (-2)\cdot(-8)=2\cdot8=16 \]
}
\subsection*{Division}
From the definition of division (see \refsec{Divisjon}), combined with what we now know about multiplication involving negative numbers, it follows that\regv

\st{\begin{center}
		\algv{
			-18:6=\text{''The number which yields $ -18 $ when multiplied by 6  ''}
		}
		$ 6\cdot(-3)=-18 $, hence $ -18:6=-3 $
\end{center}} \vsk
\st{
\begin{center}
	\algv{
		42:(-7)=\text{''The number which yields $ 42 $ when multiplied by $ -7 $''}
	}
	$ (-7)\cdot(-8)=42 $, hence $ 42:(-7)=-8 $
\end{center}
}\vsk
\st{
\begin{center}
	\algv{
		-45:(-5)&=\text{''The number which yields $ -45 $ when multiplied}\\
		&\phantom{=''}\text{by $ -5 $''}
	}
	$ (-5)\cdot9=-45 $, hence $ -45:(-5)=9 $
\end{center}
 }\vsk
\reg[Division involving negative numbers]{
Division between a positive number and a negative number yields a negative number.\vsk

Division between two negative numbers yields a positive\\ number. \vsk

The magnitude of the dividend divided by the magnitude of the divisor yields the magnitude of the quotient. 
}
\eks[1]{ \vsb
	\[ -24:6=-4 \]
}
\eks[2]{ \vsb
	\[ 24:(-2)=-12 \]
}
\eks[3]{ \vsb
	\[ -24:(-3)=8 \]
}
\eks[4]{ \vs
\[ \frac{2}{-3}=-\frac{2}{3} \]
}
\eks[5]{ \vs
	\[ \frac{-10}{7}=-\frac{10}{7} \]
}
\newpage

\newpage
\section{\negmeng \label{negmeng}}
\textit{Notice: This view of negative numbers will first come into use in \refsec{likloysfire}, a section a lot of readers can skip without loss of understanding.}\vsk

So far, we have studied negative number by the aid of number lines. Studying negative numbers as amounts is at first difficult because negative amounts makes no sense! To make an interpretation of negative numbers through the perspective of amounts, we'll use what we shall call the \textit{weight principle}. Then we look upon the numbers as amounts of forces. The positive numbers are amounts of forces acting downwards while the negative numbers are amounts of forces working upwards\footnote{From reality one can look upon the positive and the negative numbers as balloons filled with air and helium, respectively. Balloons filled with air acts with a force downwards (they fall), while balloons filled with helium acts with a force upwards (they rise).}. In this way, the results of calculations involving positive and negative numbers can be looked upon as the result of weighing the amounts. Hence, a positive number and a negative number of equal magnitude will cancel each other.
\vsk

\reg[Negative numbers as amounts]{
Negative numbers will be illustrated as a light blue amount:
\fig{negm1}
}
\eks[]{
\[ 1+(-1)=0 \]
\fig{negm2}
}

\end{document}


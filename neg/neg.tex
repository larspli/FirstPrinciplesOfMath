\documentclass[english,hidelinks,pdftex, 11 pt, class=report,crop=false]{standalone}
\usepackage[T1]{fontenc}
\usepackage[utf8]{luainputenc}
\usepackage{lmodern} % load a font with all the characters
\usepackage{geometry}
\geometry{verbose,paperwidth=16.1 cm, paperheight=24 cm, inner=2.3cm, outer=1.8 cm, bmargin=2cm, tmargin=1.8cm}
\setlength{\parindent}{0bp}
\usepackage{import}
\usepackage[subpreambles=false]{standalone}
\usepackage{amsmath}
\usepackage{amssymb}
\usepackage{esint}
\usepackage{babel}
\usepackage{tabu}
\makeatother
\makeatletter

\usepackage{titlesec}
\usepackage{ragged2e}
\RaggedRight
\raggedbottom
\frenchspacing

% Norwegian names of figures, chapters, parts and content
\addto\captionsenglish{\renewcommand{\figurename}{Figur}}
\makeatletter
\addto\captionsenglish{\renewcommand{\chaptername}{Kapittel}}
\addto\captionsenglish{\renewcommand{\partname}{Del}}

\addto\captionsenglish{\renewcommand{\contentsname}{Innhald}}

\usepackage{graphicx}
\usepackage{float}
\usepackage{subfig}
\usepackage{placeins}
\usepackage{cancel}
\usepackage{framed}
\usepackage{wrapfig}
\usepackage[subfigure]{tocloft}
\usepackage[font=footnotesize,labelfont=sl]{caption} % Figure caption
\usepackage{bm}
\usepackage[dvipsnames, table]{xcolor}
\definecolor{shadecolor}{rgb}{0.105469, 0.613281, 1}
\colorlet{shadecolor}{Emerald!15} 
\usepackage{icomma}
\makeatother
\usepackage[many]{tcolorbox}
\usepackage{multicol}
\usepackage{stackengine}

% For tabular
\usepackage{array}
\usepackage{multirow}
\usepackage{longtable} %breakable table

% Ligningsreferanser
\usepackage{mathtools}
\mathtoolsset{showonlyrefs}

% index
\usepackage{imakeidx}
\makeindex[title=Indeks]

%Footnote:
\usepackage[bottom, hang, flushmargin]{footmisc}
\usepackage{perpage} 
\MakePerPage{footnote}
\addtolength{\footnotesep}{2mm}
\renewcommand{\thefootnote}{\arabic{footnote}}
\renewcommand\footnoterule{\rule{\linewidth}{0.4pt}}
\renewcommand{\thempfootnote}{\arabic{mpfootnote}}

%colors
\definecolor{c1}{cmyk}{0,0.5,1,0}
\definecolor{c2}{cmyk}{1,0.25,1,0}
\definecolor{n3}{cmyk}{1,0.,1,0}
\definecolor{neg}{cmyk}{1,0.,0.,0}

% Lister med bokstavar
\usepackage[inline]{enumitem}

\newcounter{rg}
\numberwithin{rg}{chapter}
\newcommand{\reg}[2][]{\begin{tcolorbox}[boxrule=0.3 mm,arc=0mm,colback=blue!3] {\refstepcounter{rg}\phantomsection \large \textbf{\therg \;#1} \vspace{5 pt}}\newline #2  \end{tcolorbox}\vspace{-5pt}}

\newcommand\alg[1]{\begin{align} #1 \end{align}}

\newcommand\eks[2][]{\begin{tcolorbox}[boxrule=0.3 mm,arc=0mm,enhanced jigsaw,breakable,colback=green!3] {\large \textbf{Eksempel #1} \vspace{5 pt}\\} #2 \end{tcolorbox}\vspace{-5pt} }

\newcommand{\st}[1]{\begin{tcolorbox}[boxrule=0.0 mm,arc=0mm,enhanced jigsaw,breakable,colback=yellow!12]{ #1} \end{tcolorbox}}

\newcommand{\spr}[1]{\begin{tcolorbox}[boxrule=0.3 mm,arc=0mm,enhanced jigsaw,breakable,colback=yellow!7] {\large \textbf{Språkboksen} \vspace{5 pt}\\} #1 \end{tcolorbox}\vspace{-5pt} }

\newcommand{\sym}[1]{\colorbox{blue!15}{#1}}

\newcommand{\info}[2]{\begin{tcolorbox}[boxrule=0.3 mm,arc=0mm,enhanced jigsaw,breakable,colback=cyan!6] {\large \textbf{#1} \vspace{5 pt}\\} #2 \end{tcolorbox}\vspace{-5pt} }

\newcommand\algv[1]{\vspace{-11 pt}\begin{align*} #1 \end{align*}}

\newcommand{\regv}{\vspace{5pt}}
\newcommand{\mer}{\textsl{Merk}: }
\newcommand\vsk{\vspace{11pt}}
\newcommand\vs{\vspace{-11pt}}
\newcommand\vsb{\vspace{-16pt}}
\newcommand\sv{\vsk \textbf{Svar:} \vspace{4 pt}\\}
\newcommand\br{\\[5 pt]}
\newcommand{\asym}[1]{../fig/#1}
\newcommand\algvv[1]{\vs\vs\begin{align*} #1 \end{align*}}
\newcommand{\y}[1]{$ {#1} $}
\newcommand{\os}{\\[5 pt]}
\newcommand{\prbxl}[2]{
\parbox[l][][l]{#1\linewidth}{#2
	}}
\newcommand{\prbxr}[2]{\parbox[r][][l]{#1\linewidth}{
		\setlength{\abovedisplayskip}{5pt}
		\setlength{\belowdisplayskip}{5pt}	
		\setlength{\abovedisplayshortskip}{0pt}
		\setlength{\belowdisplayshortskip}{0pt} 
		\begin{shaded}
			\footnotesize	#2 \end{shaded}}}

\renewcommand{\cfttoctitlefont}{\Large\bfseries}
\setlength{\cftaftertoctitleskip}{0 pt}
\setlength{\cftbeforetoctitleskip}{0 pt}

\newcommand{\bs}{\\[3pt]}
\newcommand{\vn}{\\[6pt]}
\newcommand{\fig}[1]{\begin{figure}
		\centering
		\includegraphics[]{\asym{#1}}
\end{figure}}


\newcommand{\sectionbreak}{\clearpage} % New page on each section

% Equation comments
\newcommand{\cm}[1]{\llap{\color{blue} #1}}

\newcommand\fork[2]{\begin{tcolorbox}[boxrule=0.3 mm,arc=0mm,enhanced jigsaw,breakable,colback=yellow!7] {\large \textbf{#1 (forklaring)} \vspace{5 pt}\\} #2 \end{tcolorbox}\vspace{-5pt} }




%colors
\newcommand{\colr}[1]{{\color{red} #1}}
\newcommand{\colb}[1]{{\color{blue} #1}}
\newcommand{\colo}[1]{{\color{orange} #1}}
\newcommand{\colc}[1]{{\color{cyan} #1}}
\definecolor{projectgreen}{cmyk}{100,0,100,0}
\newcommand{\colg}[1]{{\color{projectgreen} #1}}

%%% SECTION HEADLINES %%%

% Our numbers
\newcommand{\likteikn}{Likskapsteiknet}
\newcommand{\talsifverd}{Tal, siffer og verdi}
\newcommand{\koordsys}{Koordinatsystem}

% Calculations
\newcommand{\adi}{Addisjon}
\newcommand{\sub}{Subtraksjon}
\newcommand{\gong}{Multiplikasjon (Gonging)}
\newcommand{\del}{Divisjon (deling)}

%Factorization and order of operations
\newcommand{\fak}{Faktorisering}
\newcommand{\rrek}{Reknerekkefølge}

%Fractions
\newcommand{\brgrpr}{Introduksjon}
\newcommand{\brvu}{Verdi, utviding og forkorting av brøk}
\newcommand{\bradsub}{Addisjon og subtraksjon}
\newcommand{\brgngheil}{Brøk gonga med heiltal}
\newcommand{\brdelheil}{Brøk delt med heiltal}
\newcommand{\brgngbr}{Brøk gonga med brøk}
\newcommand{\brkans}{Kansellering av faktorar}
\newcommand{\brdelmbr}{Deling med brøk}
\newcommand{\Rasjtal}{Rasjonale tal}

%Negative numbers
\newcommand{\negintro}{Introduksjon}
\newcommand{\negrekn}{Dei fire rekneartane med negative tal}
\newcommand{\negmeng}{Negative tal som mengde}

% Geometry 1
\newcommand{\omgr}{Omgrep}
\newcommand{\eignsk}{Eigenskapar for trekantar og firkantar}
\newcommand{\omkr}{Omkrins}
\newcommand{\area}{Areal}

%Algebra 
\newcommand{\algintro}{Introduksjon}
\newcommand{\pot}{Potensar}
\newcommand{\irrasj}{Irrasjonale tal}

%Equations
\newcommand{\ligintro}{Introduksjon}
\newcommand{\liglos}{Løysing ved dei fire rekneartane}
\newcommand{\ligloso}{Løysingsmetodane oppsummert}

%Functions
\newcommand{\fintro}{Introduksjon}
\newcommand{\lingraf}{Lineære funksjonar og grafar}

%Geometry 2
\newcommand{\geoform}{Formlar for areal og omkrins}
\newcommand{\kongogsim}{Kongruente og formlike trekantar}
\newcommand{\geofork}{Forklaringar}

% Names of rules
\newcommand{\gangdestihundre}{Å gange desimaltall med 10, 100 osv.}
\newcommand{\delmedtihundre}{Deling med 10, 100, 1\,000 osv.}
\newcommand{\ompref}{Omgjøring av prefikser}
\newcommand{\adkom}{Addisjon er kommutativ}
\newcommand{\gangkom}{Multiplikasjon er kommutativ}
\newcommand{\brdef}{Brøk som omskriving av delestykke}
\newcommand{\brtbr}{Brøk gonga med brøk}
\newcommand{\delmbr}{Brøk delt på brøk}
\newcommand{\gangpar}{Gonging med parentes (distributiv lov)}
\newcommand{\gangparsam}{Parantesar gonga saman}
\newcommand{\gangmnegto}{Gonging med negative tal I}
\newcommand{\gangmnegtre}{Gonging med negative tal II}
\newcommand{\konsttre}{Konstruksjon av trekantar}
\newcommand{\kongtre}{Kongruente trekantar}
\newcommand{\topv}{Toppvinklar}
\newcommand{\trisum}{Summen av vinklane i ein trekant}
\newcommand{\firsum}{Summen av vinklane i ein firkant}
\newcommand{\potgang}{Gonging med potensar}
\newcommand{\potdivpot}{Divisjon med potensar}
\newcommand{\potanull}{Spesialtilfellet \boldmath $a^0$}
\newcommand{\potneg}{Potens med negativ eksponent}
\newcommand{\potbr}{Brøk som grunntal}
\newcommand{\faktgr}{Faktorar som grunntal}
\newcommand{\potsomgrunn}{Potens som grunntal}
\newcommand{\arsirk}{Arealet til ein sirkel}
\newcommand{\artrap}{Arealet til eit trapes}
\newcommand{\arpar}{Arealet til eit parallellogram}
\newcommand{\pyt}{Pytagoras' setning}
\newcommand{\forform}{Forhold i formlike trekantar}
\newcommand{\vilkform}{Vilkår i formlike trekantar}
\newcommand{\omkrsirk}{Omkrinsen til ein sirkel (og $ \bm \pi $)}
\newcommand{\artri}{Arealet til ein trekant}
\newcommand{\arrekt}{Arealet til eit rektangel}
\newcommand{\liknflyt}{Flytting av ledd over likskapsteiknet}
\newcommand{\funklin}{Lineære funksjonar}

%Opg
\newcommand{\abc}[1]{
	\begin{enumerate}[label=\alph*),leftmargin=18pt]
		#1
	\end{enumerate}
}
\newcommand{\abcs}[2]{
	\begin{enumerate}[label=\alph*),start=#1,leftmargin=18pt]
		#2
	\end{enumerate}
}
\newcommand{\abcn}[1]{
	\begin{enumerate}[label=\arabic*),leftmargin=18pt]
		#1
	\end{enumerate}
}
\newcommand{\abch}[1]{
	\hspace{-2pt}	\begin{enumerate*}[label=\alph*), itemjoin=\hspace{1cm}]
		#1
	\end{enumerate*}
}
\newcommand{\abchs}[2]{
	\hspace{-2pt}	\begin{enumerate*}[label=\alph*), itemjoin=\hspace{1cm}, start=#1]
		#2
	\end{enumerate*}
}

\newcommand{\opgt}{\phantomsection \addcontentsline{toc}{section}{Oppgaver} \section*{Oppgaver for kapittel \thechapter}\vs \setcounter{section}{1}}
\newcounter{opg}
\numberwithin{opg}{section}
\newcommand{\op}[1]{\vspace{15pt} \refstepcounter{opg}\large \textbf{\color{blue}\theopg} \vspace{2 pt} \label{#1} \\}
\newcommand{\ekspop}[1]{\vsk\textbf{Gruble \thechapter.#1}\vspace{2 pt} \\}
\newcommand{\nes}{\stepcounter{section}
	\setcounter{opg}{0}}
\newcommand{\opr}[1]{\vspace{3pt}\textbf{\ref{#1}}}
\newcommand{\oeks}[1]{\begin{tcolorbox}[boxrule=0.3 mm,arc=0mm,colback=white]
		\textit{Eksempel: } #1	  
\end{tcolorbox}}
\newcommand\opgeks[2][]{\begin{tcolorbox}[boxrule=0.1 mm,arc=0mm,enhanced jigsaw,breakable,colback=white] {\footnotesize \textbf{Eksempel #1} \\} \footnotesize #2 \end{tcolorbox}\vspace{-5pt} }

%License
\newcommand{\lic}{\textit{Matematikken sine byggesteinar by Sindre Sogge Heggen is licensed under CC BY-NC-SA 4.0. To view a copy of this license, visit\\ 
		\net{http://creativecommons.org/licenses/by-nc-sa/4.0/}{http://creativecommons.org/licenses/by-nc-sa/4.0/}}}

%referances
\newcommand{\net}[2]{{\color{blue}\href{#1}{#2}}}
\newcommand{\hrs}[2]{\hyperref[#1]{\color{blue}\textsl{#2 \ref*{#1}}}}
\newcommand{\rref}[1]{\hrs{#1}{Regel}}
\newcommand{\refkap}[1]{\hrs{#1}{Kapittel}}
\newcommand{\refsec}[1]{\hrs{#1}{Seksjon}}

\usepackage{datetime2}

\usepackage[]{hyperref}


\begin{document}

\section{\negintro}
Vi har tidlegare sett at (for eksempel) talet 5 på ei tallinje ligg 5 einarlengder til høgre for 0. 
\fig{neg1}
Men kva om vi går andre veien, altså mot venstre? Dette spørsmålet svarer vi på ved å innføre \textit{negative tal}\index{tal!negativt}\index{tal!positivt}.\regv

\reg[Positive og negative tal]{
På ei tallinje gjeld følgande:
\begin{itemize}
	\item Tal plassert \textsl{til høgre} for 0 er positive tal.
	\item Tal plassert \textsl{til venstre} for 0 er negative tal.
\end{itemize}
\fig{neg2}
}\vsk 
I praksis kan vi ikkje heile tida bruke ei tallinje for å avgjere om eit tal er negativt eller positivt, og derfor bruker vi eit symbol for å vise at tal er negative. Dette symbolet er rett og slett \sym{$ - $}, altså det same symbolet som vi bruker ved subtraksjon. $ 5 $ er med dét eit positivt tal, mens $ -5 $ er eit negativt tal. På tallinja er det slik at
\begin{itemize}
	\item 5 ligg 5 einarlengder \textsl{til høgre} for 0.
	\item $ -5 $ ligg 5 einarlengder \textsl{til venstre} for 0.
\end{itemize}
\fig{neg3}
Den store forskjellen på $ 5 $ og $ -5 $ er altså på kva side av 0 tala ligg. Da $ 5 $ og $ -5 $ har same avstand til 0, seier vi at $ 5 $ og $ -5 $ har same \textit{lengde}\index{lengde}. \regv

\reg[Lengde (talverdi/absoluttverdi)\index{talverdi}\index{absoluttverdi}]{
Lengda til eit tal skrivast ved symbolet \sym{| |}.\vsk	
	
Lengda til eit positivt tal er verdien til talet.\vsk
	
Lengda til eit negativt tal er verdien til det positive talet med same siffer.
} 
\eks[1]{ \vs \vs
\[ |27|=27 \]
}
\eks[2]{ \vs \vs
\[ |-27|=27 \]
}
\info{Forteikn}{
\textit{Forteikn}\index{forteikn} er ei samlenemning for \sym{$ + $} og \sym{$ - $}. $ 5 $ har \sym{$ + $} som forteikn og $ -5 $ har \sym{$ - $} som forteikn.
}
\newpage
\section{\negrekn \label{rekmneg}}
Ved innføringa av negative tal får dei fire rekneartane nye sider som vi må sjå på trinnvis. Når vi adderer, subtraherer, multipliserer eller dividerer med negative tal vil vi ofte, for å gjere det meir tydeleg, skrive negative tal med parentes rundt. Da skriv vi for eksempel $ -4 $ som $ (-4) $. 

\subsection*{Addisjon}
Når vi adderte i \hrs{Addisjon}{seksjon} såg vi på \sym{$ + $} som vandring \textsl{mot høgre}. Negative tal gjer at vi må utvide omgrepet for \sym{$ + $}\,: \regv
\st{\begin{center}
		\begin{tabular}{cl}
			$ + $&''Like langt og i \textsl{same} retning som''	
		\end{tabular}
\end{center}}
La oss sjå på reknestykket
\[ 7+(-4) \]
Vår utvida definisjon av \sym{$ + $} seier oss no at
\[ 7+(-4)=\text{''}7 \text{ og like langt og i \textsl{same} retning som } (-4)\text{''} \]
$ (-4) $ har lengde 4 og retning \textsl{mot venstre}. Vårt reknestykke seier altså at vi skal starte på 7, og deretter gå lengda 4 \textsl{mot venstre}.
\[ 7+(-4)=3 \]
\fig{neg6}

\reg[Addisjon med negative tal]{
Å addere eit negativt tal er det same som å subtrahere talet med same talverdi.
}
\eks[1]{ \vs
	\[ 4+(-3)=4-3=1 \]
}
\eks[2]{ \vs
	\[ -8+(-3)=-8-3=-11 \]
}
\info{Merk}{
\rref{adkom} erklærer at addisjon er kommutativ. Dette er gjeldande også ved innføringa av negative tal, for eksempel er	
\[ 7+(-3)=4=-3+7 \]	
}

\subsection*{Subtraksjon}
I \hrs{Subtraksjon}{seksjon} såg vi på \sym{$ - $} som vandring \textsl{mot venstre}. Også tydinga av \sym{$ - $} må utvidast når vi jobbar med negative tal:\regv

\st{\begin{center}
	\begin{tabular}{cl}
		$ - $&''Like langt og i \textsl{motsett} retning som''	
	\end{tabular}
\end{center}}
La oss sjå på reknestykket
\[ 2-(-6) \]
Med vår utvida tyding av \sym{$-$}, kan vi skrive
\[ 2-(-6)=\text{''}2 \text{ og like langt og i \textsl{motsett} retning som } (-6)\text{''} \]
$ -6 $ har lengde 6 og retning \textsl{mot venstre}. Når vi skal gå same lengde, men i \textsl{motsett} retning, må vi altså gå lengda 6 \textsl{mot høgre}\footnote{Vi minner enda ein gong om at raudfarga på pila indikerer at ein skal vandre fra pilspissen til andre enden.}. Dette er det same som å addere 6:
\[ 2-(-6)=2+6=8 \] \vs
\fig{neg7}
\reg[Subtraksjon med negative tal]{
Å subtrahere eit negativt tal er det same som å addere det positive talet med same talverdi.
}
\eks[1]{ \vs
\[ 11-(-9)=11+9=20 \]
}
\eks[2]{ \vs
	\[ -3-(-7)=-3+7=4 \]
}
\begin{comment}
	\info{Merk}{
	Med innføringa av negative tal kan subtraksjon bli sett på som addisjon med negative tal. For eksempel er $ {7-3=7+(-3)} $ og $ {2-7=2+(-7)} $.
}
\end{comment}

\subsection*{Multiplikasjon}
I \hrs{Gonging}{seksjon} introduserte vi gonging med positive heiltal som gjentatt addisjon. Med våre utvida omgrep av addisjon og subtraksjon, kan vi no også utvide omgrepet multiplikasjon: \regv

\reg[Multiplikasjon med positive og negative tal\label{gongneg}]{ \vs
\begin{itemize}
		\item Gonging med eit positivt heiltal er det same som gjentatt addisjon.
		\item Gonging med eit negativt heiltal er det same som gjentatt subtraksjon.
\end{itemize}}
\eks[1]{ \vs \vs \label{negeksempel}
\alg{
	2\cdot3 &=\text{''Like langt og i \textsl{same} retning som 2, 3 gonger''}\\
	&=2+2+2 \\
	&=6
}
}
\eks[2]{ \vs \vs
\alg{
	(-2)\cdot3&=\text{''Like langt og i \textsl{same} retning som }(-2) \text{, 3 gonger''} \\
	&=-2-2-2\\
	&=-6
}
}
\eks[3]{ \vs \vs
\alg{
	2\cdot(-3)&=\text{''Like langt og i \textsl{motsett} retning som 2, 3 gonger''} \\
	&=-2-2-2 \\
	&=-6
}
}
\eks[4]{ \vs \vs
	\alg{
		(-3)\cdot(-4)&=\text{''Like langt og i \textsl{motsett} retning som $ -3 $, 4 gonger''}\\[-12pt] 
		&=3+3+3+3\\
		&=12
	}	 	 
}
\info{Multiplikasjon er kommutativ}{
\textsl{Eksempel 2} og \textsl{Eksempel 3} på side \pageref{negeksempel} illustrerer at \rref{gangkom} også er gjeldande ved innføringa av negative tal:
\[ (-2)\cdot3=3\cdot(-2) \]
}  \vsk
Det blir tungvint å rekne gonging som gjentatt addisjon/subtraksjon kvar gong vi har eit negativt tal involvert, men som ein direkte konsekvens av \rref{gongneg} kan vi lage oss følgande to reglar:\regv

\reg[\gangmnegto \label{gangmnegto}]{
Produktet av eit negativt og eit positivt tal er eit negativt tal. \vsk

Talverdien til faktorane gonga saman gir talverdien til produktet.
}
\eks[1]{
Rekn ut $ (-7)\cdot8 $

\sv
Sidan $ 7\cdot8=56 $, er $ (-7)\cdot8=-56 $
}
\eks[2]{
	Rekn ut $ 3\cdot(-9) $.
	
	\sv
	Sidan $ 3\cdot9=27 $, er $ 3\cdot(-9)=-27 $
}
\vsk

\reg[\gangmnegtre \label{gangmnegtre}]{
Produktet av to negative tal er eit positivt tal. \vsk

Talverdien til faktorane gonga saman gir verdien til produktet.
} 
\eks[1]{ \vsb
\[ (-5)\cdot(-10)=5\cdot10=50 \]
}
\eks[2]{
\vsb
\[ (-2)\cdot(-8)=2\cdot8=16 \]
}
\subsection*{Divisjon}
Definisjonen av divisjon (sjå \hrs{Divisjon}{seksjon}), kombinert med det vi veit om multiplikasjon med negative tal, gir oss no dette:\regv

\st{\begin{center}
		\algv{
			-18:6=\text{''Talet eg må gonge 6 med for å få $ -18 $''}
		}
		$ 6\cdot(-3)=-18 $, altså er $ -18:6=-3 $
\end{center}} \vsk
\st{
\begin{center}
	\algv{
		42:(-7)=\text{''Talet eg må gonge $ -7 $ med for å få 42''}
	}
	$ (-7)\cdot(-8)=42 $, altså er $ 42:(-7)=-8 $
\end{center}
}\vsk
\st{
\begin{center}
	\algv{
		-45:(-5)=\text{''Talet eg må gonge $ -5 $ med for å få $ -45 $''}
	}
	$ (-5)\cdot9=-45 $, altså er $ -45:(-5)=9 $
\end{center}
 }\vsk
\reg[Divisjon med negative tal]{
Divisjon mellom eit positivt og eit negativt tal gir eit negativt tal.\vsk

Divisjon mellom to negative tal gir eit positivt tal. \vsk

Talverdien til dividenden delt med talverdien til divisoren gir talverdien til kvotienten. 
}
\eks[1]{ \vsb
	\[ -24:6=-4 \]
}
\eks[2]{ \vsb
	\[ 24:(-2)=-12 \]
}
\eks[3]{ \vsb
	\[ -24:(-3)=8 \]
}
\eks[4]{ \vs
\[ \frac{2}{-3}=-\frac{2}{3} \]
}
\eks[5]{ \vs
	\[ \frac{-10}{7}=-\frac{10}{7} \]
}
\newpage

\newpage
\section{\negmeng \label{negmeng}}
\textit{Obs! Denne tolkinga av negative tal blir først brukt i \hrs{likloysfire}{seksjon}, som er ein seksjon nokre leserar utan tap av forståing kan hoppe over.}\vsk

Så langt har vi sett på negative tal ved hjelp av tallinjer. Å sjå på negative tal ved hjelp av mengder er i første omgang vanskeleg, fordi vi har likestilt mengder med antal, og negative antal gir ikkje meining! For å skape ei forståing av negative tal ut ifrå eit mengdeperspektiv, nyttar vi det vi skal kalle \textit{vektprinsippet}. Dette inneber at vi ser på tala som krefter. Dei positive tala er antal krefter som verkar nedover og dei negative tala er antal krefter som verkar oppover\footnote{Frå verkelegheita kan ein sjå på dei positive og negative tala som ballongar fylt med høvesvis luft og helium. Ballongar fylt med luft verkar med ei kraft nedover (dei dett), mens heliumbalongar verkar med ei kraft oppover (dei stig).}. Svara på reknestykker med positive og negative tal kan ein da sjå på som resultatet av ei veiing av dei forskjellige mengdene. Slik vil altså eit positivt tal og eit negativt tal med same talverdi \textsl{utlikne} kvarandre.
\vsk

\reg[Negative tal som mengde]{
Negative tal vil vi indikere som ei lyseblå mengde:
\fig{negm1}
}
\eks[]{
\[ 1+(-1)=0 \]
\fig{negm2}
}

\end{document}


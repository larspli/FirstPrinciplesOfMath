\documentclass[english,hidelinks,pdftex, 11 pt, class=report,crop=false]{standalone}
\usepackage[T1]{fontenc}
\usepackage[utf8]{luainputenc}
\usepackage{lmodern} % load a font with all the characters
\usepackage{geometry}
\geometry{verbose,paperwidth=16.1 cm, paperheight=24 cm, inner=2.3cm, outer=1.8 cm, bmargin=2cm, tmargin=1.8cm}
\setlength{\parindent}{0bp}
\usepackage{import}
\usepackage[subpreambles=false]{standalone}
\usepackage{amsmath}
\usepackage{amssymb}
\usepackage{esint}
\usepackage{babel}
\usepackage{tabu}
\makeatother
\makeatletter

\usepackage{titlesec}
\usepackage{ragged2e}
\RaggedRight
\raggedbottom
\frenchspacing

% Norwegian names of figures, chapters, parts and content
\addto\captionsenglish{\renewcommand{\figurename}{Figur}}
\makeatletter
\addto\captionsenglish{\renewcommand{\chaptername}{Kapittel}}
\addto\captionsenglish{\renewcommand{\partname}{Del}}

\addto\captionsenglish{\renewcommand{\contentsname}{Innhold}}

\usepackage{graphicx}
\usepackage{float}
\usepackage{subfig}
\usepackage{placeins}
\usepackage{cancel}
\usepackage{framed}
\usepackage{wrapfig}
\usepackage[subfigure]{tocloft}
\usepackage[font=footnotesize,labelfont=sl]{caption} % Figure caption
\usepackage{bm}
\usepackage[dvipsnames, table]{xcolor}
\definecolor{shadecolor}{rgb}{0.105469, 0.613281, 1}
\colorlet{shadecolor}{Emerald!15} 
\usepackage{icomma}
\makeatother
\usepackage[many]{tcolorbox}
\usepackage{multicol}
\usepackage{stackengine}

% For tabular
\usepackage{array}
\usepackage{multirow}
\usepackage{longtable} %breakable table

% Ligningsreferanser
\usepackage{mathtools}
\mathtoolsset{showonlyrefs}

% index
\usepackage{imakeidx}
\makeindex[title=Indeks]

%Footnote:
\usepackage[bottom, hang, flushmargin]{footmisc}
\usepackage{perpage} 
\MakePerPage{footnote}
\addtolength{\footnotesep}{2mm}
\renewcommand{\thefootnote}{\arabic{footnote}}
\renewcommand\footnoterule{\rule{\linewidth}{0.4pt}}
\renewcommand{\thempfootnote}{\arabic{mpfootnote}}

%colors
\definecolor{c1}{cmyk}{0,0.5,1,0}
\definecolor{c2}{cmyk}{1,0.25,1,0}
\definecolor{n3}{cmyk}{1,0.,1,0}
\definecolor{neg}{cmyk}{1,0.,0.,0}

% Lister med bokstavar
\usepackage{enumitem}

\newcounter{rg}
\numberwithin{rg}{chapter}
\newcommand{\reg}[2][]{\begin{tcolorbox}[boxrule=0.3 mm,arc=0mm,colback=blue!3] {\refstepcounter{rg}\phantomsection \large \textbf{\therg \;#1} \vspace{5 pt}}\newline #2  \end{tcolorbox}\vspace{-5pt}}

\newcommand\alg[1]{\begin{align} #1 \end{align}}

\newcommand\eks[2][]{\begin{tcolorbox}[boxrule=0.3 mm,arc=0mm,enhanced jigsaw,breakable,colback=green!3] {\large \textbf{Eksempel #1} \vspace{5 pt}\\} #2 \end{tcolorbox}\vspace{-5pt} }

\newcommand{\st}[1]{\begin{tcolorbox}[boxrule=0.0 mm,arc=0mm,enhanced jigsaw,breakable,colback=yellow!12]{ #1} \end{tcolorbox}}

\newcommand{\spr}[1]{\begin{tcolorbox}[boxrule=0.3 mm,arc=0mm,enhanced jigsaw,breakable,colback=yellow!7] {\large \textbf{Språkboksen} \vspace{5 pt}\\} #1 \end{tcolorbox}\vspace{-5pt} }

\newcommand{\sym}[1]{\colorbox{blue!15}{#1}}

\newcommand{\info}[2]{\begin{tcolorbox}[boxrule=0.3 mm,arc=0mm,enhanced jigsaw,breakable,colback=cyan!6] {\large \textbf{#1} \vspace{5 pt}\\} #2 \end{tcolorbox}\vspace{-5pt} }

\newcommand\algv[1]{\vspace{-11 pt}\begin{align*} #1 \end{align*}}

\newcommand{\regv}{\vspace{5pt}}
\newcommand{\mer}{\textsl{Merk}: }
\newcommand\vsk{\vspace{11pt}}
\newcommand\vs{\vspace{-11pt}}
\newcommand\vsb{\vspace{-16pt}}
\newcommand\sv{\vsk \textbf{Svar:} \vspace{4 pt}\\}
\newcommand\br{\\[5 pt]}
\newcommand{\asym}[1]{../fig/#1}
\newcommand\algvv[1]{\vs\vs\begin{align*} #1 \end{align*}}
\newcommand{\y}[1]{$ {#1} $}
\newcommand{\os}{\\[5 pt]}
\newcommand{\prbxl}[2]{
\parbox[l][][l]{#1\linewidth}{#2
	}}
\newcommand{\prbxr}[2]{\parbox[r][][l]{#1\linewidth}{
		\setlength{\abovedisplayskip}{5pt}
		\setlength{\belowdisplayskip}{5pt}	
		\setlength{\abovedisplayshortskip}{0pt}
		\setlength{\belowdisplayshortskip}{0pt} 
		\begin{shaded}
			\footnotesize	#2 \end{shaded}}}

\renewcommand{\cfttoctitlefont}{\Large\bfseries}
\setlength{\cftaftertoctitleskip}{0 pt}
\setlength{\cftbeforetoctitleskip}{0 pt}

\newcommand{\bs}{\\[3pt]}
\newcommand{\vn}{\\[6pt]}
\newcommand{\fig}[1]{\begin{figure}
		\centering
		\includegraphics[]{\asym{#1}}
\end{figure}}
\newcommand{\net}[2]{{\color{blue}\href{#1}{#2}}}

\newcommand{\hrs}[2]{\hyperref[#1]{\color{blue}\textsl{#2 \ref*{#1}}}}
\newcommand{\rref}[1]{\hyperref[#1]{\color{blue}\textsl{Regel \ref*{#1}}}}

\newcommand{\sectionbreak}{\clearpage} % New page on each section

% Equation comments
\newcommand{\cm}[1]{\llap{\color{blue} #1}}

\newcommand\fork[2]{\begin{tcolorbox}[boxrule=0.3 mm,arc=0mm,enhanced jigsaw,breakable,colback=yellow!7] {\large \textbf{#1 (forklaring)} \vspace{5 pt}\\} #2 \end{tcolorbox}\vspace{-5pt} }


%%% SECTION HEADLINES %%%

% Our numbers
\newcommand{\likteikn}{Likhetstegnet}
\newcommand{\talsifverd}{Tall, siffer og verdi}
\newcommand{\koordsys}{Koordinatsystem}

% Calculations
\newcommand{\adi}{Addisjon}
\newcommand{\sub}{Subtraksjon}
\newcommand{\gong}{Multiplikasjon (Gonging)}
\newcommand{\del}{Divisjon (deling)}

%Factorization and order of operations
\newcommand{\fak}{Faktorisering}
\newcommand{\rrek}{Regnerekkefølge}

%Fractions
\newcommand{\brgrpr}{Introduksjon}
\newcommand{\brvu}{Verdi, utviding og forkorting av brøk}
\newcommand{\bradsub}{Addisjon og subtraksjon}
\newcommand{\brgngheil}{Brøk ganget med heltall}
\newcommand{\brdelheil}{Brøk delt med heltall}
\newcommand{\brgngbr}{Brøk ganget med brøk}
\newcommand{\brkans}{Kansellering av faktorer}
\newcommand{\brdelmbr}{Deling med brøk}
\newcommand{\Rasjtal}{Rasjonale tall}

%Negative numbers
\newcommand{\negintro}{Introduksjon}
\newcommand{\negrekn}{De fire regneartane med negative tall}
\newcommand{\negmeng}{Negative tall som mengde}

% Geometry 1
\newcommand{\omgr}{Begrep}
\newcommand{\eignsk}{Egenskaper for trekanter og firkanter}
\newcommand{\omkr}{Omkrets}
\newcommand{\area}{Areal}

%Algebra 
\newcommand{\algintro}{Introduksjon}
\newcommand{\pot}{Potenser}
\newcommand{\irrasj}{Irrasjonale tall}

%Equations
\newcommand{\ligintro}{Introduksjon}
\newcommand{\liglos}{Løsing ved de fire regneartene}
\newcommand{\ligloso}{Løsingsmetodene oppsummert}

%Functions
\newcommand{\fintro}{Introduksjon}
\newcommand{\lingraf}{Lineære funksjoner og grafer}

%Geometry 2
\newcommand{\geoform}{Formler for areal og omkrets}
\newcommand{\kongogsim}{Kongruente og formlike trekanter}
\newcommand{\geofork}{Forklaringar}

% Names of rules
\newcommand{\adkom}{Addisjon er kommutativ}
\newcommand{\gangkom}{Multiplikasjon er kommutativ}
\newcommand{\brdef}{Brøk som omskriving av delestykke}
\newcommand{\brtbr}{Brøk ganget med brøk}
\newcommand{\delmbr}{Brøk delt på brøk}
\newcommand{\gangpar}{Ganging med parentes (distributiv lov)}
\newcommand{\gangparsam}{Paranteser ganget sammen}
\newcommand{\gangmnegto}{Ganging med negative tall I}
\newcommand{\gangmnegtre}{Ganging med negative tall II}
\newcommand{\konsttre}{Konstruksjon av trekanter}
\newcommand{\kongtre}{Kongruente trekanter}
\newcommand{\topv}{Toppvinkler}
\newcommand{\trisum}{Summen av vinklene i en trekant}
\newcommand{\firsum}{Summen av vinklene i en firkant}
\newcommand{\potgang}{Ganging med potenser}
\newcommand{\potdivpot}{Divisjon med potenser}
\newcommand{\potanull}{Spesialtilfellet \boldmath $a^0$}
\newcommand{\potneg}{Potens med negativ eksponent}
\newcommand{\potbr}{Brøk som grunntall}
\newcommand{\faktgr}{Faktorer som grunntall}
\newcommand{\potsomgrunn}{Potens som grunntall}
\newcommand{\arsirk}{Arealet til en sirkel}
\newcommand{\artrap}{Arealet til et trapes}
\newcommand{\arpar}{Arealet til et parallellogram}
\newcommand{\pyt}{Pytagoras' setning}
\newcommand{\forform}{Forhold i formlike trekanter}
\newcommand{\vilkform}{Vilkår i formlike trekanter}
\newcommand{\omkrsirk}{Omkretsen til en sirkel (og $ \bm \pi $)}
\newcommand{\artri}{Arealet til en trekant}
\newcommand{\arrekt}{Arealet til et rektangel}
\newcommand{\liknflyt}{Flytting av ledd over likhetstegnet}
\newcommand{\funklin}{Lineære funksjoner}

%Opg
% Opg
\newcommand{\abc}[1]{
	\begin{enumerate}[label=\alph*),leftmargin=18pt]
		#1
	\end{enumerate}
}
\newcommand{\opgt}{\phantomsection \addcontentsline{toc}{section}{Oppgaver} \section*{Oppgaver for kapittel \thechapter}\vs \setcounter{section}{1}}
\newcounter{opg}
\numberwithin{opg}{section}
\newcommand{\op}[1]{\vspace{15pt} \refstepcounter{opg}\large \textbf{\color{blue}\theopg} \vspace{2 pt} \label{#1} \\}
\newcommand{\ekspop}{\vsk\textbf{Gruble \thechapter}\vspace{2 pt} \\}
\newcommand{\nes}{\stepcounter{section}
	\setcounter{opg}{0}}
\newcommand{\opr}[1]{\vspace{3pt}\textbf{\ref{#1}}}

%License
\newcommand{\lic}{\textit{Matematikken sine byggesteiner by Sindre Sogge Heggen is licensed under CC BY-NC-SA 4.0. To view a copy of this license, visit\\ 
		\net{http://creativecommons.org/licenses/by-nc-sa/4.0/}{http://creativecommons.org/licenses/by-nc-sa/4.0/}}}

\usepackage{datetime2}
\usepackage[]{hyperref}


\begin{document}

\section{\negintro}
Vi har tidligere sett at (for eksempel) tallet 5 på ei tallinje ligger 5 enerlengder til høyre for 0. 
\fig{neg1}
Men hva om vi går andre veien, altså mot venstre? Dette spørsmålet svarer vi på ved å innføre \textit{negative tall}\index{tall!negativt}\index{tall!positivt}.\regv

\reg[Positive og negative tal]{
På en talinje gjelder følgende:
\begin{itemize}
	\item Tall plassert \textsl{til høyre} for 0 er positive tall.
	\item Tall plassert \textsl{til venstre} for 0 er negative tal.
\end{itemize}
\fig{neg2}
}\vsk 
I praksis kan vi ikke hele tiden bruke en tallinje for å avgjøre om et tall er negativt eller positivt, og derfor bruker vi et symbol for å vise at tall er negative. Dette symbolet er rett og slett \sym{$ - $}, altså det samme symbolet som vi bruker ved subtraksjon. $ 5 $ er med dét et positivt tall, mens $ -5 $ er et negativt tall. På tallinja er det slik at
\begin{itemize}
	\item 5 ligg 5 enerlengder \textsl{til høyre} for 0.
	\item $ -5 $ ligg 5 enerlengder \textsl{til venstre} for 0.
\end{itemize}
\fig{neg3}
Den store forskjellen på $ 5 $ og $ -5 $ er altså på hvilken side av 0 tallene ligger. Da $ 5 $ og $ -5 $ har samme avstand til 0, sier vi at $ 5 $ og $ -5 $ har samme \textit{lengde}\index{lengde}. \regv

\reg[Lengde (tallverdi/absoluttverdi)\index{tallverdi}\index{absoluttverdi}]{
Lengden til et tall skrives ved symbolet \sym{| |}.\vsk	
	
Lengden til et positivt tall er verdien til tallet.\vsk
	
Lengden til et negativt tall er verdien til det positive tallet med samme siffer.
} 
\eks[1]{ \vs \vs
\[ |27|=27 \]
}
\eks[2]{ \vs \vs
\[ |-27|=27 \]
}
\info{Fortegn}{
\textit{Fortegn}\index{fortegn} er en samlebetegnelse for \sym{$ + $} og \sym{$ - $}. $ 5 $ har \sym{$ + $} som fortegn og $ -5 $ har \sym{$ - $} som fortegn.
}
\newpage
\section{\negrekn \label{rekmneg}}
Ved innføringen av negative tall får de fire regneartene nye sider som vi må se på trinnvis. Når vi adderer, subtraherer, multipliserer eller dividerer med negative tall vil vi ofte, for tydeligheten sin skyld, skrive negative tall med parentes rundt. Da skriver vi for eksempel $ -4 $ som $ (-4) $. 

\subsection*{Addisjon}
Når vi adderte i \hrs{Addisjon}{seksjon} så vi på \sym{$ + $} som vandring \textsl{mot høyre}. Negative tall gjør at vi må utvide begrepet for \sym{$ + $}\,: \regv
\st{\begin{center}
		\begin{tabular}{cl}
			$ + $&''Like langt og i \textsl{samme} retning som''	
		\end{tabular}
\end{center}}
La oss se på regnestykket
\[ 7+(-4) \]
Vår utvidede definisjon av \sym{$ + $} sier oss nå at
\[ 7+(-4)=\text{''}7 \text{ og like langt og i \textsl{samme} retning som } (-4)\text{''} \]
$ (-4) $ har lengde 4 og retning \textsl{mot venstre}. Vårt regnestykke sier altså at vi skal starte på 7, og deretter gå lengden 4 \textsl{mot venstre}.
\[ 7+(-4)=3 \]
\fig{neg6}

\reg[Addisjon med negative tall]{
Å addere et negativt tall er det samme som å subtrahere tallet med samme tallverdi.
}
\eks[1]{ \vs
	\[ 4+(-3)=4-3=1 \]
}
\eks[2]{ \vs
	\[ -8+(-3)=-8-3=-11 \]
}
\info{Merk}{
\rref{adkom} erklærer at addisjon er kommutativ. Dette er gjeldende også ved innføringen av negative tall, for eksempel er	
\[ 7+(-3)=4=-3+7 \]	
}

\subsection*{Subtraksjon}
I \hrs{Subtraksjon}{seksjon} så vi på \sym{$ - $} som vandring \textsl{mot venstre}. Også betydningen av \sym{$ - $} må utvides når vi jobber med negative tall:\regv

\st{\begin{center}
	\begin{tabular}{cl}
		$ - $&''Like langt og i \textsl{motsatt} retning som''	
	\end{tabular}
\end{center}}
La oss se på regnestykket
\[ 2-(-6) \]
Med vår utvidede betydning av \sym{$-$}, kan vi skrive
\[ 2-(-6)=\text{''}2 \text{ og like langt og i \textsl{motsatt} retning som } (-6)\text{''} \]
$ -6 $ har lengde 6 og retning \textsl{mot venstre}. Når vi skal gå samme lengde, men i \textsl{motsatt} retning, må vi altså gå lengden 6 \textsl{mot høyre}\footnote{Vi minner enda en gang om at rødfargen på pila indikerer at man skal vandre fra pilspissen til andre enden.}. Dette er det samme som å addere 6:
\[ 2-(-6)=2+6=8 \] \vs
\fig{neg7}
\reg[Subtraksjon med negative tall]{
Å subtrahere et negativt tall er det samme som å addere det positive tallet med samme tallverdi.
}
\eks[1]{ \vs
\[ 11-(-9)=11+9=20 \]
}
\eks[2]{ \vs
	\[ -3-(-7)=-3+7=4 \]
}
\begin{comment}
	\info{Merk}{
	Med innføringen av negative tall kan subtraksjon bli sett på som addisjon med negative tal. For eksempel er $ {7-3=7+(-3)} $ og $ {2-7=2+(-7)} $.
	}
\end{comment}

\subsection*{Multiplikasjon}
I \hrs{Gonging}{seksjon} introduserte vi ganging med positive heltall som gjentatt addisjon. Med våre utvidede begrep av addisjon og subtraksjon, kan vi nå også utvide begrepet multiplikasjon: \regv

\reg[Multiplikasjon med positive og negative tall\label{gongneg}]{ \vs
\begin{itemize}
		\item Ganging med et positivt heltall er det samme som gjentatt addisjon.
		\item Ganging med et negativt heltal er det samme som gjentatt subtraksjon.
\end{itemize}}
\eks[1]{ \vs \vs \label{negeksempel}
\alg{
	2\cdot3 &=\text{''Like langt og i \textsl{samme} retning som 2, 3 ganger''}\\
	&=2+2+2 \\
	&=6
}
}
\eks[2]{ \vs \vs
\alg{
	(-2)\cdot3&=\text{''Like langt og i \textsl{samme} retning som }(-2) \text{, 3 ganger''} \\[-12pt] 
	&=-2-2-2\\
	&=-6
}
}
\eks[3]{ \vs \vs
\alg{
	2\cdot(-3)&=\text{''Like langt og i \textsl{motsatt} retning som 2, 3 ganger''} \\
	&=-2-2-2 \\
	&=-6
}
}
\eks[4]{ \vs \vs
	\alg{
		(-3)\cdot(-4)&=\text{''Like langt og i \textsl{motsatt} retning som $ -3 $, 4 ganger''}\\[-12pt] 
		&=3+3+3+3\\
		&=12
	}	 	 
}
\info{Multiplikasjon er kommutativ}{
\textsl{Eksempel 2} og \textsl{Eksempel 3} på side \pageref{negeksempel} illustrerer at \rref{gangkom} også er gjeldende ved innføringen av negative tall:
\[ (-2)\cdot3=3\cdot(-2) \]
}  \vsk
Det blir tungvint å regne ganging som gjentatt addisjon/subtraksjon hver gang vi har et negativt tall involvert, men som en direkte konsekvens av \rref{gongneg} kan vi lage oss følgende to regler:\regv

\reg[\gangmnegto \label{gangmnegto}]{
Produktet av et negativt og et positivt tall er et negativt tall. \vsk

Tallverdien til faktorene ganget sammen gir tallverdien til\\ produktet.
}
\eks[1]{
Regn ut $ (-7)\cdot8 $

\sv
Siden $ 7\cdot8=56 $, er $ (-7)\cdot8=-56 $
}
\eks[2]{
	Regn ut $ 3\cdot(-9) $.
	
	\sv
	Siden $ 3\cdot9=27 $, er $ 3\cdot(-9)=-27 $
}
\vsk

\reg[\gangmnegtre \label{gangmnegtre}]{
Produktet av to negative tall er et positivt tall. \vsk

Tallverdien til faktorene ganget sammen gir verdien til produktet.
} 
\eks[1]{ \vsb
\[ (-5)\cdot(-10)=5\cdot10=50 \]
}
\eks[2]{
\vsb
\[ (-2)\cdot(-8)=2\cdot8=16 \]
}
\subsection*{Divisjon}
Definisjonen av divisjon (se \hrs{Divisjon}{seksjon}), kombinert med det vi vet om multiplikasjon med negative tall, gir oss nå dette:\regv

\st{\begin{center}
		\algv{
			-18:6=\text{''Tallet jeg må gange 6 med for å få $ -18 $''}
		}
		$ 6\cdot(-3)=-18 $, altså er $ -18:6=-3 $
\end{center}} \vsk
\st{
\begin{center}
	\algv{
		42:(-7)=\text{''Tallet jeg må gange $ -7 $ med for å få 42''}
	}
	$ (-7)\cdot(-8)=42 $, altså er $ 42:(-7)=-8 $
\end{center}
}\vsk
\st{
\begin{center}
	\algv{
		-45:(-5)=\text{''Tallet jeg må gange $ -5 $ med for å få $ -45 $''}
	}
	$ (-5)\cdot9=-45 $, altså er $ -45:(-5)=9 $
\end{center}
 }\vsk
\reg[Divisjon med negative tal]{
Divisjon mellom et positivt og et negativt tall gir et negativt tal.\vsk

Divisjon mellom to negative tall gir et positivt tall. \vsk

Tallverdien til dividenden delt med tallverdien til divisoren gir tallverdien til kvotienten. 
}
\eks[1]{ \vsb
	\[ -24:6=-4 \]
}
\eks[2]{ \vsb
	\[ 24:(-2)=-12 \]
}
\eks[3]{ \vsb
	\[ -24:(-3)=8 \]
}
\eks[4]{ \vsb
\[ \frac{2}{-3}=-\frac{2}{3} \]
}
\eks[5]{ \vsb
	\[ \frac{-10}{7}=-\frac{10}{7} \]
}
\newpage

\newpage
\section{\negmeng \label{negmeng}}
\textit{Obs! Denne tolkningen av negative tall blir først brukt i \hrs{likloysfire}{seksjon}, som er en seksjon noen lesere uten tap av forståelse kan hoppe over.}\vsk

Så langt har vi sett på negative tall ved hjelp av tallinjer. Å se på negative tall ved hjelp av mengder er i første omgang vanskelig, fordi vi har likestilt mengder med antall, og negative antall gir ikke mening! For å skape en forståelse av negative tall ut ifra et mengdeperspektiv, bruker vi det vi skal kalle \textit{vektprinsippet}. Dette innebærer at vi ser på tallene som krefter. De positive tallene er antall krefter som virker nedover og de negative tallene er antall krefter som virker oppover\footnote{Fra virkeligheten kan man se på de positive og negative tallene som ballonger fylt med henholdsvis luft og helium. Ballonger fylt med luft virker med en kraft nedover (de faller), mens heliumballonger virker med en kraft oppover (de stiger).}. Svarene på regnestykker med positive og negative tall kan man da se på som resultatet av en veiing av de forskjellige mengdene. Slik vil altså et positivt tall og et negativt tall med samme tallverdi \textsl{utligne} hverandre.
\vsk

\reg[Negative tall som mengde]{
Negative tall vil vi indikere som en lyseblå mengde:
\fig{negm1}
}
\eks[]{
\[ 1+(-1)=0 \]
\fig{negm2}
}

\end{document}


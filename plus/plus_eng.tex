\input{../doc}
\input{../preamb_eng}
\begin{document}
\newpage
\section{\adi \label{Addisjon}}	
\subsection*{Addition with amounts}
When we have an amount and wish to add more we use the symbol \sym{$ + $}. If we have 2 and want to add 3, we write
\[ 2+3=5 \]
\fig{plusm1}
The order in which we add have no impact on the outcome; starting off with 2 and then adding 3 is the same as starting off with 3 and then adding 2:
\[ 3+2=5 \]
\fig{plusm1a}
\spr{
	A calculation involving addition includes two or more \textit{terms}\index{term} and one \textit{sum}\index{sum}. In the calculation
	\[ 2+3=5 \]
	both $ 2 $ and $ 3 $ are terms while $ 5 $ is the sum.\vsk
	
	Common ways of saying $ 2+3 $ are
	\begin{itemize}
		\item ''2 plus 3''
		\item ''2 added to 3''
	 	\item ''2 and 3 added''
	\end{itemize}

} \vsk
\newpage
\reg[Addition is commutative \label{adkom}]{
Summen er den same uansett rekkefølge på ledda.
}

\eks{ \vs \vsb
\alg{
2+5 &= 7 =5+2  \vn
6+3 &=9=3+6
}
} 

\subsection*{Addition on the number line: moving to the right}
On a number line addition with positive numbers involves moving \textsl{to \\the right}:\regv
\eks[1]{
\[ 2+7=9 \]
\fig{plus2}
}
\eks[2]{
	\[ 4+11=15 \]
\fig{plus3}
}
\info{Interpretation of \sym{$\bm=$}}{
\sym{+} gives us the possibility of expressing numbers in different ways, for example is $ {5=2+3} $ and $ {5=1+4} $. In this context \sym{=} means ''have the same value as''. This will also be the case for subtraction, multiplication and divison, which we'll look at in the next three sections.
}


\end{document}
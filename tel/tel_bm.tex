\documentclass[english,hidelinks,pdftex, 11 pt, class=report,crop=false]{standalone}
\usepackage[T1]{fontenc}
\usepackage[utf8]{luainputenc}
\usepackage{lmodern} % load a font with all the characters
\usepackage{geometry}
\geometry{verbose,paperwidth=16.1 cm, paperheight=24 cm, inner=2.3cm, outer=1.8 cm, bmargin=2cm, tmargin=1.8cm}
\setlength{\parindent}{0bp}
\usepackage{import}
\usepackage[subpreambles=false]{standalone}
\usepackage{amsmath}
\usepackage{amssymb}
\usepackage{esint}
\usepackage{babel}
\usepackage{tabu}
\makeatother
\makeatletter

\usepackage{titlesec}
\usepackage{ragged2e}
\RaggedRight
\raggedbottom
\frenchspacing

% Norwegian names of figures, chapters, parts and content
\addto\captionsenglish{\renewcommand{\figurename}{Figur}}
\makeatletter
\addto\captionsenglish{\renewcommand{\chaptername}{Kapittel}}
\addto\captionsenglish{\renewcommand{\partname}{Del}}

\addto\captionsenglish{\renewcommand{\contentsname}{Innhald}}

\usepackage{graphicx}
\usepackage{float}
\usepackage{subfig}
\usepackage{placeins}
\usepackage{cancel}
\usepackage{framed}
\usepackage{wrapfig}
\usepackage[subfigure]{tocloft}
\usepackage[font=footnotesize,labelfont=sl]{caption} % Figure caption
\usepackage{bm}
\usepackage[dvipsnames, table]{xcolor}
\definecolor{shadecolor}{rgb}{0.105469, 0.613281, 1}
\colorlet{shadecolor}{Emerald!15} 
\usepackage{icomma}
\makeatother
\usepackage[many]{tcolorbox}
\usepackage{multicol}
\usepackage{stackengine}

% For tabular
\usepackage{array}
\usepackage{multirow}
\usepackage{longtable} %breakable table

% Ligningsreferanser
\usepackage{mathtools}
\mathtoolsset{showonlyrefs}

% index
\usepackage{imakeidx}
\makeindex[title=Indeks]

%Footnote:
\usepackage[bottom, hang, flushmargin]{footmisc}
\usepackage{perpage} 
\MakePerPage{footnote}
\addtolength{\footnotesep}{2mm}
\renewcommand{\thefootnote}{\arabic{footnote}}
\renewcommand\footnoterule{\rule{\linewidth}{0.4pt}}
\renewcommand{\thempfootnote}{\arabic{mpfootnote}}

%colors
\definecolor{c1}{cmyk}{0,0.5,1,0}
\definecolor{c2}{cmyk}{1,0.25,1,0}
\definecolor{n3}{cmyk}{1,0.,1,0}
\definecolor{neg}{cmyk}{1,0.,0.,0}

% Lister med bokstavar
\usepackage[inline]{enumitem}

\newcounter{rg}
\numberwithin{rg}{chapter}
\newcommand{\reg}[2][]{\begin{tcolorbox}[boxrule=0.3 mm,arc=0mm,colback=blue!3] {\refstepcounter{rg}\phantomsection \large \textbf{\therg \;#1} \vspace{5 pt}}\newline #2  \end{tcolorbox}\vspace{-5pt}}

\newcommand\alg[1]{\begin{align} #1 \end{align}}

\newcommand\eks[2][]{\begin{tcolorbox}[boxrule=0.3 mm,arc=0mm,enhanced jigsaw,breakable,colback=green!3] {\large \textbf{Eksempel #1} \vspace{5 pt}\\} #2 \end{tcolorbox}\vspace{-5pt} }

\newcommand{\st}[1]{\begin{tcolorbox}[boxrule=0.0 mm,arc=0mm,enhanced jigsaw,breakable,colback=yellow!12]{ #1} \end{tcolorbox}}

\newcommand{\spr}[1]{\begin{tcolorbox}[boxrule=0.3 mm,arc=0mm,enhanced jigsaw,breakable,colback=yellow!7] {\large \textbf{Språkboksen} \vspace{5 pt}\\} #1 \end{tcolorbox}\vspace{-5pt} }

\newcommand{\sym}[1]{\colorbox{blue!15}{#1}}

\newcommand{\info}[2]{\begin{tcolorbox}[boxrule=0.3 mm,arc=0mm,enhanced jigsaw,breakable,colback=cyan!6] {\large \textbf{#1} \vspace{5 pt}\\} #2 \end{tcolorbox}\vspace{-5pt} }

\newcommand\algv[1]{\vspace{-11 pt}\begin{align*} #1 \end{align*}}

\newcommand{\regv}{\vspace{5pt}}
\newcommand{\mer}{\textsl{Merk}: }
\newcommand\vsk{\vspace{11pt}}
\newcommand\vs{\vspace{-11pt}}
\newcommand\vsb{\vspace{-16pt}}
\newcommand\sv{\vsk \textbf{Svar:} \vspace{4 pt}\\}
\newcommand\br{\\[5 pt]}
\newcommand{\asym}[1]{../fig/#1}
\newcommand\algvv[1]{\vs\vs\begin{align*} #1 \end{align*}}
\newcommand{\y}[1]{$ {#1} $}
\newcommand{\os}{\\[5 pt]}
\newcommand{\prbxl}[2]{
\parbox[l][][l]{#1\linewidth}{#2
	}}
\newcommand{\prbxr}[2]{\parbox[r][][l]{#1\linewidth}{
		\setlength{\abovedisplayskip}{5pt}
		\setlength{\belowdisplayskip}{5pt}	
		\setlength{\abovedisplayshortskip}{0pt}
		\setlength{\belowdisplayshortskip}{0pt} 
		\begin{shaded}
			\footnotesize	#2 \end{shaded}}}

\renewcommand{\cfttoctitlefont}{\Large\bfseries}
\setlength{\cftaftertoctitleskip}{0 pt}
\setlength{\cftbeforetoctitleskip}{0 pt}

\newcommand{\bs}{\\[3pt]}
\newcommand{\vn}{\\[6pt]}
\newcommand{\fig}[1]{\begin{figure}
		\centering
		\includegraphics[]{\asym{#1}}
\end{figure}}


\newcommand{\sectionbreak}{\clearpage} % New page on each section

% Equation comments
\newcommand{\cm}[1]{\llap{\color{blue} #1}}

\newcommand\fork[2]{\begin{tcolorbox}[boxrule=0.3 mm,arc=0mm,enhanced jigsaw,breakable,colback=yellow!7] {\large \textbf{#1 (forklaring)} \vspace{5 pt}\\} #2 \end{tcolorbox}\vspace{-5pt} }




%colors
\newcommand{\colr}[1]{{\color{red} #1}}
\newcommand{\colb}[1]{{\color{blue} #1}}
\newcommand{\colo}[1]{{\color{orange} #1}}
\newcommand{\colc}[1]{{\color{cyan} #1}}
\definecolor{projectgreen}{cmyk}{100,0,100,0}
\newcommand{\colg}[1]{{\color{projectgreen} #1}}

%%% SECTION HEADLINES %%%

% Our numbers
\newcommand{\likteikn}{Likskapsteiknet}
\newcommand{\talsifverd}{Tal, siffer og verdi}
\newcommand{\koordsys}{Koordinatsystem}

% Calculations
\newcommand{\adi}{Addisjon}
\newcommand{\sub}{Subtraksjon}
\newcommand{\gong}{Multiplikasjon (Gonging)}
\newcommand{\del}{Divisjon (deling)}

%Factorization and order of operations
\newcommand{\fak}{Faktorisering}
\newcommand{\rrek}{Reknerekkefølge}

%Fractions
\newcommand{\brgrpr}{Introduksjon}
\newcommand{\brvu}{Verdi, utviding og forkorting av brøk}
\newcommand{\bradsub}{Addisjon og subtraksjon}
\newcommand{\brgngheil}{Brøk gonga med heiltal}
\newcommand{\brdelheil}{Brøk delt med heiltal}
\newcommand{\brgngbr}{Brøk gonga med brøk}
\newcommand{\brkans}{Kansellering av faktorar}
\newcommand{\brdelmbr}{Deling med brøk}
\newcommand{\Rasjtal}{Rasjonale tal}

%Negative numbers
\newcommand{\negintro}{Introduksjon}
\newcommand{\negrekn}{Dei fire rekneartane med negative tal}
\newcommand{\negmeng}{Negative tal som mengde}

% Geometry 1
\newcommand{\omgr}{Omgrep}
\newcommand{\eignsk}{Eigenskapar for trekantar og firkantar}
\newcommand{\omkr}{Omkrins}
\newcommand{\area}{Areal}

%Algebra 
\newcommand{\algintro}{Introduksjon}
\newcommand{\pot}{Potensar}
\newcommand{\irrasj}{Irrasjonale tal}

%Equations
\newcommand{\ligintro}{Introduksjon}
\newcommand{\liglos}{Løysing ved dei fire rekneartane}
\newcommand{\ligloso}{Løysingsmetodane oppsummert}

%Functions
\newcommand{\fintro}{Introduksjon}
\newcommand{\lingraf}{Lineære funksjonar og grafar}

%Geometry 2
\newcommand{\geoform}{Formlar for areal og omkrins}
\newcommand{\kongogsim}{Kongruente og formlike trekantar}
\newcommand{\geofork}{Forklaringar}

% Names of rules
\newcommand{\gangdestihundre}{Å gange desimaltall med 10, 100 osv.}
\newcommand{\delmedtihundre}{Deling med 10, 100, 1\,000 osv.}
\newcommand{\ompref}{Omgjøring av prefikser}
\newcommand{\adkom}{Addisjon er kommutativ}
\newcommand{\gangkom}{Multiplikasjon er kommutativ}
\newcommand{\brdef}{Brøk som omskriving av delestykke}
\newcommand{\brtbr}{Brøk gonga med brøk}
\newcommand{\delmbr}{Brøk delt på brøk}
\newcommand{\gangpar}{Gonging med parentes (distributiv lov)}
\newcommand{\gangparsam}{Parantesar gonga saman}
\newcommand{\gangmnegto}{Gonging med negative tal I}
\newcommand{\gangmnegtre}{Gonging med negative tal II}
\newcommand{\konsttre}{Konstruksjon av trekantar}
\newcommand{\kongtre}{Kongruente trekantar}
\newcommand{\topv}{Toppvinklar}
\newcommand{\trisum}{Summen av vinklane i ein trekant}
\newcommand{\firsum}{Summen av vinklane i ein firkant}
\newcommand{\potgang}{Gonging med potensar}
\newcommand{\potdivpot}{Divisjon med potensar}
\newcommand{\potanull}{Spesialtilfellet \boldmath $a^0$}
\newcommand{\potneg}{Potens med negativ eksponent}
\newcommand{\potbr}{Brøk som grunntal}
\newcommand{\faktgr}{Faktorar som grunntal}
\newcommand{\potsomgrunn}{Potens som grunntal}
\newcommand{\arsirk}{Arealet til ein sirkel}
\newcommand{\artrap}{Arealet til eit trapes}
\newcommand{\arpar}{Arealet til eit parallellogram}
\newcommand{\pyt}{Pytagoras' setning}
\newcommand{\forform}{Forhold i formlike trekantar}
\newcommand{\vilkform}{Vilkår i formlike trekantar}
\newcommand{\omkrsirk}{Omkrinsen til ein sirkel (og $ \bm \pi $)}
\newcommand{\artri}{Arealet til ein trekant}
\newcommand{\arrekt}{Arealet til eit rektangel}
\newcommand{\liknflyt}{Flytting av ledd over likskapsteiknet}
\newcommand{\funklin}{Lineære funksjonar}

%Opg
\newcommand{\abc}[1]{
	\begin{enumerate}[label=\alph*),leftmargin=18pt]
		#1
	\end{enumerate}
}
\newcommand{\abcs}[2]{
	\begin{enumerate}[label=\alph*),start=#1,leftmargin=18pt]
		#2
	\end{enumerate}
}
\newcommand{\abcn}[1]{
	\begin{enumerate}[label=\arabic*),leftmargin=18pt]
		#1
	\end{enumerate}
}
\newcommand{\abch}[1]{
	\hspace{-2pt}	\begin{enumerate*}[label=\alph*), itemjoin=\hspace{1cm}]
		#1
	\end{enumerate*}
}
\newcommand{\abchs}[2]{
	\hspace{-2pt}	\begin{enumerate*}[label=\alph*), itemjoin=\hspace{1cm}, start=#1]
		#2
	\end{enumerate*}
}

\newcommand{\opgt}{\phantomsection \addcontentsline{toc}{section}{Oppgaver} \section*{Oppgaver for kapittel \thechapter}\vs \setcounter{section}{1}}
\newcounter{opg}
\numberwithin{opg}{section}
\newcommand{\op}[1]{\vspace{15pt} \refstepcounter{opg}\large \textbf{\color{blue}\theopg} \vspace{2 pt} \label{#1} \\}
\newcommand{\ekspop}[1]{\vsk\textbf{Gruble \thechapter.#1}\vspace{2 pt} \\}
\newcommand{\nes}{\stepcounter{section}
	\setcounter{opg}{0}}
\newcommand{\opr}[1]{\vspace{3pt}\textbf{\ref{#1}}}
\newcommand{\oeks}[1]{\begin{tcolorbox}[boxrule=0.3 mm,arc=0mm,colback=white]
		\textit{Eksempel: } #1	  
\end{tcolorbox}}
\newcommand\opgeks[2][]{\begin{tcolorbox}[boxrule=0.1 mm,arc=0mm,enhanced jigsaw,breakable,colback=white] {\footnotesize \textbf{Eksempel #1} \\} \footnotesize #2 \end{tcolorbox}\vspace{-5pt} }

%License
\newcommand{\lic}{\textit{Matematikken sine byggesteinar by Sindre Sogge Heggen is licensed under CC BY-NC-SA 4.0. To view a copy of this license, visit\\ 
		\net{http://creativecommons.org/licenses/by-nc-sa/4.0/}{http://creativecommons.org/licenses/by-nc-sa/4.0/}}}

%referances
\newcommand{\net}[2]{{\color{blue}\href{#1}{#2}}}
\newcommand{\hrs}[2]{\hyperref[#1]{\color{blue}\textsl{#2 \ref*{#1}}}}
\newcommand{\rref}[1]{\hrs{#1}{Regel}}
\newcommand{\refkap}[1]{\hrs{#1}{Kapittel}}
\newcommand{\refsec}[1]{\hrs{#1}{Seksjon}}

\usepackage{datetime2}

\usepackage[]{hyperref}


\begin{document}
\newpage
\section{Likhetstegnet, mengder og tallinjer}
\index{tall}
\subsection*{\likteikn}
Som navnet tilsier, viser \textit{likhetstegnet} \index{likhetstegnet} \sym{$ = $} til at noe er likt. I hvilken grad og når man kan si at noe er likt er en filosofisk diskusjon, og innledningsvis er vi bare prisgitt dette: Hvilken likhet \sym{$=$} sikter til må bli forstått ut ifra konteksten tegnet blir brukt i. Med denne forstelsen av \sym{$ = $} kan vi studere noen grunnleggende egenskaper for tallene våre, og så komme tilbake til mer presise betydninger av tegnet. \regv
\spr{
Vanlige måter å sei \sym{$=$} på er
\begin{itemize}
	\item ''er lik'' \\
	\item ''er det samme som''
\end{itemize}
}
\subsection*{Mengder og tallinjer}
Tall kan representere så mangt. I denne boka skal vi holde oss til to måter å tolke tallene på; tall som en \textsl{mengde} og tall som en \textsl{plassering på ei linje}. Alle representasjoner av tall tar egentleg utganspunkt i hva forståelsen er av tallene 0 og 1.

\subsubsection*{Tall som mengde}
	Når vi snakkar om en mengde, vil tallet 0 vere\footnote{I \hrs{Rekneartane}{kapittel} skal vi se at det også er andre tolkninger av 0.} knytt til ''ingenting''. En figur der det ikke er noe til stade vil slik vere det samme som 0:
	\[ =0 \]
	1 vil vi tegne som en rute:
	\fig{rut1}

Andre tall vil da vere definert ut ifra hvor mange enerruter (enere) man har:
	\fig{rut2}
\newpage	
\subsubsection*{Tall som plassering på ei linje}
	Når vi plasserer tall på ei linje, vil 0 vere utgangspunket vårt:
	\fig{lin0a}
	Så plasserer vi 1 en viss lengde til høyre for 0:
	\fig{lin0b}
	Andre tall vil da vere definert ut ifra hvor mange enerlengder (enere) vi er unna 0:
	\fig{lin1}
\subsection*{Positive heiltal}
Vi skal straks se at tall ikke nødvendigvis trenger å være \textsl{hele} antal enere, men tallene som \textsl{er} det har et eget navn:\regv

\reg[Positive heiltal]{
Tall som er et helt antall enere kalles \textit{positive\footnotemark heltall}\index{positive heltall}. De\\ positive heltallene er
\[ 1, 2, 3, 4, 5 \text{ og så videre.} \]
Positive heltal blir også kalt \textit{naturlige tal}\index{tall!naturlige}.
}
\info{Hva med 0?}{
Noen forfattare inkluderer også 0 i begrepet naturlige tal. I noen sammenhenger vil dette lønne seg, i andre ikke.
}
\footnotetext{Hva ordet positiv innebærer skal vi gjøre greie for i \hrs{Negtal}{kapittel}.}

\newpage
\section{\talsifverd}
Tallene våre er bygd opp av \textit{sifrene}\index{siffer} $ 0, 1, 2 , 3, 4, 5, 6, 7, 8 $ og $ 9 $, og \textsl{plasseringen} av dem. Sifrene og deres plassering definerer\footnote{Etter hvert skal vi også se at \textit{fortegn} er med på å definere verdien til tallet (se \hrs{Negtal}{kapittel}).} \textit{verdien} \index{verdi} til tallet.
\subsection*{Heiltal større enn 10}
La oss som et eksempel skrive tallet \textsl{fjorten} ved hjelp av sifrene våre.
\fig{tel}
Vi kan nå lage en gruppe med 10 enere, i tillegg har vi da 4 enere. Da skriver vi fjorten slik:
\[ \text{fjorten}=14 \]
\fig{tel2_bm}\vsk

\fig{tel2t}
\newpage
\subsection*{Desimaltall}
I mange tilfeller har vi ikke et helt antall enere, og da vil det vere behov for å dele 1 inn i mindre biter. La oss starte med å tegne en ener:
\fig{maal}
\fig{des1}
Så deler vi eneren vår inn i 10 mindre biter:
\fig{maal1}
\fig{des1a}
Siden vi har delt 1 inn i 10 biter, kaller vi en slik bit for \textit{en tidel}:
\fig{maal1a_bm}
\fig{des1b_bm}
\begin{comment}
\eks{\vs
	\fig{maal2}
	\fig{des2}
}\vsk
\end{comment}
Tideler skriver vi ved hjelp av  \textit{desimaltegnet} \sym{,}  :
\fig{maal1b}
\fig{des1c}
\eks[]{\vs 
	\fig{maal2a}
	\fig{des3}
}\regv
\spr{
På engelsk bruker man punktum \sym{.} som desimaltegn i staden for komma \sym{,}\,:\vsb
\alg{
	3,5&\quad(norsk) \\
	3.5&\quad (english)
}
}
\newpage
\subsection*{Titallssystemet}
Vi har nå sett hvordan vi kan uttrykke verdien til tall ved å plassere \\siffer etter antall tiere, enere og tideler, og det stopper selvsagt ikke der: \regv

\reg[Titalssystemet]{
Verdien til et tall er gitt av siffera $ 0, 1, 2, 3, 4, 5, 6, 7, 8  $ og $ 9 $, og plasseringen av dem. Med sifferet som angir enere som utgangspunkt vil
\begin{itemize}
	\item siffer til venstre (i rekkefølge) indikere antall tiere, \\hundrere, tusener osv.
	\item siffer til høyre (i rekkefølge) indikere antall tideler, \\hundredeler, tusendeler osv.
\end{itemize}
}
\eks[1]{\vs 
	\fig{maal3_bm}
}
\eks[2]{ \vs \vs
\fig{titalsys_bm}
}

\newpage
\section{\koordsys \label{Koord}}

\prbxl{0.5}{I mange tilfeller er det nyttig å bruke to tallinjer samtidig. Dette kallar vi et \textit{koordinatsystem}\index{koordinatsystem}. Vi plasserer da én tallinje som går \textsl{horisontalt} og én som går \textsl{vertikalt}. En plassering i et koordinatsystem kaller vi et \textit{punkt}\index{punkt}. 
 }\qquad
\prbxr{0.4}{Strengt tatt finnes det mange typer koordinatsystem, men i denne boka bruker vi ordet om bare én sort, nemlig det \textit{kartesiske koordinatsystem}. Det er oppkalt etter den franske filosofen og matematikeren René Descartes.}
\st{Et punkt skriver vi som to tall inni en parentes. De to tallene blir kalt \textit{førstekoordinaten} og \textit{andrekoordinaten}.
	\begin{itemize}
		\item Førstekoordinaten forteller oss hvor langt vi skal gå langs horisontalaksen.
		\item Andrekoordinaten forteller oss hvor langt vi skal gå langs vertikalaksen.
\end{itemize}
I figuren ser vi punktene $ (2, 3) $, $ (5, 1) $ og $ (0, 0) $. Punktet der aksene møtes, altså $ (0, 0) $, kalles \textit{origo}.
\fig{kord}
}

\end{document}
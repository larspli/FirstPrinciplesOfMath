\documentclass[english,hidelinks,pdftex, 11 pt, class=report,crop=false]{standalone}
\usepackage[T1]{fontenc}
\usepackage[utf8]{luainputenc}
\usepackage{lmodern} % load a font with all the characters
\usepackage{geometry}
\geometry{verbose,paperwidth=16.1 cm, paperheight=24 cm, inner=2.3cm, outer=1.8 cm, bmargin=2cm, tmargin=1.8cm}
\setlength{\parindent}{0bp}
\usepackage{import}
\usepackage[subpreambles=false]{standalone}
\usepackage{amsmath}
\usepackage{amssymb}
\usepackage{esint}
\usepackage{babel}
\usepackage{tabu}
\makeatother
\makeatletter

\usepackage{titlesec}
\usepackage{ragged2e}
\RaggedRight
\raggedbottom
\frenchspacing

% Norwegian names of figures, chapters, parts and content
\addto\captionsenglish{\renewcommand{\figurename}{Figure}}
\makeatletter
\addto\captionsenglish{\renewcommand{\chaptername}{Chapter}}
%\addto\captionsenglish{\renewcommand{\partname}{Part}}

%\addto\captionsenglish{\renewcommand{\contentsname}{Content}}

\usepackage{graphicx}
\usepackage{float}
\usepackage{subfig}
\usepackage{placeins}
\usepackage{cancel}
\usepackage{framed}
\usepackage{wrapfig}
\usepackage[subfigure]{tocloft}
\usepackage[font=footnotesize,labelfont=sl]{caption} % Figure caption
\usepackage{bm}
\usepackage[dvipsnames, table]{xcolor}
\definecolor{shadecolor}{rgb}{0.105469, 0.613281, 1}
\colorlet{shadecolor}{Emerald!15} 
\usepackage{icomma}
\makeatother
\usepackage[many]{tcolorbox}
\usepackage{multicol}
\usepackage{stackengine}

% For tabular
\usepackage{array}
\usepackage{multirow}
\usepackage{longtable} %breakable table

% Ligningsreferanser
\usepackage{mathtools}
\mathtoolsset{showonlyrefs}

% index
\usepackage{imakeidx}
\makeindex[title=Index]

%Footnote:
\usepackage[bottom, hang, flushmargin]{footmisc}
\usepackage{perpage} 
\MakePerPage{footnote}
\addtolength{\footnotesep}{2mm}
\renewcommand{\thefootnote}{\arabic{footnote}}
\renewcommand\footnoterule{\rule{\linewidth}{0.4pt}}
\renewcommand{\thempfootnote}{\arabic{mpfootnote}}

%colors
\definecolor{c1}{cmyk}{0,0.5,1,0}
\definecolor{c2}{cmyk}{1,0.25,1,0}
\definecolor{n3}{cmyk}{1,0.,1,0}
\definecolor{neg}{cmyk}{1,0.,0.,0}

% Lister med bokstavar
\usepackage{enumitem}

\newcounter{rg}
\numberwithin{rg}{chapter}
\newcommand{\reg}[2][]{\begin{tcolorbox}[boxrule=0.3 mm,arc=0mm,colback=blue!3] {\refstepcounter{rg}\phantomsection \large \textbf{\therg \;#1} \vspace{5 pt}}\newline #2  \end{tcolorbox}\vspace{-5pt}}

\newcommand\alg[1]{\begin{align} #1 \end{align}}

\newcommand\eks[2][]{\begin{tcolorbox}[boxrule=0.3 mm,arc=0mm,enhanced jigsaw,breakable,colback=green!3] {\large \textbf{Example #1} \vspace{5 pt}\\} #2 \end{tcolorbox}\vspace{-5pt} }

\newcommand{\st}[1]{\begin{tcolorbox}[boxrule=0.0 mm,arc=0mm,enhanced jigsaw,breakable,colback=yellow!12]{ #1} \end{tcolorbox}}

\newcommand{\spr}[1]{\begin{tcolorbox}[boxrule=0.3 mm,arc=0mm,enhanced jigsaw,breakable,colback=yellow!7] {\large \textbf{The language box} \vspace{5 pt}\\} #1 \end{tcolorbox}\vspace{-5pt} }

\newcommand{\sym}[1]{\colorbox{blue!15}{#1}}

\newcommand{\info}[2]{\begin{tcolorbox}[boxrule=0.3 mm,arc=0mm,enhanced jigsaw,breakable,colback=cyan!6] {\large \textbf{#1} \vspace{5 pt}\\} #2 \end{tcolorbox}\vspace{-5pt} }

\newcommand\algv[1]{\vspace{-11 pt}\begin{align*} #1 \end{align*}}

\newcommand{\regv}{\vspace{5pt}}
\newcommand{\mer}{\textsl{Notice}: }
\newcommand{\merk}{Notice}
\newcommand\vsk{\vspace{11pt}}
\newcommand\vs{\vspace{-11pt}}
\newcommand\vsb{\vspace{-16pt}}
\newcommand\sv{\vsk \textbf{Answer} \vspace{4 pt}\\}
\newcommand\br{\\[5 pt]}
\newcommand{\asym}[1]{../fig/#1}
\newcommand\algvv[1]{\vs\vs\begin{align*} #1 \end{align*}}
\newcommand{\y}[1]{$ {#1} $}
\newcommand{\os}{\\[5 pt]}
\newcommand{\prbxl}[2]{
\parbox[l][][l]{#1\linewidth}{#2
	}}
\newcommand{\prbxr}[2]{\parbox[r][][l]{#1\linewidth}{
		\setlength{\abovedisplayskip}{5pt}
		\setlength{\belowdisplayskip}{5pt}	
		\setlength{\abovedisplayshortskip}{0pt}
		\setlength{\belowdisplayshortskip}{0pt} 
		\begin{shaded}
			\footnotesize	#2 \end{shaded}}}

\renewcommand{\cfttoctitlefont}{\Large\bfseries}
\setlength{\cftaftertoctitleskip}{0 pt}
\setlength{\cftbeforetoctitleskip}{0 pt}

\newcommand{\bs}{\\[3pt]}
\newcommand{\vn}{\\[6pt]}
\newcommand{\fig}[1]{\begin{figure}
		\centering
		\includegraphics[]{\asym{#1}}
\end{figure}}

\newcommand{\sectionbreak}{\clearpage} % New page on each section

% Equation comments
\newcommand{\cm}[1]{\llap{\color{blue} #1}}

\newcommand\fork[2]{\begin{tcolorbox}[boxrule=0.3 mm,arc=0mm,enhanced jigsaw,breakable,colback=yellow!7] {\large \textbf{#1 (explanation)} \vspace{5 pt}\\} #2 \end{tcolorbox}\vspace{-5pt} }


%%% SECTION HEADLINES %%%

% Our numbers
\newcommand{\likteikn}{The equal sign}
\newcommand{\talsifverd}{Numbers, digits and values}
\newcommand{\koordsys}{Coordinate systems}

% Calculations
\newcommand{\adi}{Addition}
\newcommand{\sub}{Subtraction}
\newcommand{\gong}{Multiplication}
\newcommand{\del}{Division}

%Factorization and order of operations
\newcommand{\fak}{Factorization}
\newcommand{\rrek}{Order of operations}

%Fractions
\newcommand{\brgrpr}{Introduction}
\newcommand{\brvu}{Values, expanding and simplifying}
\newcommand{\bradsub}{Addition and subtraction}
\newcommand{\brgngheil}{Fractions multiplied by integers}
\newcommand{\brdelheil}{Fractions divided by integers}
\newcommand{\brgngbr}{Fractions multiplied by fractions}
\newcommand{\brkans}{Cancelation of fractions}
\newcommand{\brdelmbr}{Division by fractions}
\newcommand{\Rasjtal}{Rational numbers}

%Negative numbers
\newcommand{\negintro}{Introduction}
\newcommand{\negrekn}{The elementary operations}
\newcommand{\negmeng}{Negative numbers as amounts}

% Geometry 1
\newcommand{\omgr}{Terms}
\newcommand{\eignsk}{Attributes of triangles and quadrilaterals}
\newcommand{\omkr}{Perimeter}
\newcommand{\area}{Area}

%Algebra 
\newcommand{\algintro}{Introduction}
\newcommand{\pot}{Powers}
\newcommand{\irrasj}{Irrational numbers}

%Equations
\newcommand{\ligintro}{Introduction}
\newcommand{\liglos}{Solving with the elementary operations}
\newcommand{\ligloso}{Solving with elementary operations summarized}

%Functions
\newcommand{\fintro}{Introduction}
\newcommand{\lingraf}{Linear functions and graphs}

%Geometry 2
\newcommand{\geoform}{Formulas of area and perimeter}
\newcommand{\kongogsim}{Congruent and similar triangles}
\newcommand{\geofork}{Explanations}

% Names of rules
\newcommand{\adkom}{Addition is commutative}
\newcommand{\gangkom}{Multiplication is commutative}
\newcommand{\brdef}{Fractions as rewriting of division}
\newcommand{\brtbr}{Fractions multiplied by fractions}
\newcommand{\delmbr}{Fractions divided by fractions}
\newcommand{\gangpar}{Distributive law}
\newcommand{\gangparsam}{Paranthesis multiplied together}
\newcommand{\gangmnegto}{Multiplication by negative numbers I}
\newcommand{\gangmnegtre}{Multiplication by negative numbers II}
\newcommand{\konsttre}{Unique construction of triangles}
\newcommand{\kongtre}{Congruent triangles}
\newcommand{\topv}{Vertical angles}
\newcommand{\trisum}{The sum of angles in a triangle}
\newcommand{\firsum}{The sum of angles in a quadrilateral}
\newcommand{\potgang}{Multiplication by powers}
\newcommand{\potdivpot}{Division by powers}
\newcommand{\potanull}{The special case of \boldmath $a^0$}
\newcommand{\potneg}{Powers with negative exponents}
\newcommand{\potbr}{Fractions as base}
\newcommand{\faktgr}{Factors as base}
\newcommand{\potsomgrunn}{Powers as base}
\newcommand{\arsirk}{The area of a circle}
\newcommand{\artrap}{The area of a trapezoid}
\newcommand{\arpar}{The area of a parallelogram}
\newcommand{\pyt}{Pythagoras's theorem}
\newcommand{\forform}{Ratios in similar triangles}
\newcommand{\vilkform}{Terms of similar triangles}
\newcommand{\omkrsirk}{The perimeter of a circle (and the value of $ \bm \pi $)}
\newcommand{\artri}{The area of a triangle}
\newcommand{\arrekt}{The area of a rectangle}
\newcommand{\liknflyt}{Moving terms across the equal sign}
\newcommand{\funklin}{Linear functions}

%License
\newcommand{\lic}{\textit{First Principles of Math by Sindre Sogge Heggen is licensed under CC BY-NC-SA 4.0. To view a copy of this license, visit\\ 
		\net{http://creativecommons.org/licenses/by-nc-sa/4.0/}{http://creativecommons.org/licenses/by-nc-sa/4.0/}}}

%referances
\newcommand{\net}[2]{{\color{blue}\href{#1}{#2}}}
\newcommand{\hrs}[2]{\hyperref[#1]{\color{blue}\textsl{#2 \ref*{#1}}}}
\newcommand{\rref}[1]{\hrs{#1}{Rule}}
\newcommand{\refkap}[1]{\hrs{#1}{Chapter}}
\newcommand{\refsec}[1]{\hrs{#1}{Section}}

\usepackage{datetime2}
\usepackage[]{hyperref}


\begin{document}
\newpage
\section{The equal sign, amounts and number lines}
\index{tal}
\subsection*{The equal sign}
As the name implies, the \textit{equal sign} \index{equal sign} \sym{$ = $} refers to things that are the same. In what sense some things are the same is a philosophical question and initially we are bound to this: What equality \sym{$=$} points to must be understood by the context in which the sign is used. With this understanding of \sym{=} we can study some basic properties of our numbers and then later return to more precise meanings of the sign. \regv
\spr{
Common ways of expressing \sym{$=$} is
\begin{itemize}
	\item ''equals'' \\
	\item ''is the same as''
\end{itemize}
}
\subsection*{Amounts and number lines}
There are so many things numbers can represent, however, in this book we shall stick to two ways of interpreting a number; a number as an \textsl{amount} and a number as a \textsl{placement on a line}. All representations of numbers relies on the understanding of 0 and 1.

\subsubsection*{Numbers as amounts}
	Talking about an amount, the number 0 is\footnote{In \refkap{Rekneartane} we'll se that there are also other interpretations of 0.} connected to ''nothing''. A figure showing nothing will therefore equal 0:
	\[ =0 \]
	1 we'll draw like a box:
	\fig{rut1}
In this way, other numbers are defined by how many one-boxes (ones) we have:
	\fig{rut2}
\newpage	
\subsubsection*{Numbers as placements on a line}
	When placing numbers on a line, 0 is our starting point:
	\fig{lin0a}
	Now we place 1 a certain length to the right of 0:
	\fig{lin0b}
	Other numbers are now defined by how many one-lengths (ones) we are away from 0:
	\fig{lin1}
\subsection*{Positive integers}
We'll soon see that numbers do not necessarily have to be a \textsl{whole} amount of ones, but those who \textsl{are} have their own name:\regv

\reg[Positive integers]{
Numbers which are a whole amount of ones are called \\\textit{positive\footnotemark integers}\index{positive integers}. The positive integers are
\[ 1, 2, 3, 4, 5 \text{ and so on.} \]
Positive integers are also called \textit{natural numbers}\index{numbers!natural}.
}
\info{What about 0?}{
Some authors also include 0 in the definition of positive\\ integers/natural numbers. This is in some cases beneficial, in others not.
}
\footnotetext{We'll see what the the word \textit{positiv}e refers to in chapter \hrs{Negtal}{chapter}.}

\newpage
\section{Numbers, digits and value}
Our numbers consists of the \textit{digits}\index{digits} $ 0, 1, 2 , 3, 4, 5, 6, 7, 8 $ and $ 9 $ and their \textsl{position}. The digits and their positions defines\footnote{Later on, we'll also see that \textit{signs} have an impact on a numbers value  (see \refkap{Negtal}).} the \textit{value} \index{value} of numbers.
\subsection*{Integers larger then 10}
Let's, as an example, write the number \textsl{fourteen} by our digits.
\fig{tel_eng}
We can now make a group of 10 ones, then we also have 4 ones. By this, we write fourteen as
\[ \text{fourteen}=14 \]
\fig{tel2_eng}\vsk

\fig{tel2t}
\newpage
\subsection*{Decimal numbers}
Sometimes we don't have a whole amount of ones, and this brings the need of dividing 1 into smaller pieces. Let's start off by drawing a one:
\fig{maal}
\fig{des1}
Now we divide our one into 10 smaller pieces:
\fig{maal1}
\fig{des1a}
Since we have divided 1 into 10 pieces, we name one such piece \textit{a tenth}:
\fig{maal1a_eng}
\fig{des1b_eng}
\begin{comment}
\eks{\vs
	\fig{maal2}
	\fig{des2}
}\vsk
\end{comment}
We indicate tenths by using the \textit{decimal mark} \sym{.}  :
\fig{maal1b_eng}
\fig{des1c_eng}
\eks[]{\vs 
	\fig{maal2a_eng}
	\fig{des3_eng}
}\regv
\spr{
In a lot of countires, comma \sym{,} is used as decimal mark instead of dot.\vsb
\alg{
	3,5&\quad(other) \\
	3.5&\quad (english)
}
}
\newpage
\subsection*{Base-10 positional notation}
So far we have seen how we can express the value of a number by placing digits according to the amount of tens, ones and tenths, and the pattern continues: \regv

\reg[Base-10 positional notation]{
The value of a number is given by the digits $ 0, 1, 2, 3, 4, 5, 6, 7, 8  $ and $ 9 $ and their position. In respect of the digit indicating ones, 
\begin{itemize}
	\item digits to the left (respectively) indicate amount of tens, \\hundeds, thousands etc.
	\item digits to the left (respectively) indicate amount of tenths, \\hundedths, thousandths etc.
\end{itemize}
}
\eks[1]{\vs 
	\fig{maal3_eng}
}
\eks[2]{ \vs \vs
\fig{titalsys_eng}
}

\newpage
\section{Coordinate systems \label{Koord}}

\prbxl{0.5}{Two number lines can be put together to form a \textit{coordinate system}\index{coordinate system}. In that case we place one number line \textsl{horizontally} and one \textsl{vertically}. A position in a coordinate system is called a  \textit{point}\index{point}. 
 }\qquad
\prbxr{0.4}{In fact, there are a lot of types of coordinate systems but in this book we'll use the term about the \textit{cartesian coordinate system}. It is named after the french mathematician and philosopher René Descartes.}
\st{A point is written as two numbers inside a bracket. We shall call\\ these two numbers the \textit{first coordinate} and the \textit{second \\coordinate}.
	\begin{itemize}
		\item The first coordinate tells how many units to move along the horizontal axis.
		\item The second coordinate tells how many units to move along the vertical axis.
\end{itemize}
In the figure, the points $ (2,3) $, $ (5,1) $ and $ (0, 0) $ are shown. The point where the axis intersect, that is $ (0, 0) $, is called \textit{origo}.
\fig{kord}
}

\end{document}
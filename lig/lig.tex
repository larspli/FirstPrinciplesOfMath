\documentclass[english,hidelinks,pdftex, 11 pt, class=report,crop=false]{standalone}
\usepackage[T1]{fontenc}
\usepackage[utf8]{luainputenc}
\usepackage{lmodern} % load a font with all the characters
\usepackage{geometry}
\geometry{verbose,paperwidth=16.1 cm, paperheight=24 cm, inner=2.3cm, outer=1.8 cm, bmargin=2cm, tmargin=1.8cm}
\setlength{\parindent}{0bp}
\usepackage{import}
\usepackage[subpreambles=false]{standalone}
\usepackage{amsmath}
\usepackage{amssymb}
\usepackage{esint}
\usepackage{babel}
\usepackage{tabu}
\makeatother
\makeatletter

\usepackage{titlesec}
\usepackage{ragged2e}
\RaggedRight
\raggedbottom
\frenchspacing

% Norwegian names of figures, chapters, parts and content
\addto\captionsenglish{\renewcommand{\figurename}{Figur}}
\makeatletter
\addto\captionsenglish{\renewcommand{\chaptername}{Kapittel}}
\addto\captionsenglish{\renewcommand{\partname}{Del}}

\addto\captionsenglish{\renewcommand{\contentsname}{Innhald}}

\usepackage{graphicx}
\usepackage{float}
\usepackage{subfig}
\usepackage{placeins}
\usepackage{cancel}
\usepackage{framed}
\usepackage{wrapfig}
\usepackage[subfigure]{tocloft}
\usepackage[font=footnotesize,labelfont=sl]{caption} % Figure caption
\usepackage{bm}
\usepackage[dvipsnames, table]{xcolor}
\definecolor{shadecolor}{rgb}{0.105469, 0.613281, 1}
\colorlet{shadecolor}{Emerald!15} 
\usepackage{icomma}
\makeatother
\usepackage[many]{tcolorbox}
\usepackage{multicol}
\usepackage{stackengine}

% For tabular
\usepackage{array}
\usepackage{multirow}
\usepackage{longtable} %breakable table

% Ligningsreferanser
\usepackage{mathtools}
\mathtoolsset{showonlyrefs}

% index
\usepackage{imakeidx}
\makeindex[title=Indeks]

%Footnote:
\usepackage[bottom, hang, flushmargin]{footmisc}
\usepackage{perpage} 
\MakePerPage{footnote}
\addtolength{\footnotesep}{2mm}
\renewcommand{\thefootnote}{\arabic{footnote}}
\renewcommand\footnoterule{\rule{\linewidth}{0.4pt}}
\renewcommand{\thempfootnote}{\arabic{mpfootnote}}

%colors
\definecolor{c1}{cmyk}{0,0.5,1,0}
\definecolor{c2}{cmyk}{1,0.25,1,0}
\definecolor{n3}{cmyk}{1,0.,1,0}
\definecolor{neg}{cmyk}{1,0.,0.,0}

% Lister med bokstavar
\usepackage[inline]{enumitem}

\newcounter{rg}
\numberwithin{rg}{chapter}
\newcommand{\reg}[2][]{\begin{tcolorbox}[boxrule=0.3 mm,arc=0mm,colback=blue!3] {\refstepcounter{rg}\phantomsection \large \textbf{\therg \;#1} \vspace{5 pt}}\newline #2  \end{tcolorbox}\vspace{-5pt}}

\newcommand\alg[1]{\begin{align} #1 \end{align}}

\newcommand\eks[2][]{\begin{tcolorbox}[boxrule=0.3 mm,arc=0mm,enhanced jigsaw,breakable,colback=green!3] {\large \textbf{Eksempel #1} \vspace{5 pt}\\} #2 \end{tcolorbox}\vspace{-5pt} }

\newcommand{\st}[1]{\begin{tcolorbox}[boxrule=0.0 mm,arc=0mm,enhanced jigsaw,breakable,colback=yellow!12]{ #1} \end{tcolorbox}}

\newcommand{\spr}[1]{\begin{tcolorbox}[boxrule=0.3 mm,arc=0mm,enhanced jigsaw,breakable,colback=yellow!7] {\large \textbf{Språkboksen} \vspace{5 pt}\\} #1 \end{tcolorbox}\vspace{-5pt} }

\newcommand{\sym}[1]{\colorbox{blue!15}{#1}}

\newcommand{\info}[2]{\begin{tcolorbox}[boxrule=0.3 mm,arc=0mm,enhanced jigsaw,breakable,colback=cyan!6] {\large \textbf{#1} \vspace{5 pt}\\} #2 \end{tcolorbox}\vspace{-5pt} }

\newcommand\algv[1]{\vspace{-11 pt}\begin{align*} #1 \end{align*}}

\newcommand{\regv}{\vspace{5pt}}
\newcommand{\mer}{\textsl{Merk}: }
\newcommand\vsk{\vspace{11pt}}
\newcommand\vs{\vspace{-11pt}}
\newcommand\vsb{\vspace{-16pt}}
\newcommand\sv{\vsk \textbf{Svar:} \vspace{4 pt}\\}
\newcommand\br{\\[5 pt]}
\newcommand{\asym}[1]{../fig/#1}
\newcommand\algvv[1]{\vs\vs\begin{align*} #1 \end{align*}}
\newcommand{\y}[1]{$ {#1} $}
\newcommand{\os}{\\[5 pt]}
\newcommand{\prbxl}[2]{
\parbox[l][][l]{#1\linewidth}{#2
	}}
\newcommand{\prbxr}[2]{\parbox[r][][l]{#1\linewidth}{
		\setlength{\abovedisplayskip}{5pt}
		\setlength{\belowdisplayskip}{5pt}	
		\setlength{\abovedisplayshortskip}{0pt}
		\setlength{\belowdisplayshortskip}{0pt} 
		\begin{shaded}
			\footnotesize	#2 \end{shaded}}}

\renewcommand{\cfttoctitlefont}{\Large\bfseries}
\setlength{\cftaftertoctitleskip}{0 pt}
\setlength{\cftbeforetoctitleskip}{0 pt}

\newcommand{\bs}{\\[3pt]}
\newcommand{\vn}{\\[6pt]}
\newcommand{\fig}[1]{\begin{figure}
		\centering
		\includegraphics[]{\asym{#1}}
\end{figure}}


\newcommand{\sectionbreak}{\clearpage} % New page on each section

% Equation comments
\newcommand{\cm}[1]{\llap{\color{blue} #1}}

\newcommand\fork[2]{\begin{tcolorbox}[boxrule=0.3 mm,arc=0mm,enhanced jigsaw,breakable,colback=yellow!7] {\large \textbf{#1 (forklaring)} \vspace{5 pt}\\} #2 \end{tcolorbox}\vspace{-5pt} }




%colors
\newcommand{\colr}[1]{{\color{red} #1}}
\newcommand{\colb}[1]{{\color{blue} #1}}
\newcommand{\colo}[1]{{\color{orange} #1}}
\newcommand{\colc}[1]{{\color{cyan} #1}}
\definecolor{projectgreen}{cmyk}{100,0,100,0}
\newcommand{\colg}[1]{{\color{projectgreen} #1}}

%%% SECTION HEADLINES %%%

% Our numbers
\newcommand{\likteikn}{Likskapsteiknet}
\newcommand{\talsifverd}{Tal, siffer og verdi}
\newcommand{\koordsys}{Koordinatsystem}

% Calculations
\newcommand{\adi}{Addisjon}
\newcommand{\sub}{Subtraksjon}
\newcommand{\gong}{Multiplikasjon (Gonging)}
\newcommand{\del}{Divisjon (deling)}

%Factorization and order of operations
\newcommand{\fak}{Faktorisering}
\newcommand{\rrek}{Reknerekkefølge}

%Fractions
\newcommand{\brgrpr}{Introduksjon}
\newcommand{\brvu}{Verdi, utviding og forkorting av brøk}
\newcommand{\bradsub}{Addisjon og subtraksjon}
\newcommand{\brgngheil}{Brøk gonga med heiltal}
\newcommand{\brdelheil}{Brøk delt med heiltal}
\newcommand{\brgngbr}{Brøk gonga med brøk}
\newcommand{\brkans}{Kansellering av faktorar}
\newcommand{\brdelmbr}{Deling med brøk}
\newcommand{\Rasjtal}{Rasjonale tal}

%Negative numbers
\newcommand{\negintro}{Introduksjon}
\newcommand{\negrekn}{Dei fire rekneartane med negative tal}
\newcommand{\negmeng}{Negative tal som mengde}

% Geometry 1
\newcommand{\omgr}{Omgrep}
\newcommand{\eignsk}{Eigenskapar for trekantar og firkantar}
\newcommand{\omkr}{Omkrins}
\newcommand{\area}{Areal}

%Algebra 
\newcommand{\algintro}{Introduksjon}
\newcommand{\pot}{Potensar}
\newcommand{\irrasj}{Irrasjonale tal}

%Equations
\newcommand{\ligintro}{Introduksjon}
\newcommand{\liglos}{Løysing ved dei fire rekneartane}
\newcommand{\ligloso}{Løysingsmetodane oppsummert}

%Functions
\newcommand{\fintro}{Introduksjon}
\newcommand{\lingraf}{Lineære funksjonar og grafar}

%Geometry 2
\newcommand{\geoform}{Formlar for areal og omkrins}
\newcommand{\kongogsim}{Kongruente og formlike trekantar}
\newcommand{\geofork}{Forklaringar}

% Names of rules
\newcommand{\gangdestihundre}{Å gange desimaltall med 10, 100 osv.}
\newcommand{\delmedtihundre}{Deling med 10, 100, 1\,000 osv.}
\newcommand{\ompref}{Omgjøring av prefikser}
\newcommand{\adkom}{Addisjon er kommutativ}
\newcommand{\gangkom}{Multiplikasjon er kommutativ}
\newcommand{\brdef}{Brøk som omskriving av delestykke}
\newcommand{\brtbr}{Brøk gonga med brøk}
\newcommand{\delmbr}{Brøk delt på brøk}
\newcommand{\gangpar}{Gonging med parentes (distributiv lov)}
\newcommand{\gangparsam}{Parantesar gonga saman}
\newcommand{\gangmnegto}{Gonging med negative tal I}
\newcommand{\gangmnegtre}{Gonging med negative tal II}
\newcommand{\konsttre}{Konstruksjon av trekantar}
\newcommand{\kongtre}{Kongruente trekantar}
\newcommand{\topv}{Toppvinklar}
\newcommand{\trisum}{Summen av vinklane i ein trekant}
\newcommand{\firsum}{Summen av vinklane i ein firkant}
\newcommand{\potgang}{Gonging med potensar}
\newcommand{\potdivpot}{Divisjon med potensar}
\newcommand{\potanull}{Spesialtilfellet \boldmath $a^0$}
\newcommand{\potneg}{Potens med negativ eksponent}
\newcommand{\potbr}{Brøk som grunntal}
\newcommand{\faktgr}{Faktorar som grunntal}
\newcommand{\potsomgrunn}{Potens som grunntal}
\newcommand{\arsirk}{Arealet til ein sirkel}
\newcommand{\artrap}{Arealet til eit trapes}
\newcommand{\arpar}{Arealet til eit parallellogram}
\newcommand{\pyt}{Pytagoras' setning}
\newcommand{\forform}{Forhold i formlike trekantar}
\newcommand{\vilkform}{Vilkår i formlike trekantar}
\newcommand{\omkrsirk}{Omkrinsen til ein sirkel (og $ \bm \pi $)}
\newcommand{\artri}{Arealet til ein trekant}
\newcommand{\arrekt}{Arealet til eit rektangel}
\newcommand{\liknflyt}{Flytting av ledd over likskapsteiknet}
\newcommand{\funklin}{Lineære funksjonar}

%Opg
\newcommand{\abc}[1]{
	\begin{enumerate}[label=\alph*),leftmargin=18pt]
		#1
	\end{enumerate}
}
\newcommand{\abcs}[2]{
	\begin{enumerate}[label=\alph*),start=#1,leftmargin=18pt]
		#2
	\end{enumerate}
}
\newcommand{\abcn}[1]{
	\begin{enumerate}[label=\arabic*),leftmargin=18pt]
		#1
	\end{enumerate}
}
\newcommand{\abch}[1]{
	\hspace{-2pt}	\begin{enumerate*}[label=\alph*), itemjoin=\hspace{1cm}]
		#1
	\end{enumerate*}
}
\newcommand{\abchs}[2]{
	\hspace{-2pt}	\begin{enumerate*}[label=\alph*), itemjoin=\hspace{1cm}, start=#1]
		#2
	\end{enumerate*}
}

\newcommand{\opgt}{\phantomsection \addcontentsline{toc}{section}{Oppgaver} \section*{Oppgaver for kapittel \thechapter}\vs \setcounter{section}{1}}
\newcounter{opg}
\numberwithin{opg}{section}
\newcommand{\op}[1]{\vspace{15pt} \refstepcounter{opg}\large \textbf{\color{blue}\theopg} \vspace{2 pt} \label{#1} \\}
\newcommand{\ekspop}[1]{\vsk\textbf{Gruble \thechapter.#1}\vspace{2 pt} \\}
\newcommand{\nes}{\stepcounter{section}
	\setcounter{opg}{0}}
\newcommand{\opr}[1]{\vspace{3pt}\textbf{\ref{#1}}}
\newcommand{\oeks}[1]{\begin{tcolorbox}[boxrule=0.3 mm,arc=0mm,colback=white]
		\textit{Eksempel: } #1	  
\end{tcolorbox}}
\newcommand\opgeks[2][]{\begin{tcolorbox}[boxrule=0.1 mm,arc=0mm,enhanced jigsaw,breakable,colback=white] {\footnotesize \textbf{Eksempel #1} \\} \footnotesize #2 \end{tcolorbox}\vspace{-5pt} }

%License
\newcommand{\lic}{\textit{Matematikken sine byggesteinar by Sindre Sogge Heggen is licensed under CC BY-NC-SA 4.0. To view a copy of this license, visit\\ 
		\net{http://creativecommons.org/licenses/by-nc-sa/4.0/}{http://creativecommons.org/licenses/by-nc-sa/4.0/}}}

%referances
\newcommand{\net}[2]{{\color{blue}\href{#1}{#2}}}
\newcommand{\hrs}[2]{\hyperref[#1]{\color{blue}\textsl{#2 \ref*{#1}}}}
\newcommand{\rref}[1]{\hrs{#1}{Regel}}
\newcommand{\refkap}[1]{\hrs{#1}{Kapittel}}
\newcommand{\refsec}[1]{\hrs{#1}{Seksjon}}

\usepackage{datetime2}

\usepackage[]{hyperref}

\begin{document}

\section{\ligintro}
Eit kvart matematisk uttrykk som inneheld \sym{$=$} er ei \textit{likning}\index{likning}, likevel er ordet likning tradisjonelt knytt til at vi har eit \textit{ukjend} tal.\vsk

Sei at vi ønsker å finne eit tal som er slik at viss vi legg til $4$, så får vi $7$. Dette talet kan vi kalle for kva som helst, men det vanlegaste er å kalle det for $ x $, som altså er det ukjende talet vårt. Likninga vår kan no skrivast slik:
\[ x+4=7 \]
$ x $-verdien\footnote{I andre tilfelle kan det vere fleire verdiar.} som gjer at det blir same verdi på begge sider av likskapsteiknet kallast \textit{løysinga} av likninga.
Det er alltids lov til å sjå eller prøve seg fram for å finne verdien til $ x $. Kanskje har du allereie merka at $ {x=3} $ er løysinga av likninga, sidan
\[ 3+4=7 \]
Men dei fleste likningar er det vanskelig å sjå eller gjette seg fram til svaret på, og da må vi ty til meir generelle løysingsmetodar. Eigentleg er det berre eitt prinsipp vi følg: \regv
\st{Vi kan alltid utføre ein matematisk operasjon på den eine sida av likskapsteiknet, så lenge vi utfører den også på den andre sida.}
Dei matematiske operasjonane vi har presentert i denne boka er dei fire rekneartane. Med desse lyd prinsippet slik:\regv
\st{Vi kan alltid legge til, trekke ifrå, gonge eller dele med eit tal på den eine sida av likskapsteiknet, så lenge vi gjer det også på den andre sida.}\regv
Prinsippet følg av tydinga til \sym{$=$}. Når to uttrykk har same verdi, må dei naudsynleg fortsette å ha lik verdi, så lenge vi utfører dei same matematiske operasjonane på dei. I komande seksjon skal vi likevel konkretisere dette prinsippet for kvar enkelt rekneoperasjon, men viss du føler dette allereie gir god meining kan du med fordel hoppe til \hrs{ligsaml}{seksjon}.
\section{\liglos \label{likloysfire}}
\textit{I figurane til denne seksjonen skal vi forstå likningar ut ifrå eit vektprinsipp. \sym{$=$} vil da indikere\footnote{\sym{$\neq$} er symbolet for ''er \textsl{ikkje} lik''.} at det er like mykje vekt (lik verdi) på venstre side som på høgre side.}
\begin{figure}
	\centering
	\includegraphics[]{\asym{lig}}\qquad
	\includegraphics[]{\asym{lig1}}
\end{figure}
\subsection*{Addisjon og subtraksjon; tal som skifter side}
\subsubsection*{Første eksempel}
Vi har allereie funne løysinga på denne likninga, men lat oss løyse den på ein annan måte\footnote{\textsl{Merk:} I tidlegare figurar har det vore samsvar mellom størrelsen på rutene og (tal)verdien til talet dei symboliserer. Dette gjeld ikkje rutene som representerer $ x $.}:
\[ x+4=7 \]
\fig{lig2}
Det blir tydeleg kva verdien til $ x $ er viss $ x $ står aleine på ei av sidene, og $ x $ blir isolert på venstresida viss vi tar bort 4. Men skal vi ta bort 4 fra venstresida, må vi ta bort 4 fra høgresida òg, skal begge sidene ha same verdi.
\[ x+4-{\color{red}4}=7-{\color{red}4}  \]
\fig{lig2b}
Da $ 4-{\color{red}4}=0 $ og $ 7-{\color{red}4}=3 $, får vi at
\[ x=3 \]
\fig{lig2c}
Dette kunne vi ha skrive noko meir kortfatta slik:

\prbxl{0.5}{
	\alg{
		x+4 &= 7 \\
		x&= 7-4 \\
		x &= 3
}}
\prbxr{0.5}{Mellom første og andre linje er det vanleg å seie at \textsl{4 har skifta side, og derfor også fortegn (fra $ + $ til $ - $).}}
\textbf{Andre eksempel}\os
Lat oss gå videre til å sjå på ei litt vanskelegare likning\footnote{Legg merke til at figuren illustrerer $ {4x+(-2)} $ (sjå \hrs{negmeng}{seksjon}) på venstre side. Men  $ {4x+(-2)} $ er det same som $ {4x-2} $ (sjå \hrs{rekmneg}{seksjon}).}:
\[ 4x-2=3x+5 \]
\fig{lig4}
For å skaffe eit uttrykk med $ x $ berre på éi side, tar vi vekk $ 3x $ på begge sider:
\[ 4x-2-{\color{red}3x}=3x+5-{\color{red}3x} \]
\fig{lig5}
Da får vi at
\[ x-2=5 \]
\fig{lig5b}
For å isolere $ x $, legg vi til 2 på venstre side. Da må vi også legge til $ 2 $ på høgre side:
\[ x-2+{\color{blue}2}=5+{\color{blue}2} \]
\fig{lig6}
\newpage
Da får vi at
\[ x=7 \]
\fig{lig7}
Stega vi har tatt kan oppsummerast slik:
\begin{flalign*}
&& 4x-2&=3x+5 && \llap{1. figur} \\
&& 4x-{\color{red}3x}-2&=3x-{\color{red}3x}+5 &&  \llap{2. figur} \\
&& x -2 &= 5 &&\llap{3. figur}\\
&& x-2+\color{blue}2&=  5+\color{blue}2 &&\llap{4. figur}\\
&& x &= 7 &&\llap{5. figur}
\end{flalign*}
Dette kan vi på ein forenkla måte skrive slik:
\alg{
4x-2&=3x+5 \\
4x-{\color{red}3x}&=5+\color{blue}2\\
x &= 7
}

\reg[Flytting av tal over likskapsteiknet \label{bytt}]{I ei likning ønsker vi å samle alle $x$-ledd og alle kjente ledd på kvar si side av likskapsteiknet. Skifter eit ledd side, skifter det forteikn.}

\eks[1]{Løys likninga
\[ 3x+3 =2x+5 \]
\sv	
	 \vs \vs \vs \vs
	\begin{align*}
	3x-2x &=5-3 \\
	x &=2
	\end{align*}  \vspace{-20pt}}

\eks[2]{Løys likninga
\[ -4x-3 =-5x+12  \]	
\sv
	 \vs \vs \vs \vs
	\begin{align*}
	-4x+5x &=12+3 \\
	x &=15
	\end{align*}}
\subsection*{Gonging og deling}	
\subsubsection{Deling}
Hittil har vi sett på likningar der vi endte opp med éin $x $ på den eine sida av likhetsteiknet. Ofte har vi fleire $ x $-ar, som for eksempel i likninga
\[ 3x=6 \]
\fig{lig8}
Deler vi venstresiden vår i tre like grupper, får vi éin $ x $ i kvar gruppe.  Deler vi også høgresida inn i tre like grupper, må alle gruppene ha den same verdien
\[ \frac{3x}{3}=\frac{6}{3} \]
\fig{lig9}
Altså er
\[ x=2 \]
\fig{lig10}
Lat oss oppsummere utrekninga vår:\\ 
\prbxl{0.6}{\begin{flalign*}
	&& 3x&=6 && \llap{1. figur} \br
	&& \frac{3x}{3}&=\frac{6}{3} && \llap{2. figur} \br
	&& x&=2 && \llap{3. figur}
	\end{flalign*}}\qquad\qquad
\prbxr{0.25}{Du huskar kanskje at vi gjerne skriv
\[ \frac{\cancel{3}x}{\cancel{3}} \] 
}
\newpage
\reg[Deling på begge sider av ei likning \label{ligdel}]{Vi kan dele begge sider av ei likning med det same talet.}
\eks[1]{ Løys likninga
	\[ 	4x = 20  \]
	\sv \vs \vs \vsb
	\begin{align*}
	\frac{\cancel{4}x}{\cancel{4}}&=\frac{20}{4} \\
	x &=5
	\end{align*}
	\vspace{-20 pt}
	}
	
\eks[2]{Løys likninga
	\[2x+6 =3x-2 \]
\sv \vs \vs \vs \vs
	\begin{flalign*}
	&& 2x-3x &= -2-6 &&\\
	&&-x &= -8&& \\
	&& \frac{\cancel{-1}x}{\cancel{-1}} &= \frac{-8}{-1} &&\cm{($-x=-1x$)}\\
	&& x &= 8&&
	\end{flalign*}}

\subsection*{Gonging}
Det siste tilfellet vi skal sjå på er når likningar inneheld brøkdelar av den ukjende, som for eksempel i likninga
\[ \frac{x}{3}=4 \]
\fig{lig11}
Vi kan få éin $ x $ på venstresida viss vi legg til to eksemplar av $ \frac{x}{3} $. Likninga fortel oss at $ \frac{x}{3} $ har same verdi som 4. Dette betyr at 
for kvar $ \frac{x}{3} $ vi legg til på venstresida, må vi legge til 4 på høgresida, skal sidene ha same verdi.
\[ \frac{x}{3}+\frac{x}{3}+\frac{x}{3}=4+4+4 \]
\fig{lig12}
Vi legger no merke til at \y{\frac{x}{3}+\frac{x}{3}+\frac{x}{3}=\frac{x}{3}\cdot3} og at \y{4+4+4=4\cdot3}:
\[ \frac{x}{3}\cdot 3 = 4\cdot 3 \]
\fig{lig12b}
Og da $ \frac{x}{3}\cdot3=x $ og $ 4\cdot3=12 $, har vi at
\[ x=12 \]
\fig{lig13}
Ei oppsummering av stega våre kan vi skrive slik:
\begin{flalign*}
&& \frac{x}{3}&=4 && \llap{1. figur} \br 
&& \frac{x}{3}+\frac{x}{3}+\frac{x}{3} &= 4+4+4  &&\llap{2. figur} \br
&& \frac{x}{3}\cdot 3&=4\cdot3 && \llap{3. figur} \\
&& x&=12 && \llap{4. figur}
\end{flalign*}
Dette kan vi kortare skrive som
\alg{
\frac{x}{3}&= 4 \br 
\frac{x}{\cancel{3}}\cdot \cancel{3} &= 4\cdot 3 \br
x &= 12
}
\newpage
\reg[Gonging på begge sider av ei likning]{
Vi kan gonge begge sider av ei likning med det same talet.
}
\eks[1]{
Løys likninga
\[ \frac{x}{5}=2 \]
\sv \vsb \vs

\algv{
\frac{x}{\cancel{5}}\cdot\cancel{5}&=2\cdot5 \\
x &= 10
}
}
\eks[2]{
Løys likninga 
\[ \frac{7x}{10}-5=13+\frac{x}{10} \]
\sv \vsb \vs

\algv{
\frac{7x}{10}-\frac{x}{10}&=13+5\br
\frac{6x}{10}&=18 \br
\frac{6x}{\cancel{10}}\cdot \cancel{10}&=18\cdot10 \\
6x&=180 \br
\frac{\cancel{6}x}{\cancel{6}}&=\frac{180}{6}\br
x&=30
}
}
\newpage
\section{\ligloso \label{ligsaml}}
\reg[Løysingsmetodar for likningar \label{lsmlig}]{
Vi kan alltid
\begin{itemize}
\item addere eller subtrahere begge sider av ei likning med det same talet. 
Dette er ekvivalent til å flytte eit ledd fra den eine sida av likninga til den andre, så lenge vi også skiftar forteikn på leddet.
\item gonge eller dele begge sider av ei likning med det same tallet.
\end{itemize}
}
\eks[1]{
Løys likninga 
\[ 3x-4=6+2x \]

\sv \vs \vs

\algv{
3x-2x&=6+4 \\
x&=10
}
}
\eks[2]{
Løys likninga
\[ 9-7x=8x+3 \]
\sv \vs \vs
\algv{
 9-7x&=-8x+3 \\
  8x-7x&=3-9 \\
  x&=-6
}
}
\newpage
\eks[3]{
	Løys likninga
	\[ 10x-20=7x-5 \]
	\sv \vs \vs
	\algv{
		10x-20&=7x-5 \\
		10x-7x&=20-5 \\
		3x&=15 \\
		\frac{\cancel{3}x}{\cancel{3}}&=\frac{15}{3} \\
		x&=5
	}
}
\eks[4]{
Løys likninga
\[ 15-4x=x+5 \]

\sv \vs \vsb
\alg{
15-5&=x+4x \\
10&=5x\\
\frac{10}{5}&=\frac{\cancel{5}x}{\cancel{5}}\\
2&=x
}
}	

\eks[5]{
Løys likninga
\[ \frac{4x}{9}-20=8-\frac{3x}{9} \]

\sv \vs \vsb
\alg{
\frac{4x}{9}+\frac{3x}{9}&=20+8 \\
\frac{\cancel{7}x}{9\cdot \cancel{7}}&=\frac{28}{7} \br
\frac{x}{\cancel{9}}\cdot \cancel{9}&=4\cdot 9\\
x&=36
}
}

\newpage
\eks[6]{ Løys likninga
	\[ \frac{1}{3}x+\frac{1}{6}=\frac{5}{12}x+2  \vs\]	
	
	\sv
	Får å unngå brøkar, gongar vi begge sider med fellesnemnaren 12:
	\begin{align}
	\left(\frac{1}{3}x+\frac{1}{6}\right)12&=\left(\frac{5}{12}x+2\right)12 \br	
	\frac{1}{3}x\cdot12+\frac{1}{6}\cdot12&=\frac{5}{12}x\cdot12+2\cdot12 \tag{$ \ast $}\br	
	4x+2 &= 5x+24 \\
	4x-5x &= 24-2 \\
	-x &= 22 \\
	\frac{\cancel{-1}\,x}{\cancel{-1}} &= \frac{22}{-1} \\
	x &= -22
	\end{align} 
}
\info{Tips}{
	Mange liker å lage seg ein regel om at ''{vi kan gonge eller dele alle ledd med det same talet}''. I eksempelet over kunne vi da hoppa direkte til andre linje i utrekninga.
}
\eks[7]{
Løys likninga
\[ 3-\frac{6}{x}= 2+\frac{5}{2x} \vs\]

\sv
Vi gongar begge sider med fellesnemnaren $ 2x $:
\alg{
	 2x\left(3-\frac{6}{x}\right)&= 2x\left(2+\frac{5}{2x}\right) \br
6x -12 &= 4x+5\\
6x-4x &= 5+12\\
2x &= 17\\
x&=\frac{17}{2}
}
}
\section{Power equations}
Let's solve the equation
\[ x^2=9 \]
This is called a \textit{power equation}\index{power equation}. In general, power equations are difficult to solve applying the four elementary operations only. Regarding power equations, we can raise both sides to the inverse power of the exponent associated with $ x $:
\[ \left(x^2\right)^\frac{1}{2}=9^\frac{1}{2} \]
By \rref{potsomgrunn}, we have
\alg{
	x^{2\cdot\frac{1}{2}}&=9^\frac{1}{2} \\
	x&=9^\frac{1}{2}
}
Since $ 3^2=9 $, we have $ 9^\frac{1}{2}=3 $. Now observe this: \vsk

\textit{The principle stated on page \pageref{principle} says we can, like we just did, carry out a mathematical operation on both sides of an equation. However, sticking to this principle does not guarantee that all solutions are found.} \\ \vsk

Concerning our equation, we know that $ {x=3} $ is a solution. For the sake of it, we can confirm this by the calculation
\[ 3^2=3\cdot3=9 \]
But we also have
\[ (-3)^2=(-3)(-3)=9 \]
Hence, $ -3 $ is also a solution of our original equation!

\reg[Power equations]{
	An eqation which can be written as
	\[ x^a=b \]
	where $ a $ and $ b $ are constants,
	is a \textit{power} equation.
} 
\newpage
\eks[1]{
	Solve the equation 
	\[ x^2+5= 21\]
	\sv \vs \vs \vs
	\algv{
		x^2+5&= 21\\
		x^2 &= 21-5\\
		x^2 &= 16
	}
	Since $ {4\cdot4 =16} $ og $ {(-4)\cdot(-4)=16} $, er både $ {x=4} $ og ${ x=-4} $ løsninger av likninga.
}
\eks[2]{
	Løys likninga:
	\[ 5x^3 -35 = 4x^3-8 \]
	\sv \vs \vs
	\algv{
		5x^3 -35 &= 4x^3-8 \\
		5x^3-4x^3 &= -8+35 \\
		x^3 &= 27 
	}
	Siden $ {3\cdot3\cdot3=27} $, er $ {x=3} $ løsningen av likninga.
}
\end{document}

\begin{comment}
\eks[]{
Gitt eit trapes med to parallelle sider $ a $ og $ b $ og høgde $ h $. Da er arealet $ A $ gitt som
\[ A=\frac{(a+b)h}{2} \]
Løys likninga med hensyn på $ b $.

\sv \vs \vsb
\alg{
 A&=\frac{(a+b)h}{2} \br
 2A&=(a+b)h \\
 2A&=ah+bh \\
 2A-ah&=bh \\
 \frac{2A-ah}{h}&=b
}
}
\end{comment}
\end{document}



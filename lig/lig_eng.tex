\documentclass[english,hidelinks,pdftex, 11 pt, class=report,crop=false]{standalone}
\usepackage[T1]{fontenc}
\usepackage[utf8]{luainputenc}
\usepackage{lmodern} % load a font with all the characters
\usepackage{geometry}
\geometry{verbose,paperwidth=16.1 cm, paperheight=24 cm, inner=2.3cm, outer=1.8 cm, bmargin=2cm, tmargin=1.8cm}
\setlength{\parindent}{0bp}
\usepackage{import}
\usepackage[subpreambles=false]{standalone}
\usepackage{amsmath}
\usepackage{amssymb}
\usepackage{esint}
\usepackage{babel}
\usepackage{tabu}
\makeatother
\makeatletter

\usepackage{titlesec}
\usepackage{ragged2e}
\RaggedRight
\raggedbottom
\frenchspacing

% Norwegian names of figures, chapters, parts and content
\addto\captionsenglish{\renewcommand{\figurename}{Figure}}
\makeatletter
\addto\captionsenglish{\renewcommand{\chaptername}{Chapter}}
%\addto\captionsenglish{\renewcommand{\partname}{Part}}

%\addto\captionsenglish{\renewcommand{\contentsname}{Content}}

\usepackage{graphicx}
\usepackage{float}
\usepackage{subfig}
\usepackage{placeins}
\usepackage{cancel}
\usepackage{framed}
\usepackage{wrapfig}
\usepackage[subfigure]{tocloft}
\usepackage[font=footnotesize,labelfont=sl]{caption} % Figure caption
\usepackage{bm}
\usepackage[dvipsnames, table]{xcolor}
\definecolor{shadecolor}{rgb}{0.105469, 0.613281, 1}
\colorlet{shadecolor}{Emerald!15} 
\usepackage{icomma}
\makeatother
\usepackage[many]{tcolorbox}
\usepackage{multicol}
\usepackage{stackengine}

% For tabular
\usepackage{array}
\usepackage{multirow}
\usepackage{longtable} %breakable table

% Ligningsreferanser
\usepackage{mathtools}
\mathtoolsset{showonlyrefs}

% index
\usepackage{imakeidx}
\makeindex[title=Index]

%Footnote:
\usepackage[bottom, hang, flushmargin]{footmisc}
\usepackage{perpage} 
\MakePerPage{footnote}
\addtolength{\footnotesep}{2mm}
\renewcommand{\thefootnote}{\arabic{footnote}}
\renewcommand\footnoterule{\rule{\linewidth}{0.4pt}}
\renewcommand{\thempfootnote}{\arabic{mpfootnote}}

%colors
\definecolor{c1}{cmyk}{0,0.5,1,0}
\definecolor{c2}{cmyk}{1,0.25,1,0}
\definecolor{n3}{cmyk}{1,0.,1,0}
\definecolor{neg}{cmyk}{1,0.,0.,0}

% Lister med bokstavar
\usepackage{enumitem}

\newcounter{rg}
\numberwithin{rg}{chapter}
\newcommand{\reg}[2][]{\begin{tcolorbox}[boxrule=0.3 mm,arc=0mm,colback=blue!3] {\refstepcounter{rg}\phantomsection \large \textbf{\therg \;#1} \vspace{5 pt}}\newline #2  \end{tcolorbox}\vspace{-5pt}}

\newcommand\alg[1]{\begin{align} #1 \end{align}}

\newcommand\eks[2][]{\begin{tcolorbox}[boxrule=0.3 mm,arc=0mm,enhanced jigsaw,breakable,colback=green!3] {\large \textbf{Example #1} \vspace{5 pt}\\} #2 \end{tcolorbox}\vspace{-5pt} }

\newcommand{\st}[1]{\begin{tcolorbox}[boxrule=0.0 mm,arc=0mm,enhanced jigsaw,breakable,colback=yellow!12]{ #1} \end{tcolorbox}}

\newcommand{\spr}[1]{\begin{tcolorbox}[boxrule=0.3 mm,arc=0mm,enhanced jigsaw,breakable,colback=yellow!7] {\large \textbf{The language box} \vspace{5 pt}\\} #1 \end{tcolorbox}\vspace{-5pt} }

\newcommand{\sym}[1]{\colorbox{blue!15}{#1}}

\newcommand{\info}[2]{\begin{tcolorbox}[boxrule=0.3 mm,arc=0mm,enhanced jigsaw,breakable,colback=cyan!6] {\large \textbf{#1} \vspace{5 pt}\\} #2 \end{tcolorbox}\vspace{-5pt} }

\newcommand\algv[1]{\vspace{-11 pt}\begin{align*} #1 \end{align*}}

\newcommand{\regv}{\vspace{5pt}}
\newcommand{\mer}{\textsl{Notice}: }
\newcommand{\merk}{Notice}
\newcommand\vsk{\vspace{11pt}}
\newcommand\vs{\vspace{-11pt}}
\newcommand\vsb{\vspace{-16pt}}
\newcommand\sv{\vsk \textbf{Answer} \vspace{4 pt}\\}
\newcommand\br{\\[5 pt]}
\newcommand{\asym}[1]{../fig/#1}
\newcommand\algvv[1]{\vs\vs\begin{align*} #1 \end{align*}}
\newcommand{\y}[1]{$ {#1} $}
\newcommand{\os}{\\[5 pt]}
\newcommand{\prbxl}[2]{
\parbox[l][][l]{#1\linewidth}{#2
	}}
\newcommand{\prbxr}[2]{\parbox[r][][l]{#1\linewidth}{
		\setlength{\abovedisplayskip}{5pt}
		\setlength{\belowdisplayskip}{5pt}	
		\setlength{\abovedisplayshortskip}{0pt}
		\setlength{\belowdisplayshortskip}{0pt} 
		\begin{shaded}
			\footnotesize	#2 \end{shaded}}}

\renewcommand{\cfttoctitlefont}{\Large\bfseries}
\setlength{\cftaftertoctitleskip}{0 pt}
\setlength{\cftbeforetoctitleskip}{0 pt}

\newcommand{\bs}{\\[3pt]}
\newcommand{\vn}{\\[6pt]}
\newcommand{\fig}[1]{\begin{figure}
		\centering
		\includegraphics[]{\asym{#1}}
\end{figure}}

\newcommand{\sectionbreak}{\clearpage} % New page on each section

% Equation comments
\newcommand{\cm}[1]{\llap{\color{blue} #1}}

\newcommand\fork[2]{\begin{tcolorbox}[boxrule=0.3 mm,arc=0mm,enhanced jigsaw,breakable,colback=yellow!7] {\large \textbf{#1 (explanation)} \vspace{5 pt}\\} #2 \end{tcolorbox}\vspace{-5pt} }


%%% SECTION HEADLINES %%%

% Our numbers
\newcommand{\likteikn}{The equal sign}
\newcommand{\talsifverd}{Numbers, digits and values}
\newcommand{\koordsys}{Coordinate systems}

% Calculations
\newcommand{\adi}{Addition}
\newcommand{\sub}{Subtraction}
\newcommand{\gong}{Multiplication}
\newcommand{\del}{Division}

%Factorization and order of operations
\newcommand{\fak}{Factorization}
\newcommand{\rrek}{Order of operations}

%Fractions
\newcommand{\brgrpr}{Introduction}
\newcommand{\brvu}{Values, expanding and simplifying}
\newcommand{\bradsub}{Addition and subtraction}
\newcommand{\brgngheil}{Fractions multiplied by integers}
\newcommand{\brdelheil}{Fractions divided by integers}
\newcommand{\brgngbr}{Fractions multiplied by fractions}
\newcommand{\brkans}{Cancelation of fractions}
\newcommand{\brdelmbr}{Division by fractions}
\newcommand{\Rasjtal}{Rational numbers}

%Negative numbers
\newcommand{\negintro}{Introduction}
\newcommand{\negrekn}{The elementary operations}
\newcommand{\negmeng}{Negative numbers as amounts}

% Geometry 1
\newcommand{\omgr}{Terms}
\newcommand{\eignsk}{Attributes of triangles and quadrilaterals}
\newcommand{\omkr}{Perimeter}
\newcommand{\area}{Area}

%Algebra 
\newcommand{\algintro}{Introduction}
\newcommand{\pot}{Powers}
\newcommand{\irrasj}{Irrational numbers}

%Equations
\newcommand{\ligintro}{Introduction}
\newcommand{\liglos}{Solving with the elementary operations}
\newcommand{\ligloso}{Solving with elementary operations summarized}

%Functions
\newcommand{\fintro}{Introduction}
\newcommand{\lingraf}{Linear functions and graphs}

%Geometry 2
\newcommand{\geoform}{Formulas of area and perimeter}
\newcommand{\kongogsim}{Congruent and similar triangles}
\newcommand{\geofork}{Explanations}

% Names of rules
\newcommand{\adkom}{Addition is commutative}
\newcommand{\gangkom}{Multiplication is commutative}
\newcommand{\brdef}{Fractions as rewriting of division}
\newcommand{\brtbr}{Fractions multiplied by fractions}
\newcommand{\delmbr}{Fractions divided by fractions}
\newcommand{\gangpar}{Distributive law}
\newcommand{\gangparsam}{Paranthesis multiplied together}
\newcommand{\gangmnegto}{Multiplication by negative numbers I}
\newcommand{\gangmnegtre}{Multiplication by negative numbers II}
\newcommand{\konsttre}{Unique construction of triangles}
\newcommand{\kongtre}{Congruent triangles}
\newcommand{\topv}{Vertical angles}
\newcommand{\trisum}{The sum of angles in a triangle}
\newcommand{\firsum}{The sum of angles in a quadrilateral}
\newcommand{\potgang}{Multiplication by powers}
\newcommand{\potdivpot}{Division by powers}
\newcommand{\potanull}{The special case of \boldmath $a^0$}
\newcommand{\potneg}{Powers with negative exponents}
\newcommand{\potbr}{Fractions as base}
\newcommand{\faktgr}{Factors as base}
\newcommand{\potsomgrunn}{Powers as base}
\newcommand{\arsirk}{The area of a circle}
\newcommand{\artrap}{The area of a trapezoid}
\newcommand{\arpar}{The area of a parallelogram}
\newcommand{\pyt}{Pythagoras's theorem}
\newcommand{\forform}{Ratios in similar triangles}
\newcommand{\vilkform}{Terms of similar triangles}
\newcommand{\omkrsirk}{The perimeter of a circle (and the value of $ \bm \pi $)}
\newcommand{\artri}{The area of a triangle}
\newcommand{\arrekt}{The area of a rectangle}
\newcommand{\liknflyt}{Moving terms across the equal sign}
\newcommand{\funklin}{Linear functions}

%License
\newcommand{\lic}{\textit{First Principles of Math by Sindre Sogge Heggen is licensed under CC BY-NC-SA 4.0. To view a copy of this license, visit\\ 
		\net{http://creativecommons.org/licenses/by-nc-sa/4.0/}{http://creativecommons.org/licenses/by-nc-sa/4.0/}}}

%referances
\newcommand{\net}[2]{{\color{blue}\href{#1}{#2}}}
\newcommand{\hrs}[2]{\hyperref[#1]{\color{blue}\textsl{#2 \ref*{#1}}}}
\newcommand{\rref}[1]{\hrs{#1}{Rule}}
\newcommand{\refkap}[1]{\hrs{#1}{Chapter}}
\newcommand{\refsec}[1]{\hrs{#1}{Section}}

\usepackage{datetime2}
\usepackage[]{hyperref}

\begin{document}

\section{\ligintro}
Even though every mathematical expression involving \sym{$=$} is an \textit{equation}\index{equation}, the word is, traditionally, closely linked to the presence of an \\\textit{unknown} number.\vsk

Say we want to find the number which when added by $4$ results in $7$. The name of this unknown number is free to choose but commonly it's called $ x $. Our equation can now be written as
\[ x+4=7 \]
The $ x $-value\footnote{In other cases it can be several values.} which results in the same values on each side of the equal sign is the \textit{solution} of the equation.
It is nothing wrong done by simply observing what the value of $ x $ must be. Probably you have already realized that $ {x=3} $ is the solution of the equation, since
\[ 3+4=7 \]
However, most equations are difficult to solve simply by observing, and it is therefore vice to take the advantage of more general methods. In reality, there is only one principle to follow: \regv
\st{ \label{principle}
We can always carry out a mathematical operation on one of the sides of the equal sign, as long as we carry out the operation on the other side too.}
The mathematical operations presented in this book is the four elementary operations. Concerning these the principal sounds:\regv
\st{We can always add, subtract, multiply or divide by a number on one side of the equal sign, as long as we also do it on the other side.}\regv
The principle follows from the meaning of \sym{$=$}. When two expressions are of equal value, their values are necessarily still equal as long as we carry out identical mathematical operations on them. Anyways, in the coming section we'll specify this principle for every single elementary operation. If you already feel things make sense, you can, without no great loss of insight, skip to \refsec{ligsaml}.
\section{\liglos \label{likloysfire}}
\textit{In the figures of this section we'll understand equations from what we call the weight principle. In that case, \sym{$=$} indicates\footnote{\sym{$\neq$} symbols ''not equal''.} there is equal weight (equal value) on the left and the right side.} \vs
\begin{figure}
	\centering
	\includegraphics[]{\asym{lig}}\qquad
	\includegraphics[]{\asym{lig1}}
\end{figure} \vs \vspace{-3pt}
\subsection*{Addition and subtraction; moving terms}
\subsubsection*{First example} \vspace{-3pt}
We have already found the solution of this equation, but let's now solve it in a different way\footnote{\mer In earlier figures, there have been a correspondence between the size of the boxes and the (absolute) value of the number they represent. This does not apply to the boxes representing $ x $.}: \vs
\[ x+4=7 \]
\fig{lig2}
The value of $ x $ becomes clear if $ x $ is alone on one of the sides, and we can isolate $ x $ on the left side by removing 4. But if we are to remove 4 from the left side, we must also remove 4 from the right side, in order to preserve equal values on both sides. \vspace{-3pt}
\[ x+4-{\color{red}4}=7-{\color{red}4}  \]
\fig{lig2b}
Since $ 4-{\color{red}4}=0 $ and $ 7-{\color{red}4}=3 $, we get 
\[ x=3 \]
\fig{lig2c}
In a more abbreviated way this can be written as

\prbxl{0.5}{
	\alg{
		x+4 &= 7 \\
		x&= 7-4 \\
		x &= 3
}}
\prbxr{0.5}{Between the first and second line it is common to say that \textsl{4 has shifted side and therefore also sign (from $ + $ to $ - $).}}
\textbf{Second example}\os
Let's move on to a somehow more complex equation\footnote{Notice that the figure illustrates $ {4x+(-2)} $ (see \refsec{negmeng}) on the left side. However, $ {4x+(-2)} $ equals $ {4x-2} $ (see \refsec{rekmneg}).}:
\[ 4x-2=3x+5 \]
\fig{lig4}
To get an expression with $ x $ exclusively on one side, we remove $ 3x $ on both sides:
\[ 4x-2-{\color{red}3x}=3x+5-{\color{red}3x} \]
\fig{lig5}
Now,
\[ x-2=5 \]
\fig{lig5b}
To isolate $ x $ we add 2 on the left side. Then we must also add $ 2 $ on the right side:
\[ x-2+{\color{blue}2}=5+{\color{blue}2} \]
\fig{lig6}
\newpage
Hence
\[ x=7 \]
\fig{lig7}
The steps we have made can be summarized in this way:
\begin{flalign*}
&& 4x-2&=3x+5 && \llap{1. figure} \\
&& 4x-{\color{red}3x}-2&=3x-{\color{red}3x}+5 &&  \llap{2. figure} \\
&& x -2 &= 5 &&\llap{3. figure}\\
&& x-2+\color{blue}2&=  5+\color{blue}2 &&\llap{4. figure}\\
&& x &= 7 &&\llap{5. figure}
\end{flalign*}
In a more abbreviated way we can write
\alg{
4x-2&=3x+5 \\
4x-{\color{red}3x}&=5+\color{blue}2\\
x &= 7
}

\reg[Moving numbers across the equal sign \label{bytt}]{To solve an equation, we gather all $x$-terms and all known terms on respective sides of the equal sign. A term which shifts side, also shifts sign.}

\eks[1]{Solve the equation
\[ 3x+3 =2x+5 \]
\sv	
	 \vs \vs \vs \vs
	\begin{align*}
	3x-2x &=5-3 \\
	x &=2
	\end{align*}  \vspace{-20pt}}

\eks[2]{Solve the equation
\[ -4x-3 =-5x+12  \]	
\sv
	 \vs \vs \vs \vs
	\begin{align*}
	-4x+5x &=12+3 \\
	x &=15
	\end{align*}}
\subsection*{Multiplication and division}	
\subsubsection{Division}
So far we have studied equations resulting in a single instance of $x $ on one side of the equal sign. Often there are several instances of $ x $, as, for example, in the equation
\[ 3x=6 \]
\fig{lig8}
If we separate the left side into three equal groups, we get a single $ x $ in each group. And by separating the right side into three equal groups, all groups present are of equal value
\[ \frac{3x}{3}=\frac{6}{3} \]
\fig{lig9}
Therefore
\[ x=2 \]
\fig{lig10}
Let's summarize our calculations:
\begin{flalign*}
&& 3x&=6 && \llap{1. figure} \br
&& \frac{\cancel{3}x}{\cancel{3}}&=\frac{6}{3} && \llap{2. figure} \br
&& x&=2 && \llap{3. figure}
\end{flalign*}
\newpage
\reg[Division on both sides of an equation \label{ligdel}]{We can divide both sides of an equation by the same number.}
\eks[1]{ Solve the equation
	\[ 	4x = 20  \]
	\sv \vs \vs \vsb
	\begin{align*}
	\frac{\cancel{4}x}{\cancel{4}}&=\frac{20}{4} \\
	x &=5
	\end{align*}
	\vspace{-20 pt}
	}
	
\eks[2]{Solve the equation
	\[2x+6 =3x-2 \]
\sv \vs \vs \vs \vs
	\begin{flalign*}
	&& 2x-3x &= -2-6 &&\\
	&&-x &= -8&& \\
	&& \frac{\cancel{-1}x}{\cancel{-1}} &= \frac{-8}{-1} &&\cm{($-x=-1x$)}\\
	&& x &= 8&&
	\end{flalign*}}

\subsubsection*{Multiplication}
Let's solve the equation
\[ \frac{x}{3}=4 \]
\fig{lig11}
We can get a unit $ x $ on the left side if we add two more instances of $ \frac{x}{3} $. The equation informs that $ \frac{x}{3} $ equals 4, this implies that for every instance of $ \frac{x}{3} $ we add to the left side, we must add 4 to the right side, in order to keep the balance.
\[ \frac{x}{3}+\frac{x}{3}+\frac{x}{3}=4+4+4 \]
\fig{lig12}
Now we notice that \y{\frac{x}{3}+\frac{x}{3}+\frac{x}{3}=\frac{x}{3}\cdot3} and that \y{4+4+4=4\cdot3}:
\[ \frac{x}{3}\cdot 3 = 4\cdot 3 \]
\fig{lig12b}
Since $ \frac{x}{3}\cdot3=x $ and $ 4\cdot3=12 $, we have
\[ x=12 \]
\fig{lig13}
Our steps can be summarized in the following way:
\begin{flalign*}
&& \frac{x}{3}&=4 && \llap{1. figure} \br 
&& \frac{x}{3}+\frac{x}{3}+\frac{x}{3} &= 4+4+4  &&\llap{2. figure} \br
&& \frac{x}{3}\cdot 3&=4\cdot3 && \llap{3. figure} \\
&& x&=12 && \llap{4. figure}
\end{flalign*}
In a more abbreviated form this can be written as
\alg{
\frac{x}{3}&= 4 \br 
\frac{x}{\cancel{3}}\cdot \cancel{3} &= 4\cdot 3 \br
x &= 12
}
\newpage
\reg[Multiplication on both sides of an equation]{
We can multiply both sides of an equation by the same number.
}
\eks[1]{
Solve the equation
\[ \frac{x}{5}=2 \]
\sv \vsb \vs

\algv{
\frac{x}{\cancel{5}}\cdot\cancel{5}&=2\cdot5 \\
x &= 10
}
}
\eks[2]{
Solve the equation
\[ \frac{7x}{10}-5=13+\frac{x}{10} \]
\sv \vsb \vs

\algv{
\frac{7x}{10}-\frac{x}{10}&=13+5\br
\frac{6x}{10}&=18 \br
\frac{6x}{\cancel{10}}\cdot \cancel{10}&=18\cdot10 \\
6x&=180 \br
\frac{\cancel{6}x}{\cancel{6}}&=\frac{180}{6}\br
x&=30
}
}
\newpage
\section{\ligloso \label{ligsaml}}
\reg[Solving methods with elementary operations \label{lsmlig}]{
We can always
\begin{itemize}
\item add or subtract both sides of an equation by the same number. This is equivalent to shifting a term from one side to the other, also shifting the terms sign.
\item multiply or divide both sides of an equation by the same number.
\end{itemize}
}
\eks[1]{
Solve the equation 
\[ 3x-4=6+2x \]

\sv \vs \vs

\algv{
3x-2x&=6+4 \\
x&=10
}
}
\eks[2]{
Solve the equation
\[ 9-7x=8x+3 \]
\sv \vs \vs
\algv{
 9-7x&=-8x+3 \\
  8x-7x&=3-9 \\
  x&=-6
}
}
\newpage
\eks[3]{
	Solve the equation
	\[ 10x-20=7x-5 \]
	\sv \vs \vs
	\algv{
		10x-20&=7x-5 \\
		10x-7x&=20-5 \\
		3x&=15 \\
		\frac{\cancel{3}x}{\cancel{3}}&=\frac{15}{3} \\
		x&=5
	}
}
\eks[4]{
Solve the equation
\[ 15-4x=x+5 \]

\sv \vs \vsb
\alg{
15-5&=x+4x \\
10&=5x\\
\frac{10}{5}&=\frac{\cancel{5}x}{\cancel{5}}\\
2&=x
}
}	

\eks[5]{
Solve the equation
\[ \frac{4x}{9}-20=8-\frac{3x}{9} \]

\sv \vs \vsb
\alg{
\frac{4x}{9}+\frac{3x}{9}&=20+8 \\
\frac{\cancel{7}x}{9\cdot \cancel{7}}&=\frac{28}{7} \br
\frac{x}{\cancel{9}}\cdot \cancel{9}&=4\cdot 9\\
x&=36
}
}

\newpage
\eks[6]{
Solve the equation	
	\[ \frac{1}{3}x+\frac{1}{6}=\frac{5}{12}x+2  \vs\]	
	
	\sv
	To avoid fractions, we multiply both sides by the common denominator 12:
	\begin{align}
	\left(\frac{1}{3}x+\frac{1}{6}\right)12&=\left(\frac{5}{12}x+2\right)12 \br	
	\frac{1}{3}x\cdot12+\frac{1}{6}\cdot12&=\frac{5}{12}x\cdot12+2\cdot12 \tag{$ \ast $}\br	
	4x+2 &= 5x+24 \\
	4x-5x &= 24-2 \\
	-x &= 22 \\
	\frac{\cancel{-1}\,x}{\cancel{-1}} &= \frac{22}{-1} \\
	x &= -22
	\end{align} 
}
\info{Tip}{
	Some like to make the rule that ''{we can multiply or divide all terms by the same number}''. In that case, we could have jumped to the second line in the calculations of the example above.
}
\newpage
\eks[7]{
Solve the equation
\[ 3-\frac{6}{x}= 2+\frac{5}{2x} \vs\]

\sv
We multiply both sides by the common denominator $ 2x $:
\alg{
	 2x\left(3-\frac{6}{x}\right)&= 2x\left(2+\frac{5}{2x}\right) \br
6x -12 &= 4x+5\\
6x-4x &= 5+12\\
2x &= 17\\
x&=\frac{17}{2}
}
}
\section{Power equations}
Let's solve the equation
\[ x^2=9 \]
This is called a \textit{power equation}\index{power equation}. In general, power equations are difficult to solve applying the four elementary operations only. Applying power-rules, we raise both sides to the power of the inverse of the exponent associated with $ x $:
\[ \left(x^2\right)^\frac{1}{2}=9^\frac{1}{2} \]
By \rref{potsomgrunn}, we have
\alg{
x^{2\cdot\frac{1}{2}}&=9^\frac{1}{2} \\
x&=9^\frac{1}{2}
}
Since $ 3^2=9 $, we have $ 9^\frac{1}{2}=3 $. Now observe this: \vsk

\textit{The principle stated on page \pageref{principle} says we can, like we just did, carry out a mathematical operation on both sides of an equation. However, sticking to this principle does not guarantee that all solutions are found.} \\ \vsk

Concerning our equation, we know that $ {x=3} $ is a solution. For the sake of it, we can confirm this by the calculation
\[ 3^2=3\cdot3=9 \]
But we also have
\[ (-3)^2=(-3)(-3)=9 \]
Hence, $ -3 $ is also a solution of our original equation!

\reg[Power equations]{
An equation which can be written as
\[ x^a=b \]
where $ a $ and $ b $ are constants,
is a \textit{power} equation. \vsk

The equation has $ a $ distinct solutions.
} 
\newpage
\eks[1]{
	Solve the equation 
	\[ x^2+5= 21\]
	\sv \vs \vs \vs
	\algv{
		x^2+5&= 21\\
		x^2 &= 21-5\\
		x^2 &= 16
	}
	Since $ {4\cdot4 =16} $ and $ {(-4)\cdot(-4)=16} $, we have
	\[ x=4\qquad\vee\qquad x=-4 \]
}
\eks[2]{
Solve the equation
\[ 3x^2+1=7 \]
\sv \vs \vs \vs
\alg{
3x^2&=7-1 \\
3x^2&=6 \\
\frac{\cancel{3}x^2}{\cancel{3}}&=\frac{6}{3}\\
x^2&=2}
Hence,
\[ x=\sqrt{2}\qquad\vee\qquad x=-\sqrt{2} \]
}
\info{\merk}{
Although the equation
\[ x^a=b \]
has $ a $ solutions, they are not necessarily all \textit{real}\footnote{As earlier mentioned, \textit{real} and \textit{imaginary} numbers lie outside the scope of this book}. Concerning this book, it means we settle with finding all rational or irrational numbers which solves the equation. For example,
\[ x^3=8 \]
has 3 solutions, but we settle with the solution $ x=2 $.
}
\end{document}



\input{../doc}
\input{../preamb_eng}
\begin{document}

\section{\ligintro}
Even though every mathematical expression involving \sym{$=$} is an \textit{equation}\index{equation}, the word is, traditionally, closely linked to having an \textit{unknown} number.\vsk

Say we want to find the number which when added by $4$ results in $7$. The name of this unknown number is free to chose, but it is most common to call it $ x $. Our equation can now be written as
\[ x+4=7 \]
The $ x $-value\footnote{In other cases it can be several values.} which results in the same values on each side of the equal sign is the \textit{solution} of the equation.
It it is nothing wrong done by simply observing what the value of $ x $ must be. Maybe you have already realized that $ {x=3} $ is the solution of the equation, since
\[ 3+4=7 \]
However, most equations are difficult to solve simply by observing, and it is therefore vice to take the advantage of more general methods. In reality, there are only one principle to follow: \regv
\st{
We can always carry out one mathematical operation on one of the sides of the equal sign, as long as we carry out the operation on the other side too.}
The mathematical operations presented in this book is the four elementary operations. Concerning these we the principle sounds\regv
\st{We can always add, subtract, multiply or divide by a number on one side of the equal sign, as long as we also do it on the other side.}\regv
The principle follows from the meaning of \sym{$=$}. When two expressions are of equal value, their values are necessarily still equal as long as we carry out identical mathematical operations on them. Still, in the coming section we'll specify this principle for every single elementary operation. If you already feel things make sense you can, without no great loss of insight, skip to section to 
\refsec{ligsaml}.
\section{\liglos \label{likloysfire}}
\textit{In the figures of this section we'll understand equations from what we call the weight principle. In that case, \sym{$=$} indicates\footnote{\sym{$\neq$} symbols ''not equal''.} there is equally much weight (equal value) on the left and the right side.} \vs
\begin{figure}
	\centering
	\includegraphics[]{\asym{lig}}\qquad
	\includegraphics[]{\asym{lig1}}
\end{figure} \vs \vspace{-3pt}
\subsection*{Addition and subtraction; numbers changing sides}
\subsubsection*{First example} \vspace{-3pt}
We have already found the solution of this equation, but let's now solve it in a different way\footnote{\textsl{Notice:} In earlier figures there have been a correspondence between the size of the boxes and the (absolute) value of the number they represent. This does not apply to the boxes representing $ x $.}: \vs
\[ x+4=7 \]
\fig{lig2}
The value of $ x $ becomes clear if $ x $ is alone on one of the sides, and we can isolate $ x $ on the left side by removing 4. But if we are to remove 4 from the left side, we must also remove 4 from the right side in order to preserve equal values on both sides. \vspace{-3pt}
\[ x+4-{\color{red}4}=7-{\color{red}4}  \]
\fig{lig2b}
Since $ 4-{\color{red}4}=0 $ and $ 7-{\color{red}4}=3 $, we get 
\[ x=3 \]
\fig{lig2c}
In a more abbreviated way this can be written as

\prbxl{0.5}{
	\alg{
		x+4 &= 7 \\
		x&= 7-4 \\
		x &= 3
}}
\prbxr{0.5}{Between the first and second line it is common to say that \textsl{4 as changed side and therefore also sign (from $ + $ to $ - $).}}
\textbf{Second example}\os
Let's move on to a somehow more complex equation\footnote{Notice that the figure illustrates $ {4x+(-2)} $ (see \refsec{negmeng}) on the left side. However, $ {4x+(-2)} $ equals $ {4x-2} $ (see \refsec{rekmneg}).}:
\[ 4x-2=3x+5 \]
\fig{lig4}
To get an expression with $ x $ exclusively on one side, we remove $ 3x $ on both sides:
\[ 4x-2-{\color{red}3x}=3x+5-{\color{red}3x} \]
\fig{lig5}
Now,
\[ x-2=5 \]
\fig{lig5b}
To isolate $ x $ we add 2 on the left side. Then we must also add $ 2 $ on the right side:
\[ x-2+{\color{blue}2}=5+{\color{blue}2} \]
\fig{lig6}
\newpage
Hence
\[ x=7 \]
\fig{lig7}
The steps we have made can be summarized in this way:
\begin{flalign*}
&& 4x-2&=3x+5 && \llap{1. figure} \\
&& 4x-{\color{red}3x}-2&=3x-{\color{red}3x}+5 &&  \llap{2. figure} \\
&& x -2 &= 5 &&\llap{3. figure}\\
&& x-2+\color{blue}2&=  5+\color{blue}2 &&\llap{4. figure}\\
&& x &= 7 &&\llap{5. figure}
\end{flalign*}
In a more abbreviated way we can write
\alg{
4x-2&=3x+5 \\
4x-{\color{red}3x}&=5+\color{blue}2\\
x &= 7
}

\reg[Changing numbers across the equal sign \label{bytt}]{To solve an equation, we gather all $x$-terms and all known terms on respective sides of the equal sign. A term which changes sides, also changes sign.}

\eks[1]{Solve the equation
\[ 3x+3 =2x+5 \]
\sv	
	 \vs \vs \vs \vs
	\begin{align*}
	3x-2x &=5-3 \\
	x &=2
	\end{align*}  \vspace{-20pt}}

\eks[2]{Solve the equation
\[ -4x-3 =-5x+12  \]	
\sv
	 \vs \vs \vs \vs
	\begin{align*}
	-4x+5x &=12+3 \\
	x &=15
	\end{align*}}
\subsection*{Gonging og deling}	
\subsubsection{Divisjon}
So far we have studied equations which resulted in a single instance of $x $ on one side of the equal sign. Often there are several instances of $ x $, as, for example, in the equation
\[ 3x=6 \]
\fig{lig8}
If we separate the left side into three equal groups, we get a single $ x $ in each group. And by separating the right side into three equal groups, all groups present are of equal value
\[ \frac{3x}{3}=\frac{6}{3} \]
\fig{lig9}
Therefore
\[ x=2 \]
\fig{lig10}
Let's summarize our calculations:\\
\prbxl{0.6}{\begin{flalign*}
	&& 3x&=6 && \llap{1. figure} \br
	&& \frac{3x}{3}&=\frac{6}{3} && \llap{2. figure} \br
	&& x&=2 && \llap{3. figure}
	\end{flalign*}}\qquad\qquad
\prbxr{0.25}{Du huskar kanskje at vi gjerne skriv
\[ \frac{\cancel{3}x}{\cancel{3}} \] 
}
\newpage
\reg[Deling på begge sider av ei likning \label{ligdel}]{We can divide both sides of an equation by the same number.}
\eks[1]{ Solve the equation
	\[ 	4x = 20  \]
	\sv \vs \vs \vsb
	\begin{align*}
	\frac{\cancel{4}x}{\cancel{4}}&=\frac{20}{4} \\
	x &=5
	\end{align*}
	\vspace{-20 pt}
	}
	
\eks[2]{Solve the equation
	\[2x+6 =3x-2 \]
\sv \vs \vs \vs \vs
	\begin{flalign*}
	&& 2x-3x &= -2-6 &&\\
	&&-x &= -8&& \\
	&& \frac{\cancel{-1}x}{\cancel{-1}} &= \frac{-8}{-1} &&\cm{($-x=-1x$)}\\
	&& x &= 8&&
	\end{flalign*}}

\subsection*{Gonging}
Let's solve the equation
\[ \frac{x}{3}=4 \]
\fig{lig11}
We can get one single $ x $ on the left side if we add two more instances of $ \frac{x}{3} $. The equation informs that $ \frac{x}{3} $ equals 4, this implies that for every instance of $ \frac{x}{3} $ we add to the left side, we must add 4 to the right side, in order to keep the balance.
\[ \frac{x}{3}+\frac{x}{3}+\frac{x}{3}=4+4+4 \]
\fig{lig12}
Now we notice that \y{\frac{x}{3}+\frac{x}{3}+\frac{x}{3}=\frac{x}{3}\cdot3} and that \y{4+4+4=4\cdot3}:
\[ \frac{x}{3}\cdot 3 = 4\cdot 3 \]
\fig{lig12b}
Since $ \frac{x}{3}\cdot3=x $ and $ 4\cdot3=12 $, we have
\[ x=12 \]
\fig{lig13}
Our steps can be summarized in the following way:
\begin{flalign*}
&& \frac{x}{3}&=4 && \llap{1. figure} \br 
&& \frac{x}{3}+\frac{x}{3}+\frac{x}{3} &= 4+4+4  &&\llap{2. figure} \br
&& \frac{x}{3}\cdot 3&=4\cdot3 && \llap{3. figure} \\
&& x&=12 && \llap{4. figure}
\end{flalign*}
In a more abbreviated form this can be written as
\alg{
\frac{x}{3}&= 4 \br 
\frac{x}{\cancel{3}}\cdot \cancel{3} &= 4\cdot 3 \br
x &= 12
}
\newpage
\reg[Gonging på begge sider av ei likning]{
We can multiply both sides of an equation by the same number.
}
\eks[1]{
Solve the equation
\[ \frac{x}{5}=2 \]
\sv \vsb \vs

\algv{
\frac{x}{\cancel{5}}\cdot\cancel{5}&=2\cdot5 \\
x &= 10
}
}
\eks[2]{
Solve the equation
\[ \frac{7x}{10}-5=13+\frac{x}{10} \]
\sv \vsb \vs

\algv{
\frac{7x}{10}-\frac{x}{10}&=13+5\br
\frac{6x}{10}&=18 \br
\frac{6x}{\cancel{10}}\cdot \cancel{10}&=18\cdot10 \\
6x&=180 \br
\frac{\cancel{6}x}{\cancel{6}}&=\frac{180}{6}\br
x&=30
}
}
\newpage
\section{\ligloso \label{ligsaml}}
\reg[Løysingsmetodar for likningar \label{lsmlig}]{
We can always
\begin{itemize}
\item add or subtract both sides of an equation by the same number. This is equaivalent to changing a term from one side to the other, as long as the terms sign is also changed.
\item multiply or divide both sides of an equation by the same number.
\end{itemize}
}
\eks[1]{
Solve the equation 
\[ 3x-4=6+2x \]

\sv \vs \vs

\algv{
3x-2x&=6+4 \\
x&=10
}
}
\eks[2]{
Solve the equation
\[ 9-7x=8x+3 \]
\sv \vs \vs
\algv{
 9-7x&=-8x+3 \\
  8x-7x&=3-9 \\
  x&=-6
}
}
\newpage
\eks[3]{
	Solve the equation
	\[ 10x-20=7x-5 \]
	\sv \vs \vs
	\algv{
		10x-20&=7x-5 \\
		10x-7x&=20-5 \\
		3x&=15 \\
		\frac{\cancel{3}x}{\cancel{3}}&=\frac{15}{3} \\
		x&=5
	}
}
\eks[4]{
Solve the equation
\[ 15-4x=x+5 \]

\sv \vs \vsb
\alg{
15-5&=x+4x \\
10&=5x\\
\frac{10}{5}&=\frac{\cancel{5}x}{\cancel{5}}\\
2&=x
}
}	

\eks[5]{
Solve the equation
\[ \frac{4x}{9}-20=8-\frac{3x}{9} \]

\sv \vs \vsb
\alg{
\frac{4x}{9}+\frac{3x}{9}&=20+8 \\
\frac{\cancel{7}x}{9\cdot \cancel{7}}&=\frac{28}{7} \br
\frac{x}{\cancel{9}}\cdot \cancel{9}&=4\cdot 9\\
x&=36
}
}

\newpage
\eks[6]{
Solve the equation	
	\[ \frac{1}{3}x+\frac{1}{6}=\frac{5}{12}x+2  \vs\]	
	
	\sv
	To avoid fractions, we multiply both sides by the common denominator 12:
	\begin{align}
	\left(\frac{1}{3}x+\frac{1}{6}\right)12&=\left(\frac{5}{12}x+2\right)12 \br	
	\frac{1}{3}x\cdot12+\frac{1}{6}\cdot12&=\frac{5}{12}x\cdot12+2\cdot12 \tag{$ \ast $}\br	
	4x+2 &= 5x+24 \\
	4x-5x &= 24-2 \\
	-x &= 22 \\
	\frac{\cancel{-1}\,x}{\cancel{-1}} &= \frac{22}{-1} \\
	x &= -22
	\end{align} 
}
\info{Tip}{
	There are som who like to make the rule that ''{we can multiply or divide all terms by the same number}''. In that case, we could have jumped to the second line in the calculations of the example above.
}
\newpage
\eks[7]{
Solve the equation
\[ 3-\frac{6}{x}= 2+\frac{5}{2x} \vs\]

\sv
We multiply both sides by the common denominator $ 2x $:
\alg{
	 2x\left(3-\frac{6}{x}\right)&= 2x\left(2+\frac{5}{2x}\right) \br
6x -12 &= 4x+5\\
6x-4x &= 5+12\\
2x &= 17\\
x&=\frac{17}{2}
}
}
\begin{comment}
\eks[]{
Gitt eit trapes med to parallelle sider $ a $ og $ b $ og høgde $ h $. Da er arealet $ A $ gitt som
\[ A=\frac{(a+b)h}{2} \]
Løys likninga med hensyn på $ b $.

\sv \vs \vsb
\alg{
 A&=\frac{(a+b)h}{2} \br
 2A&=(a+b)h \\
 2A&=ah+bh \\
 2A-ah&=bh \\
 \frac{2A-ah}{h}&=b
}
}
\end{comment}
\end{document}



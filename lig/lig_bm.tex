\documentclass[english,hidelinks,pdftex, 11 pt, class=report,crop=false]{standalone}
\usepackage[T1]{fontenc}
\usepackage[utf8]{luainputenc}
\usepackage{lmodern} % load a font with all the characters
\usepackage{geometry}
\geometry{verbose,paperwidth=16.1 cm, paperheight=24 cm, inner=2.3cm, outer=1.8 cm, bmargin=2cm, tmargin=1.8cm}
\setlength{\parindent}{0bp}
\usepackage{import}
\usepackage[subpreambles=false]{standalone}
\usepackage{amsmath}
\usepackage{amssymb}
\usepackage{esint}
\usepackage{babel}
\usepackage{tabu}
\makeatother
\makeatletter

\usepackage{titlesec}
\usepackage{ragged2e}
\RaggedRight
\raggedbottom
\frenchspacing

% Norwegian names of figures, chapters, parts and content
\addto\captionsenglish{\renewcommand{\figurename}{Figur}}
\makeatletter
\addto\captionsenglish{\renewcommand{\chaptername}{Kapittel}}
\addto\captionsenglish{\renewcommand{\partname}{Del}}

\addto\captionsenglish{\renewcommand{\contentsname}{Innhald}}

\usepackage{graphicx}
\usepackage{float}
\usepackage{subfig}
\usepackage{placeins}
\usepackage{cancel}
\usepackage{framed}
\usepackage{wrapfig}
\usepackage[subfigure]{tocloft}
\usepackage[font=footnotesize,labelfont=sl]{caption} % Figure caption
\usepackage{bm}
\usepackage[dvipsnames, table]{xcolor}
\definecolor{shadecolor}{rgb}{0.105469, 0.613281, 1}
\colorlet{shadecolor}{Emerald!15} 
\usepackage{icomma}
\makeatother
\usepackage[many]{tcolorbox}
\usepackage{multicol}
\usepackage{stackengine}

% For tabular
\usepackage{array}
\usepackage{multirow}
\usepackage{longtable} %breakable table

% Ligningsreferanser
\usepackage{mathtools}
\mathtoolsset{showonlyrefs}

% index
\usepackage{imakeidx}
\makeindex[title=Indeks]

%Footnote:
\usepackage[bottom, hang, flushmargin]{footmisc}
\usepackage{perpage} 
\MakePerPage{footnote}
\addtolength{\footnotesep}{2mm}
\renewcommand{\thefootnote}{\arabic{footnote}}
\renewcommand\footnoterule{\rule{\linewidth}{0.4pt}}
\renewcommand{\thempfootnote}{\arabic{mpfootnote}}

%colors
\definecolor{c1}{cmyk}{0,0.5,1,0}
\definecolor{c2}{cmyk}{1,0.25,1,0}
\definecolor{n3}{cmyk}{1,0.,1,0}
\definecolor{neg}{cmyk}{1,0.,0.,0}

% Lister med bokstavar
\usepackage[inline]{enumitem}

\newcounter{rg}
\numberwithin{rg}{chapter}
\newcommand{\reg}[2][]{\begin{tcolorbox}[boxrule=0.3 mm,arc=0mm,colback=blue!3] {\refstepcounter{rg}\phantomsection \large \textbf{\therg \;#1} \vspace{5 pt}}\newline #2  \end{tcolorbox}\vspace{-5pt}}

\newcommand\alg[1]{\begin{align} #1 \end{align}}

\newcommand\eks[2][]{\begin{tcolorbox}[boxrule=0.3 mm,arc=0mm,enhanced jigsaw,breakable,colback=green!3] {\large \textbf{Eksempel #1} \vspace{5 pt}\\} #2 \end{tcolorbox}\vspace{-5pt} }

\newcommand{\st}[1]{\begin{tcolorbox}[boxrule=0.0 mm,arc=0mm,enhanced jigsaw,breakable,colback=yellow!12]{ #1} \end{tcolorbox}}

\newcommand{\spr}[1]{\begin{tcolorbox}[boxrule=0.3 mm,arc=0mm,enhanced jigsaw,breakable,colback=yellow!7] {\large \textbf{Språkboksen} \vspace{5 pt}\\} #1 \end{tcolorbox}\vspace{-5pt} }

\newcommand{\sym}[1]{\colorbox{blue!15}{#1}}

\newcommand{\info}[2]{\begin{tcolorbox}[boxrule=0.3 mm,arc=0mm,enhanced jigsaw,breakable,colback=cyan!6] {\large \textbf{#1} \vspace{5 pt}\\} #2 \end{tcolorbox}\vspace{-5pt} }

\newcommand\algv[1]{\vspace{-11 pt}\begin{align*} #1 \end{align*}}

\newcommand{\regv}{\vspace{5pt}}
\newcommand{\mer}{\textsl{Merk}: }
\newcommand\vsk{\vspace{11pt}}
\newcommand\vs{\vspace{-11pt}}
\newcommand\vsb{\vspace{-16pt}}
\newcommand\sv{\vsk \textbf{Svar:} \vspace{4 pt}\\}
\newcommand\br{\\[5 pt]}
\newcommand{\asym}[1]{../fig/#1}
\newcommand\algvv[1]{\vs\vs\begin{align*} #1 \end{align*}}
\newcommand{\y}[1]{$ {#1} $}
\newcommand{\os}{\\[5 pt]}
\newcommand{\prbxl}[2]{
\parbox[l][][l]{#1\linewidth}{#2
	}}
\newcommand{\prbxr}[2]{\parbox[r][][l]{#1\linewidth}{
		\setlength{\abovedisplayskip}{5pt}
		\setlength{\belowdisplayskip}{5pt}	
		\setlength{\abovedisplayshortskip}{0pt}
		\setlength{\belowdisplayshortskip}{0pt} 
		\begin{shaded}
			\footnotesize	#2 \end{shaded}}}

\renewcommand{\cfttoctitlefont}{\Large\bfseries}
\setlength{\cftaftertoctitleskip}{0 pt}
\setlength{\cftbeforetoctitleskip}{0 pt}

\newcommand{\bs}{\\[3pt]}
\newcommand{\vn}{\\[6pt]}
\newcommand{\fig}[1]{\begin{figure}
		\centering
		\includegraphics[]{\asym{#1}}
\end{figure}}


\newcommand{\sectionbreak}{\clearpage} % New page on each section

% Equation comments
\newcommand{\cm}[1]{\llap{\color{blue} #1}}

\newcommand\fork[2]{\begin{tcolorbox}[boxrule=0.3 mm,arc=0mm,enhanced jigsaw,breakable,colback=yellow!7] {\large \textbf{#1 (forklaring)} \vspace{5 pt}\\} #2 \end{tcolorbox}\vspace{-5pt} }




%colors
\newcommand{\colr}[1]{{\color{red} #1}}
\newcommand{\colb}[1]{{\color{blue} #1}}
\newcommand{\colo}[1]{{\color{orange} #1}}
\newcommand{\colc}[1]{{\color{cyan} #1}}
\definecolor{projectgreen}{cmyk}{100,0,100,0}
\newcommand{\colg}[1]{{\color{projectgreen} #1}}

%%% SECTION HEADLINES %%%

% Our numbers
\newcommand{\likteikn}{Likskapsteiknet}
\newcommand{\talsifverd}{Tal, siffer og verdi}
\newcommand{\koordsys}{Koordinatsystem}

% Calculations
\newcommand{\adi}{Addisjon}
\newcommand{\sub}{Subtraksjon}
\newcommand{\gong}{Multiplikasjon (Gonging)}
\newcommand{\del}{Divisjon (deling)}

%Factorization and order of operations
\newcommand{\fak}{Faktorisering}
\newcommand{\rrek}{Reknerekkefølge}

%Fractions
\newcommand{\brgrpr}{Introduksjon}
\newcommand{\brvu}{Verdi, utviding og forkorting av brøk}
\newcommand{\bradsub}{Addisjon og subtraksjon}
\newcommand{\brgngheil}{Brøk gonga med heiltal}
\newcommand{\brdelheil}{Brøk delt med heiltal}
\newcommand{\brgngbr}{Brøk gonga med brøk}
\newcommand{\brkans}{Kansellering av faktorar}
\newcommand{\brdelmbr}{Deling med brøk}
\newcommand{\Rasjtal}{Rasjonale tal}

%Negative numbers
\newcommand{\negintro}{Introduksjon}
\newcommand{\negrekn}{Dei fire rekneartane med negative tal}
\newcommand{\negmeng}{Negative tal som mengde}

% Geometry 1
\newcommand{\omgr}{Omgrep}
\newcommand{\eignsk}{Eigenskapar for trekantar og firkantar}
\newcommand{\omkr}{Omkrins}
\newcommand{\area}{Areal}

%Algebra 
\newcommand{\algintro}{Introduksjon}
\newcommand{\pot}{Potensar}
\newcommand{\irrasj}{Irrasjonale tal}

%Equations
\newcommand{\ligintro}{Introduksjon}
\newcommand{\liglos}{Løysing ved dei fire rekneartane}
\newcommand{\ligloso}{Løysingsmetodane oppsummert}

%Functions
\newcommand{\fintro}{Introduksjon}
\newcommand{\lingraf}{Lineære funksjonar og grafar}

%Geometry 2
\newcommand{\geoform}{Formlar for areal og omkrins}
\newcommand{\kongogsim}{Kongruente og formlike trekantar}
\newcommand{\geofork}{Forklaringar}

% Names of rules
\newcommand{\gangdestihundre}{Å gange desimaltall med 10, 100 osv.}
\newcommand{\delmedtihundre}{Deling med 10, 100, 1\,000 osv.}
\newcommand{\ompref}{Omgjøring av prefikser}
\newcommand{\adkom}{Addisjon er kommutativ}
\newcommand{\gangkom}{Multiplikasjon er kommutativ}
\newcommand{\brdef}{Brøk som omskriving av delestykke}
\newcommand{\brtbr}{Brøk gonga med brøk}
\newcommand{\delmbr}{Brøk delt på brøk}
\newcommand{\gangpar}{Gonging med parentes (distributiv lov)}
\newcommand{\gangparsam}{Parantesar gonga saman}
\newcommand{\gangmnegto}{Gonging med negative tal I}
\newcommand{\gangmnegtre}{Gonging med negative tal II}
\newcommand{\konsttre}{Konstruksjon av trekantar}
\newcommand{\kongtre}{Kongruente trekantar}
\newcommand{\topv}{Toppvinklar}
\newcommand{\trisum}{Summen av vinklane i ein trekant}
\newcommand{\firsum}{Summen av vinklane i ein firkant}
\newcommand{\potgang}{Gonging med potensar}
\newcommand{\potdivpot}{Divisjon med potensar}
\newcommand{\potanull}{Spesialtilfellet \boldmath $a^0$}
\newcommand{\potneg}{Potens med negativ eksponent}
\newcommand{\potbr}{Brøk som grunntal}
\newcommand{\faktgr}{Faktorar som grunntal}
\newcommand{\potsomgrunn}{Potens som grunntal}
\newcommand{\arsirk}{Arealet til ein sirkel}
\newcommand{\artrap}{Arealet til eit trapes}
\newcommand{\arpar}{Arealet til eit parallellogram}
\newcommand{\pyt}{Pytagoras' setning}
\newcommand{\forform}{Forhold i formlike trekantar}
\newcommand{\vilkform}{Vilkår i formlike trekantar}
\newcommand{\omkrsirk}{Omkrinsen til ein sirkel (og $ \bm \pi $)}
\newcommand{\artri}{Arealet til ein trekant}
\newcommand{\arrekt}{Arealet til eit rektangel}
\newcommand{\liknflyt}{Flytting av ledd over likskapsteiknet}
\newcommand{\funklin}{Lineære funksjonar}

%Opg
\newcommand{\abc}[1]{
	\begin{enumerate}[label=\alph*),leftmargin=18pt]
		#1
	\end{enumerate}
}
\newcommand{\abcs}[2]{
	\begin{enumerate}[label=\alph*),start=#1,leftmargin=18pt]
		#2
	\end{enumerate}
}
\newcommand{\abcn}[1]{
	\begin{enumerate}[label=\arabic*),leftmargin=18pt]
		#1
	\end{enumerate}
}
\newcommand{\abch}[1]{
	\hspace{-2pt}	\begin{enumerate*}[label=\alph*), itemjoin=\hspace{1cm}]
		#1
	\end{enumerate*}
}
\newcommand{\abchs}[2]{
	\hspace{-2pt}	\begin{enumerate*}[label=\alph*), itemjoin=\hspace{1cm}, start=#1]
		#2
	\end{enumerate*}
}

\newcommand{\opgt}{\phantomsection \addcontentsline{toc}{section}{Oppgaver} \section*{Oppgaver for kapittel \thechapter}\vs \setcounter{section}{1}}
\newcounter{opg}
\numberwithin{opg}{section}
\newcommand{\op}[1]{\vspace{15pt} \refstepcounter{opg}\large \textbf{\color{blue}\theopg} \vspace{2 pt} \label{#1} \\}
\newcommand{\ekspop}[1]{\vsk\textbf{Gruble \thechapter.#1}\vspace{2 pt} \\}
\newcommand{\nes}{\stepcounter{section}
	\setcounter{opg}{0}}
\newcommand{\opr}[1]{\vspace{3pt}\textbf{\ref{#1}}}
\newcommand{\oeks}[1]{\begin{tcolorbox}[boxrule=0.3 mm,arc=0mm,colback=white]
		\textit{Eksempel: } #1	  
\end{tcolorbox}}
\newcommand\opgeks[2][]{\begin{tcolorbox}[boxrule=0.1 mm,arc=0mm,enhanced jigsaw,breakable,colback=white] {\footnotesize \textbf{Eksempel #1} \\} \footnotesize #2 \end{tcolorbox}\vspace{-5pt} }

%License
\newcommand{\lic}{\textit{Matematikken sine byggesteinar by Sindre Sogge Heggen is licensed under CC BY-NC-SA 4.0. To view a copy of this license, visit\\ 
		\net{http://creativecommons.org/licenses/by-nc-sa/4.0/}{http://creativecommons.org/licenses/by-nc-sa/4.0/}}}

%referances
\newcommand{\net}[2]{{\color{blue}\href{#1}{#2}}}
\newcommand{\hrs}[2]{\hyperref[#1]{\color{blue}\textsl{#2 \ref*{#1}}}}
\newcommand{\rref}[1]{\hrs{#1}{Regel}}
\newcommand{\refkap}[1]{\hrs{#1}{Kapittel}}
\newcommand{\refsec}[1]{\hrs{#1}{Seksjon}}

\usepackage{datetime2}

\usepackage[]{hyperref}

\begin{document}

\section{I\ligintro}
Ethvert matematisk uttrykk som inneholder \sym{$=$} er en \textit{likning}\index{likning}, likevel er ordet likning tradisjonelt knytt til at vi har et \textit{ukjent} tall.\vsk

Si at vi ønsker å finne et tall som er slik at hvis vi legger til $4$, så får vi $7$. Dette talet kan vi kalle for kva som helst, men det vanligste er å kalle det for $ x $, som altså er det ukjente talet vårt. Likningen vår kan nå skrives slik:
\[ x+4=7 \]
$ x $-verdien\footnote{I andre tilfeller kan det være flere verdier.} som gjør at det blir samme verdi på begge sider av likhetstegnet kalles \textit{løsningen} av likningen.
Det er alltids lov til å se eller prøve seg fram for å finne verdien til $ x $. Kanskje har du allerede merket at $ {x=3} $ er løningen av likningen, siden
\[ 3+4=7 \]
Men de fleste likninger er det vanskelig å se eller gjette seg fram til svaret på, og da må vi ty til mer generelle løsningsmetodar. Egentlig er det bare ett prinsipp vi følger: \regv
\st{Vi kan alltid utføre en matematisk operasjon på den ene siden av likhetstegnet, så lenge vi utfører den også på den andre siden.}
De matematiske operasjonane vi har presentert i denne boka er de fire rekneartene. Med disse lyder prinsippet slik:\regv
\st{Vi kan alltid legge til, trekke ifra, gange eller dele med et tall på den ene siden av likhetstegnet, så lenge vi gjør det også på den andre siden.}\regv
Prinsippet følger av betydningen til \sym{$=$}. Når to uttrykk har samme verdi, må de nødvendigvis fortsette å ha lik verdi, så lenge vi utfører de samme matematiske operasjonane på dem. I kommende seksjon skal vi likevel konkretisere dette prinsippet for hver enkelt rekneoperasjon, men hvis du føler dette allerede gir god meining kan du med fordel hoppe til \hrs{ligsaml}{seksjon}.
\section{\liglos \label{likloysfire}}
\textit{I figurene til denne seksjonen skal vi forstå likninger ut ifra et vektprinsipp. \sym{$=$} vil da indikere\footnote{\sym{$\neq$} er symbolet for ''er \textsl{ikkje} lik''.} at det er like mykje vekt (lik verdi) på venstre side som på høyre side.} \vspace{-5pt}
\begin{figure}
	\centering
	\includegraphics[]{\asym{lig}}\qquad
	\includegraphics[]{\asym{lig1}}
\end{figure}
\subsection*{Addisjon og subtraksjon; tall som skifter side}
\subsubsection*{Første eksempel}
Vi har allerede funnet løsningen på denne likningen, men la oss løse den på en annen måte\footnote{\textsl{Merk:} I tidligere figurer har det vært samsvar mellom størrelsen på rutene og (tall)verdien til tallet de symboliserer. Dette gjelder ikke rutene som representerer $ x $.}: \vspace{-3pt}
\[ x+4=7 \]
\fig{lig2}
Det blir tydelig hva verdien til $ x $ er hvis $ x $ står alene på en av sidene, og $ x $ blir isolert på venstresiden hvis vi tar bort 4. Men skal vi ta bort 4 fra venstresiden, må vi ta bort 4 fra høyresiden også, skal begge sidene ha samme verdi. \vs
\[ x+4-{\color{red}4}=7-{\color{red}4}  \]
\fig{lig2b}
Siden $ 4-{\color{red}4}=0 $ og $ 7-{\color{red}4}=3 $, får vi at \vspace{-3pt}
\[ x=3 \]
\fig{lig2c}
Dette kunne vi ha skrevet noe mer kortfattet slik:

\prbxl{0.5}{
	\alg{
		x+4 &= 7 \\
		x&= 7-4 \\
		x &= 3
}}
\prbxr{0.5}{Mellom første og andre linje er det vanlig å si at \textsl{4 har skiftet side, og derfor også fortegn (fra $ + $ til $ - $).}}
\textbf{Andre eksempel}\os
La oss gå videre til å se på en litt vanskeligere likning\footnote{Legg merke til at figuren illustrerer $ {4x+(-2)} $ (se \hrs{negmeng}{seksjon}) på venstre side. Men  $ {4x+(-2)} $ er det samme som $ {4x-2} $ (se \hrs{rekmneg}{seksjon}).}:
\[ 4x-2=3x+5 \]
\fig{lig4}
For å skaffe et uttrykk med $ x $ bare på én side, tar vi vekk $ 3x $ på begge sider:
\[ 4x-2-{\color{red}3x}=3x+5-{\color{red}3x} \]
\fig{lig5}
Da får vi at
\[ x-2=5 \]
\fig{lig5b}
For å isolere $ x $, legger vi til 2 på venstre side. Da må vi også legge til $ 2 $ på høyre side:
\[ x-2+{\color{blue}2}=5+{\color{blue}2} \]
\fig{lig6}
\newpage
Da får vi at
\[ x=7 \]
\fig{lig7}
Stegene vi har tatt kan oppsummeres slik:
\begin{flalign*}
&& 4x-2&=3x+5 && \llap{1. figur} \\
&& 4x-{\color{red}3x}-2&=3x-{\color{red}3x}+5 &&  \llap{2. figur} \\
&& x -2 &= 5 &&\llap{3. figur}\\
&& x-2+\color{blue}2&=  5+\color{blue}2 &&\llap{4. figur}\\
&& x &= 7 &&\llap{5. figur}
\end{flalign*}
Dette kan vi på en forenklet måte skrive slik:
\alg{
4x-2&=3x+5 \\
4x-{\color{red}3x}&=5+\color{blue}2\\
x &= 7
}

\reg[Flytting av tall over likhetstegnet \label{bytt}]{I en likning ønsker vi å samle alle $x$-ledd og alle kjente ledd på hver sin side av likhetstegnet. Skifter et ledd side, skifter det fortegn.}

\eks[1]{Løs likningen
\[ 3x+3 =2x+5 \]
\sv	
	 \vs \vs \vs \vs
	\begin{align*}
	3x-2x &=5-3 \\
	x &=2
	\end{align*}  \vspace{-20pt}}
\eks[2]{Løs likningen
\[ -4x-3 =-5x+12 \\ \]	
\sv
	 \vs \vs \vs \vs
	\begin{align*}
	-4x+5x &=12+3 \\
	x &=15
	\end{align*}}
\subsection*{Ganging og deling}	
\subsubsection{Deling}
Hittil har vi sett på likninger der vi endte opp med én $x $ på den ene siden av likhetstegnet. Ofte har vi flere $ x $-er, som for eksempel i likningen
\[ 3x=6 \]
\fig{lig8}
Deler vi venstresiden vår i tre like grupper, får vi én $ x $ i hver gruppe. Deler vi også høyresiden inn i tre like grupper, må alle gruppene ha den samme verdien
\[ \frac{3x}{3}=\frac{6}{3} \]
\fig{lig9}
Altså er
\[ x=2 \]
\fig{lig10}
La oss oppsummere utregningen vår:\\ 
\prbxl{0.6}{\begin{flalign*}
	&& 3x&=6 && \llap{1. figur} \br
	&& \frac{3x}{3}&=\frac{6}{3} && \llap{2. figur} \br
	&& x&=2 && \llap{3. figur}
	\end{flalign*}}\qquad\qquad
\prbxr{0.25}{Du husker kanskje at vi gjerne skriver
\[ \frac{\cancel{3}x}{\cancel{3}} \] 
}
\newpage
\reg[Deling på begge sider av en likning \label{ligdel}]{Vi kan dele begge sider av en likning med det samme tallet.}
\eks[1]{ Løs likningen\vspace{-3pt}
	\[ 	4x = 20  \]
	\sv \vs \vs \vsb
	\begin{align*}
	\frac{\cancel{4}x}{\cancel{4}}&=\frac{20}{4} \\
	x &=5
	\end{align*}
	\vspace{-20 pt}
	}
	
\eks[2]{Løs likningen \vspace{-3pt}
	\[2x+6 =3x-2 \]
\sv \vs \vs \vs \vs
	\begin{flalign*}
	&& 2x-3x &= -2-6 &&\\
	&&-x &= -8&& \\
	&& \frac{\cancel{-1}x}{\cancel{-1}} &= \frac{-8}{-1} &&\cm{($-x=-1x$)}\\
	&& x &= 8&&
	\end{flalign*}}

\subsection*{Ganging}
Det siste tilfellet vi skal se på er når likninger inneholder brøkdeler av den ukjente, som for eksempel i likningen
\[ \frac{x}{3}=4 \]
\fig{lig11}
Vi kan få én $ x $ på venstresiden hvis vi legger til to eksemplar av $ \frac{x}{3} $. Likningen forteller oss at $ \frac{x}{3} $ har samme verdi som 4. Dette betyr at 
for hver $ \frac{x}{3} $ vi legger til på venstresiden, må vi legge til 4 på høyresiden, skal sidene ha samme verdi.
\[ \frac{x}{3}+\frac{x}{3}+\frac{x}{3}=4+4+4 \]
\fig{lig12}
Vi legger nå merke til at \y{\frac{x}{3}+\frac{x}{3}+\frac{x}{3}=\frac{x}{3}\cdot3} og at \y{4+4+4=4\cdot3}:
\[ \frac{x}{3}\cdot 3 = 4\cdot 3 \]
\fig{lig12b}
Og da $ \frac{x}{3}\cdot3=x $ og $ 4\cdot3=12 $, har vi at
\[ x=12 \]
\fig{lig13}
En oppsummering av stegene våre kan vi skrive slik:
\begin{flalign*}
&& \frac{x}{3}&=4 && \llap{1. figur} \br 
&& \frac{x}{3}+\frac{x}{3}+\frac{x}{3} &= 4+4+4  &&\llap{2. figur} \br
&& \frac{x}{3}\cdot 3&=4\cdot3 && \llap{3. figur} \\
&& x&=12 && \llap{4. figur}
\end{flalign*}
Dette kan vi kortere skrive som
\alg{
\frac{x}{3}&= 4 \br 
\frac{x}{\cancel{3}}\cdot \cancel{3} &= 4\cdot 3 \br
x &= 12
}
\newpage
\reg[Ganging på begge sider av en likning]{
Vi kan gange begge sider av en likning med det samme talet.
}
\eks[1]{
Løs likningen
\[ \frac{x}{5}=2 \]
\sv \vsb \vs

\algv{
\frac{x}{\cancel{5}}\cdot\cancel{5}&=2\cdot5 \\
x &= 10
}
}
\eks[2]{
Løs likningen 
\[ \frac{7x}{10}-5=13+\frac{x}{10} \]
\sv \vsb \vs

\algv{
\frac{7x}{10}-\frac{x}{10}&=13+5\br
\frac{6x}{10}&=18 \br
\frac{6x}{\cancel{10}}\cdot \cancel{10}&=18\cdot10 \\
6x&=180 \br
\frac{\cancel{6}x}{\cancel{6}}&=\frac{180}{6}\br
x&=30
}
}
\newpage
\section{\ligloso \label{ligsaml}}
\reg[Løysningsmetoder for likninger \label{lsmlig}]{
Vi kan alltid
\begin{itemize}
\item addere eller subtrahere begge sider av en likning med det samme tallet. 
Dette er ekvivalent til å flytte et ledd fra den ene siden av likningen til den andre, så lenge vi også skifter fortegn på leddet.
\item gange eller dele begge sider av en likning med det samme tallet.
\end{itemize}
}
\eks[1]{
Løs likningen 
\[ 3x-4=6+2x \]

\sv \vs \vs

\algv{
3x-2x&=6+4 \\
x&=10
}
}
\eks[2]{
Løs likningen
\[ 9-7x=8x+3 \]
\sv \vs \vs
\algv{
 9-7x&=-8x+3 \\
  8x-7x&=3-9 \\
  x&=-6
}
}
\newpage
\eks[3]{
	Løs likningen
	\[ 10x-20=7x-5 \]
	\sv \vs \vs
	\algv{
		10x-20&=7x-5 \\
		10x-7x&=20-5 \\
		3x&=15 \\
		\frac{\cancel{3}x}{\cancel{3}}&=\frac{15}{3} \\
		x&=5
	}
}
\eks[4]{
Løy likningen
\[ 15-4x=x+5 \]

\sv \vs \vsb
\alg{
15-5&=x+4x \\
10&=5x\\
\frac{10}{5}&=\frac{\cancel{5}x}{\cancel{5}}\\
2&=x
}
}	

\eks[5]{
Løs likningen
\[ \frac{4x}{9}-20=8-\frac{3x}{9} \]

\sv \vs \vsb
\alg{
\frac{4x}{9}+\frac{3x}{9}&=20+8 \\
\frac{\cancel{7}x}{9\cdot \cancel{7}}&=\frac{28}{7} \br
\frac{x}{\cancel{9}}\cdot \cancel{9}&=4\cdot 9\\
x&=36
}
}

\newpage
\eks[6]{ Løs likningen
	\[ \frac{1}{3}x+\frac{1}{6}=\frac{5}{12}x+2  \vs\]	
	
	\sv
	Får å unngå brøker, ganger vi begge sider med fellesnevneren 12:
	\begin{align}
	\left(\frac{1}{3}x+\frac{1}{6}\right)12&=\left(\frac{5}{12}x+2\right)12 \br	
	\frac{1}{3}x\cdot12+\frac{1}{6}\cdot12&=\frac{5}{12}x\cdot12+2\cdot12 \tag{$ \ast $}\br	
	4x+2 &= 5x+24 \\
	4x-5x &= 24-2 \\
	-x &= 22 \\
	\frac{\cancel{-1}\,x}{\cancel{-1}} &= \frac{22}{-1} \\
	x &= -22
	\end{align} 
}
\info{Tips}{
	Mange liker å lage seg ein regel om at ''{vi kan gange eller dele alle ledd med det samme tallet}''. I eksempelet over kunne vi da hoppet direkte til andre linje i utrekningen.
}
\eks[7]{
Løs likningen
\[ 3-\frac{6}{x}= 2+\frac{5}{2x} \vs\]

\sv
Vi ganger begge sider med fellesnevneren $ 2x $:
\alg{
	 2x\left(3-\frac{6}{x}\right)&= 2x\left(2+\frac{5}{2x}\right) \br
6x -12 &= 4x+5\\
6x-4x &= 5+12\\
2x &= 17\\
x&=\frac{17}{2}
}
}
\begin{comment}
\eks[]{
Gitt eit trapes med to parallelle sider $ a $ og $ b $ og høgde $ h $. Da er arealet $ A $ gitt som
\[ A=\frac{(a+b)h}{2} \]
Løys likninga med hensyn på $ b $.

\sv \vs \vsb
\alg{
 A&=\frac{(a+b)h}{2} \br
 2A&=(a+b)h \\
 2A&=ah+bh \\
 2A-ah&=bh \\
 \frac{2A-ah}{h}&=b
}
}
\end{comment}
\end{document}



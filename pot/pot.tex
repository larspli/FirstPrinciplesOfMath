\documentclass[english,hidelinks,pdftex, 11 pt, class=report,crop=false]{standalone}
\usepackage[T1]{fontenc}
\usepackage[utf8]{luainputenc}
\usepackage{lmodern} % load a font with all the characters
\usepackage{geometry}
\geometry{verbose,paperwidth=16.1 cm, paperheight=24 cm, inner=2.3cm, outer=1.8 cm, bmargin=2cm, tmargin=1.8cm}
\setlength{\parindent}{0bp}
\usepackage{import}
\usepackage[subpreambles=false]{standalone}
\usepackage{amsmath}
\usepackage{amssymb}
\usepackage{esint}
\usepackage{babel}
\usepackage{tabu}
\makeatother
\makeatletter

\usepackage{titlesec}
\usepackage{ragged2e}
\RaggedRight
\raggedbottom
\frenchspacing

% Norwegian names of figures, chapters, parts and content
\addto\captionsenglish{\renewcommand{\figurename}{Figur}}
\makeatletter
\addto\captionsenglish{\renewcommand{\chaptername}{Kapittel}}
\addto\captionsenglish{\renewcommand{\partname}{Del}}

\addto\captionsenglish{\renewcommand{\contentsname}{Innhald}}

\usepackage{graphicx}
\usepackage{float}
\usepackage{subfig}
\usepackage{placeins}
\usepackage{cancel}
\usepackage{framed}
\usepackage{wrapfig}
\usepackage[subfigure]{tocloft}
\usepackage[font=footnotesize,labelfont=sl]{caption} % Figure caption
\usepackage{bm}
\usepackage[dvipsnames, table]{xcolor}
\definecolor{shadecolor}{rgb}{0.105469, 0.613281, 1}
\colorlet{shadecolor}{Emerald!15} 
\usepackage{icomma}
\makeatother
\usepackage[many]{tcolorbox}
\usepackage{multicol}
\usepackage{stackengine}

% For tabular
\usepackage{array}
\usepackage{multirow}
\usepackage{longtable} %breakable table

% Ligningsreferanser
\usepackage{mathtools}
\mathtoolsset{showonlyrefs}

% index
\usepackage{imakeidx}
\makeindex[title=Indeks]

%Footnote:
\usepackage[bottom, hang, flushmargin]{footmisc}
\usepackage{perpage} 
\MakePerPage{footnote}
\addtolength{\footnotesep}{2mm}
\renewcommand{\thefootnote}{\arabic{footnote}}
\renewcommand\footnoterule{\rule{\linewidth}{0.4pt}}
\renewcommand{\thempfootnote}{\arabic{mpfootnote}}

%colors
\definecolor{c1}{cmyk}{0,0.5,1,0}
\definecolor{c2}{cmyk}{1,0.25,1,0}
\definecolor{n3}{cmyk}{1,0.,1,0}
\definecolor{neg}{cmyk}{1,0.,0.,0}

% Lister med bokstavar
\usepackage[inline]{enumitem}

\newcounter{rg}
\numberwithin{rg}{chapter}
\newcommand{\reg}[2][]{\begin{tcolorbox}[boxrule=0.3 mm,arc=0mm,colback=blue!3] {\refstepcounter{rg}\phantomsection \large \textbf{\therg \;#1} \vspace{5 pt}}\newline #2  \end{tcolorbox}\vspace{-5pt}}

\newcommand\alg[1]{\begin{align} #1 \end{align}}

\newcommand\eks[2][]{\begin{tcolorbox}[boxrule=0.3 mm,arc=0mm,enhanced jigsaw,breakable,colback=green!3] {\large \textbf{Eksempel #1} \vspace{5 pt}\\} #2 \end{tcolorbox}\vspace{-5pt} }

\newcommand{\st}[1]{\begin{tcolorbox}[boxrule=0.0 mm,arc=0mm,enhanced jigsaw,breakable,colback=yellow!12]{ #1} \end{tcolorbox}}

\newcommand{\spr}[1]{\begin{tcolorbox}[boxrule=0.3 mm,arc=0mm,enhanced jigsaw,breakable,colback=yellow!7] {\large \textbf{Språkboksen} \vspace{5 pt}\\} #1 \end{tcolorbox}\vspace{-5pt} }

\newcommand{\sym}[1]{\colorbox{blue!15}{#1}}

\newcommand{\info}[2]{\begin{tcolorbox}[boxrule=0.3 mm,arc=0mm,enhanced jigsaw,breakable,colback=cyan!6] {\large \textbf{#1} \vspace{5 pt}\\} #2 \end{tcolorbox}\vspace{-5pt} }

\newcommand\algv[1]{\vspace{-11 pt}\begin{align*} #1 \end{align*}}

\newcommand{\regv}{\vspace{5pt}}
\newcommand{\mer}{\textsl{Merk}: }
\newcommand\vsk{\vspace{11pt}}
\newcommand\vs{\vspace{-11pt}}
\newcommand\vsb{\vspace{-16pt}}
\newcommand\sv{\vsk \textbf{Svar:} \vspace{4 pt}\\}
\newcommand\br{\\[5 pt]}
\newcommand{\asym}[1]{../fig/#1}
\newcommand\algvv[1]{\vs\vs\begin{align*} #1 \end{align*}}
\newcommand{\y}[1]{$ {#1} $}
\newcommand{\os}{\\[5 pt]}
\newcommand{\prbxl}[2]{
\parbox[l][][l]{#1\linewidth}{#2
	}}
\newcommand{\prbxr}[2]{\parbox[r][][l]{#1\linewidth}{
		\setlength{\abovedisplayskip}{5pt}
		\setlength{\belowdisplayskip}{5pt}	
		\setlength{\abovedisplayshortskip}{0pt}
		\setlength{\belowdisplayshortskip}{0pt} 
		\begin{shaded}
			\footnotesize	#2 \end{shaded}}}

\renewcommand{\cfttoctitlefont}{\Large\bfseries}
\setlength{\cftaftertoctitleskip}{0 pt}
\setlength{\cftbeforetoctitleskip}{0 pt}

\newcommand{\bs}{\\[3pt]}
\newcommand{\vn}{\\[6pt]}
\newcommand{\fig}[1]{\begin{figure}
		\centering
		\includegraphics[]{\asym{#1}}
\end{figure}}


\newcommand{\sectionbreak}{\clearpage} % New page on each section

% Equation comments
\newcommand{\cm}[1]{\llap{\color{blue} #1}}

\newcommand\fork[2]{\begin{tcolorbox}[boxrule=0.3 mm,arc=0mm,enhanced jigsaw,breakable,colback=yellow!7] {\large \textbf{#1 (forklaring)} \vspace{5 pt}\\} #2 \end{tcolorbox}\vspace{-5pt} }




%colors
\newcommand{\colr}[1]{{\color{red} #1}}
\newcommand{\colb}[1]{{\color{blue} #1}}
\newcommand{\colo}[1]{{\color{orange} #1}}
\newcommand{\colc}[1]{{\color{cyan} #1}}
\definecolor{projectgreen}{cmyk}{100,0,100,0}
\newcommand{\colg}[1]{{\color{projectgreen} #1}}

%%% SECTION HEADLINES %%%

% Our numbers
\newcommand{\likteikn}{Likskapsteiknet}
\newcommand{\talsifverd}{Tal, siffer og verdi}
\newcommand{\koordsys}{Koordinatsystem}

% Calculations
\newcommand{\adi}{Addisjon}
\newcommand{\sub}{Subtraksjon}
\newcommand{\gong}{Multiplikasjon (Gonging)}
\newcommand{\del}{Divisjon (deling)}

%Factorization and order of operations
\newcommand{\fak}{Faktorisering}
\newcommand{\rrek}{Reknerekkefølge}

%Fractions
\newcommand{\brgrpr}{Introduksjon}
\newcommand{\brvu}{Verdi, utviding og forkorting av brøk}
\newcommand{\bradsub}{Addisjon og subtraksjon}
\newcommand{\brgngheil}{Brøk gonga med heiltal}
\newcommand{\brdelheil}{Brøk delt med heiltal}
\newcommand{\brgngbr}{Brøk gonga med brøk}
\newcommand{\brkans}{Kansellering av faktorar}
\newcommand{\brdelmbr}{Deling med brøk}
\newcommand{\Rasjtal}{Rasjonale tal}

%Negative numbers
\newcommand{\negintro}{Introduksjon}
\newcommand{\negrekn}{Dei fire rekneartane med negative tal}
\newcommand{\negmeng}{Negative tal som mengde}

% Geometry 1
\newcommand{\omgr}{Omgrep}
\newcommand{\eignsk}{Eigenskapar for trekantar og firkantar}
\newcommand{\omkr}{Omkrins}
\newcommand{\area}{Areal}

%Algebra 
\newcommand{\algintro}{Introduksjon}
\newcommand{\pot}{Potensar}
\newcommand{\irrasj}{Irrasjonale tal}

%Equations
\newcommand{\ligintro}{Introduksjon}
\newcommand{\liglos}{Løysing ved dei fire rekneartane}
\newcommand{\ligloso}{Løysingsmetodane oppsummert}

%Functions
\newcommand{\fintro}{Introduksjon}
\newcommand{\lingraf}{Lineære funksjonar og grafar}

%Geometry 2
\newcommand{\geoform}{Formlar for areal og omkrins}
\newcommand{\kongogsim}{Kongruente og formlike trekantar}
\newcommand{\geofork}{Forklaringar}

% Names of rules
\newcommand{\gangdestihundre}{Å gange desimaltall med 10, 100 osv.}
\newcommand{\delmedtihundre}{Deling med 10, 100, 1\,000 osv.}
\newcommand{\ompref}{Omgjøring av prefikser}
\newcommand{\adkom}{Addisjon er kommutativ}
\newcommand{\gangkom}{Multiplikasjon er kommutativ}
\newcommand{\brdef}{Brøk som omskriving av delestykke}
\newcommand{\brtbr}{Brøk gonga med brøk}
\newcommand{\delmbr}{Brøk delt på brøk}
\newcommand{\gangpar}{Gonging med parentes (distributiv lov)}
\newcommand{\gangparsam}{Parantesar gonga saman}
\newcommand{\gangmnegto}{Gonging med negative tal I}
\newcommand{\gangmnegtre}{Gonging med negative tal II}
\newcommand{\konsttre}{Konstruksjon av trekantar}
\newcommand{\kongtre}{Kongruente trekantar}
\newcommand{\topv}{Toppvinklar}
\newcommand{\trisum}{Summen av vinklane i ein trekant}
\newcommand{\firsum}{Summen av vinklane i ein firkant}
\newcommand{\potgang}{Gonging med potensar}
\newcommand{\potdivpot}{Divisjon med potensar}
\newcommand{\potanull}{Spesialtilfellet \boldmath $a^0$}
\newcommand{\potneg}{Potens med negativ eksponent}
\newcommand{\potbr}{Brøk som grunntal}
\newcommand{\faktgr}{Faktorar som grunntal}
\newcommand{\potsomgrunn}{Potens som grunntal}
\newcommand{\arsirk}{Arealet til ein sirkel}
\newcommand{\artrap}{Arealet til eit trapes}
\newcommand{\arpar}{Arealet til eit parallellogram}
\newcommand{\pyt}{Pytagoras' setning}
\newcommand{\forform}{Forhold i formlike trekantar}
\newcommand{\vilkform}{Vilkår i formlike trekantar}
\newcommand{\omkrsirk}{Omkrinsen til ein sirkel (og $ \bm \pi $)}
\newcommand{\artri}{Arealet til ein trekant}
\newcommand{\arrekt}{Arealet til eit rektangel}
\newcommand{\liknflyt}{Flytting av ledd over likskapsteiknet}
\newcommand{\funklin}{Lineære funksjonar}

%Opg
\newcommand{\abc}[1]{
	\begin{enumerate}[label=\alph*),leftmargin=18pt]
		#1
	\end{enumerate}
}
\newcommand{\abcs}[2]{
	\begin{enumerate}[label=\alph*),start=#1,leftmargin=18pt]
		#2
	\end{enumerate}
}
\newcommand{\abcn}[1]{
	\begin{enumerate}[label=\arabic*),leftmargin=18pt]
		#1
	\end{enumerate}
}
\newcommand{\abch}[1]{
	\hspace{-2pt}	\begin{enumerate*}[label=\alph*), itemjoin=\hspace{1cm}]
		#1
	\end{enumerate*}
}
\newcommand{\abchs}[2]{
	\hspace{-2pt}	\begin{enumerate*}[label=\alph*), itemjoin=\hspace{1cm}, start=#1]
		#2
	\end{enumerate*}
}

\newcommand{\opgt}{\phantomsection \addcontentsline{toc}{section}{Oppgaver} \section*{Oppgaver for kapittel \thechapter}\vs \setcounter{section}{1}}
\newcounter{opg}
\numberwithin{opg}{section}
\newcommand{\op}[1]{\vspace{15pt} \refstepcounter{opg}\large \textbf{\color{blue}\theopg} \vspace{2 pt} \label{#1} \\}
\newcommand{\ekspop}[1]{\vsk\textbf{Gruble \thechapter.#1}\vspace{2 pt} \\}
\newcommand{\nes}{\stepcounter{section}
	\setcounter{opg}{0}}
\newcommand{\opr}[1]{\vspace{3pt}\textbf{\ref{#1}}}
\newcommand{\oeks}[1]{\begin{tcolorbox}[boxrule=0.3 mm,arc=0mm,colback=white]
		\textit{Eksempel: } #1	  
\end{tcolorbox}}
\newcommand\opgeks[2][]{\begin{tcolorbox}[boxrule=0.1 mm,arc=0mm,enhanced jigsaw,breakable,colback=white] {\footnotesize \textbf{Eksempel #1} \\} \footnotesize #2 \end{tcolorbox}\vspace{-5pt} }

%License
\newcommand{\lic}{\textit{Matematikken sine byggesteinar by Sindre Sogge Heggen is licensed under CC BY-NC-SA 4.0. To view a copy of this license, visit\\ 
		\net{http://creativecommons.org/licenses/by-nc-sa/4.0/}{http://creativecommons.org/licenses/by-nc-sa/4.0/}}}

%referances
\newcommand{\net}[2]{{\color{blue}\href{#1}{#2}}}
\newcommand{\hrs}[2]{\hyperref[#1]{\color{blue}\textsl{#2 \ref*{#1}}}}
\newcommand{\rref}[1]{\hrs{#1}{Regel}}
\newcommand{\refkap}[1]{\hrs{#1}{Kapittel}}
\newcommand{\refsec}[1]{\hrs{#1}{Seksjon}}

\usepackage{datetime2}

\usepackage[]{hyperref}

\begin{document}

\section{\pot \label{Potensar}}
\fig{pot}
Ein potens består av eit \textit{grunntal}\index{grunntal} og ein \textit{eksponent}\index{eksponent}. For eksempel er $2^{3}$ ein potens med grunntal 2 og
eksponent 3. Ein positiv, heiltals eksponent seier kor mange eksemplar
av grunntalet som skal gongast saman, altså er
\[ 2^3 =2\cdot2\cdot2 \]

\reg[Potenstall]{
$ {a^n} $ er eit potenstal med grunntal $ a $ og eksponent $ n $. 
\vsk

Viss $ n $ er eit naturleg tal, vil $ a^n $ svare til $ n $ eksemplar av $ a $ multiplisert med kvarandre. \vsk

\textit{Merk: } $ a^1=a $
}
\eks[1]{\vs \vs
\algv{
5^3 &= 5\cdot5\cdot5 \\
&= 125
}
}
\eks[2]{\vs \vs
	\[ c^4 = c\cdot c \cdot c \cdot c \]
}
\eks[3]{ \vs \vs
\algv{
(-7)^2 &= (-7)\cdot(-7) \\
&= 49
}
}\vsk \vsk

\spr{
Vanlege måtar å seie $ 2^3 $ på er
\begin{itemize}
	\item ''2 i tredje''
	\item ''2 opphøgd i 3''
\end{itemize}
I programmeringsspråk brukast gjerne symbolet \sym{\^{}} eller symbola \sym{**} mellom grunntall og eksponent.
}
\newpage
\info{Merk}{
Dei komande sidene vil innehalde reglar for potensar med tilhøyrande forklaringar. Sjølv om det er ønskeleg at dei har ei så generell form som mogleg, har vi i forklaringane valgt å bruke eksempel der eksponentane ikkje er variablar. Å bruke variablar som eksponentar ville gitt mykje mindre leservenlege uttrykk, og vi vil påstå at dei generelle tilfella kjem godt til synes også ved å studere konkrete tilfelle. 
} \vsk \vsk

\reg[\potgang \label{potgang}]{
\begin{equation}
a^{m}\cdot a^{n}=a^{m+n}	
\end{equation}
}
\eks[1]{\vs \vs
\algv{3^{5}\cdot3^{2}&=3^{5+2}\\&=3^{7}}
}
\eks[2]{\vs \vs
\algv{
b^4\cdot b^{11}&= b^{3+11}\\
&=b^{14}
}
}
\eks[3]{ \vs \vs
\algv{
a^5\cdot a^{-7} &= a^{5-7} \\
&= a^{-2} 
}
(Sjå \rref{potneg} for korleis potens med negativ eksponent kan tolkast.)	
} 
\newpage
\fork{\ref{potgang} \potgang}{
	Lat oss sjå på tilfellet 
	\[ a^{2}\cdot a^{3} \]
	Vi har at
	\algv{
		a^{2} & =2\cdot2\vn
		a^{3} & =2\cdot2\cdot2
	}
	
	Med andre ord kan vi skrive 
	\begin{align*}
	a^{2}\cdot a^{3} & =\overbrace{a \cdot a}^{a^{2}}\cdot\overbrace{a\cdot a\cdot a}^{a^{3}}\\
	& =a^{5}
	\end{align*}
} \vsk \vsk

\reg[\potdivpot \label{potdivpot}]{\vs
\[ \frac{a^{m}}{a^{n}}=a^{m-n} \] }

\eks[1]{\vspace{-20 pt}
\[
\frac{3^{5}}{3^{2}}=3^{5-2}=3^{3}
\]
} 
\eks[2]{ \vs \vsb
	\alg{
		\frac{2^{4}\cdot a^{7}}{a^{6}\cdot2^{2}}&=2^{4-2}\cdot a^{7-6}\\
		&=2^{2}a \\
		&=4a
	}
}
\newpage
\fork{\ref{potdivpot} \potdivpot}{
	Lat
	oss undersøke brøken
	\[ \frac{a^{5}}{a^{2}} \]
	Vi skriv
	ut potensane i tellar og nemnar: 
	\begin{align*}
	\frac{a^{5}}{a^{2}} & =\frac{a\cdot a\cdot a\cdot a\cdot a}{a\cdot a}\br
	& =\frac{\bcancel{a}\cdot\bcancel{a}\cdot a\cdot a\cdot a}{\bcancel{a}\cdot\bcancel{a}}\\
	& =a\cdot a\cdot a\\
	& =a^{3}
	\end{align*}
	Dette kunne vi ha skrive som
	\begin{align*}
	\frac{a^{5}}{a^{2}} & =a^{5-2}\\
	& =a^{3}
	\end{align*}
} \vsk \vsk

\reg[\potanull \label{pota0}]{\vs \vs
\[
a^{0}=1
\]
}
\eks[1]{\vs \vs\[
1000^{0}=1
\]}
\eks[2]{\vs \vs\[
(-b)^{0}=1
\]}
\fork{\ref{pota0} \potanull}{
	Eit tal delt på seg sjølv er alltid lik 1, derfor er 
	\[
	\frac{a^{n}}{a^{n}}=1
	\]
	Av dette, og \rref{potdivpot}, har vi at
	\algv{
		1&=\frac{a^{n}}{a^{n}}
		\\& =a^{n-n}\\
		& =a^{0}
	}
} \vsk \vsk

\reg[\potneg \label{potneg}]{
	\[ a^{-n}=\frac{1}{a^n} \]
}
\eks[1]{ \vs \vs
	\alg{
		a^{-8}&=\frac{1}{a^8}  
	}	
}
\eks[2]{ \vs \vs
\alg{
(-4)^{-3}&=\frac{1}{(-4)^3} 
=-\frac{1}{64}
}
}
\fork{\ref{potneg} \potneg}{
	Av \rref{pota0} har vi at $ a^0=1 $. Altså er
	\alg{
		\frac{1}{a^n}=\frac{a^0}{a^n}
	}
	Av \rref{potdivpot}  er
	\algv{
		\frac{a^0}{a^n}&=a^{0-n} \\
		&=a^{-n}
	}
} \vsk \vsk



\reg[\potbr \label{potbr}]{\vs
\begin{equation}\label{pbrg}
\left(\frac{a}{b}\right)^{m}=\frac{a^{m}}{b^{m}}
\end{equation}} 
\eks[1]{ \vs \vs
\alg{
\left(\frac{3}{4}\right)^2=\frac{3^2}{4^2} 
=\frac{9}{16}
}
}
\eks[2]{ \vs \vs
	\alg{
		\left(\frac{a}{7}\right)^3=\frac{a^3}{7^3} 
		=\frac{a^3}{343}
	}
}
\newpage
\fork{\ref{potbr} \potbr}{
	Lat oss studere
	\[ \left(\frac{a}{b}\right)^3 \]
	Vi har at
	\begin{align*}
	\left(\frac{a}{b}\right)^3 	&=\frac{a}{b}\cdot \frac{a}{b}\cdot \frac{a}{b}\br
	& =\frac{a\cdot a\cdot a}{b\cdot b\cdot b}\br
	& =\frac{a^{3}}{b^{3}}
	\end{align*}
}\vsk \vsk

\reg[\faktgr \label{faktgr}]{
\begin{equation}\label{key}
\left(ab\right)^{m}=a^{m}b^{m}
\end{equation}
}
\eks[1]{ \vs \vs \vs
\alg{
(3a)^5&=3^5a^5 \\
&=243a^5 
}	
}
\eks[2]{\vs\vs
\[
(ab)^{4}=a^{4}b^{4}
\]
}
\fork{\ref{faktgr} \faktgr}{
	Lat oss
	bruke ${(a\cdot b)^{3}}$ som eksempel. Vi har at
\alg{
	(a\cdot b)^{3}&=(a\cdot b)\cdot(a\cdot b)\cdot(a\cdot b) \\
	&=a\cdot a\cdot a \cdot b \cdot b \cdot b \\
	&=a^3b^3
}
}\vsk \vsk

\newpage
\reg[\potsomgrunn \label{potsomgrunn}]{\vs
\begin{equation}
\left(a^{m}\right)^{n}=a^{m\cdot n}
\end{equation}}
\eks[1]{ \vs \vs
\alg{
\left(c^4\right)^5&=c^{4\cdot5}\\
&=c^{20}	
}	
}
\eks[2]{ \vs \vs 
\alg{
\left(3^\frac{5}{4}\right)^8&=3^{\frac{5}{4}\cdot8} \\
&=3^{10}
}	
}
\fork{\ref{potsomgrunn} \potsomgrunn}{
	Lat oss bruke $\left(a^{3}\right)^{4}$ som eksempel. Vi har at
	\begin{align*}
	\left(a^{3}\right)^{4} & =a^{3}\cdot a^{3}\cdot a^{3}\cdot a^{3}
	\end{align*}
	
	
	Av \rref{potgang} er
	\algv{
		a^{3}\cdot a^{3}\cdot a^{3}\cdot a^{3} & =a^{3+3+3+3}\\
		& =a^{3\cdot4}\\
		&=a^{12}
	}	
}

\newpage
\reg[\textit{n}-rot]{ \vs
\[ a^\frac{1}{n}=\sqrt[n]{a} \]
Symbolet \sym{$ \sqrt{\phantom{a}} $} kallast eit \textit{rotteikn}\index{rotteikn}. For eksponenten $ \frac{1}{2} $ er det vanleg å utelate 2 i rotteiknet:
\[ a^\frac{1}{2}=\sqrt{a} \]
}
\eks{
Av \rref{potsomgrunn} har vi at
\alg{
\left(a^b\right)^\frac{1}{b}&=a^{b\cdot \frac{1}{b}} \\
&=a	
}
For eksempel er	
\algv{
9^\frac{1}{2}=\sqrt{9}=3 &\text{, sidan } 3^2 =9 \vn
125^\frac{1}{3}=\sqrt[3]{125}=5 &\text{, sidan } 5^3 =125 \vn	
16^\frac{1}{4}=\sqrt[4]{16}=2 &\text{, sidan } 2^4 =16
}	
}
\spr{
$\sqrt{9} $ kallast ''kvadratrota til 9'' \vsk

$ \sqrt[5]{9} $ kallast ''femterota til 9''.
}
\newpage
\section{\irrasj}
\reg[Irrasjonale tal]{
Eit tal som \textsl{ikkje} er eit rasjonalt tal, er eit irrasjonalt tal\index{tal!irrasjonalt}\footnotemark.\vsk

Verdien til eit irrasjonalt tal har uendeleg mange desimalar med eit ikkje-repeterande mønster.
}
\footnotetext{Strengt tatt er irrasjonale tal alle \textit{reelle} tal som ikkje er rasjonale tal. Men for å forklare kva \textit{reelle} tal er, må vi forklare kva \textit{imaginære} tal er, og det har vi valgt å ikkje gjere i denne boka. }
\eks[1]{
$ \sqrt{2} $ er eit irrasjonalt tal.
\[ \sqrt{2}=1.414213562373... \]
}












\end{document}
 

\documentclass[english,hidelinks,pdftex, 11 pt, class=report,crop=false]{standalone}
\usepackage[T1]{fontenc}
\usepackage[utf8]{luainputenc}
\usepackage{lmodern} % load a font with all the characters
\usepackage{geometry}
\geometry{verbose,paperwidth=16.1 cm, paperheight=24 cm, inner=2.3cm, outer=1.8 cm, bmargin=2cm, tmargin=1.8cm}
\setlength{\parindent}{0bp}
\usepackage{import}
\usepackage[subpreambles=false]{standalone}
\usepackage{amsmath}
\usepackage{amssymb}
\usepackage{esint}
\usepackage{babel}
\usepackage{tabu}
\makeatother
\makeatletter

\usepackage{titlesec}
\usepackage{ragged2e}
\RaggedRight
\raggedbottom
\frenchspacing

% Norwegian names of figures, chapters, parts and content
\addto\captionsenglish{\renewcommand{\figurename}{Figur}}
\makeatletter
\addto\captionsenglish{\renewcommand{\chaptername}{Kapittel}}
\addto\captionsenglish{\renewcommand{\partname}{Del}}

\addto\captionsenglish{\renewcommand{\contentsname}{Innhold}}

\usepackage{graphicx}
\usepackage{float}
\usepackage{subfig}
\usepackage{placeins}
\usepackage{cancel}
\usepackage{framed}
\usepackage{wrapfig}
\usepackage[subfigure]{tocloft}
\usepackage[font=footnotesize,labelfont=sl]{caption} % Figure caption
\usepackage{bm}
\usepackage[dvipsnames, table]{xcolor}
\definecolor{shadecolor}{rgb}{0.105469, 0.613281, 1}
\colorlet{shadecolor}{Emerald!15} 
\usepackage{icomma}
\makeatother
\usepackage[many]{tcolorbox}
\usepackage{multicol}
\usepackage{stackengine}

% For tabular
\usepackage{array}
\usepackage{multirow}
\usepackage{longtable} %breakable table

% Ligningsreferanser
\usepackage{mathtools}
\mathtoolsset{showonlyrefs}

% index
\usepackage{imakeidx}
\makeindex[title=Indeks]

%Footnote:
\usepackage[bottom, hang, flushmargin]{footmisc}
\usepackage{perpage} 
\MakePerPage{footnote}
\addtolength{\footnotesep}{2mm}
\renewcommand{\thefootnote}{\arabic{footnote}}
\renewcommand\footnoterule{\rule{\linewidth}{0.4pt}}
\renewcommand{\thempfootnote}{\arabic{mpfootnote}}

%colors
\definecolor{c1}{cmyk}{0,0.5,1,0}
\definecolor{c2}{cmyk}{1,0.25,1,0}
\definecolor{n3}{cmyk}{1,0.,1,0}
\definecolor{neg}{cmyk}{1,0.,0.,0}

% Lister med bokstavar
\usepackage{enumitem}

\newcounter{rg}
\numberwithin{rg}{chapter}
\newcommand{\reg}[2][]{\begin{tcolorbox}[boxrule=0.3 mm,arc=0mm,colback=blue!3] {\refstepcounter{rg}\phantomsection \large \textbf{\therg \;#1} \vspace{5 pt}}\newline #2  \end{tcolorbox}\vspace{-5pt}}

\newcommand\alg[1]{\begin{align} #1 \end{align}}

\newcommand\eks[2][]{\begin{tcolorbox}[boxrule=0.3 mm,arc=0mm,enhanced jigsaw,breakable,colback=green!3] {\large \textbf{Eksempel #1} \vspace{5 pt}\\} #2 \end{tcolorbox}\vspace{-5pt} }

\newcommand{\st}[1]{\begin{tcolorbox}[boxrule=0.0 mm,arc=0mm,enhanced jigsaw,breakable,colback=yellow!12]{ #1} \end{tcolorbox}}

\newcommand{\spr}[1]{\begin{tcolorbox}[boxrule=0.3 mm,arc=0mm,enhanced jigsaw,breakable,colback=yellow!7] {\large \textbf{Språkboksen} \vspace{5 pt}\\} #1 \end{tcolorbox}\vspace{-5pt} }

\newcommand{\sym}[1]{\colorbox{blue!15}{#1}}

\newcommand{\info}[2]{\begin{tcolorbox}[boxrule=0.3 mm,arc=0mm,enhanced jigsaw,breakable,colback=cyan!6] {\large \textbf{#1} \vspace{5 pt}\\} #2 \end{tcolorbox}\vspace{-5pt} }

\newcommand\algv[1]{\vspace{-11 pt}\begin{align*} #1 \end{align*}}

\newcommand{\regv}{\vspace{5pt}}
\newcommand{\mer}{\textsl{Merk}: }
\newcommand\vsk{\vspace{11pt}}
\newcommand\vs{\vspace{-11pt}}
\newcommand\vsb{\vspace{-16pt}}
\newcommand\sv{\vsk \textbf{Svar:} \vspace{4 pt}\\}
\newcommand\br{\\[5 pt]}
\newcommand{\asym}[1]{../fig/#1}
\newcommand\algvv[1]{\vs\vs\begin{align*} #1 \end{align*}}
\newcommand{\y}[1]{$ {#1} $}
\newcommand{\os}{\\[5 pt]}
\newcommand{\prbxl}[2]{
\parbox[l][][l]{#1\linewidth}{#2
	}}
\newcommand{\prbxr}[2]{\parbox[r][][l]{#1\linewidth}{
		\setlength{\abovedisplayskip}{5pt}
		\setlength{\belowdisplayskip}{5pt}	
		\setlength{\abovedisplayshortskip}{0pt}
		\setlength{\belowdisplayshortskip}{0pt} 
		\begin{shaded}
			\footnotesize	#2 \end{shaded}}}

\renewcommand{\cfttoctitlefont}{\Large\bfseries}
\setlength{\cftaftertoctitleskip}{0 pt}
\setlength{\cftbeforetoctitleskip}{0 pt}

\newcommand{\bs}{\\[3pt]}
\newcommand{\vn}{\\[6pt]}
\newcommand{\fig}[1]{\begin{figure}
		\centering
		\includegraphics[]{\asym{#1}}
\end{figure}}
\newcommand{\net}[2]{{\color{blue}\href{#1}{#2}}}

\newcommand{\hrs}[2]{\hyperref[#1]{\color{blue}\textsl{#2 \ref*{#1}}}}
\newcommand{\rref}[1]{\hyperref[#1]{\color{blue}\textsl{Regel \ref*{#1}}}}

\newcommand{\sectionbreak}{\clearpage} % New page on each section

% Equation comments
\newcommand{\cm}[1]{\llap{\color{blue} #1}}

\newcommand\fork[2]{\begin{tcolorbox}[boxrule=0.3 mm,arc=0mm,enhanced jigsaw,breakable,colback=yellow!7] {\large \textbf{#1 (forklaring)} \vspace{5 pt}\\} #2 \end{tcolorbox}\vspace{-5pt} }


%%% SECTION HEADLINES %%%

% Our numbers
\newcommand{\likteikn}{Likhetstegnet}
\newcommand{\talsifverd}{Tall, siffer og verdi}
\newcommand{\koordsys}{Koordinatsystem}

% Calculations
\newcommand{\adi}{Addisjon}
\newcommand{\sub}{Subtraksjon}
\newcommand{\gong}{Multiplikasjon (Gonging)}
\newcommand{\del}{Divisjon (deling)}

%Factorization and order of operations
\newcommand{\fak}{Faktorisering}
\newcommand{\rrek}{Regnerekkefølge}

%Fractions
\newcommand{\brgrpr}{Introduksjon}
\newcommand{\brvu}{Verdi, utviding og forkorting av brøk}
\newcommand{\bradsub}{Addisjon og subtraksjon}
\newcommand{\brgngheil}{Brøk ganget med heltall}
\newcommand{\brdelheil}{Brøk delt med heltall}
\newcommand{\brgngbr}{Brøk ganget med brøk}
\newcommand{\brkans}{Kansellering av faktorer}
\newcommand{\brdelmbr}{Deling med brøk}
\newcommand{\Rasjtal}{Rasjonale tall}

%Negative numbers
\newcommand{\negintro}{Introduksjon}
\newcommand{\negrekn}{De fire regneartane med negative tall}
\newcommand{\negmeng}{Negative tall som mengde}

% Geometry 1
\newcommand{\omgr}{Begrep}
\newcommand{\eignsk}{Egenskaper for trekanter og firkanter}
\newcommand{\omkr}{Omkrets}
\newcommand{\area}{Areal}

%Algebra 
\newcommand{\algintro}{Introduksjon}
\newcommand{\pot}{Potenser}
\newcommand{\irrasj}{Irrasjonale tall}

%Equations
\newcommand{\ligintro}{Introduksjon}
\newcommand{\liglos}{Løsing ved de fire regneartene}
\newcommand{\ligloso}{Løsingsmetodene oppsummert}

%Functions
\newcommand{\fintro}{Introduksjon}
\newcommand{\lingraf}{Lineære funksjoner og grafer}

%Geometry 2
\newcommand{\geoform}{Formler for areal og omkrets}
\newcommand{\kongogsim}{Kongruente og formlike trekanter}
\newcommand{\geofork}{Forklaringar}

% Names of rules
\newcommand{\adkom}{Addisjon er kommutativ}
\newcommand{\gangkom}{Multiplikasjon er kommutativ}
\newcommand{\brdef}{Brøk som omskriving av delestykke}
\newcommand{\brtbr}{Brøk ganget med brøk}
\newcommand{\delmbr}{Brøk delt på brøk}
\newcommand{\gangpar}{Ganging med parentes (distributiv lov)}
\newcommand{\gangparsam}{Paranteser ganget sammen}
\newcommand{\gangmnegto}{Ganging med negative tall I}
\newcommand{\gangmnegtre}{Ganging med negative tall II}
\newcommand{\konsttre}{Konstruksjon av trekanter}
\newcommand{\kongtre}{Kongruente trekanter}
\newcommand{\topv}{Toppvinkler}
\newcommand{\trisum}{Summen av vinklene i en trekant}
\newcommand{\firsum}{Summen av vinklene i en firkant}
\newcommand{\potgang}{Ganging med potenser}
\newcommand{\potdivpot}{Divisjon med potenser}
\newcommand{\potanull}{Spesialtilfellet \boldmath $a^0$}
\newcommand{\potneg}{Potens med negativ eksponent}
\newcommand{\potbr}{Brøk som grunntall}
\newcommand{\faktgr}{Faktorer som grunntall}
\newcommand{\potsomgrunn}{Potens som grunntall}
\newcommand{\arsirk}{Arealet til en sirkel}
\newcommand{\artrap}{Arealet til et trapes}
\newcommand{\arpar}{Arealet til et parallellogram}
\newcommand{\pyt}{Pytagoras' setning}
\newcommand{\forform}{Forhold i formlike trekanter}
\newcommand{\vilkform}{Vilkår i formlike trekanter}
\newcommand{\omkrsirk}{Omkretsen til en sirkel (og $ \bm \pi $)}
\newcommand{\artri}{Arealet til en trekant}
\newcommand{\arrekt}{Arealet til et rektangel}
\newcommand{\liknflyt}{Flytting av ledd over likhetstegnet}
\newcommand{\funklin}{Lineære funksjoner}

%Opg
% Opg
\newcommand{\abc}[1]{
	\begin{enumerate}[label=\alph*),leftmargin=18pt]
		#1
	\end{enumerate}
}
\newcommand{\opgt}{\phantomsection \addcontentsline{toc}{section}{Oppgaver} \section*{Oppgaver for kapittel \thechapter}\vs \setcounter{section}{1}}
\newcounter{opg}
\numberwithin{opg}{section}
\newcommand{\op}[1]{\vspace{15pt} \refstepcounter{opg}\large \textbf{\color{blue}\theopg} \vspace{2 pt} \label{#1} \\}
\newcommand{\ekspop}{\vsk\textbf{Gruble \thechapter}\vspace{2 pt} \\}
\newcommand{\nes}{\stepcounter{section}
	\setcounter{opg}{0}}
\newcommand{\opr}[1]{\vspace{3pt}\textbf{\ref{#1}}}

%License
\newcommand{\lic}{\textit{Matematikken sine byggesteiner by Sindre Sogge Heggen is licensed under CC BY-NC-SA 4.0. To view a copy of this license, visit\\ 
		\net{http://creativecommons.org/licenses/by-nc-sa/4.0/}{http://creativecommons.org/licenses/by-nc-sa/4.0/}}}

\usepackage{datetime2}
\usepackage[]{hyperref}


\begin{document}
\section{\brgrpr \label{brgrpr}}
\reg[\brdef \label{brdef}]{
	En brøk\index{brøk} er en annen måte å skrive et delestykke på. I en brøk kaller vi dividenden for \textit{teller}\index{teller} og divisoren for \textit{nevner}\index{nevner}.
	\begin{figure}
		\centering
		\includegraphics[]{\asym{br0_bm}}
	\end{figure}
}\vsk

\spr{
Vanlige måter å si $ \frac{1}{4} $ på er\footnote{I tillegg har vi utsagnene fra språkboksen på side \pageref{sprakdiv}.}
\begin{itemize}
	\item ''én firedel''
	\item ''1 av 4''
	\item ''1 over 4''
\end{itemize}
}
\subsubsection{Brøk som mengde}
La oss se på brøken $ \frac{1}{4} $ som en mengde. Vi starter da med å tenke på tallet 1 som en rute\footnote{Av praktiske årsaker velger vi oss her en enerrute som er større enn den vi brukte i \hrs{Talavare}{Kapittel}.}: 
\begin{figure}[hbt]
	\centering
	\includegraphics{\asym{br1}}
\end{figure}
\newpage
Så  deler vi denne ruten inn i fire mindre ruter som er like store. Hver av disse rutene blir da $ \frac{1}{4} $ (1 av 4):
\begin{figure}[hbt]
	\centering
	\includegraphics{\asym{br2}}
\end{figure} 

Har vi én slik rute, har vi altså 1 firedel:
\begin{figure}[hbt]
	\centering
	\includegraphics{\asym{br3a}}
\end{figure} 
Men skal man bare ut ifra en figur kunne se hvor stor en brøk er, må man vite hvor stor 1 er, og for å få dette lettere til syne skal vi også ta med de ''tomme'' rutene:
\fig{br3}
Slik vil de blå og de tomme rutene fortelle oss hvor mange biter 1 er delt inn i, mens de blå rutene alene forteller oss hvor mange slike biter det \textsl{egentlig} er. Slik kan vi seie at\regv
 
\st{ \vs
\alg{
	\text{antall blå ruter}&=\text{teller} \\	
	\text{antall blå ruter}+\text{antall tommme ruter}&=\text{nevner} 
}
\fig{br0a}
}

\newpage
\subsubsection{Brøk på tallinja}
På tallinja deler vi lengden mellom 0 og 1 inn i like mange lengder som nevneren angir. Har vi en brøk med 4 i nevner, deler vi lengden mellom 0 og 1 inn i 4 like lengder:
\fig{br9}
Tallinja er også fin å bruke for å tegne inn brøker som er større enn 1:
\fig{br9a}
\subsubsection{Teller og nevner oppsummert}
Selv om vi har vært innom det allerede, er det så avgjørende å forstå hva telleren og nevneren sier oss at vi tar en kort oppsummering:
\begin{itemize}
	\item Nevneren forteller hvor mange biter 1 er delt inn i.
	\item Telleren forteller hvor mange slike biter det er.
\end{itemize}
\newpage
\section{\brvu}
\reg[Verdien til en brøk]{Verdien til en brøk finn vi ved å dele telleren med nevneren.}
\eks{
Finn verdien til $ \dfrac{1}{4} $.

\sv \vsb
\[ \frac{1}{4}=0,25 \]
}
\subsubsection{Brøker med samme verdi}
Brøker kan ha samme verdi selv om de ser forskjellige ut. Hvis du regner ut $ 1:2 $, $ 2:4 $ og $ 4:8 $, får du i alle tilfeller 0,5 som svar. Dette betyr at
\alg{
\frac{1}{2}=\frac{2}{4}=\frac{4}{8}=0,5
} \\[5pt]
\fig{br12} \vsk
\fig{br12a}
\subsubsection{Utviding}
At brøker kan se forskjellige ut, men ha samme verdi, betyr at vi kan endre på utseendet til en brøk uten å endre verdien. La oss som eksempel gjøre om $ \frac{3}{5} $ til en brøk med samme verdi, men med 10 som nevner:
\begin{itemize}
	\item $ \frac{3}{5} $ kan vi gjøre om til en brøk med 10 i nevner om vi deler hver femdel inn i 2 like biter, for da blir 1 til sammen delt inn i\\ $ {5\cdot2=10} $ biter.
	\item Telleren i $ \frac{3}{5} $ forteller at der er 3 femdeler. Når disse blir delt i to, blir de totalt til $ 3\cdot2=6 $ tidelar. Altså har $ \frac{3}{5} $ samme verdi som $ \frac{6}{10} $.
\end{itemize}
\[ \frac{3}{5}=\frac{3\cdot2}{5\cdot2}=\frac{6}{10} \]
\fig{br13}
\fig{br13a}

\subsubsection{Forkorting}
Legg nå merke til at vi også kan gå ''andre veien''. $ \frac{6}{10} $ kan vi gjøre om til en brøk med 5 i nevner ved å dele både teller og nevner med 2:
\[ \frac{6}{10}=\frac{6:2}{10:2}=\frac{3}{5} \]

\reg[Utviding og forkorting av brøk]{
Vi kan gange eller dele teller og nevner med det samme tallet uten at brøken endrer verdi. \vsk

Å gange med et tall større enn 1 kalles å \textit{utvide}\index{brøk!utviding av} brøken. Å dele med et tall større enn 1 kalles å \textit{forkorte}\index{brøk!forkorting av} brøken.
}
\eks[1]{Utvid $ \frac{3}{5} $ til en brøk med 20 som nevner.
	
	\sv
	Da $ {5\cdot4=20} $, ganger vi både teller og nevner med 4:
	\alg{
		\frac{3}{5} &= \frac{3\cdot4}{5\cdot4} \br
		&= \frac{12}{20}
	}
}
\eks[2]{
	Utvid $ \frac{150}{50} $ til en brøk med 100 som nevner.\os
	
	\sv
	Da $ {50\cdot2=100} $, ganger vi både teller og nevner med 2:
	\alg{
		\frac{150}{50} &= \frac{150\cdot2}{50\cdot2} \br
		&= \frac{300}{100}
	}
	
}
\eks[3]{
	Forkort $ \frac{18}{30} $ til en brøk med 5 som nevner.
	
	\sv
	Da $ 30:6=5 $, deler vi både teller og nevner med 6:
	\alg{
		\frac{18}{30}&=\frac{18:6}{30:6} \br
		&=\frac{3}{5}
	} 
}
\newpage
\section{\bradsub}
Addisjon og subtraksjon av brøker handlar i stor grad om nevnerne. Husk nå at nevnerne forteller oss om inndelingen av 1. Hvis brøker har lik nevner, representerer de et antal biter med lik størrelse. Da gir det mening å regne addisjon eller subtraksjon mellom tellerene. Hvis brøker har ulike nevnere, representerer de et antall biter med ulik størrelse, og da gir ikkje addisjon eller subtraksjon mellom tellerene direkte mening.
\subsubsection{Lik nevner}
Om vi for eksempel har 2 seksdeler og adderer 3 seksdeler, ender vi opp med 5 seksdeler:
\[ \frac{2}{6}+\frac{3}{6}=\frac{5}{6} \]
\fig{br4}
\fig{br4a}
\regv
\reg[Addisjon/subtraksjon av brøkar med lik nevner \label{bradliknemn}]{
Når vi regner addisjon/subtraksjon mellom brøker med lik nevner, finner vi summen/differansen av tellerene og beholder nevneren.
}
\eks[1]{\vsb \vs
\alg{
\frac{2}{7}+\frac{8}{7}&=\frac{2+8}{7} \br
&= \frac{10}{7}
}
}
\newpage
\eks[2]{\vsb \vs
	\alg{
		\frac{7}{9}-\frac{5}{9}&=\frac{7-5}{9} \br
		&= \frac{2}{9}
	}
}
\subsection*{Ulike nevnere}
La oss se på regnestykket\footnote{Vi minner om at rødfargen på pila indikerer at man skal vandre fra pilspissen til andre enden.}
\[ \frac{3}{5}-\frac{1}{2} \]
\fig{br7}
\fig{br7t}
Skal vi skrive differansen som en brøk, må vi sørge for at brøkene har samme nevner. De to brøkene våre kan begge ha 10 som nevner:
\alg{
\frac{3}{5}=\frac{3\cdot2}{5\cdot2}=\frac{6}{10}\qquad\quad\qquad \frac{1}{2}=\frac{1\cdot5}{2\cdot5}=\frac{5}{10}
}
Dette betyr at
\[ \frac{3}{5}-\frac{1}{2}=\frac{6}{10}-\frac{5}{10} \]
\fig{br7a}
\fig{br7ta}
Det vi har gjort, er å utvide begge brøkene slik at de har samme nevner, nemlig 10. Når nevnerne i brøkene er like, kan vi regne ut subtraksjons-stykket for tellerene:
\alg{
\frac{3}{5}-\frac{1}{2}&=\frac{6}{10}-\frac{5}{10}\br
&=\frac{1}{10}
}
\reg[Addisjon/subtraksjon av brøkar med ulik\\ nevner]{
Når vi regner addisjon/subtraksjon mellom brøker med ulik nevner, må vi utvide brøkene slik at de har lik nevner, for så å bruke \rref{bradliknemn}.
}
\eks[1]{
Regn ut
\[ \frac{2}{\colb{9}}+\frac{6}{\colo{7}} \]
Begge nevnerne kan bli $ 63 $ hvis vi ganger med rett heltal. Vi utvider derfor til brøker med 63 i nevner:
\alg{
\frac{2\cdot\colo{7}}{9\cdot \colo{7}}+\frac{6\cdot\colb{9}}{7\cdot\colb{9}}&=\frac{14}{63}+\frac{54}{63} \br
&=\frac{68}{63}
}
}
\newpage
\info{Fellesnevner}{
I \textsl{Eksempel 1} over blir 63 kalt en \textit{fellesnevner}\index{fellesnevner}. Dette fordi det finnes heltall vi kan gange nevnerne med som gir oss tallet 63:
\alg{
	9\cdot7 &= 63 \\
	7\cdot 9 &= 63 
}
Hvis vi ganger sammen alle nevnerene i et regnestykke, finner vi alltid en fellesnevner, men vi sparer oss for store tall om vi finner den \textsl{minste} fellesnevneren. Ta for eksempel regnestykket
\[ \frac{7}{6}+\frac{5}{3} \]
Her kan vi bruke fellesnevneren $ {6\cdot3=18} $, men det er bedre å merke seg at $ 6\cdot1=3\cdot2=6 $ også er en fellesnevner. Altså er
\alg{
	\frac{7}{6}+\frac{5}{3}&=\frac{7}{6}+\frac{5\cdot2}{3\cdot2}\br
	&=\frac{7}{6}+\frac{10}{6}\br
	&=\frac{17}{6}
}

}
\eks[2]{
	Regn ut
	\[ \frac{3}{2}-\frac{5}{8}+\frac{10}{4} \]
	
	\sv
	Alle nevnerne kan bli 8 hvis vi ganger med rett heltall. Vi utvider derfor til brøker med 8 i nevner:
	\alg{
		\frac{3}{2}-\frac{5}{8}+\frac{10}{4} &= \frac{3\cdot4}{2\cdot4}-\frac{5}{8}+\frac{10\cdot2}{4\cdot2} \br
		&= \frac{12}{8}-\frac{5}{8}+\frac{20}{8} \br
		&= \frac{27}{8}
	}
}
\newpage
\section{\brgngheil}
I \hrs{Gonging}{seksjon} så vi at ganging med heltall er det samme som gjentatt addisjon. Skal vi for eksempel regne ut $ \frac{2}{5}\cdot 3 $, kan vi derfor regne slik:
\alg{
\frac{2}{5}\cdot 3 &= \frac{2}{5}+\frac{2}{5}+\frac{2}{5}\br
&= \frac{2+2+2}{5} \br
&= \frac{6}{5}
}
\fig{br14}
Men vi vet også at \y{2+2+2=2\cdot3}, og derfor kan vi forenkle regnestykket vårt:
\alg{
	\frac{2}{5}\cdot 3 &= \frac{2\cdot3}{5}\br
	&= \frac{6}{5} \label{brok5o61}
}
Multiplikasjon mellom heltall og brøk er også kommutativ\footnote{Husk at $ \frac{2}{5} $ bare er en omskriving av $ 2:5 $. }:
\alg{
3\cdot\frac{2}{5}&=3\cdot2:5 \\
&=6:5 \\
&=\frac{6}{5}
}

\reg[\label{brhel}Brøk ganget med heltal]{
Når vi ganger en brøk med et heltall, ganger vi heltallet med telleren i brøken.
}
\newpage
\eks[1]{\vsb \vs
\alg{
\frac{1}{3}\cdot 4&=\frac{1\cdot 4}{3} \br &=\frac{4}{3}
}
}
\eks[2]{ \vs
\vsb
\alg{
3\cdot\frac{2}{5}&=\frac{3\cdot 2}{5} \br &=\frac{6}{5}
}
}
\info{En tolkning av ganging med brøk \label{brtolk}}{
Av \rref{brhel} kan vi også danne en tolkning av hva å gange med en brøk innebærer. For eksempel, å gange $ 3 $ med $ \frac{2}{5} $ kan tolkes på disse to måtene:
\begin{itemize}
	\item Vi ganger $ 3 $ med $ 2 $, og deler produktet med $ 5 $:
	\[ 3\cdot2=6\quad,\quad 6:5=\frac{6}{5}\]
	\item Vi deler $ 3 $ med $ 5 $, og ganger kvotienten med $ 2 $:
	\[ 3:5=\frac{3}{5} \quad,\quad \frac{3}{5}\cdot2=\frac{3\cdot2}{5}=\frac{6}{5}\]
	
\end{itemize}
}

\newpage
\section{\brdelheil}
Det er nå viktig å huske på to ting:
\begin{itemize}
	\item Deling kan man se på som en lik fordeling av et antal
	\item I en brøk er det telleren som forteller noe om antallet (nevneren forteller om inndelingen av 1)
\end{itemize}
\subsubsection{Tilfellet der telleren er delelig med divisoren}
La oss regne ut
\[ \frac{6}{8}:2 \]
\fig{br15}
\fig{br15t}
Vi har her 6 åttedeler som vi skal fordele likt på 2. Dette blir $ 4:2=3 $ åttedeler.
\fig{br15a}
\fig{br15ta}
Altså er
\[ \frac{6}{8}:2=\frac{3}{8} \]
%\fig{br15b}
\newpage
\subsubsection{Tilfellet der telleren ikke er delelig med divisoren}
Hva nå om vi skal dele $ \frac{3}{4} $ på 2? 
\[ \frac{3}{4}:2 \]
\fig{br5}
\fig{br5t}
Saken er at vi alltid kan utivde brøken vår slik at telleren blir delelig med divisoren. Siden vi skal dele med 2, utvider vi altså brøken vår med 2:
\[ \frac{3}{4}=\frac{3\cdot2}{4\cdot2}=\frac{6}{8} \]
\fig{br5c}
\fig{br5ct}
Nå har vi 6 åttedeler. 6 åttedeler delt på 2 blir 3 åttedeler:
\fig{br5a}
\fig{br5at}
Altså er
\[ \frac{3}{4}:2=\frac{3}{8} \]
%\fig{br5b}
Rent matematisk har vi rett og slett ganget nevneren til $ \frac{3}{4} $ med 2:
\alg{
\frac{3}{4}:2 &= \frac{3}{4\cdot2} \br
&= \frac{3}{8}
}
\reg[Brøk delt med heltal \label{brdeenk}]{
Når vi deler en brøk med et heltall, ganger vi nevneren med heltallet.
}
\eks[1]{\vsb
\algv{
\frac{5}{3}:6 &= \frac{5}{3\cdot 6} \br
&= \frac{5}{18}
}
}
\info{Unntak}{
Innledningsvis av denne seksjonen fant vi at
\[ \frac{4}{8}:2=\frac{2}{8} \]
Da ganget vi ikke nevneren med 2, slik \rref{brdeenk} tilsier. Om vi gjør det, får vi 
\alg{
\frac{4}{8}:2=\frac{4}{8\cdot2}=\frac{4}{16}
}
Men 
\[ \dfrac{2}{8}=\dfrac{2\cdot2}{8\cdot2}=\dfrac{4}{16} \]
De to svarene har altså samme verdi. Saken er at skal vi dele en brøk på et heltall, og telleren er delelig med heltallet, kan vi direkte dele telleren på heltallet. I slike tilfeller er det altså ikke feil, men heller ikke nødvendig å bruke \rref{brdeenk}.
}
\section{\brgngbr \label{brgngbr}}		
Vi har sett\footnote{Sjå tekstboksen med tittelen \textit{En tolkning av ganging med brøk} på s. \pageref{brtolk}.} hvordan å gange med en brøk innebærer å gange det andre tallet med telleren, og så dele produktet med nevneren. La oss bruke dette til å regne ut
\[  {\frac{5}{4}\cdot\frac{3}{2}}\] 
Med tolkningen akkurat nevnt, skal vi nå gange $ \frac{5}{4} $ først med 3, og så dele produktet med 2. Av \rref{brhel} er
\alg{
\frac{5}{4}\cdot3 =\frac{5\cdot 3}{4}
}
Og av \rref{brdeenk} er
\alg{
\frac{5\cdot3}{4}:2=\frac{5\cdot3}{4\cdot2}
}
Altså er 
\alg{
\frac{5}{4}\cdot \frac{3}{2}=\frac{5\cdot 3}{4\cdot2}
}
\reg[\brtbr\label{brtbr}]{
Når vi ganger to brøker med hverandre, ganger vi teller med teller og nevner med nevner.
}
\eks[1]{
\algvv{
\frac{4}{7}\cdot \frac{6}{9}&= \frac{4\cdot6}{7\cdot9} \br
&= \frac{24}{63}
}
}
\eks[2]{
\algvv{
	\frac{1}{2}\cdot\frac{9}{10}&=\frac{1\cdot9}{2\cdot10} \br
		&=\frac{9}{20}
	}

}
\newpage
\section{\brkans}
Når telleren og nevneren har lik verdi, er verdien til brøken alltid 1. For eksempel er \y{\frac{3}{3}=1}, \y{\frac{25}{25}=1} osv. Dette kan vi utnytte for å forenkle brøkuttrykk. \vsk

La oss forenkle brøkuttrykket
\[ \frac{8\cdot 5}{9\cdot8} \]
Da $ {8\cdot5}={5\cdot8} $, kan vi skrive
\[ \frac{8\cdot5}{9\cdot8}=\frac{5\cdot8}{9\cdot8} \]
Og som vi nylig har sett (\rref{brtbr}) er
\[\frac{5\cdot8}{9\cdot8}= \frac{5}{9}\cdot\frac{8}{8}\]
Siden $ {\frac{8}{8}=1} $, har vi at
\alg{
	\frac{5}{9}\cdot\frac{8}{8} &= \frac{5}{9}\cdot1 \br
	&= \frac{5}{9}
}
Når bare ganging er til stede i brøker, kan man alltid omrokkere slik vi har gjort over, men når man har forstått hva omrokkeringen ender med, er det bedre å bruke \textit{kansellering}\index{kansellering}. Man setter da en strek over to og to like faktorer for å indikere at de utgjør en brøk med verdien 1. Tilfellet vi akkurat så på skriver vi da som 
\[ \frac{\cancel{8}\cdot 5}{9\cdot\cancel{8}}=\frac{5}{9} \]
\newpage
\reg[Kansellering av faktorar\label{kans}]{Når bare ganging er til stede i en brøk, kan vi kansellere par av like faktorer i teller og nevner.}
\eks[1]{
	Kanseller så mange faktorer som mulig i brøken
	\[ \frac{3\cdot12\cdot7}{7\cdot 4 \cdot 12} \]
	\sv \vsb
	\algvv{
		\frac{3\cdot\cancel{12}\cdot\cancel{7}}{\cancel{7}\cdot 4 \cdot \cancel{12}} &= \frac{3}{4} 
	}
}
\eks[2]{
	Forkort brøken $ \frac{12}{42} $.
	
	\sv 
	Vi legger merke til at $ 6 $ er en faktor i både $ 12 $ og $ 42 $, altså er	
	\alg{
		\frac{12}{42} &= \frac{\cancel{6}\cdot2}{\cancel{6}\cdot7}\br
		&= \frac{2}{7}
	}
}
\eks[3]{
	Forkort brøken $ \frac{48}{16}  $.
	
	\sv
	Vi legger merke til at $ 16 $ er en faktor i $ 48 $, altså er
	\alg{
		\frac{48}{16}&=
		\frac{3\cdot\cancel{16}}{\cancel{16}}\br
		&= \frac{3}{1}\br
		&= 3
	}
	\textsl{Merk}: Hvis alle faktorer er kansellert i teller eller nevner, er dette det samme som at tallet 1 står der.
}
\newpage
\info{Forkorting via primtalsfaktorisering}{
	Det er ikke alltid like lett å legge merke til en felles faktor, slik vi har gjort i \textsl{Eksempel 2} og \textsl{Eksempel 3} over. Vil man vere helt sikker på at man ikke har oversett felles faktorer, kan man alltid primtalsfaktorisere (se \refsec{primtalsfakt}) både teller og nevner. For eksempel har vi at
	\alg{
		\frac{12}{42}&=\frac{\cancel{2}\cdot2\cdot\cancel{3}}{\cancel{2}\cdot\cancel{3}\cdot7}\br
		&=\frac{2}{7}
	}
}

\info{Brøker forenkler utregninger}{
	Desmialtallet $ 0,125 $ kan vi skrive som brøken $ \frac{1}{8} $. Regnestykket
	\[ 0,125\cdot16 \]
	vil for de fleste av oss ta en stund å løse for hand med vanlige multiplikasjonsregler. Men bruker vi brøkuttrykket får vi at
	\alg{
		0,125\cdot16 &= \frac{1}{8}\cdot16 \br
		&= \frac{2 \cdot\cancel{8}}{\cancel{8}}\\
		&= 2
	}
} 
\newpage
\info{''Å stryke nuller''}{
	Et tall som 3000 kan vi skrive som $ 3\cdot10\cdot10\cdot10 $, mens 700 kan vi skrive som $ 7\cdot10\cdot10 $. Brøken $ \frac{3000}{700} $ kan vi derfor forkorte slik:
	\alg{
		\frac{3000}{700}&= \frac{3\cdot\cancel{10}\cdot\cancel{10}\cdot10}{7\cdot\cancel{10}\cdot\cancel{10}}\br
		&= \frac{3\cdot10}{7} \br
		&= \frac{30}{7}
	}
	I praksis er dette det samme som ''å stryke nuller'':
	\alg{
		\frac{30\cancel{00}}{7\cancel{00}}&= \frac{30}{7}
	}
	\textsl{Obs!} Nuller er de eneste sifrene vi kan ''stryke'' på denne måten, for eksempel kan vi ikke forkorte $ \frac{123}{13} $ på nokon som helst måte. I tillegg kan vi bare ''stryke'' nuller som står som bakerste siffer, for eksempel kan vi ikke ''stryke'' nuller i brøken $ \frac{101}{10} $.  
}
\newpage
\section{\brdelmbr}
\subsubsection{Deling ved å se på tallinja}
La oss regne ut $ 4:\frac{2}{3}  $. Siden brøken vi deler 4 på har 3 i nevner, kan det være en idé å gjøre om også 4 til en brøk med $ 3 $ i nevner. Vi har at 
\[ 4=\dfrac{12}{3} \]
\fig{br10c4}
Husk nå at en betydning av $ 4:\frac{2}{3} $ er
\begin{center}
	''Hvor mange ganger $ \frac{2}{3} $ går på 4.''
\end{center}
Ved å se på tallinja, finner vi at $ \frac{2}{3} $ går 6 ganger på 4. Altså er
\[ 4:\frac{2}{3}=6 \]
\fig{br10c}
\newpage
\subsubsection{En generell metode}
Vi kan ikke se på en tallinje hver gang vi skal dele med brøker, så nå skal vi komme fram til en generell regnemetode ved igjen å bruke $ 4:\frac{2}{3} $ som eksempel. For denne metoden bruker vi denne betydningen av divisjon:
\begin{center}
	$  4:\dfrac{2}{3}= $ ''Tallet vi må gange $ \frac{2}{3} $ med for å få 4.''
\end{center}
For å finne dette tallet starter vi med å gange $ \frac{2}{3} $ med tallet som gjør at produktet blir 1. Dette tallet er \textit{den omvendte brøken}\index{brøk!omvendt} av $ \frac{2}{3} $, som er $ \frac{3}{2} $:
\[ \frac{2}{3}\cdot\frac{3}{2}=1 \]
Nå gjenstår det bare å gange med 4 for å få 4:
\[ \frac{2}{3}\cdot\frac{3}{2}\cdot4=4 \]
For å få 4, må vi altså gange $ \dfrac{2}{3} $ med $ \dfrac{3}{2}\cdot4 $. Dette betyr at
\alg{
4:\frac{2}{3}&=\frac{3}{2}\cdot4 \br
&=6
}
\reg[\delmbr \label{delmbr}]{
Når vi deler et tall med en brøk, ganger vi tallet med den omvendte brøken.
}
\eks[1]{\vsb
	\algv{
			6:\frac{\color{c1}2}{\color{c2}9}&=6\cdot\frac{\color{c2}9}{\color{c1}2}\\
		&=27
	}
}
\eks[2]{\vsb
	\algv{
		\frac{4}{3}:\frac{\color{c1}5}{\color{c2}8}&= \frac{4}{3}\cdot\frac{\color{c2}8}{\color{c1}5} \br
		&= \frac{32}{15}
	}
}
\newpage
\eks[3]{\vsb
	\algv{
		\frac{3}{5}:\frac{3}{10}&= \frac{3}{5}\cdot\frac{10}{3} \br
		&= \frac{30}{15} }
	Her bør vi også se at brøken kan forkortes:
	\alg{
		\frac{30}{15}&= \frac{2\cdot\cancel{15}}{ \cancel{15}} \br
		&= 2
	}
\mer Vi kan spare oss for store tall hvis vi kansellerer faktorer underveis i utregninger:
	\alg{
		\frac{3}{5}\cdot\frac{10}{3}  &= \frac{\cancel{3}\cdot2\cdot\cancel{5}}{\cancel{5}\cdot\cancel{3}} \br
		&= 2 
	}
}
\newpage
\section{\Rasjtal \label{Rasjtal}}
\reg[Rasjonale tall]{
Et hvert tall som kan bli skrevet som en brøk med heltalls teller og nevner, er et \textit{rasjonalt tall}\index{tall!rasjonalt}.
}
\info{Merk}{Rasjonale tall gir oss en samlebetegnelse for
\begin{itemize}
	\item \textbf{Heltall} \\
	For eksempel $ 4=\frac{4}{1} $.
	\item \textbf{Desimaltall med endelig antall desimaler}\\
	For eksempel $ 0,2=\frac{1}{5} $.
	\item \textbf{Desimaltall med repeterende desimalmønster} \\
	For eksempel \footnote{\sym{$ \bar{3} $} indikerer at 3 fortsetter i det uendelige. En annen måte å indikere dette på er å bruke symbolet \sym{...}\,. Altså er $ 0,08\bar{3}=0,08333333 ... $} $ 0,08\bar{3}=\frac{1}{12} $. 
\end{itemize}}

\end{document}



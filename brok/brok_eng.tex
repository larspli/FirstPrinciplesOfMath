\documentclass[english,hidelinks,pdftex, 11 pt, class=report,crop=false]{standalone}
\usepackage[T1]{fontenc}
\usepackage[utf8]{luainputenc}
\usepackage{lmodern} % load a font with all the characters
\usepackage{geometry}
\geometry{verbose,paperwidth=16.1 cm, paperheight=24 cm, inner=2.3cm, outer=1.8 cm, bmargin=2cm, tmargin=1.8cm}
\setlength{\parindent}{0bp}
\usepackage{import}
\usepackage[subpreambles=false]{standalone}
\usepackage{amsmath}
\usepackage{amssymb}
\usepackage{esint}
\usepackage{babel}
\usepackage{tabu}
\makeatother
\makeatletter

\usepackage{titlesec}
\usepackage{ragged2e}
\RaggedRight
\raggedbottom
\frenchspacing

% Norwegian names of figures, chapters, parts and content
\addto\captionsenglish{\renewcommand{\figurename}{Figure}}
\makeatletter
\addto\captionsenglish{\renewcommand{\chaptername}{Chapter}}
%\addto\captionsenglish{\renewcommand{\partname}{Part}}

%\addto\captionsenglish{\renewcommand{\contentsname}{Content}}

\usepackage{graphicx}
\usepackage{float}
\usepackage{subfig}
\usepackage{placeins}
\usepackage{cancel}
\usepackage{framed}
\usepackage{wrapfig}
\usepackage[subfigure]{tocloft}
\usepackage[font=footnotesize,labelfont=sl]{caption} % Figure caption
\usepackage{bm}
\usepackage[dvipsnames, table]{xcolor}
\definecolor{shadecolor}{rgb}{0.105469, 0.613281, 1}
\colorlet{shadecolor}{Emerald!15} 
\usepackage{icomma}
\makeatother
\usepackage[many]{tcolorbox}
\usepackage{multicol}
\usepackage{stackengine}

% For tabular
\usepackage{array}
\usepackage{multirow}
\usepackage{longtable} %breakable table

% Ligningsreferanser
\usepackage{mathtools}
\mathtoolsset{showonlyrefs}

% index
\usepackage{imakeidx}
\makeindex[title=Index]

%Footnote:
\usepackage[bottom, hang, flushmargin]{footmisc}
\usepackage{perpage} 
\MakePerPage{footnote}
\addtolength{\footnotesep}{2mm}
\renewcommand{\thefootnote}{\arabic{footnote}}
\renewcommand\footnoterule{\rule{\linewidth}{0.4pt}}
\renewcommand{\thempfootnote}{\arabic{mpfootnote}}

%colors
\definecolor{c1}{cmyk}{0,0.5,1,0}
\definecolor{c2}{cmyk}{1,0.25,1,0}
\definecolor{n3}{cmyk}{1,0.,1,0}
\definecolor{neg}{cmyk}{1,0.,0.,0}

% Lister med bokstavar
\usepackage{enumitem}

\newcounter{rg}
\numberwithin{rg}{chapter}
\newcommand{\reg}[2][]{\begin{tcolorbox}[boxrule=0.3 mm,arc=0mm,colback=blue!3] {\refstepcounter{rg}\phantomsection \large \textbf{\therg \;#1} \vspace{5 pt}}\newline #2  \end{tcolorbox}\vspace{-5pt}}

\newcommand\alg[1]{\begin{align} #1 \end{align}}

\newcommand\eks[2][]{\begin{tcolorbox}[boxrule=0.3 mm,arc=0mm,enhanced jigsaw,breakable,colback=green!3] {\large \textbf{Example #1} \vspace{5 pt}\\} #2 \end{tcolorbox}\vspace{-5pt} }

\newcommand{\st}[1]{\begin{tcolorbox}[boxrule=0.0 mm,arc=0mm,enhanced jigsaw,breakable,colback=yellow!12]{ #1} \end{tcolorbox}}

\newcommand{\spr}[1]{\begin{tcolorbox}[boxrule=0.3 mm,arc=0mm,enhanced jigsaw,breakable,colback=yellow!7] {\large \textbf{The language box} \vspace{5 pt}\\} #1 \end{tcolorbox}\vspace{-5pt} }

\newcommand{\sym}[1]{\colorbox{blue!15}{#1}}

\newcommand{\info}[2]{\begin{tcolorbox}[boxrule=0.3 mm,arc=0mm,enhanced jigsaw,breakable,colback=cyan!6] {\large \textbf{#1} \vspace{5 pt}\\} #2 \end{tcolorbox}\vspace{-5pt} }

\newcommand\algv[1]{\vspace{-11 pt}\begin{align*} #1 \end{align*}}

\newcommand{\regv}{\vspace{5pt}}
\newcommand{\mer}{\textsl{Notice}: }
\newcommand{\merk}{Notice}
\newcommand\vsk{\vspace{11pt}}
\newcommand\vs{\vspace{-11pt}}
\newcommand\vsb{\vspace{-16pt}}
\newcommand\sv{\vsk \textbf{Answer} \vspace{4 pt}\\}
\newcommand\br{\\[5 pt]}
\newcommand{\asym}[1]{../fig/#1}
\newcommand\algvv[1]{\vs\vs\begin{align*} #1 \end{align*}}
\newcommand{\y}[1]{$ {#1} $}
\newcommand{\os}{\\[5 pt]}
\newcommand{\prbxl}[2]{
\parbox[l][][l]{#1\linewidth}{#2
	}}
\newcommand{\prbxr}[2]{\parbox[r][][l]{#1\linewidth}{
		\setlength{\abovedisplayskip}{5pt}
		\setlength{\belowdisplayskip}{5pt}	
		\setlength{\abovedisplayshortskip}{0pt}
		\setlength{\belowdisplayshortskip}{0pt} 
		\begin{shaded}
			\footnotesize	#2 \end{shaded}}}

\renewcommand{\cfttoctitlefont}{\Large\bfseries}
\setlength{\cftaftertoctitleskip}{0 pt}
\setlength{\cftbeforetoctitleskip}{0 pt}

\newcommand{\bs}{\\[3pt]}
\newcommand{\vn}{\\[6pt]}
\newcommand{\fig}[1]{\begin{figure}
		\centering
		\includegraphics[]{\asym{#1}}
\end{figure}}

\newcommand{\sectionbreak}{\clearpage} % New page on each section

% Equation comments
\newcommand{\cm}[1]{\llap{\color{blue} #1}}

\newcommand\fork[2]{\begin{tcolorbox}[boxrule=0.3 mm,arc=0mm,enhanced jigsaw,breakable,colback=yellow!7] {\large \textbf{#1 (explanation)} \vspace{5 pt}\\} #2 \end{tcolorbox}\vspace{-5pt} }


%%% SECTION HEADLINES %%%

% Our numbers
\newcommand{\likteikn}{The equal sign}
\newcommand{\talsifverd}{Numbers, digits and values}
\newcommand{\koordsys}{Coordinate systems}

% Calculations
\newcommand{\adi}{Addition}
\newcommand{\sub}{Subtraction}
\newcommand{\gong}{Multiplication}
\newcommand{\del}{Division}

%Factorization and order of operations
\newcommand{\fak}{Factorization}
\newcommand{\rrek}{Order of operations}

%Fractions
\newcommand{\brgrpr}{Introduction}
\newcommand{\brvu}{Values, expanding and simplifying}
\newcommand{\bradsub}{Addition and subtraction}
\newcommand{\brgngheil}{Fractions multiplied by integers}
\newcommand{\brdelheil}{Fractions divided by integers}
\newcommand{\brgngbr}{Fractions multiplied by fractions}
\newcommand{\brkans}{Cancelation of fractions}
\newcommand{\brdelmbr}{Division by fractions}
\newcommand{\Rasjtal}{Rational numbers}

%Negative numbers
\newcommand{\negintro}{Introduction}
\newcommand{\negrekn}{The elementary operations}
\newcommand{\negmeng}{Negative numbers as amounts}

% Geometry 1
\newcommand{\omgr}{Terms}
\newcommand{\eignsk}{Attributes of triangles and quadrilaterals}
\newcommand{\omkr}{Perimeter}
\newcommand{\area}{Area}

%Algebra 
\newcommand{\algintro}{Introduction}
\newcommand{\pot}{Powers}
\newcommand{\irrasj}{Irrational numbers}

%Equations
\newcommand{\ligintro}{Introduction}
\newcommand{\liglos}{Solving with the elementary operations}
\newcommand{\ligloso}{Solving with elementary operations summarized}

%Functions
\newcommand{\fintro}{Introduction}
\newcommand{\lingraf}{Linear functions and graphs}

%Geometry 2
\newcommand{\geoform}{Formulas of area and perimeter}
\newcommand{\kongogsim}{Congruent and similar triangles}
\newcommand{\geofork}{Explanations}

% Names of rules
\newcommand{\adkom}{Addition is commutative}
\newcommand{\gangkom}{Multiplication is commutative}
\newcommand{\brdef}{Fractions as rewriting of division}
\newcommand{\brtbr}{Fractions multiplied by fractions}
\newcommand{\delmbr}{Fractions divided by fractions}
\newcommand{\gangpar}{Distributive law}
\newcommand{\gangparsam}{Paranthesis multiplied together}
\newcommand{\gangmnegto}{Multiplication by negative numbers I}
\newcommand{\gangmnegtre}{Multiplication by negative numbers II}
\newcommand{\konsttre}{Unique construction of triangles}
\newcommand{\kongtre}{Congruent triangles}
\newcommand{\topv}{Vertical angles}
\newcommand{\trisum}{The sum of angles in a triangle}
\newcommand{\firsum}{The sum of angles in a quadrilateral}
\newcommand{\potgang}{Multiplication by powers}
\newcommand{\potdivpot}{Division by powers}
\newcommand{\potanull}{The special case of \boldmath $a^0$}
\newcommand{\potneg}{Powers with negative exponents}
\newcommand{\potbr}{Fractions as base}
\newcommand{\faktgr}{Factors as base}
\newcommand{\potsomgrunn}{Powers as base}
\newcommand{\arsirk}{The area of a circle}
\newcommand{\artrap}{The area of a trapezoid}
\newcommand{\arpar}{The area of a parallelogram}
\newcommand{\pyt}{Pythagoras's theorem}
\newcommand{\forform}{Ratios in similar triangles}
\newcommand{\vilkform}{Terms of similar triangles}
\newcommand{\omkrsirk}{The perimeter of a circle (and the value of $ \bm \pi $)}
\newcommand{\artri}{The area of a triangle}
\newcommand{\arrekt}{The area of a rectangle}
\newcommand{\liknflyt}{Moving terms across the equal sign}
\newcommand{\funklin}{Linear functions}

%License
\newcommand{\lic}{\textit{First Principles of Math by Sindre Sogge Heggen is licensed under CC BY-NC-SA 4.0. To view a copy of this license, visit\\ 
		\net{http://creativecommons.org/licenses/by-nc-sa/4.0/}{http://creativecommons.org/licenses/by-nc-sa/4.0/}}}

%referances
\newcommand{\net}[2]{{\color{blue}\href{#1}{#2}}}
\newcommand{\hrs}[2]{\hyperref[#1]{\color{blue}\textsl{#2 \ref*{#1}}}}
\newcommand{\rref}[1]{\hrs{#1}{Rule}}
\newcommand{\refkap}[1]{\hrs{#1}{Chapter}}
\newcommand{\refsec}[1]{\hrs{#1}{Section}}

\usepackage{datetime2}
\usepackage[]{hyperref}


\begin{document}
\section{\brgrpr \label{brgrpr}}
\reg[\brdef \label{brdef}]{
	A fraction\index{fraction} is a different way of writing a division. In a fraction the dividend is called the \textit{numerator}\index{numerator} and the divisor the denominator\index{denominator}.
	\begin{figure}
		\centering
		\includegraphics[]{\asym{br0_eng}}
	\end{figure}
}\vsk

\spr{
Common ways of saying $ \frac{1}{4} $ are\footnote{We also have the expressions from the language box on page \pageref{sprakdiv}.}
\begin{itemize}
	\item ''one fourth''
	\item ''1 of 4''
	\item ''1 over 4''
\end{itemize}
}
\subsubsection{Fractions as amounts}
Let us present $ \frac{1}{4} $ as an amount. We then think of the number 1 as a \\box\footnote{For practical reasons, we choose a unit box larger than the one used in \refkap{Talavare}.}: 
\begin{figure}[hbt]
	\centering
	\includegraphics{\asym{br1}}
\end{figure}
\newpage
Further, we divide this box into four smaller, equal-sized boxes. The sum of these boxes equals 1.
\begin{figure}[hbt]
	\centering
	\includegraphics{\asym{br2}}
\end{figure} 
One such box equals $ \frac{1}{4} $:
\begin{figure}[hbt]
	\centering
	\includegraphics{\asym{br3a}}
\end{figure} 
However, if you from a figure only are to see how large a fraction is, the size of 1 must be known, and to make this more apparent we'll also include the ''empty'' boxes:
\fig{br3}
In this way, the blue and the empty boxes tell us how many pieces 1 is divided into, while
the blue boxes alone tells how many of these boxes are \textsl{actually} present. In other words,\regv
 
\st{ \vs
\alg{
	\text{number of blue boxes}&=\text{numerator} \\
	\text{number of blue boxes}+\text{number of empty boxes}&=\text{denomiator}
}
\fig{br0a}
}

\newpage
\subsubsection{Fractions on the number line}
On the number line, we divide the length between 0 and 1 into as many pieces as the denominator indicates. In the case of a fraction with denominator 4, we divide the length between 0 and 1 into 4 equal lengths:
\fig{br9}
Moreover, fractions larger than 1 are easily presented on the number line:
\fig{br9a}
\subsubsection{Numerator and denominator summarized}
Although already mentioned, the interpretations of the numerator and the denominator are of such importance that we shortly summarize them:
\begin{itemize}
	\item The denominator tells how many pieces 1 is divided into.
	\item The numerator tells how many of these pieces are present.
\end{itemize}
\newpage
\section{\brvu}
\reg[The value of a fraction]{The value of a fraction is given by dividing the numerator by the denominator.}
\eks{
Find the value of $ \dfrac{1}{4} $.

\sv \vsb
\[ \frac{1}{4}=0.25 \]
}
\subsubsection{Fractions with equal value}
Fractions can have the same value even though they look different. If you calculate $ 1:2 $, $ 2:4 $ and $ 4:8 $, you will in every case end up with 0.5 as the answer. This means that
\alg{
\frac{1}{2}=\frac{2}{4}=\frac{4}{8}=0,5
} \\[5pt]
\fig{br12} \vsk
\fig{br12a}
\subsubsection{Expanding}
The fact that fractions can look different but have the same value, implies that we can change a fraction's look without changing its value. Let's, as an example, change $ \frac{3}{5} $ into a fraction of equal value but with denominator 10:
\begin{itemize}
	\item We can make $ \frac{3}{5} $ into a fraction with denominator 10 if we\\ divide each fifth into 2 equal pieces. In that case, 1 is divided into $ {5\cdot2=10} $ pieces in total.
	\item The numerator of $ \frac{3}{5} $ indicates that there are 3 fifths. When these are divided by 2, they make up $ 3\cdot2=6 $ tenths. Hence $ \frac{3}{5} $ equals $ \frac{6}{10} $.
\end{itemize}
\[ \frac{3}{5}=\frac{3\cdot2}{5\cdot2}=\frac{6}{10} \]
\fig{br13}
\fig{br13a}

\subsubsection{Simplifying}
Notice that we can also go ''the opposite way''. We can change $ \frac{6}{10} $ into a fraction with denominator 5 by dividing both the numerator and the denominator by 2:
\[ \frac{6}{10}=\frac{6:2}{10:2}=\frac{3}{5} \]

\reg[Expanding of fractions]{
We can either multiply or divide both the numerator and the denominator by the same number without alternating the fractions value. \vsk

Multiplying by a number larger than 1 is called \textit{expanding}\index{fraction!expansion of} the fraction. Dividing by a number larger than 1 is called \textit{simplifying}\index{fraction!simplifying of} the fraction.
}
\eks[1]{Expand $ \frac{3}{5} $ into a fraction with denominator 20.
	
	\sv
	Since $ {5\cdot4=20} $, we multiply both the numerator and the denominator by 4:
	\alg{
		\frac{3}{5} &= \frac{3\cdot4}{5\cdot4} \br
		&= \frac{12}{20}
	}
}
\eks[2]{
	Expand $ \frac{150}{50} $ into a fraction with denominator 100.\os
	
	\sv
	Since $ {50\cdot2=100} $, we multiply both the numerator and the denominator by 2:
	\alg{
		\frac{150}{50} &= \frac{150\cdot2}{50\cdot2} \br
		&= \frac{300}{100}
	}	
}
\eks[3]{
	Simplify $ \frac{18}{30} $ into a fraction with denominator 5.
	
	\sv
	Since $ 30:6=5 $, we divide both the numerator and the denominator by 6:
	\alg{
		\frac{18}{30}&=\frac{18:6}{30:6} \br
		&=\frac{3}{5}
	} 
}
\newpage
\section{\bradsub}
Addition and subtraction of fractions are in large parts focused around the denominators. Recall that the denominators indicate the partitioning of 1. If fractions have equal denominators, they represent amounts of equal-sized pieces. In this case it makes sense calculating addition or subtraction of the numerators. However, if fractions have unequal denominators, they represent amounts of different-sized pieces, and hence addition and subtraction of the numerators makes no sense directly.
\subsubsection{Equal denominators}
If we, for example, have 2 sixths and add 3 sixths, the sum is 5 sixths:
\[ \frac{2}{6}+\frac{3}{6}=\frac{5}{6} \]
\fig{br4}
\fig{br4a}
\regv
\reg[Addition/subtraction of fractions with equal denominators \label{bradliknemn}]{
When adding/subtracting fractions with equal denominators, we find the sum/difference of the numerators and keep the denominator.
}
\eks[1]{\vsb \vs
\alg{
\frac{2}{7}+\frac{8}{7}&=\frac{2+8}{7} \br
&= \frac{10}{7}
}
}
\newpage
\eks[2]{\vsb \vs
	\alg{
		\frac{7}{9}-\frac{5}{9}&=\frac{7-5}{9} \br
		&= \frac{2}{9}
	}
}
\subsection*{Unequal denominators}
Let's examine the calculation\footnote{Recall that the red-colored arrow indicates that you shall start at the arrowhead and then move to the other end.}
\[ \frac{3}{5}-\frac{1}{2} \]
\fig{br7}
\fig{br7t}
To write the difference as a single fraction, the two terms need to have denominators of equal value. Both of the fractions can have denominator 10:
\alg{
\frac{3}{5}=\frac{3\cdot2}{5\cdot2}=\frac{6}{10}\qquad\quad\qquad \frac{1}{2}=\frac{1\cdot5}{2\cdot5}=\frac{5}{10}
}
Hence
\[ \frac{3}{5}-\frac{1}{2}=\frac{6}{10}-\frac{5}{10} \]
\fig{br7a}
\fig{br7ta}
Summarized, we have expanded the fractions such that they have  denominators of equal value, that is 10. When the denominators are equal, we can calculate the difference of the numerators:
\alg{
\frac{3}{5}-\frac{1}{2}&=\frac{6}{10}-\frac{5}{10}\br
&=\frac{1}{10}
}
\reg[Addition/subtraction of fractions with unequal\\denominators]{
When calculating addition/subtraction of fractions, we must expand the fractions such that they have denominators of equal value, and then apply \rref{bradliknemn}.
}
\eks[1]{
Calculate
\[ \frac{2}{9}+\frac{6}{7} \]
Both denominators can be transformed into $ 63 $ if multiplied by a fitting integer. Therefore, we expand the fractions as follows:
\alg{
\frac{2\cdot7}{9\cdot 7}+\frac{6\cdot9}{7\cdot9}&=\frac{14}{63}+\frac{54}{63} \br
&=\frac{68}{63}
}
}
\newpage
\info{Common denominator}{
In \textsl{Example 1} above, 63 is called a \textit{common denominator}\index{common denominator} because there exists integers which, when multiplied by the original denominators, results in 63:
\alg{
	9\cdot7 &= 63 \\
	7\cdot 9 &= 63 
}
Multiplying together the original denominators always results in a common denominator but one can avoid large numbers by finding the \textsl{smallest} common denominator. Take, for example,
\[ \frac{7}{6}+\frac{5}{3} \]
$ {6\cdot3=18} $ is a common denominator, but it's worth noticing that $ 6\cdot1=3\cdot2=6 $ is too. Hence,
\alg{
	\frac{7}{6}+\frac{5}{3}&=\frac{7}{6}+\frac{5\cdot2}{3\cdot2}\br
	&=\frac{7}{6}+\frac{10}{6}=\frac{17}{6}
}
}
\eks[2]{
Calculate
	\[ \frac{3}{2}-\frac{5}{8}+\frac{10}{4} \]
	
	\sv
	All denominators can be transformed into 8 if multiplied by a fitting integer. Therefore, we expand the fractions as follows:
	\alg{
		\frac{3}{2}-\frac{5}{8}+\frac{10}{4} &= \frac{3\cdot4}{2\cdot4}-\frac{5}{8}+\frac{10\cdot2}{4\cdot2} \br
		&= \frac{12}{8}-\frac{5}{8}+\frac{20}{8} \br
		&= \frac{27}{8}
	}
}
\newpage
\section{\brgngheil}
In \refsec{Gonging} we observed that multiplying by an integer corresponds to repeated addition. Hence, if we are to calculate $ \frac{2}{5}\cdot 3 $, we can write
\alg{
\frac{2}{5}\cdot 3 &= \frac{2}{5}+\frac{2}{5}+\frac{2}{5}\br
&= \frac{2+2+2}{5} \br
&= \frac{6}{5}
}
\fig{br14}
Noticing that \y{2+2+2=2\cdot3}, we get
\alg{
	\frac{2}{5}\cdot 3 &= \frac{2\cdot3}{5}\br
	&= \frac{6}{5} \label{brok5o61}
}
Multiplication of integers and fractions are also commutative\footnote{Recall that $ \frac{2}{5} $ corresponds to $ 2:5 $. }:
\alg{
3\cdot\frac{2}{5}&=3\cdot2:5 \\
&=6:5 \\
&=\frac{6}{5}
}

\reg[\label{brhel}Fractions multiplied by integers]{
When multiplying a fraction by an integer, we multiply the numerator by the integer.
}
\newpage
\eks[1]{\vsb \vs
\alg{
\frac{1}{3}\cdot 4&=\frac{1\cdot 4}{3} \br &=\frac{4}{3}
}
}
\eks[2]{ \vs
\vsb
\alg{
3\cdot\frac{2}{5}&=\frac{3\cdot 2}{5} \br &=\frac{6}{5}
}
}
\info{An interpretation of multiplying by a fraction \label{brtolk}}{
By \rref{brhel} we can make an interpretation of multiplying by a fraction. For example, multiplying $ 3 $ by $ \frac{2}{5} $ can be interpreted in these two following ways:
\begin{itemize}
	\item We multiply $ 3 $ by $ 2 $ and divide by $ 5 $:
	\[ (3\cdot2):5=\frac{3\cdot2}{5}=\frac{6}{5}\]
	\item We divide $ 3 $ by $ 5 $ and multiply the quotient by $ 2 $:
	\[ 3:5=\frac{3}{5} \quad,\quad \frac{3}{5}\cdot2=\frac{3\cdot2}{5}=\frac{6}{5}\]
	
\end{itemize}
}

\newpage
\section{\brdelheil}
It is now important to recall two things:
\begin{itemize}
	\item Division can be interpreted as an equal distribution of amounts 
	\item In a fraction, it is the numerator which indicates the amount (the denominator indicates the partitioning of 1)
\end{itemize}
\subsubsection{When the numerator is divisible by the divisor}
Let's calculate
\[ \frac{6}{8}:2 \]
\fig{br15}
\fig{br15t}
We have 6 eights which are to be equally distributed into 2 groups. This results in $ 6:2=3 $ eights.
\fig{br15a}
\fig{br15ta}
Thus
\[ \frac{6}{8}:2=\frac{3}{8} \]
%\fig{br15b}
\newpage
\subsubsection{When the numerator is not divisible by the denominator}
What if we are to divide $ \frac{3}{4} $ by 2? 
\[ \frac{3}{4}:2 \]
\fig{br5}
\fig{br5t}
Thing is, we can always expand the fraction such that the numerator becomes divisible by the divisor. Since 2 is the divisor, we expand the fraction by 2:
\[ \frac{3}{4}=\frac{3\cdot2}{4\cdot2}=\frac{6}{8} \]
\fig{br5c}
\fig{br5ct}
Now we have 6 eights. 6 eights divided by 2 equals 3 eights:
\fig{br5a}
\fig{br5at}
Hence
\[ \frac{3}{4}:2=\frac{3}{8} \]
In effect, we have multiplied $ \frac{3}{4} $ by 2:
\alg{
\frac{3}{4}:2 &= \frac{3}{4\cdot2} \br
&= \frac{3}{8}
}
\reg[Fractions divided by integers \label{brdeenk}]{
When dividing a fraction by an integer, we multiply the denominator by the integer.
}
\eks[1]{\vsb
\algv{
\frac{5}{3}:6 &= \frac{5}{3\cdot 6} \br
&= \frac{5}{18}
}
}
\info{\merk}{
At the start of this section we found that
\[ \frac{4}{8}:2=\frac{2}{8} \]
In that case, there were no need to multiply the denominator by 2, such as \rref{brdeenk} implies. However, if we do, we have 
\alg{
\frac{4}{8}:2=\frac{4}{8\cdot2}=\frac{4}{16}
}
Now,
\[ \dfrac{2}{8}=\dfrac{2\cdot2}{8\cdot2}=\dfrac{4}{16} \]
Hence, unsurprisingly, the two answers are of equal value.
}
\section{\brgngbr \label{brgngbr}}		
We have seen that\footnote{Look at the text box titled ''\textit{An interpretation of multiplying by a fraction}'' on page \pageref{brtolk}.} multiplying a number by a fraction involves multiplying the number by the numerator and then dividing the product by the denominator. Let us apply this to calculate
\[  {\frac{5}{4}\cdot\frac{3}{2}}\] 
Firstly, we multiply $ \frac{5}{4} $ by 3, then we divide the resulting product by 2. By \rref{brhel}, we have
\alg{
\frac{5}{4}\cdot3 =\frac{5\cdot 3}{4}
}
And by \rref{brdeenk}, we get
\alg{
\frac{5\cdot3}{4}:2=\frac{5\cdot3}{4\cdot2}
}
Hence
\alg{
\frac{5}{4}\cdot \frac{3}{2}=\frac{5\cdot 3}{4\cdot2}
}
\reg[\brtbr\label{brtbr}]{
When multiplying a fraction by a fraction, we multiply numerator by numerator and denominator by denominator.
}
\eks[1]{
\algvv{
\frac{4}{7}\cdot \frac{6}{9}&= \frac{4\cdot6}{7\cdot9} \br
&= \frac{24}{63}
}
}
\eks[2]{
\algvv{
	\frac{1}{2}\cdot\frac{9}{10}&=\frac{1\cdot9}{2\cdot10} \br
		&=\frac{9}{20}
	}

}
\newpage
\section{\brkans}
When the numerator and the denominator are of equal value, the fractions value always equals 1. For example, \y{\frac{3}{3}=1}, \y{\frac{25}{25}=1} etc. We can exploit this fact to simplify expressions involving fractions. \vsk

Let us simplify the expression
\[ \frac{8\cdot 5}{9\cdot8} \]
Since $ {8\cdot5}={5\cdot8} $, we can write
\[ \frac{8\cdot5}{9\cdot8}=\frac{5\cdot8}{9\cdot8} \]
And, as recently seen (\rref{brtbr}), we have
\[\frac{5\cdot8}{9\cdot8}= \frac{5}{9}\cdot\frac{8}{8}\]
Since $ {\frac{8}{8}=1} $,
\algv{
	\frac{5}{9}\cdot\frac{8}{8} &= \frac{5}{9}\cdot1 \br
	&= \frac{5}{9}
}
When multiplication is exclusively present in a fraction, you can always shuffle the way we did in the above expressions. However, when you have understood the outcome of the shuffling, it is better to apply \textit{cancellation}\index{cancellation}. You then draw a line across two and two equal factors, thus indicating that they constitute a fraction which equals 1. Hence, our most recent example can be simplified to
\[ \frac{\cancel{8}\cdot 5}{9\cdot\cancel{8}}=\frac{5}{9} \]
\newpage
\reg[Cancellation of factors\label{kans}]{When multiplication is exclusively present in a fraction, we can cancel pair of equal factors in numerator and denominator.}
\eks[1]{
	Cancel as many factors as possible in the fraction.
	\[ \frac{3\cdot12\cdot7}{7\cdot 4 \cdot 12} \]
	\sv \vsb
	\algvv{
		\frac{3\cdot\cancel{12}\cdot\cancel{7}}{\cancel{7}\cdot 4 \cdot \cancel{12}} &= \frac{3}{4} 
	}
}
\eks[1]{
	Simplify the fraction $ \frac{12}{42} $.
	
	\sv \vsb
	\algv{
		\frac{12}{42} &= \frac{\cancel{6}\cdot2}{\cancel{6}\cdot7}\br
		&= \frac{2}{7}
	}
}
\eks[2]{
	Simplify the fraction $ \frac{48}{16}  $.
	
	\sv\vsb
	\algv{
		\frac{48}{16}&=
		\frac{3\cdot\cancel{16}}{\cancel{16}}\br
		&= \frac{3}{1}\br
		&= 3
	}
	\mer If all factors are canceled in the numerator or the denominator, 1 takes their place.
}
\newpage
\info{Fractions simplify calculations}{
	The decimal number $ 0.125 $ can be written as the fraction $ \frac{1}{8} $. The calculation
	\[ 0.125\cdot16 \]
	is, for the most of us, rather strenuous to carry out. However, exploiting the nature of fractions, we have
	\alg{
		0.125\cdot16 &= \frac{1}{8}\cdot16 \br
		&= \frac{2 \cdot\cancel{8}}{\cancel{8}}\\
		&= 2
	}
} \vsk
\info{''Cancelling zeros''}{
	A number such as 3000 equals $ 3\cdot10\cdot10\cdot10 $, while 700 equals \\$ 7\cdot10\cdot10 $. Hence, we can simplify $ \frac{3000}{700} $ like this:
	\alg{
		\frac{3000}{700}&= \frac{3\cdot\cancel{10}\cdot\cancel{10}\cdot10}{7\cdot\cancel{10}\cdot\cancel{10}}\br
		&= \frac{3\cdot10}{7} \br
		&= \frac{30}{7}
	}
	In practice, this is the same as ''cancelling zeros'':
	\alg{
		\frac{30\cancel{00}}{7\cancel{00}}&= \frac{30}{7}
	}
	\textsl{Aware!} Zeros are the only digits we can ''cancel'' this way. For example, $ \frac{123}{13} $ cannot be simplified in any way. Also, we can only ''cancel'' zeros which are right-most situated, e.g. we cannot ''cancel'' zeros in the fraction $ \frac{101}{10} $.  
}
\newpage
\section{\brdelmbr}
\subsubsection{Division by studying the number line}
Let's calculate $ 4:\frac{2}{3}  $. Since the fraction have denominator 3, it could be wise to transform also 4 into a fraction with denominator $ 3 $.
\[ 4=\dfrac{12}{3} \]
\fig{br10c4}
Recall that one of the interpretations of $ 4:\frac{2}{3} $ is
\begin{center}
	''The number of $ \frac{2}{3}$'s added to make 4.''
\end{center}
By studying a number line, we find that 6 instances of $ \frac{2}{3} $ added together equals 4. Hence
\[ 4:\frac{2}{3}=6 \]
\fig{br10c}
\newpage
\subsubsection{A general method}
We can't study the number line every time we are to divide by a fraction, so here we shall find a general method, again with $ {4:\frac{2}{3}} $ as our example. In this case, we apply the following interpretation of division:
\begin{center}
	$  4:\dfrac{2}{3}= $ ''The number to multiply $ \frac{2}{3} $ by to make 4.''
\end{center}
We begin the search of this number by multiplying $ \frac{2}{3} $ by the number which results in the product equal to 1. This number is the \textit{inverted fraction}\index{fraction!inverted} of $ \frac{2}{3} $, namely $ \frac{3}{2} $:
\[ \frac{2}{3}\cdot\frac{3}{2}=1 \]
Now we only have to multiply by 4 to make 4:
\[ \frac{2}{3}\cdot\frac{3}{2}\cdot4=4 \]
Therefore, to make 4 we must multiply $ \dfrac{2}{3} $ by $ \dfrac{3}{2}\cdot4 $. Consequently,
\alg{
4:\frac{2}{3}&=\frac{3}{2}\cdot4 \br
&=6
}
\reg[\delmbr \label{delmbr}]{
When dividing a number by a fraction, we multiply the number by the inverted fraction.
}
\eks[1]{\vsb
	\algv{
			6:\frac{\color{c1}2}{\color{c2}9}&=6\cdot\frac{\color{c2}9}{\color{c1}2}\\
		&=27
	}
}
\eks[2]{\vsb
	\algv{
		\frac{4}{3}:\frac{\color{c1}5}{\color{c2}8}&= \frac{4}{3}\cdot\frac{\color{c2}8}{\color{c1}5} \br
		&= \frac{32}{15}
	}
}
\newpage
\eks[3]{\vsb
	\algv{
		\frac{3}{5}:\frac{3}{10}&= \frac{3}{5}\cdot\frac{10}{3} \br
		&= \frac{30}{15} }
	In this case we should also observe that the fraction can be simplified:
	\algv{
		\frac{30}{15}&= \frac{2\cdot\cancel{15}}{ \cancel{15}} \br
		&= 2
	}
\mer Canceling factors along the way saves the labor of working with large numbers:
	\alg{
		\frac{3}{5}\cdot\frac{10}{3}  &= \frac{\cancel{3}\cdot2\cdot\cancel{5}}{\cancel{5}\cdot\cancel{3}} \br
		&= 2 
	}
}
\newpage
\section{\Rasjtal \label{Rasjtal}}
\reg[Rational numbers]{
Any number which can be expressed as a fraction is a \textit{rational number}\index{number!rational}.
}
\info{Merk}{Rational numbers is a collective name of
\begin{itemize}
	\item \textbf{Integers} \\
	For example $ 4=\frac{4}{1} $.
	\item \textbf{Decimal numbers with a finite number of digits}\\
	For example $ 0,2=\frac{1}{5} $.
	\item \textbf{Decimal numbers with infinite digits in a repeating manner} \\
	For example \footnote{\sym{$ \bar{3} $} indicates that 3 repeats infinite. Another way of expressing this is by using \sym{...}\,. That is, $ 0.08\bar{3}=0.08333333 ... $} $ 0.08\bar{3}=\frac{1}{12} $. 
\end{itemize}}

\end{document}



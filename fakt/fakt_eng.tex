\input{../doc}
\input{../preamb_eng}


\begin{document}
%\newpage
\section{\fak \label{Faktorisering}}
If an integer dividend and an integer divisor results in an integer \\quotient, we say that the dividend is \textit{divisible} by the divisor. For example is $ 6 $ divisible with $ 3 $ because $ {6:3=2} $, and $ 40 $ is divisible with $ 10 $ because $ {40:10=4} $. The concept of divisibility contributes to the definition of \textit{prime numbers}:\regv

\reg[Primtal \index{numbers!prime}]{
	A natural number larger than 1, and only divisible by itself and 1, is a prime number.
}

\eks[]{
	The first five prime numbers are $ 2, 3, 5 , 7  $ and $ 11 $.
} \vsk \vspace{5pt}

\reg[Factorization]{\index{factorization}
	Factorization involves writing a number as the product of other numbers.
}
\eks{
	Factorize 24 in three different ways.
	
	\sv  \vsb \vsb 
	\alg{
		24&=2\cdot 12 \br
		24&=3\cdot 8 \br
		24&=2\cdot 3 \cdot 4
	}
}  \vsk \vspace{5pt}

\reg[Prime factorization]{\index{prime factorization}
	Factorization involving prime factors only is called prime\\ factorization.
}
\eks[]{
	Prime factorize 12.
	
	\sv \vs \vs
	\[ 12= 2 \cdot2\cdot3 \]
}

\end{document}


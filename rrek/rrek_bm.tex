\documentclass[english,hidelinks,pdftex, 11 pt, class=report,crop=false]{standalone}
\usepackage[T1]{fontenc}
\usepackage[utf8]{luainputenc}
\usepackage{lmodern} % load a font with all the characters
\usepackage{geometry}
\geometry{verbose,paperwidth=16.1 cm, paperheight=24 cm, inner=2.3cm, outer=1.8 cm, bmargin=2cm, tmargin=1.8cm}
\setlength{\parindent}{0bp}
\usepackage{import}
\usepackage[subpreambles=false]{standalone}
\usepackage{amsmath}
\usepackage{amssymb}
\usepackage{esint}
\usepackage{babel}
\usepackage{tabu}
\makeatother
\makeatletter

\usepackage{titlesec}
\usepackage{ragged2e}
\RaggedRight
\raggedbottom
\frenchspacing

% Norwegian names of figures, chapters, parts and content
\addto\captionsenglish{\renewcommand{\figurename}{Figur}}
\makeatletter
\addto\captionsenglish{\renewcommand{\chaptername}{Kapittel}}
\addto\captionsenglish{\renewcommand{\partname}{Del}}

\addto\captionsenglish{\renewcommand{\contentsname}{Innhald}}

\usepackage{graphicx}
\usepackage{float}
\usepackage{subfig}
\usepackage{placeins}
\usepackage{cancel}
\usepackage{framed}
\usepackage{wrapfig}
\usepackage[subfigure]{tocloft}
\usepackage[font=footnotesize,labelfont=sl]{caption} % Figure caption
\usepackage{bm}
\usepackage[dvipsnames, table]{xcolor}
\definecolor{shadecolor}{rgb}{0.105469, 0.613281, 1}
\colorlet{shadecolor}{Emerald!15} 
\usepackage{icomma}
\makeatother
\usepackage[many]{tcolorbox}
\usepackage{multicol}
\usepackage{stackengine}

% For tabular
\usepackage{array}
\usepackage{multirow}
\usepackage{longtable} %breakable table

% Ligningsreferanser
\usepackage{mathtools}
\mathtoolsset{showonlyrefs}

% index
\usepackage{imakeidx}
\makeindex[title=Indeks]

%Footnote:
\usepackage[bottom, hang, flushmargin]{footmisc}
\usepackage{perpage} 
\MakePerPage{footnote}
\addtolength{\footnotesep}{2mm}
\renewcommand{\thefootnote}{\arabic{footnote}}
\renewcommand\footnoterule{\rule{\linewidth}{0.4pt}}
\renewcommand{\thempfootnote}{\arabic{mpfootnote}}

%colors
\definecolor{c1}{cmyk}{0,0.5,1,0}
\definecolor{c2}{cmyk}{1,0.25,1,0}
\definecolor{n3}{cmyk}{1,0.,1,0}
\definecolor{neg}{cmyk}{1,0.,0.,0}

% Lister med bokstavar
\usepackage[inline]{enumitem}

\newcounter{rg}
\numberwithin{rg}{chapter}
\newcommand{\reg}[2][]{\begin{tcolorbox}[boxrule=0.3 mm,arc=0mm,colback=blue!3] {\refstepcounter{rg}\phantomsection \large \textbf{\therg \;#1} \vspace{5 pt}}\newline #2  \end{tcolorbox}\vspace{-5pt}}

\newcommand\alg[1]{\begin{align} #1 \end{align}}

\newcommand\eks[2][]{\begin{tcolorbox}[boxrule=0.3 mm,arc=0mm,enhanced jigsaw,breakable,colback=green!3] {\large \textbf{Eksempel #1} \vspace{5 pt}\\} #2 \end{tcolorbox}\vspace{-5pt} }

\newcommand{\st}[1]{\begin{tcolorbox}[boxrule=0.0 mm,arc=0mm,enhanced jigsaw,breakable,colback=yellow!12]{ #1} \end{tcolorbox}}

\newcommand{\spr}[1]{\begin{tcolorbox}[boxrule=0.3 mm,arc=0mm,enhanced jigsaw,breakable,colback=yellow!7] {\large \textbf{Språkboksen} \vspace{5 pt}\\} #1 \end{tcolorbox}\vspace{-5pt} }

\newcommand{\sym}[1]{\colorbox{blue!15}{#1}}

\newcommand{\info}[2]{\begin{tcolorbox}[boxrule=0.3 mm,arc=0mm,enhanced jigsaw,breakable,colback=cyan!6] {\large \textbf{#1} \vspace{5 pt}\\} #2 \end{tcolorbox}\vspace{-5pt} }

\newcommand\algv[1]{\vspace{-11 pt}\begin{align*} #1 \end{align*}}

\newcommand{\regv}{\vspace{5pt}}
\newcommand{\mer}{\textsl{Merk}: }
\newcommand\vsk{\vspace{11pt}}
\newcommand\vs{\vspace{-11pt}}
\newcommand\vsb{\vspace{-16pt}}
\newcommand\sv{\vsk \textbf{Svar:} \vspace{4 pt}\\}
\newcommand\br{\\[5 pt]}
\newcommand{\asym}[1]{../fig/#1}
\newcommand\algvv[1]{\vs\vs\begin{align*} #1 \end{align*}}
\newcommand{\y}[1]{$ {#1} $}
\newcommand{\os}{\\[5 pt]}
\newcommand{\prbxl}[2]{
\parbox[l][][l]{#1\linewidth}{#2
	}}
\newcommand{\prbxr}[2]{\parbox[r][][l]{#1\linewidth}{
		\setlength{\abovedisplayskip}{5pt}
		\setlength{\belowdisplayskip}{5pt}	
		\setlength{\abovedisplayshortskip}{0pt}
		\setlength{\belowdisplayshortskip}{0pt} 
		\begin{shaded}
			\footnotesize	#2 \end{shaded}}}

\renewcommand{\cfttoctitlefont}{\Large\bfseries}
\setlength{\cftaftertoctitleskip}{0 pt}
\setlength{\cftbeforetoctitleskip}{0 pt}

\newcommand{\bs}{\\[3pt]}
\newcommand{\vn}{\\[6pt]}
\newcommand{\fig}[1]{\begin{figure}
		\centering
		\includegraphics[]{\asym{#1}}
\end{figure}}


\newcommand{\sectionbreak}{\clearpage} % New page on each section

% Equation comments
\newcommand{\cm}[1]{\llap{\color{blue} #1}}

\newcommand\fork[2]{\begin{tcolorbox}[boxrule=0.3 mm,arc=0mm,enhanced jigsaw,breakable,colback=yellow!7] {\large \textbf{#1 (forklaring)} \vspace{5 pt}\\} #2 \end{tcolorbox}\vspace{-5pt} }




%colors
\newcommand{\colr}[1]{{\color{red} #1}}
\newcommand{\colb}[1]{{\color{blue} #1}}
\newcommand{\colo}[1]{{\color{orange} #1}}
\newcommand{\colc}[1]{{\color{cyan} #1}}
\definecolor{projectgreen}{cmyk}{100,0,100,0}
\newcommand{\colg}[1]{{\color{projectgreen} #1}}

%%% SECTION HEADLINES %%%

% Our numbers
\newcommand{\likteikn}{Likskapsteiknet}
\newcommand{\talsifverd}{Tal, siffer og verdi}
\newcommand{\koordsys}{Koordinatsystem}

% Calculations
\newcommand{\adi}{Addisjon}
\newcommand{\sub}{Subtraksjon}
\newcommand{\gong}{Multiplikasjon (Gonging)}
\newcommand{\del}{Divisjon (deling)}

%Factorization and order of operations
\newcommand{\fak}{Faktorisering}
\newcommand{\rrek}{Reknerekkefølge}

%Fractions
\newcommand{\brgrpr}{Introduksjon}
\newcommand{\brvu}{Verdi, utviding og forkorting av brøk}
\newcommand{\bradsub}{Addisjon og subtraksjon}
\newcommand{\brgngheil}{Brøk gonga med heiltal}
\newcommand{\brdelheil}{Brøk delt med heiltal}
\newcommand{\brgngbr}{Brøk gonga med brøk}
\newcommand{\brkans}{Kansellering av faktorar}
\newcommand{\brdelmbr}{Deling med brøk}
\newcommand{\Rasjtal}{Rasjonale tal}

%Negative numbers
\newcommand{\negintro}{Introduksjon}
\newcommand{\negrekn}{Dei fire rekneartane med negative tal}
\newcommand{\negmeng}{Negative tal som mengde}

% Geometry 1
\newcommand{\omgr}{Omgrep}
\newcommand{\eignsk}{Eigenskapar for trekantar og firkantar}
\newcommand{\omkr}{Omkrins}
\newcommand{\area}{Areal}

%Algebra 
\newcommand{\algintro}{Introduksjon}
\newcommand{\pot}{Potensar}
\newcommand{\irrasj}{Irrasjonale tal}

%Equations
\newcommand{\ligintro}{Introduksjon}
\newcommand{\liglos}{Løysing ved dei fire rekneartane}
\newcommand{\ligloso}{Løysingsmetodane oppsummert}

%Functions
\newcommand{\fintro}{Introduksjon}
\newcommand{\lingraf}{Lineære funksjonar og grafar}

%Geometry 2
\newcommand{\geoform}{Formlar for areal og omkrins}
\newcommand{\kongogsim}{Kongruente og formlike trekantar}
\newcommand{\geofork}{Forklaringar}

% Names of rules
\newcommand{\gangdestihundre}{Å gange desimaltall med 10, 100 osv.}
\newcommand{\delmedtihundre}{Deling med 10, 100, 1\,000 osv.}
\newcommand{\ompref}{Omgjøring av prefikser}
\newcommand{\adkom}{Addisjon er kommutativ}
\newcommand{\gangkom}{Multiplikasjon er kommutativ}
\newcommand{\brdef}{Brøk som omskriving av delestykke}
\newcommand{\brtbr}{Brøk gonga med brøk}
\newcommand{\delmbr}{Brøk delt på brøk}
\newcommand{\gangpar}{Gonging med parentes (distributiv lov)}
\newcommand{\gangparsam}{Parantesar gonga saman}
\newcommand{\gangmnegto}{Gonging med negative tal I}
\newcommand{\gangmnegtre}{Gonging med negative tal II}
\newcommand{\konsttre}{Konstruksjon av trekantar}
\newcommand{\kongtre}{Kongruente trekantar}
\newcommand{\topv}{Toppvinklar}
\newcommand{\trisum}{Summen av vinklane i ein trekant}
\newcommand{\firsum}{Summen av vinklane i ein firkant}
\newcommand{\potgang}{Gonging med potensar}
\newcommand{\potdivpot}{Divisjon med potensar}
\newcommand{\potanull}{Spesialtilfellet \boldmath $a^0$}
\newcommand{\potneg}{Potens med negativ eksponent}
\newcommand{\potbr}{Brøk som grunntal}
\newcommand{\faktgr}{Faktorar som grunntal}
\newcommand{\potsomgrunn}{Potens som grunntal}
\newcommand{\arsirk}{Arealet til ein sirkel}
\newcommand{\artrap}{Arealet til eit trapes}
\newcommand{\arpar}{Arealet til eit parallellogram}
\newcommand{\pyt}{Pytagoras' setning}
\newcommand{\forform}{Forhold i formlike trekantar}
\newcommand{\vilkform}{Vilkår i formlike trekantar}
\newcommand{\omkrsirk}{Omkrinsen til ein sirkel (og $ \bm \pi $)}
\newcommand{\artri}{Arealet til ein trekant}
\newcommand{\arrekt}{Arealet til eit rektangel}
\newcommand{\liknflyt}{Flytting av ledd over likskapsteiknet}
\newcommand{\funklin}{Lineære funksjonar}

%Opg
\newcommand{\abc}[1]{
	\begin{enumerate}[label=\alph*),leftmargin=18pt]
		#1
	\end{enumerate}
}
\newcommand{\abcs}[2]{
	\begin{enumerate}[label=\alph*),start=#1,leftmargin=18pt]
		#2
	\end{enumerate}
}
\newcommand{\abcn}[1]{
	\begin{enumerate}[label=\arabic*),leftmargin=18pt]
		#1
	\end{enumerate}
}
\newcommand{\abch}[1]{
	\hspace{-2pt}	\begin{enumerate*}[label=\alph*), itemjoin=\hspace{1cm}]
		#1
	\end{enumerate*}
}
\newcommand{\abchs}[2]{
	\hspace{-2pt}	\begin{enumerate*}[label=\alph*), itemjoin=\hspace{1cm}, start=#1]
		#2
	\end{enumerate*}
}

\newcommand{\opgt}{\phantomsection \addcontentsline{toc}{section}{Oppgaver} \section*{Oppgaver for kapittel \thechapter}\vs \setcounter{section}{1}}
\newcounter{opg}
\numberwithin{opg}{section}
\newcommand{\op}[1]{\vspace{15pt} \refstepcounter{opg}\large \textbf{\color{blue}\theopg} \vspace{2 pt} \label{#1} \\}
\newcommand{\ekspop}[1]{\vsk\textbf{Gruble \thechapter.#1}\vspace{2 pt} \\}
\newcommand{\nes}{\stepcounter{section}
	\setcounter{opg}{0}}
\newcommand{\opr}[1]{\vspace{3pt}\textbf{\ref{#1}}}
\newcommand{\oeks}[1]{\begin{tcolorbox}[boxrule=0.3 mm,arc=0mm,colback=white]
		\textit{Eksempel: } #1	  
\end{tcolorbox}}
\newcommand\opgeks[2][]{\begin{tcolorbox}[boxrule=0.1 mm,arc=0mm,enhanced jigsaw,breakable,colback=white] {\footnotesize \textbf{Eksempel #1} \\} \footnotesize #2 \end{tcolorbox}\vspace{-5pt} }

%License
\newcommand{\lic}{\textit{Matematikken sine byggesteinar by Sindre Sogge Heggen is licensed under CC BY-NC-SA 4.0. To view a copy of this license, visit\\ 
		\net{http://creativecommons.org/licenses/by-nc-sa/4.0/}{http://creativecommons.org/licenses/by-nc-sa/4.0/}}}

%referances
\newcommand{\net}[2]{{\color{blue}\href{#1}{#2}}}
\newcommand{\hrs}[2]{\hyperref[#1]{\color{blue}\textsl{#2 \ref*{#1}}}}
\newcommand{\rref}[1]{\hrs{#1}{Regel}}
\newcommand{\refkap}[1]{\hrs{#1}{Kapittel}}
\newcommand{\refsec}[1]{\hrs{#1}{Seksjon}}

\usepackage{datetime2}

\usepackage[]{hyperref}


\begin{document}
%\newpage
\section{\rrek}
\subsection*{Prioriteringen av rekneartene}
Se på følgende regnestykke:
\[ 2+3\cdot4 \]
Et slikt regnestykke \textsl{kunne} man tolket på to måter:
\begin{enumerate}
	\item ''2 pluss 3 er 5. 5 ganget med 4 er 20. Svaret er 20.''
	\item ''3 ganget med 4 er 12. 2 pluss 12 er 14. Svaret er 14.''
\end{enumerate}
Men svarene blir ikke like! Det er altså behov for å ha noen regler om hva vi skal regne ut først. Den ene regelen er at vi må regne ut ganging eller deling \textsl{før} vi legger sammen eller trekker ifra, dette betyr at \regv
\st{ \vs
\alg{
	2+3\cdot 4&=\text{''Regn ut }3\cdot4\text{, og legg sammen med 2''}  \\
	&= 2+12 \\
	&= 14
}
}
Men hva om vi ønsket å legge saman $ 2 $ og $ 3 $ først, og så gange summen med 4? Å fortelle at noe skal regnes ut først gjør vi ved hjelp av parenteser: \regv
\st{\vs
\alg{
(2+3)\cdot4&=\text{''Legg sammen 2 og 3, og gang med 4 etterpå''} \\
&= 5\cdot 4 \\
&= 20
}
}\regv

\reg[Regnerekkefølge \label{rrek}]{ \vspace{-5pt}
\begin{enumerate}
	\item Uttrykk med parentes
	\item Multiplikasjon eller divisjon
	\item Addisjon eller subtraksjon
\end{enumerate}
} 
\newpage
\eks[1]{
Regn ut
\[ 23-(3+9)+4\cdot 7 \]
\sv \vs \vs
\algv{
&& 23-(3+9)+4\cdot 7&=23-12+4\cdot7 &&\text{Parantes} \\
&&&=23-12+28 &&\text{Ganging} \\
&&&=39 &&\text{Addisjon og subtraksjon}
}
}
\eks[2]{
	Regn ut
	\[ 18:(7-5)-3 \]
	\sv \vs \vs
	\algv{
		&& 18:(7-5)-3&=18:2-3 &&\text{Parantes} \\
		&&&=9-3 &&\text{Deling} \\
		&&&=6 &&\text{Addisjon og subtraksjon}
	}
}
\subsection*{Ganging med parentes}
Hvor mange ruter ser vi i figuren under?
\fig{gang}
To måter man kan tenke på er disse:
\begin{enumerate}
	\item Det er $ 2\cdot4 =8 $ lilla ruter og $ 3\cdot4=12 $ grønne ruter. Til sammen er det $ 8+12 =20 $ ruter. Dette kan vi skrive som
\[ 2\cdot 4 + 3\cdot 4 = 20  \]
	\item Det er $ 2+3=5 $ ruter bortover og 4 ruter oppover. Altså er det $ 5\cdot4 =20 $ ruter totalt. Dette kan vi skrive som
	\[ (2+3)\cdot 4 = 20 \]
\end{enumerate}
Av disse to utregningene har vi at
\[ (2+3)\cdot4 = 2\cdot 4+ 3\cdot4 \]
\reg[\gangpar \label{gangpar}]{
Når et parentesuttrykk er en faktor, kan vi gange de andre faktorene med hvert enkelt ledd i parentesuttrykket.	 
%\fig{gang1}
}
\eks[1]{
\vs
\[ ({\color{orange}4}+{\color{ForestGreen}7})\cdot {\color{blue}8}={\color{orange}4}\cdot{\color{blue}8}+{\color{ForestGreen}7}\cdot{\color{blue}8} \]	
}
\eks[2]{ \vsb \vs
\alg{
(10-7)\cdot2 &= 10\cdot 2-7\cdot2\\
&=20-14 \\
&=6
}	
\mer Her vil det selvsagt være raskere å regne slik:
\[ (10-7)\cdot 2=3\cdot 2 =6 \]
}
\eks[2]{
Regn ut $ 12\cdot 3 $.

\sv
\vsb \vsb
\alg{
12\cdot 3&= (10+2)\cdot 3 \\
&=10\cdot 3 +2\cdot 3 \\
&=30 +6 \\
&=36
}	
}
\info{Obs!}{
Vi introduserte parenteser som en indikator på hva som skulle regnes ut først, men \rref{gangpar} gir en alternativ og likeverdig betydning av parenteser. Regelen kommer spesielt til nytte i algebraregning (sjå \hrs{Del2}{Del}).
}
\newpage
\subsubsection{Å gange med 0}
Vi har tidligere sett at 0 kan skrives som en differanse mellom to tall, og dette kan vi nå utnytte til å finne produktet når vi ganger med 0. La oss se på regnestykket
\[ (2-2)\cdot3 \]
Av \rref{gangpar} har vi at
\alg{
	(2-2)\cdot3 &= 2\cdot3-2\cdot3\\&=6-6\\&=0
}
Sidan $ 0=2-2 $, må dette bety at
\[ 0\cdot3=0 \]

\reg[Gonging med 0]{
	Viss 0 er ein faktor, er produktet lik 0.
}
\eks[1]{ \vsb \vs
	\alg{
		7\cdot0&=0\vn
		0\cdot219 &=0
	}
}
\subsection*{Assosiative lover}
\reg[Assosiativ lov ved addisjon]{
Plasseringen av parenteser mellom ledd har ingen påvirkning på summen.
}
\eks[]{ \vsb \vs
\alg{
(2+3)+4&=5+3=9 \vn
2+(3+4)&=2+7=9
}
\fig{asso0}
}
\vsk \vsk

\reg[Assosiativ lov ved multiplikasjon]{
Plasseringen av parenteser mellom faktorer har ingen påvirkning på produktet.
}
\eks[]{ \vsb \vs
\alg{
(2\cdot3)\cdot4 &=6\cdot 4=24 \vn
2\cdot(3\cdot4)&=2\cdot 12 =24
}
\fig{asso1}
}  \vsk

I motsetning til addisjon og multiplikasjon, er hverken subtraksjon eller divisjon assosiative:
\alg{
(12-5)-4&=7-4=3 \\
12-(5-4)&=12-1=11
}
\alg{
	(80:10):2&=8:2=4 \\
	80:(10:2)&=80:5=16
}
Vi har sett at parentesene hjelper oss med å si noeo om \textsl{prioriteringen} av regneartene, men det at subtraksjon og divisjon er ikke-assosiative fører til at vi også må ha en regel for hvilken \textsl{retning} vi skal regne i. \regv

\reg[Retning på utregninger \label{rret}]{
Regnearter som ut ifra \rref{rrek} har lik prioritet, skal regnes fra venstre mot høyre.
}
\eks[1]{ \vsb \vsb
	\alg{
		12-5-4&=(12-5)-4 \\
		&=7-4\\
		&=3
	}
}
\eks[2]{ \vsb \vsb
	\alg{
		80:10:2&=(80:10):2 \\
		&=8:2 \\
		&=4
	}
}
\eks[3]{ \vsb \vs
	\alg{
		6: 3\cdot 4 &= (6:3)\cdot4\\ 
		&=2\cdot4 \\
		&= 8
	}
}

\end{document}


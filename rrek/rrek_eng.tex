\documentclass[english,hidelinks,pdftex, 11 pt, class=report,crop=false]{standalone}
\usepackage[T1]{fontenc}
\usepackage[utf8]{luainputenc}
\usepackage{lmodern} % load a font with all the characters
\usepackage{geometry}
\geometry{verbose,paperwidth=16.1 cm, paperheight=24 cm, inner=2.3cm, outer=1.8 cm, bmargin=2cm, tmargin=1.8cm}
\setlength{\parindent}{0bp}
\usepackage{import}
\usepackage[subpreambles=false]{standalone}
\usepackage{amsmath}
\usepackage{amssymb}
\usepackage{esint}
\usepackage{babel}
\usepackage{tabu}
\makeatother
\makeatletter

\usepackage{titlesec}
\usepackage{ragged2e}
\RaggedRight
\raggedbottom
\frenchspacing

% Norwegian names of figures, chapters, parts and content
\addto\captionsenglish{\renewcommand{\figurename}{Figure}}
\makeatletter
\addto\captionsenglish{\renewcommand{\chaptername}{Chapter}}
%\addto\captionsenglish{\renewcommand{\partname}{Part}}

%\addto\captionsenglish{\renewcommand{\contentsname}{Content}}

\usepackage{graphicx}
\usepackage{float}
\usepackage{subfig}
\usepackage{placeins}
\usepackage{cancel}
\usepackage{framed}
\usepackage{wrapfig}
\usepackage[subfigure]{tocloft}
\usepackage[font=footnotesize,labelfont=sl]{caption} % Figure caption
\usepackage{bm}
\usepackage[dvipsnames, table]{xcolor}
\definecolor{shadecolor}{rgb}{0.105469, 0.613281, 1}
\colorlet{shadecolor}{Emerald!15} 
\usepackage{icomma}
\makeatother
\usepackage[many]{tcolorbox}
\usepackage{multicol}
\usepackage{stackengine}

% For tabular
\usepackage{array}
\usepackage{multirow}
\usepackage{longtable} %breakable table

% Ligningsreferanser
\usepackage{mathtools}
\mathtoolsset{showonlyrefs}

% index
\usepackage{imakeidx}
\makeindex[title=Index]

%Footnote:
\usepackage[bottom, hang, flushmargin]{footmisc}
\usepackage{perpage} 
\MakePerPage{footnote}
\addtolength{\footnotesep}{2mm}
\renewcommand{\thefootnote}{\arabic{footnote}}
\renewcommand\footnoterule{\rule{\linewidth}{0.4pt}}
\renewcommand{\thempfootnote}{\arabic{mpfootnote}}

%colors
\definecolor{c1}{cmyk}{0,0.5,1,0}
\definecolor{c2}{cmyk}{1,0.25,1,0}
\definecolor{n3}{cmyk}{1,0.,1,0}
\definecolor{neg}{cmyk}{1,0.,0.,0}

% Lister med bokstavar
\usepackage{enumitem}

\newcounter{rg}
\numberwithin{rg}{chapter}
\newcommand{\reg}[2][]{\begin{tcolorbox}[boxrule=0.3 mm,arc=0mm,colback=blue!3] {\refstepcounter{rg}\phantomsection \large \textbf{\therg \;#1} \vspace{5 pt}}\newline #2  \end{tcolorbox}\vspace{-5pt}}

\newcommand\alg[1]{\begin{align} #1 \end{align}}

\newcommand\eks[2][]{\begin{tcolorbox}[boxrule=0.3 mm,arc=0mm,enhanced jigsaw,breakable,colback=green!3] {\large \textbf{Example #1} \vspace{5 pt}\\} #2 \end{tcolorbox}\vspace{-5pt} }

\newcommand{\st}[1]{\begin{tcolorbox}[boxrule=0.0 mm,arc=0mm,enhanced jigsaw,breakable,colback=yellow!12]{ #1} \end{tcolorbox}}

\newcommand{\spr}[1]{\begin{tcolorbox}[boxrule=0.3 mm,arc=0mm,enhanced jigsaw,breakable,colback=yellow!7] {\large \textbf{The language box} \vspace{5 pt}\\} #1 \end{tcolorbox}\vspace{-5pt} }

\newcommand{\sym}[1]{\colorbox{blue!15}{#1}}

\newcommand{\info}[2]{\begin{tcolorbox}[boxrule=0.3 mm,arc=0mm,enhanced jigsaw,breakable,colback=cyan!6] {\large \textbf{#1} \vspace{5 pt}\\} #2 \end{tcolorbox}\vspace{-5pt} }

\newcommand\algv[1]{\vspace{-11 pt}\begin{align*} #1 \end{align*}}

\newcommand{\regv}{\vspace{5pt}}
\newcommand{\mer}{\textsl{Notice}: }
\newcommand{\merk}{Notice}
\newcommand\vsk{\vspace{11pt}}
\newcommand\vs{\vspace{-11pt}}
\newcommand\vsb{\vspace{-16pt}}
\newcommand\sv{\vsk \textbf{Answer} \vspace{4 pt}\\}
\newcommand\br{\\[5 pt]}
\newcommand{\asym}[1]{../fig/#1}
\newcommand\algvv[1]{\vs\vs\begin{align*} #1 \end{align*}}
\newcommand{\y}[1]{$ {#1} $}
\newcommand{\os}{\\[5 pt]}
\newcommand{\prbxl}[2]{
\parbox[l][][l]{#1\linewidth}{#2
	}}
\newcommand{\prbxr}[2]{\parbox[r][][l]{#1\linewidth}{
		\setlength{\abovedisplayskip}{5pt}
		\setlength{\belowdisplayskip}{5pt}	
		\setlength{\abovedisplayshortskip}{0pt}
		\setlength{\belowdisplayshortskip}{0pt} 
		\begin{shaded}
			\footnotesize	#2 \end{shaded}}}

\renewcommand{\cfttoctitlefont}{\Large\bfseries}
\setlength{\cftaftertoctitleskip}{0 pt}
\setlength{\cftbeforetoctitleskip}{0 pt}

\newcommand{\bs}{\\[3pt]}
\newcommand{\vn}{\\[6pt]}
\newcommand{\fig}[1]{\begin{figure}
		\centering
		\includegraphics[]{\asym{#1}}
\end{figure}}

\newcommand{\sectionbreak}{\clearpage} % New page on each section

% Equation comments
\newcommand{\cm}[1]{\llap{\color{blue} #1}}

\newcommand\fork[2]{\begin{tcolorbox}[boxrule=0.3 mm,arc=0mm,enhanced jigsaw,breakable,colback=yellow!7] {\large \textbf{#1 (explanation)} \vspace{5 pt}\\} #2 \end{tcolorbox}\vspace{-5pt} }


%%% SECTION HEADLINES %%%

% Our numbers
\newcommand{\likteikn}{The equal sign}
\newcommand{\talsifverd}{Numbers, digits and values}
\newcommand{\koordsys}{Coordinate systems}

% Calculations
\newcommand{\adi}{Addition}
\newcommand{\sub}{Subtraction}
\newcommand{\gong}{Multiplication}
\newcommand{\del}{Division}

%Factorization and order of operations
\newcommand{\fak}{Factorization}
\newcommand{\rrek}{Order of operations}

%Fractions
\newcommand{\brgrpr}{Introduction}
\newcommand{\brvu}{Values, expanding and simplifying}
\newcommand{\bradsub}{Addition and subtraction}
\newcommand{\brgngheil}{Fractions multiplied by integers}
\newcommand{\brdelheil}{Fractions divided by integers}
\newcommand{\brgngbr}{Fractions multiplied by fractions}
\newcommand{\brkans}{Cancelation of fractions}
\newcommand{\brdelmbr}{Division by fractions}
\newcommand{\Rasjtal}{Rational numbers}

%Negative numbers
\newcommand{\negintro}{Introduction}
\newcommand{\negrekn}{The elementary operations}
\newcommand{\negmeng}{Negative numbers as amounts}

% Geometry 1
\newcommand{\omgr}{Terms}
\newcommand{\eignsk}{Attributes of triangles and quadrilaterals}
\newcommand{\omkr}{Perimeter}
\newcommand{\area}{Area}

%Algebra 
\newcommand{\algintro}{Introduction}
\newcommand{\pot}{Powers}
\newcommand{\irrasj}{Irrational numbers}

%Equations
\newcommand{\ligintro}{Introduction}
\newcommand{\liglos}{Solving with the elementary operations}
\newcommand{\ligloso}{Solving with elementary operations summarized}

%Functions
\newcommand{\fintro}{Introduction}
\newcommand{\lingraf}{Linear functions and graphs}

%Geometry 2
\newcommand{\geoform}{Formulas of area and perimeter}
\newcommand{\kongogsim}{Congruent and similar triangles}
\newcommand{\geofork}{Explanations}

% Names of rules
\newcommand{\adkom}{Addition is commutative}
\newcommand{\gangkom}{Multiplication is commutative}
\newcommand{\brdef}{Fractions as rewriting of division}
\newcommand{\brtbr}{Fractions multiplied by fractions}
\newcommand{\delmbr}{Fractions divided by fractions}
\newcommand{\gangpar}{Distributive law}
\newcommand{\gangparsam}{Paranthesis multiplied together}
\newcommand{\gangmnegto}{Multiplication by negative numbers I}
\newcommand{\gangmnegtre}{Multiplication by negative numbers II}
\newcommand{\konsttre}{Unique construction of triangles}
\newcommand{\kongtre}{Congruent triangles}
\newcommand{\topv}{Vertical angles}
\newcommand{\trisum}{The sum of angles in a triangle}
\newcommand{\firsum}{The sum of angles in a quadrilateral}
\newcommand{\potgang}{Multiplication by powers}
\newcommand{\potdivpot}{Division by powers}
\newcommand{\potanull}{The special case of \boldmath $a^0$}
\newcommand{\potneg}{Powers with negative exponents}
\newcommand{\potbr}{Fractions as base}
\newcommand{\faktgr}{Factors as base}
\newcommand{\potsomgrunn}{Powers as base}
\newcommand{\arsirk}{The area of a circle}
\newcommand{\artrap}{The area of a trapezoid}
\newcommand{\arpar}{The area of a parallelogram}
\newcommand{\pyt}{Pythagoras's theorem}
\newcommand{\forform}{Ratios in similar triangles}
\newcommand{\vilkform}{Terms of similar triangles}
\newcommand{\omkrsirk}{The perimeter of a circle (and the value of $ \bm \pi $)}
\newcommand{\artri}{The area of a triangle}
\newcommand{\arrekt}{The area of a rectangle}
\newcommand{\liknflyt}{Moving terms across the equal sign}
\newcommand{\funklin}{Linear functions}

%License
\newcommand{\lic}{\textit{First Principles of Math by Sindre Sogge Heggen is licensed under CC BY-NC-SA 4.0. To view a copy of this license, visit\\ 
		\net{http://creativecommons.org/licenses/by-nc-sa/4.0/}{http://creativecommons.org/licenses/by-nc-sa/4.0/}}}

%referances
\newcommand{\net}[2]{{\color{blue}\href{#1}{#2}}}
\newcommand{\hrs}[2]{\hyperref[#1]{\color{blue}\textsl{#2 \ref*{#1}}}}
\newcommand{\rref}[1]{\hrs{#1}{Rule}}
\newcommand{\refkap}[1]{\hrs{#1}{Chapter}}
\newcommand{\refsec}[1]{\hrs{#1}{Section}}

\usepackage{datetime2}
\usepackage[]{hyperref}


\begin{document}
%\newpage
\section{\rrek}
\subsection*{Priority of the operations}
Look at the following calculation:
\[ 2+3\cdot4 \]
This \textsl{could} have been interpreted in two ways:
\begin{enumerate}
	\item ''2 plus 3 equals 5. 5 times 4 equals 20. The answer is 20.''
	\item ''3 times 4 equals 12. 2 plus 12 equals 14. The answer is.''
\end{enumerate}
But the answers are not the same! This points out the need to have rules for what to calculate first. One of these rules is that multiplication and division is to be calculated \textsl{before} addition or subtraction, which means that \regv
\st{ \vs
\alg{
	2+3\cdot 4&=\text{''Calculate }3\cdot4\text{, then add 2''}  \\
	&= 2+12 \\
	&= 14
}
}
But what if we wanted to calculate $ 2+3 $ first, then multiply the sum by 4? We use parentheses to tell that something is to be calculated first: \regv
\st{\vs
\alg{
(2+3)\cdot4&=\text{''Calculate }2+3\text{, multiply by 4 afterwards''} \\
&= 5\cdot 4 \\
&= 20
}
}\regv

\reg[Order of operations \label{rrek}]{ \vspace{-5pt}
\begin{enumerate}
	\item Expressions with parentheses
	\item Multiplication or division
	\item Addition or subtraction
\end{enumerate}
} 
\newpage
\eks[1]{
Calculate
\[ 23-(3+9)+4\cdot 7 \]
\sv \vs \vs
\algv{
&& 23-(3+9)+4\cdot 7&=23-12+4\cdot7 &&\text{Parentheses} \\
&&&=23-12+28 &&\text{Multiplication} \\
&&&=39 &&\text{Addition and subtraction}
}
}
\eks[2]{
	Calculate
	\[ 18:(7-5)-3 \]
	\sv \vs \vs
	\algv{
		&& 18:(7-5)-3&=18:2-3 &&\text{Parentheses} \\
		&&&=9-3 &&\text{Division} \\
		&&&=6 &&\text{Addition and subtraction}
	}
}
\subsection*{Multiplication involving parentheses}
How many boxes are present in this figure?
\fig{gang}
Two correct interpretations include:
\begin{enumerate}
	\item It is $ 2\cdot4 =8 $ purple boxes and $ 3\cdot4=12 $ green boxes. In total there are $ 8+12 =20 $ boxes. This we can write as
\[ 2\cdot 4 + 3\cdot 4 = 20  \]
	\item It is $ 2+3=5 $ boxes horizontally and 4 boxes vertically, so there are $ 5\cdot4 =20 $ boxes in total. This we can write as
	\[ (2+3)\cdot 4 = 20 \]
\end{enumerate}
From these two calculations it follows that
\[ (2+3)\cdot4 = 2\cdot 4+ 3\cdot4 \]
\reg[\gangpar \label{gangpar}]{
When an expression enclosed by parentheses is a factor, we can multiply the other factors with each term inside the parentheses.	 
%\fig{gang1}
}
\eks[1]{
\vs
\[ ({\color{orange}4}+{\color{ForestGreen}7})\cdot {\color{blue}8}={\color{orange}4}\cdot{\color{blue}8}+{\color{ForestGreen}7}\cdot{\color{blue}8} \]	
}
\eks[2]{ \vsb \vs
\alg{
(10-7)\cdot2 &= 10\cdot 2-7\cdot2\\
&=20-14 \\
&=6
}	
\mer Obviously, it would be easier to calculate like this:
\[ (10-7)\cdot 2=3\cdot 2 =6 \]
}
\eks[2]{
Calculate $ 12\cdot 3 $.

\sv
\vsb \vsb
\alg{
12\cdot 3&= (10+2)\cdot 3 \\
&=10\cdot 3 +2\cdot 3 \\
&=30 +6 \\
&=36
}	
}
\info{\merk}{
We introduced parentheses as an indicator of what to calculate first, but \rref{gangpar} gives an alternative and equivalent interpretation of parentheses . The rule is especially useful when working with algebra (see \hrs{Del2}{Part}\,).
}
\newpage
\subsubsection{Multiplying by 0}
Earlier we have seen that 0 can be expressed as the difference between two numbers, and this can help us calculate when multiplying by 0. Let's look at the calculation
\[ (2-2)\cdot3 \]
By \rref{gangpar}, we get
\alg{
 (2-2)\cdot3 &= 2\cdot3-2\cdot3\\&=6-6\\&=0
}
Since $ 0=2-2 $, this means that
\[ 0\cdot3=0 \]

\reg[Multiplication by 0]{
	If 0 is a factor, the product equals 0.
}
\eks[1]{ \vsb \vs
	\alg{
		7\cdot0&=0\vn
		0\cdot219 &=0
	}
}

\subsection*{Associative laws}
\reg[Associative law for addition]{
The placement of parentheses between terms has no impact on the sum.
}
\eks[]{ \vsb \vs
\alg{
(2+3)+4&=5+4=9 \vn
2+(3+4)&=2+7=9
}
\fig{asso0}
}
\vsk \vsk

\reg[Associative law for multiplication]{
The placement of parentheses between factors has no impact on the product.
}
\eks[]{ \vsb \vs
\alg{
(2\cdot3)\cdot4 &=6\cdot 4=24 \vn
2\cdot(3\cdot4)&=2\cdot 12 =24
}
\fig{asso1}
}  \vsk

Opposite to addition and multiplication, neither subtraction nor divison is associative:
\alg{
(12-5)-4&=7-4=3 \\
12-(5-4)&=12-1=11
}
\alg{
	(80:10):2&=8:2=4 \\
	80:(10:2)&=80:5=16
}
We have seen how parentheses helps indicating the \textsl{priority} of operations, but the fact that subtraction and division are non-associative brings the need of having a rule of in which \textsl{direction} to calculate. \regv

\reg[Direction of calculations \label{rret}]{
Operations which by \rref{rrek} have equal priority, are to be calculated from left to right.
}
\eks[1]{ \vsb \vsb
	\alg{
		12-5-4&=(12-5)-4 \\
		&=7-4\\
		&=3
	}
}
\eks[2]{ \vsb \vsb
	\alg{
		80:10:2&=(80:10):2 \\
		&=8:2 \\
		&=4
	}
}
\eks[3]{ \vsb \vs
	\alg{
		6: 3\cdot 4 &= (6:3)\cdot4\\ 
		&=2\cdot4 \\
		&= 8
	}
}

\end{document}


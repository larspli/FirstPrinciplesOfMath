\documentclass[english,hidelinks,pdftex, 11 pt, class=report,crop=false]{standalone}
\usepackage[T1]{fontenc}
\usepackage[utf8]{luainputenc}
\usepackage{lmodern} % load a font with all the characters
\usepackage{geometry}
\geometry{verbose,paperwidth=16.1 cm, paperheight=24 cm, inner=2.3cm, outer=1.8 cm, bmargin=2cm, tmargin=1.8cm}
\setlength{\parindent}{0bp}
\usepackage{import}
\usepackage[subpreambles=false]{standalone}
\usepackage{amsmath}
\usepackage{amssymb}
\usepackage{esint}
\usepackage{babel}
\usepackage{tabu}
\makeatother
\makeatletter

\usepackage{titlesec}
\usepackage{ragged2e}
\RaggedRight
\raggedbottom
\frenchspacing

% Norwegian names of figures, chapters, parts and content
\addto\captionsenglish{\renewcommand{\figurename}{Figur}}
\makeatletter
\addto\captionsenglish{\renewcommand{\chaptername}{Kapittel}}
\addto\captionsenglish{\renewcommand{\partname}{Del}}

\addto\captionsenglish{\renewcommand{\contentsname}{Innhald}}

\usepackage{graphicx}
\usepackage{float}
\usepackage{subfig}
\usepackage{placeins}
\usepackage{cancel}
\usepackage{framed}
\usepackage{wrapfig}
\usepackage[subfigure]{tocloft}
\usepackage[font=footnotesize,labelfont=sl]{caption} % Figure caption
\usepackage{bm}
\usepackage[dvipsnames, table]{xcolor}
\definecolor{shadecolor}{rgb}{0.105469, 0.613281, 1}
\colorlet{shadecolor}{Emerald!15} 
\usepackage{icomma}
\makeatother
\usepackage[many]{tcolorbox}
\usepackage{multicol}
\usepackage{stackengine}

% For tabular
\usepackage{array}
\usepackage{multirow}
\usepackage{longtable} %breakable table

% Ligningsreferanser
\usepackage{mathtools}
\mathtoolsset{showonlyrefs}

% index
\usepackage{imakeidx}
\makeindex[title=Indeks]

%Footnote:
\usepackage[bottom, hang, flushmargin]{footmisc}
\usepackage{perpage} 
\MakePerPage{footnote}
\addtolength{\footnotesep}{2mm}
\renewcommand{\thefootnote}{\arabic{footnote}}
\renewcommand\footnoterule{\rule{\linewidth}{0.4pt}}
\renewcommand{\thempfootnote}{\arabic{mpfootnote}}

%colors
\definecolor{c1}{cmyk}{0,0.5,1,0}
\definecolor{c2}{cmyk}{1,0.25,1,0}
\definecolor{n3}{cmyk}{1,0.,1,0}
\definecolor{neg}{cmyk}{1,0.,0.,0}

% Lister med bokstavar
\usepackage[inline]{enumitem}

\newcounter{rg}
\numberwithin{rg}{chapter}
\newcommand{\reg}[2][]{\begin{tcolorbox}[boxrule=0.3 mm,arc=0mm,colback=blue!3] {\refstepcounter{rg}\phantomsection \large \textbf{\therg \;#1} \vspace{5 pt}}\newline #2  \end{tcolorbox}\vspace{-5pt}}

\newcommand\alg[1]{\begin{align} #1 \end{align}}

\newcommand\eks[2][]{\begin{tcolorbox}[boxrule=0.3 mm,arc=0mm,enhanced jigsaw,breakable,colback=green!3] {\large \textbf{Eksempel #1} \vspace{5 pt}\\} #2 \end{tcolorbox}\vspace{-5pt} }

\newcommand{\st}[1]{\begin{tcolorbox}[boxrule=0.0 mm,arc=0mm,enhanced jigsaw,breakable,colback=yellow!12]{ #1} \end{tcolorbox}}

\newcommand{\spr}[1]{\begin{tcolorbox}[boxrule=0.3 mm,arc=0mm,enhanced jigsaw,breakable,colback=yellow!7] {\large \textbf{Språkboksen} \vspace{5 pt}\\} #1 \end{tcolorbox}\vspace{-5pt} }

\newcommand{\sym}[1]{\colorbox{blue!15}{#1}}

\newcommand{\info}[2]{\begin{tcolorbox}[boxrule=0.3 mm,arc=0mm,enhanced jigsaw,breakable,colback=cyan!6] {\large \textbf{#1} \vspace{5 pt}\\} #2 \end{tcolorbox}\vspace{-5pt} }

\newcommand\algv[1]{\vspace{-11 pt}\begin{align*} #1 \end{align*}}

\newcommand{\regv}{\vspace{5pt}}
\newcommand{\mer}{\textsl{Merk}: }
\newcommand\vsk{\vspace{11pt}}
\newcommand\vs{\vspace{-11pt}}
\newcommand\vsb{\vspace{-16pt}}
\newcommand\sv{\vsk \textbf{Svar:} \vspace{4 pt}\\}
\newcommand\br{\\[5 pt]}
\newcommand{\asym}[1]{../fig/#1}
\newcommand\algvv[1]{\vs\vs\begin{align*} #1 \end{align*}}
\newcommand{\y}[1]{$ {#1} $}
\newcommand{\os}{\\[5 pt]}
\newcommand{\prbxl}[2]{
\parbox[l][][l]{#1\linewidth}{#2
	}}
\newcommand{\prbxr}[2]{\parbox[r][][l]{#1\linewidth}{
		\setlength{\abovedisplayskip}{5pt}
		\setlength{\belowdisplayskip}{5pt}	
		\setlength{\abovedisplayshortskip}{0pt}
		\setlength{\belowdisplayshortskip}{0pt} 
		\begin{shaded}
			\footnotesize	#2 \end{shaded}}}

\renewcommand{\cfttoctitlefont}{\Large\bfseries}
\setlength{\cftaftertoctitleskip}{0 pt}
\setlength{\cftbeforetoctitleskip}{0 pt}

\newcommand{\bs}{\\[3pt]}
\newcommand{\vn}{\\[6pt]}
\newcommand{\fig}[1]{\begin{figure}
		\centering
		\includegraphics[]{\asym{#1}}
\end{figure}}


\newcommand{\sectionbreak}{\clearpage} % New page on each section

% Equation comments
\newcommand{\cm}[1]{\llap{\color{blue} #1}}

\newcommand\fork[2]{\begin{tcolorbox}[boxrule=0.3 mm,arc=0mm,enhanced jigsaw,breakable,colback=yellow!7] {\large \textbf{#1 (forklaring)} \vspace{5 pt}\\} #2 \end{tcolorbox}\vspace{-5pt} }




%colors
\newcommand{\colr}[1]{{\color{red} #1}}
\newcommand{\colb}[1]{{\color{blue} #1}}
\newcommand{\colo}[1]{{\color{orange} #1}}
\newcommand{\colc}[1]{{\color{cyan} #1}}
\definecolor{projectgreen}{cmyk}{100,0,100,0}
\newcommand{\colg}[1]{{\color{projectgreen} #1}}

%%% SECTION HEADLINES %%%

% Our numbers
\newcommand{\likteikn}{Likskapsteiknet}
\newcommand{\talsifverd}{Tal, siffer og verdi}
\newcommand{\koordsys}{Koordinatsystem}

% Calculations
\newcommand{\adi}{Addisjon}
\newcommand{\sub}{Subtraksjon}
\newcommand{\gong}{Multiplikasjon (Gonging)}
\newcommand{\del}{Divisjon (deling)}

%Factorization and order of operations
\newcommand{\fak}{Faktorisering}
\newcommand{\rrek}{Reknerekkefølge}

%Fractions
\newcommand{\brgrpr}{Introduksjon}
\newcommand{\brvu}{Verdi, utviding og forkorting av brøk}
\newcommand{\bradsub}{Addisjon og subtraksjon}
\newcommand{\brgngheil}{Brøk gonga med heiltal}
\newcommand{\brdelheil}{Brøk delt med heiltal}
\newcommand{\brgngbr}{Brøk gonga med brøk}
\newcommand{\brkans}{Kansellering av faktorar}
\newcommand{\brdelmbr}{Deling med brøk}
\newcommand{\Rasjtal}{Rasjonale tal}

%Negative numbers
\newcommand{\negintro}{Introduksjon}
\newcommand{\negrekn}{Dei fire rekneartane med negative tal}
\newcommand{\negmeng}{Negative tal som mengde}

% Geometry 1
\newcommand{\omgr}{Omgrep}
\newcommand{\eignsk}{Eigenskapar for trekantar og firkantar}
\newcommand{\omkr}{Omkrins}
\newcommand{\area}{Areal}

%Algebra 
\newcommand{\algintro}{Introduksjon}
\newcommand{\pot}{Potensar}
\newcommand{\irrasj}{Irrasjonale tal}

%Equations
\newcommand{\ligintro}{Introduksjon}
\newcommand{\liglos}{Løysing ved dei fire rekneartane}
\newcommand{\ligloso}{Løysingsmetodane oppsummert}

%Functions
\newcommand{\fintro}{Introduksjon}
\newcommand{\lingraf}{Lineære funksjonar og grafar}

%Geometry 2
\newcommand{\geoform}{Formlar for areal og omkrins}
\newcommand{\kongogsim}{Kongruente og formlike trekantar}
\newcommand{\geofork}{Forklaringar}

% Names of rules
\newcommand{\gangdestihundre}{Å gange desimaltall med 10, 100 osv.}
\newcommand{\delmedtihundre}{Deling med 10, 100, 1\,000 osv.}
\newcommand{\ompref}{Omgjøring av prefikser}
\newcommand{\adkom}{Addisjon er kommutativ}
\newcommand{\gangkom}{Multiplikasjon er kommutativ}
\newcommand{\brdef}{Brøk som omskriving av delestykke}
\newcommand{\brtbr}{Brøk gonga med brøk}
\newcommand{\delmbr}{Brøk delt på brøk}
\newcommand{\gangpar}{Gonging med parentes (distributiv lov)}
\newcommand{\gangparsam}{Parantesar gonga saman}
\newcommand{\gangmnegto}{Gonging med negative tal I}
\newcommand{\gangmnegtre}{Gonging med negative tal II}
\newcommand{\konsttre}{Konstruksjon av trekantar}
\newcommand{\kongtre}{Kongruente trekantar}
\newcommand{\topv}{Toppvinklar}
\newcommand{\trisum}{Summen av vinklane i ein trekant}
\newcommand{\firsum}{Summen av vinklane i ein firkant}
\newcommand{\potgang}{Gonging med potensar}
\newcommand{\potdivpot}{Divisjon med potensar}
\newcommand{\potanull}{Spesialtilfellet \boldmath $a^0$}
\newcommand{\potneg}{Potens med negativ eksponent}
\newcommand{\potbr}{Brøk som grunntal}
\newcommand{\faktgr}{Faktorar som grunntal}
\newcommand{\potsomgrunn}{Potens som grunntal}
\newcommand{\arsirk}{Arealet til ein sirkel}
\newcommand{\artrap}{Arealet til eit trapes}
\newcommand{\arpar}{Arealet til eit parallellogram}
\newcommand{\pyt}{Pytagoras' setning}
\newcommand{\forform}{Forhold i formlike trekantar}
\newcommand{\vilkform}{Vilkår i formlike trekantar}
\newcommand{\omkrsirk}{Omkrinsen til ein sirkel (og $ \bm \pi $)}
\newcommand{\artri}{Arealet til ein trekant}
\newcommand{\arrekt}{Arealet til eit rektangel}
\newcommand{\liknflyt}{Flytting av ledd over likskapsteiknet}
\newcommand{\funklin}{Lineære funksjonar}

%Opg
\newcommand{\abc}[1]{
	\begin{enumerate}[label=\alph*),leftmargin=18pt]
		#1
	\end{enumerate}
}
\newcommand{\abcs}[2]{
	\begin{enumerate}[label=\alph*),start=#1,leftmargin=18pt]
		#2
	\end{enumerate}
}
\newcommand{\abcn}[1]{
	\begin{enumerate}[label=\arabic*),leftmargin=18pt]
		#1
	\end{enumerate}
}
\newcommand{\abch}[1]{
	\hspace{-2pt}	\begin{enumerate*}[label=\alph*), itemjoin=\hspace{1cm}]
		#1
	\end{enumerate*}
}
\newcommand{\abchs}[2]{
	\hspace{-2pt}	\begin{enumerate*}[label=\alph*), itemjoin=\hspace{1cm}, start=#1]
		#2
	\end{enumerate*}
}

\newcommand{\opgt}{\phantomsection \addcontentsline{toc}{section}{Oppgaver} \section*{Oppgaver for kapittel \thechapter}\vs \setcounter{section}{1}}
\newcounter{opg}
\numberwithin{opg}{section}
\newcommand{\op}[1]{\vspace{15pt} \refstepcounter{opg}\large \textbf{\color{blue}\theopg} \vspace{2 pt} \label{#1} \\}
\newcommand{\ekspop}[1]{\vsk\textbf{Gruble \thechapter.#1}\vspace{2 pt} \\}
\newcommand{\nes}{\stepcounter{section}
	\setcounter{opg}{0}}
\newcommand{\opr}[1]{\vspace{3pt}\textbf{\ref{#1}}}
\newcommand{\oeks}[1]{\begin{tcolorbox}[boxrule=0.3 mm,arc=0mm,colback=white]
		\textit{Eksempel: } #1	  
\end{tcolorbox}}
\newcommand\opgeks[2][]{\begin{tcolorbox}[boxrule=0.1 mm,arc=0mm,enhanced jigsaw,breakable,colback=white] {\footnotesize \textbf{Eksempel #1} \\} \footnotesize #2 \end{tcolorbox}\vspace{-5pt} }

%License
\newcommand{\lic}{\textit{Matematikken sine byggesteinar by Sindre Sogge Heggen is licensed under CC BY-NC-SA 4.0. To view a copy of this license, visit\\ 
		\net{http://creativecommons.org/licenses/by-nc-sa/4.0/}{http://creativecommons.org/licenses/by-nc-sa/4.0/}}}

%referances
\newcommand{\net}[2]{{\color{blue}\href{#1}{#2}}}
\newcommand{\hrs}[2]{\hyperref[#1]{\color{blue}\textsl{#2 \ref*{#1}}}}
\newcommand{\rref}[1]{\hrs{#1}{Regel}}
\newcommand{\refkap}[1]{\hrs{#1}{Kapittel}}
\newcommand{\refsec}[1]{\hrs{#1}{Seksjon}}

\usepackage{datetime2}

\usepackage[]{hyperref}


\begin{document}
\section{\geoform}

Ein \textit{formel}\index{formel} er ei likning der ein variabel (som oftast) står aleine på éi side av likskapsteiknet. I \refsec{Areal} såg vi på formlar for arealet til rektangel og trekantar, men da brukte vi ord i staden for symbol. Her skal vi gjengi desse to formlane i ei meir algebraisk form, etterfulgt av andre klassiske formlar for areal, omkrins og volum.\regv

\reg[\arrekt\; (\ref{arrekt}) \label{arrekt2}]{ \index{areal!til rektangel}
Arealet $ A $ til eit rektangel med grunnlinje $ g $ og høgde $ h $ er
	\[A = g h \]
	\fig{tri12_alg}
}
\eks[1]{
	Finn arealet til rektangelet.
	\fig{tri12ba} \vsb
	\sv
	Arealet $ A $ til rektangelet er 
	\[ A =b\cdot 2 =2b \]	
}
\eks[2]{
	Finn arealet til kvadratet.
	\fig{tri12ca} \vsb
	\sv
	Arealet $ A $ til kvadratet er 
	\[ A =a\cdot a =a^2 \]	
}

\reg[\artri\;(\ref{artri})]{ \index{til trekant}
	Arealet $ A $ til ein trekant med grunnlinje $ g $ og høyde $ h $ er
	\[ A = \frac{g h}{2} \]
\fig{triar0}
}
\eks{
Kven av trekantane har størst areal? \vs
\begin{figure}
	\centering
	\subfloat{\includegraphics[]{\asym{triar0b}}}\qquad
	\subfloat{\includegraphics[]{\asym{triar0a}}}
	\qquad
	\subfloat{\includegraphics[]{\asym{triar0c}}}
\end{figure}
\sv

Vi let $ A_1 $, $ A_2 $ og $ A_3 $ vere areala til høvesvis trekanten til venstre, i midten og til høgre. Da har vi at
\alg{
A_1 &=\frac{4\cdot3}{2}=6 \vn
A_2 &=\frac{2\cdot 3}{2}=3\vn
A_3&=\frac{2\cdot5}{2}=5
}	
Altså er det trekanten til venstre som har størst areal.
}

\reg[\arpar \label{arpar}]{\index{til parallellogram}
	Arealet $ A $ til eit parallellogram med grunnlinje $ g $ og høgde $ h $ er
	\[ A = g h  \]
	\fig{tri21}
}
\eks{
Finn arealet til parallellogrammet
\fig{tri21b} \vsb
\sv
Arealet $ A $ til parallellogrammet er
\alg{
A&=5\cdot 2=10
}		
}
\fork{\ref{arpar} \arpar}{
	Av eit parallellogram kan vi alltid lage oss to trekantar ved å teikne inn ein av diagonalane:
	\fig{tri21c}
	Dei farga trekantane på figuren over har begge grunnlinje $ g $ og høgde $ h $. Da veit vi at begge har areal lik $ \frac{g h}{2} $.
	Arealet $ A $ til parallellogrammet blir dermed
	\alg{
		A&=\frac{g h}{2}+\frac{g h}{2} \\
		&=g\cdot h	
	}	
}
\reg[\artrap \label{artrap}]{\index{areal!til trapes}
	Arealet $ A $ til eit trapes med parallelle sider $ a $ og $ b $ og høgde $ h $ er
	\[ A = \frac{h(a+b)}{2}\]
	\fig{tri21a}
}
\eks{
Finn arealet til trapeset.
\fig{tri21e} \vsb
\sv 
Arealet $ A $ til trapeset er
\alg{
A&=\frac{3(6+4)}{2}\br
&=\frac{3\cdot10}{2}\br
&=15
}	
}
\info{Merk}{
Når ein tek utgangspunkt i ei grunnlinje og ei høgde, er arealformlane for eit parallellogram og eit rektangel identiske. Å anvende \rref{artrap} på eit parallellogram vil også resultere i eit uttrykk tilsvarande $ gh $. Dette er fordi eit parallellogram berre er eit spesialtilfelle av eit trapes (og eit rektangel er berre eit spesialtilfelle av eit parallellogram).
}
\newpage
\fork{\ref{artrap} \artrap}{
	Også for eit trapes får vi to trekanter viss vi teikner ein av diagonalene:
	\fig{tri21d}
	I figuren over er
	\alg{
		\text{Arealet til den blå trekanten}&= \frac{a h}{2} \br
		\text{Arealet til den grøne trekanten}&= \frac{b  h}{2}
	}
	Arealet $ A $ til trapeset blir dermed
	\alg{
		A&=\frac{ah}{2}+\frac{bh}{2}\br &=\frac{h(a+b)}{2} 	
	}	
}
\newpage
\reg[\omkrsirk \label{omkrsirk}]{\index{omkrins!til sirkel}
	Omkrinsen $ O $ til ein sirkel med radius $ r $ er
	\[ O = 2\pi r \]
	\fig{tri22}
	$ \pi = 3.141592653589793...
	$.
}
\eks[1]{
	Finn omkrinsen til sirkelen.
	\fig{tri22b}
	\sv 
	Omkrinsen $ O $ er 
	\algv{
		O &= 2 \pi \cdot 3 \\
		&= 6\pi 
	}
} \vsk

\reg[\arsirk \label{arsirk}]{ \index{areal!til sirkel}
	Arealet $ A $ til ein sirkel med radius $ r $ er
	\[ A = \pi r^2 \]
	\fig{tri22a}
}
\eks{
	Finn arealet til sirkelen.
	\fig{tri22c} \vsb
	\sv
	Arealet $ A $ til sirkelen er 
	\[ A=\pi\cdot 5^2 =25\pi \]
}
\newpage
\fork{\ref{arsirk} \arsirk}{
	I figuren under har vi delt opp ein sirkel i 4, 10 og 50 (like store) sektorar, og lagt desse bitene etter kvarandre.
	\fig{tri19}
	\fig{tri19a}
	\fig{tri19b}
	I kvart tilfelle må dei små sirkelbogene til saman utgjere heile boga, altså omkrinsen, til sirkelen. Viss sirkelen har radius $ r $, betyr dette at summen av bogene er $ 2\pi r $. Og når vi har like mange sektorar med bogen vendt opp som sektorar med bogen vend ned, må totallengda av bogenene vere $ \pi r $ både oppe og nede. \vsk
	
	Men jo fleire sektorar vi deler sirkelen inn i, jo meir liknar sammansettinga av dei på eit rektangel (i figuren under har vi 100 sektorar). Grunnlinja $ g $ til dette ''rektangelet'' vil vere tilnærma lik $ \pi r $, mens høgda vil vere tilnærma lik $ r $.
	\fig{tri19c}
	Arealet  $ A $ til ''rektangelet'', altså sirkelen, blir da
	\[ A\approx g h \approx \pi r\cdot r = \pi r^2 \]
}
\reg[\pyt \label{pyt}]{
	I ein rettvinkla trekant er arealet til kvadratet danna av \\hypotenusen lik summen av areala til kvadrata danna av\\ katetane.\regv
	
	\parbox[l][][l]{0.7\linewidth}{\[ a^2+b^2=c^2 \]
	}
	\parbox[r]{0.2\linewidth}{\includegraphics[]{\asym{tri26g}}} \vs
	\fig{tri26j}
}
\eks[1]{
	Finn lengda til $ c $.
	\fig{tri26h} \vs \vs
	\sv
	Vi veit at
	\[ c^2=a^2+b^2 \]
	der $ a $ og $ b $ er lengdene til dei kortaste sidene i trekanten. Dermed er
	\algv{
		c^2 &= 4^2 + 3^2 \\
		&= 16+9 \\
		&=25
	}
	Altså har vi at
	\[ c=5\qquad\vee\qquad c=-5 \]
	Da $ c $ er ei lengde, er $ c=5 $.
}
\newpage
\fork{\ref{pyt} \pyt}{ \label{pytforklaringintro}
	Under har vi teikna to kvadrat som er like store, men som er inndelt i forskjellige former.
	\fig{tri26d}
	Vi observerer no følgande:
	\begin{enumerate}
		\item Arealet til det raude kvadratet er $ a^2 $, arealet til det lilla kvadratet er $ b^2 $ og arealet til det blå kvadratet er $ c^2 $.
		\item Arealet til eit oransje rektangel er $ ab $ og arealet til ein grøn trekant er $ \frac{ab}{2} $.
		\item Om vi tek bort dei to oransje rektangla og dei fire grøne trekantane, er det igjen (av pkt. 2) eit like stort areal til venstre som til høgre.
		\fig{tri26e}
		Dette betyr at
		\begin{equation}\label{pytforkl}
		a^2+b^2=c^2
		\end{equation}
	\end{enumerate}
	Gitt ein trekant med sidelengder $ a, b $ og $ c $, der $ c $ er den lengste sidelengda.
	Så lenge trekanten er rettvinkla, kan vi alltid lage to kvadrat med sidelengder $ {a+b} $, slik som i første figur. \eqref{pytforkl} gjeld dermed for alle rettvinkla trekanter.
}
\newpage
\reg[\volforml \label{volforml}]{ 
	Volumet $ V $ til ei firkanta prisme eller ein sylinder med grunnflate $ G $ og høgde $ h $ er
	\[ V = G\cdot h \]
	\begin{figure}
		\centering
		\footnotesize
		\stackunder[6pt]{\includegraphics[scale=0.7]{\asym{vol3c}}}{Firkanta prisme}\qquad\qquad
		\stackunder[6pt]{\includegraphics[scale=0.7]{\asym{vol3b}}}{Sylinder}
	\end{figure}
	Volumet $ V $ til ei kjegle eller ei pyramide med grunnflate $ G $ og høgde $ h $ er
	\[ V = \frac{G\cdot h}{3} \]
	\begin{figure}
		\centering
		\footnotesize
		\stackunder[6pt]{\includegraphics[scale=0.7]{\asym{vol3}}}{Kjegle}\qquad \qquad    
		\stackunder[6pt]{\includegraphics[scale=0.7]{\asym{vol3a}}}{Firkanta pyramide}
	\end{figure}
}\vsk
\info{Merk}{
	Formlane frå \rref{volforml} gjeld også for prismer, sylindrar, kjegler og pyramider som heller (er skeive). Vis grunnflata er plassert horisontalt, er høgda den vertikale avstanden mellom grunnflata og toppen til figuren.
	\fig{vol4} 
	(For spisse gjenstandar som kjegler og pyramider finst det sjølvsagt bare eitt valg av grunnflate.)
}
\newpage
\eks[1]{
	\fig{vol5}
	Ein sylinder har radius 7 og høgde 5.
	\abc{
		\item Finn grunnflata til sylinderen.
		\item Finn volumet til sylinderen.
	}
	
	\sv \vs
	\abc{
		\item Av \rref{arrekt2} har vi at
		\algv{
			\text{grunnflate}&= \pi \cdot 7^2  \\
			&= 49 \pi
		} 
		\item Dermed er
		\algv{
			\text{volumet til sylinderen}&= 49\pi\cdot 6 \\
			&= 294\pi
		}
	}
}
\newpage
\eks[2]{
	Ei firkanta pyramide har lengde 2, bredde 3 og høgde 5.
	\fig{vol6}
	\abc{
		\item Finn grunnflata til pyramiden.
		\item Finn volumet til pyramiden.
	}
	\sv  \vs
	\abc{
		\item Av \rref{arrekt2} har vi at
		\algv{
			\text{grunnflate}&=2\cdot 3 \\
			&= 6 
		}
		\item Dermed er
		\algv{
			\text{volumet til pyramiden}&= 6\cdot 5 \\
			&= 30
		}
	}
} \vsk

\reg[\volkule \label{volkule}]{
	Volumet $ V $ til ei kule med radius $ r $ er:
	\[ V = \frac{4\cdot\pi\cdot r^3}{3} \]
	\begin{figure}
		\centering
		\includegraphics[scale=0.7]{\asym{vol3d}}
	\end{figure}
}
\newpage	
\section{\kongogsim}
\reg[\konsttre \label{konsttre}]{
Ein trekant $ \triangle ABC $, som vist i figuren under, kan bli unikt konstruert viss ein av følgande kriterium er oppfylt:
\prbxl{0.6}{\begin{enumerate}[label=\roman*)]
		\item $ c $, $ \angle A $ og $ \angle B $ er kjende.
		\item $ a $, $ b $ og $ c $ er kjende.
		\item $ b $, $ c $ og $ \angle A $ er kjende.
\end{enumerate}}
\parbox[r][][l]{0.3\linewidth}{
\fig{geo13}
}
}
\reg[\kongtre \label{kongtre}]{ \index{trekant!kongruet}
To trekantar som har same form og størrelse er kongruente.
\fig{geo14}
At trekantane i figuren over er kongruente skrivast
\[ \triangle ABC\cong\triangle DEF \]
}
\reg[Formlike trekantar \index{trekant!formlik}]{Formlike trekantar har tre vinklar som er parvis like store.
\fig{geo8}
At trekantane i figuren over er formlike skrivast \[ \triangle ABC\sim \triangle DEF \]
} \vsk
\newpage
\textbf{Samsvarande sider}\os
Når vi studerer formlike trekantar er \textit{samsvarande sider}\index{side!samsvarande} eit viktig omgrep. Samsvarande sider er sider som i formlike trekantar står \textit{motståande} den same vinkelen.
\fig{tri1}
For dei formlike trekantane $ \triangle ABC $ og $ \triangle DEF $ har vi at
\begin{multicols}{2}
	\quad I $ \triangle ABC $ er
	\begin{itemize}
		\item $ BC $ motstående til $u$.
		\item $ AC $ motstående til $ v$
		\item $ AB$ motstående til $ w $.
	\end{itemize}
	\quad I $ \triangle DEF $ er
	
	\begin{itemize}
		\item $ FE $ motstående til $u$.
		\item $ FD $ motstående til $ v$
		\item $ ED$ motstående til $ w $.
	\end{itemize}
\end{multicols}
Dette betyr at desse er samsvarande sider:
\begin{itemize}
	\item $ BC $ og $ FE $\\
	\item $ AC $ og $ FD $ \\
	\item $ AB $ og $ ED $
\end{itemize}

\reg[\forform \label{forform}]{
	Når to trekantar er formlike, er forholdet mellom samsvarande\footnote{Vi tek det her for gitt at kva sider som er samsvarande kjem fram av figuren.} sider det same.
	\[ \frac{AB}{DE}=\frac{AC}{DF}=\frac{BC}{EF} \]
\fig{geo8b}
}
\eks[]{
	Trekantene i figuren under er formlike. Finn lengda til $EF $.
	\begin{figure}
		\centering
		\includegraphics[scale=1]{\asym{tri3a}}\quad
		\includegraphics[scale=1]{\asym{tri3b}}
	\end{figure}
	\sv
	Vi observerer at $ AB $ samsvarer med $ DE $, $ BC $ med $ EF $ og $ AC $ med $ DF $. Det betyr at
	\alg{
		\frac{DE}{AB} &= \frac{EF}{BC} \br
		\frac{10}{5}&= \frac{EF}{3} \br
		2\cdot3&=\frac{EF}{\cancel{3}}\cdot\cancel{3}\\
		6 &= EF
	}
}
\info{Merk}{
Av \hrs{forform}{Regel} har vi at for to formlike trekantar $ \triangle ABC $ og $ \triangle DEF $ er
	\[ \frac{AB}{BC}=\frac{DE}{EF}\quad,\quad \frac{AB}{AC}=\frac{DE}{DF}\quad,\quad\frac{BC}{AC}=\frac{EF}{DF} \]
}
\reg[\vilkform \label{vilkform}]{
To trekantar $ \triangle ABC $ og $ \triangle DEF $ er formlike viss ein av desse vilkåra er oppfylt:
\begin{enumerate}[label=\roman*)]
	\item To vinklar i trekantane er parvis like store.
	\item $ \displaystyle \frac{AB}{DE}=\frac{AC}{DF}=\frac{BC}{EF} $
	\item $ \dfrac{AB}{DE}=\dfrac{AC}{DF} $ og $ \angle A=\angle D $.
\end{enumerate}

\fig{geo8b}
}
\eks[1]{
$ \angle ACB =90^\circ $. 
Vis at $ \triangle ABC \sim ACD $.
\fig{geo15} \vs
\sv
$ \triangle ABC $ og $ \triangle ACD $ er begge rettvinkla og dei har $ \angle DAC $ felles. Dermed er vilkår \textsl{i} fra \rref{vilkform} oppfylt, og trekantane er da formlike.\vsk

\mer På ein tilsvarande måte kan ein vise at $ \triangle ABC \sim CBD$.
}
\newpage
\eks[2]{
Undersøk om trekantane er formlike.
\fig{geo15a} \vs
\sv
Vi har at
\alg{
\frac{AC}{FD}=\frac{18}{12}=\frac{3}{2}\quad,\quad\frac{BC}{FE}=\frac{9}{6}=\frac{3}{2}\quad,\quad\frac{AB}{DE}=\frac{12}{10}=\frac{6}{5}
}
\alg{
\frac{AC}{IG}=\frac{18}{12}=\frac{3}{2}\quad,\quad\frac{BC}{IH}=\frac{9}{6}=\frac{3}{2}\quad,\quad \frac{AC}{IG}=\frac{18}{12}=\frac{3}{2}
}
Dermed oppfyller $ \triangle ABC $ og $ \triangle GHI $ vilkår \textsl{ii} fra \rref{vilkform}, og trekantane er da formlike.	
}
\eks[3]{
Undersøk om trekantane er formlike.	\vs
\fig{geo15b}
\sv
Vi har at $ {\angle BAC=\angle EDF} $ og at
\[ \frac{ED}{AB}=\frac{8}{4}=2\quad,\quad \frac{FD}{AC}=\frac{14}{7}=2 \]
Altså er vilkår \textsl{iii} fra \rref{vilkform} oppfylt, og da er trekantane formlike.
}

\newpage
\section{\geofork}
\fork{\ref{omkrsirk} \omkrsirk}
{
\textit{Vi skal her bruke \textit{regulære} mangekantar langs vegen til ønska resultat. I regulære mangekantar har alle sidene lik lengde. Da det er utelukkande regulære mangekantar vi kjem til å bruke, vil dei bli omtala berre som mangekantar.}\vsk

Vi skal starte med  sjå på tilnærmingar for å finne omkrinsen $ O_1 $ av ein sirkel med radius 1. 
\fig{geo9l} \vsk

\textbf{Øvre og nedre grense}\os
Ein god vane når ein skal prøve å finne ein størrelse, er å spørre seg om ein kan vite noko om kor stor eller liten ein \textsl{forventar} at han er. Vi startar derfor med å omslutte sirkelen med eit kvadrat med sidelengde 2:
\fig{geo9c}
Omkrinsen til sirkelen må vere mindre enn omkrinsen til kvadratet, derfor veit vi at
\alg{
	O_1&<2\cdot4  \\
	&< 8
}
Vidare innskriv vi ein sekskant. Sekskanten kan delast inn i 6 likesida trekantar som alle må ha sidelengder 1. Omkrinsen til sirkelen må vere større enn omkrinsen til sekskanten, noko som gir at
\alg{
	O_1&>6\cdot1 \\
	&> 6
}
\begin{figure}
	\centering
	\subfloat{\includegraphics[]{\asym{geo9}}}\qquad
	\subfloat{\includegraphics[]{\asym{geo9d2}}}
\end{figure}
Når vi no skal gå over til ei mykje meir nøyaktig jakt etter omkrinsen, veit vi altså at vi søker ein verdi mellom 6 og 8.\vsk

\textbf{Stadig betre tilnærmingar}\os
Vi fortsett med tanken om å innskrive ein mangekant. Av figurane under let vi oss overbevise om at dess fleire sider mangekanten har, dess betre estimat vil omkretsen til mangekanten vere for omkrinsen til sirkelen.
\begin{figure}
	\centering
	\subfloat[6-kant]{\includegraphics[]{\asym{geo9}}}\qquad\qquad
	\subfloat[12-kant]{\includegraphics[]{\asym{geo9a}}}	
\end{figure}
Da vi veit at sidelengda til ein 6-kant er 1, er det fristande å undersøke om vi kan bruke denne kunna til å finne sidelengda til andre mangekantar. Om vi innskriv også ein 12-kant i sirkelen vår (og i tillegg ein trekant), får vi ein figur som denne:
\begin{figure}
	\centering
	\subfloat[Ein 6-kant og ein 12-kant i lag med ein trekant danna av sentrum i sirkelen og ein av sidene i 12-kanten.]{\includegraphics[]{\asym{geo9g}}}\qquad\qquad
	\subfloat[Utklipp av trekant fra figur \textsl{(a)}.]{\includegraphics[]{\asym{geo9h}}}	
\end{figure}
La oss kalle sidelengda til 12-kanten for $ s_{12} $ og sidelengda til 6-kanten for $ s_6 $. Vidare legg vi merke til at punkta $ A $ og $ C $ ligg på sirkelbogen og at både $ \triangle ABC $ og $ \triangle BSC $ er rettvinkla trekantar (forklar for deg sjølv kvifor!). Vi har at
\alg{
	SC &= 1 \\
	BC &= \frac{s_6}{2} \\
	SB &= \sqrt{SC^2-BC^2} \\
	BA &= 1-SB \\
	AC &= s_{12}\\
	s_{12}^2 &= BA^2+BC^2
}
For å finne $ s_{12} $ må vi finne $ BA $, og for å finne $ BA $ må vi finne $ SB $. Vi startar derfor med å finne $ SB $. Da ${ SC=1} $ og $ {BC=\frac{s_6}{2}} $, er
\alg{
	SB &=\sqrt{1-\left( \frac{s_6}{2}\right)^2} \\
	&= \sqrt{1-\frac{s_6^2}{4}}
}
Vi går så vidare til å finne $ s_{12} $:
\alg{
	s_{12}^2 &= \left(1-SB\right)^2 + \left(\frac{s_6}{2}\right)^2 \\
	&= 1^2 - 2SB + SB^2 + \frac{s_6^2}{4}
}
Ved første augekast ser det ut som vi ikkje kan komme særleg lengre i å forenkle uttrykket på høgre side, men ein liten operasjon vil endre på dette. Hadde vi berre hatt $ -1 $ som eit ledd kunne vi slått saman $ -1 $ og $ \frac{s_6^2}{4} $ til å bli $ -SB^2 $. Derfor ''skaffar'' vi oss $ -1 $ ved å både addere og subtrahere 1 på høgresida:
\alg{
	s_{12}^2&= 1 - 2SB + SB^2 + \frac{s_6^2}{4}-1+1\\
	&= 2-2SB+SB^2-\left(1-\frac{s_6^2}{4}\right) \\
	&= 2-2SB+SB^2-SB^2\\
	&= 2-2SB\\
	&= 2-2\sqrt{1-\frac{s_6^2}{4}} \\
	&= 2-\sqrt{4}\,\sqrt{1-\frac{s_6^2}{4}} \\
	&= 2- \sqrt{4-s_6^2}
}
Altså er
\[ s_{12} = \sqrt{2- \sqrt{4-s_6^2}} \]
Sjølv om vi her har utleda relasjonen mellom sidelengdene $ s_{12} $ og $ s_6 $, er dette ein relasjon vi kunne vist for alle par av sidelengder der den eine er sidelengda til ein mangekant med dobbelt så mange sider som den andre. La $ s_n $ og $ s_{2n} $ høvesvis være sidelengda til ein mangekant og ein mangekant med dobbelt så mange sider. Da er
\begin{align}
s_{2n} = \sqrt{2- \sqrt{4-s_n^2}} \label{s2n}
\end{align}

Når vi kjenner sidelengda til ein innskriven mangekant, vil tilnærminga til omkrinsen til sirkelen vere denne sidelengda gonga med antal sidelengder i mangekanten. Ved hjelp av \eqref{s2n} kan vi stadig finne sidelengda til ein mangekant med dobbelt så mange sider som den forrige, og i tabellen under har vi funne sidelengda og tilnærminga til omkrinsen til sirkelen opp til ein 96-kant:

\begin{center}
	\renewcommand{\arraystretch}{1.5}
	\begin{tabular}{l|l|l}
		\textit{Formel for sidelengde}&\textit{Sidelengde} & \textit{Tilnærming for omkrins} \\
		\hline
		& $s_6= 1 $ & $ \;\,6\cdot s_6\;\,=6 $ \\
		$ s_{12} = \sqrt{2- \sqrt{4-s_6^2}} $ & $ s_{12}=0.517... $ & 		 $ 12\cdot s_{12}=6.211... $ \\
		$ s_{24} = \sqrt{2- \sqrt{4-s_{12}^2}} $ & $ s_{24}=0.261... $ & 		 $ 24\cdot s_{24}=6.265... $ \\
		$ s_{48} = \sqrt{2- \sqrt{4-s_{24}^2}} $ & $ s_{48}=0.130... $ & 		 $ 48\cdot s_{48}=6.278... $ \\		 
		$ s_{96} = \sqrt{2- \sqrt{4-s_{48}^2}} $ & $ s_{96}=0.065... $ & 		 $ 96\cdot s_{96}=6.282... $ \\		 		 
	\end{tabular}
\end{center}
\begin{figure}
	\centering
	\subfloat[6-kant]{\includegraphics[scale=0.75]{\asym{geo9}}}\quad
	\subfloat[12-kant]{\includegraphics[scale=0.75]{\asym{geo9a}}}\quad	
	\subfloat[24-kant]{\includegraphics[scale=0.75]{\asym{geo9i}}}\quad	\\
	\subfloat[48-kant]{\includegraphics[scale=0.75]{\asym{geo9j}}}\quad	
	\subfloat[96-kant]{\includegraphics[scale=0.75]{\asym{geo9k}}}		
\end{figure}
Utrekningane over er faktisk like langt som matematikaren \net{https://no.wikipedia.org/wiki/Arkimedes}{Arkimedes} kom allereie ca 250 f. kr!\vsk

For ei datamaskin er det ingen problem å rekne ut\footnote{For den datainteresserte skal det seiast at iterasjonsalgoritma må skrivast om for å unngå instabilitetar i utrekningane når antal sider blir mange.} dette for ein mangekant med ekstremt mange sider. Reknar vi oss fram til ein 201\,326\,592-kant  finn vi at
\[ \text{Omkrins av sirkel med radius 1}=6.283185307179586... \]
(Ved hjelp av meir avansert matematikk kan det visast at omkrinsen til ein sirkel med radius 1 er eit irrasjonalt tal, men at alle desimalane vist over er korrekte, derav likskapsteiknet.) \vsk
\newpage
\textbf{Den endelege formelen og $\bm \pi $} \os
Vi skal no komme fram til den kjende formelen for omkrinsen til ein sirkel. Også her skal vi ta for gitt at summen av sidelengdene til ein innskriven mangekant er ei tilnærming til omkrinsen som blir betre og betre dess fleire sidelengder det er.\vsk

For enkelheita si skuld skal vi bruke innskrivne firkantar for å få fram poenget vårt. Vi teiknar to sirklar som er vilkårleg store, men der den eine er større enn den andre, og innskriv ein firkant (eit kvadrat) i begge. Vi let $ R $ og $ r $ vere radien til høvesvis den største og den minste sirkelen, og $ K $ og $ k $ vere sidelengda til høvesvis den største og den minste firkanten.
\fig{geo9e2}
Begge firkantane kan delast inn i fire likebeinte trekantar:
\begin{figure}
	\centering
	\subfloat{\includegraphics[scale=1]{\asym{geo9e}}}\qquad
	\subfloat{\includegraphics[scale=1]{\asym{geo9f}}}	
\end{figure}
Da trekantane er formlike, har vi at
\begin{align}
\frac{K}{R} &=\frac{k}{r} \label{arogar}
\end{align}
Vi let $ {\tilde{O}=4K} $ og $ {\tilde{o}=4k} $ vere tilnærminga av omkrinsen til høvesvis den største og den minste sirkelen.
Ved å gonge med 4 på begge sider av \eqref{arogar} får vi at
\begin{align}
\frac{4A}{R} &= \frac{4a}{r} \br
\frac{\tilde{O}}{R} &= \frac{\tilde{o}}{r} \label{arogarto}
\end{align}
Og no merker vi oss dette:\vsk

\textsl{Sjølv om vi i kvar av dei to sirklane innskriv ein mangekant med 4, 100 eller kor mange sider det skulle vere, vil mangekantane alltid kunne delast inn i trekantar som oppfyller \eqref{arogar}. Og på same måte som vi har gjort i eksempelet over kan vi omskrive \eqref{arogar} til \eqref{arogarto} i staden.} \vsk

La oss derfor tenke oss mangekantar med så mange sider at vi godtek deira omkrins som lik omkrinsane til sirklane. Om vi da skriv omkrinsen til den største og den minste sirkelen som høvesvis $ O $ og $ o $, får vi at
\[ \frac{O}{R}=\frac{o}{r} \]
Da dei to sirklane våre er heilt vilkårleg valgt, har vi no komme fram til at \textit{alle sirklar har det same forholdet mellom omkrinsen og radiusen}. Ei enda vanlegare formulering er at \textit{alle sirklar har det same forholdet mellom omkrinsen og diameteren}. Vi let $ D $ og $ d $ vere diameteren til høvesvis sirkelen med radius $ R $ og $ r $. Da har vi at
\algv{
	\frac{O}{2R}=\frac{o}{2r} \br
	\frac{O}{D}=\frac{o}{d}
}
Forholdstalet mellom omkrinsen og diameteren i ein sirkel blir kalla $ \pi $ \index{$ \pi $}(uttalast ''pi''):
\[ \frac{O}{D}=\pi \]
Likninga over fører oss til formelen for omkrinsen til ein sirkel:
\alg{
	O&= \pi D\\
	&=2\pi r
}
Tidlegare fann vi at omkrinsen til ein sirkel med radius 1 (og diameter 2) er $ 6.283185307179586... $\,. Dette betyr at
\alg{
	\pi&= \frac{6.283185307179586...}{2} \br
	&= 3.141592653589793...
}
} 
\newpage
\fork{\ref{volforml} \volforml}{
Sjå boka \textit{Elementary geometry from an advanced standpoint}	av E. E. Moise.
} \regv

\fork{\ref{volkule} \volkule }{
\textbf{Førkunnskapar}\os
For å finne formelen for volumet til ei kule, introduserer vi tre omgrep: \textit{vertikalt tverrsnitt}, \textit{horisontalt tverrsnitt} og \textit{Cavalieris prinsipp}. Eit tverrsnitt er ei tenkt overflate som kjem til syne når ein skjær i ei tredimensjonal form. Eit vertikalt/horisontalt tverrsnitt er ei tenkt overflate som kjem til syne viss vi skjær ei tredimensjonal form enten rett vertikalt eller rett horisontalt.
\fig{tverrsnitt}
Cavalieris prinsipp lyd slik:\os

\textsl{Viss tverrsnittsareala til to tredimensjonale former er dei same langs den same høgda, har formene same volum. } \os

Dette prinsippet er illustrert i figuren under, som viser eit vertikalt tverrsnitt av to former bygd av fem prismer. Prismene i dei to formene er parvis like.
\begin{figure}
\centering
\subfloat[]{
	\includegraphics{\asym{cav}}	
} \qquad
\subfloat[]{
	\includegraphics{\asym{cavb}}	
}
\end{figure}
Det er opplagt at viss ein startar med forma vist i \textsl{(a)}, så vil ikkje volumet endre seg om ein forskyv prismene mot høgre, slik som i \textsl{(b)}.\newpage

\textbf{Volumet til ei kule}\os
Vi starter med å sjå for oss ei kule eksakt omslutta av ein  sylinder. La radiusen til kula og sylinderen vere $ r $, da er høgda til sylinderen $ 2r $.
\begin{figure}[H]
	\centering
	\includegraphics[]{\asym{volkule1}}
	\caption{\label{figkule}}
\end{figure}
Vi innfører følgande størrelsar:
\alg{
V_s &= \text{volumet til sylinderen}\\
V_k &= \text{volumet til kula}\\
V_i &= \text{volumet til forma inneklemt mellom sylinderen og kula}
}
Da har vi at
\begin{equation}
	V_k=V_s-V-i \label{volkule_Vk}
\end{equation}
Tenk no at vi skjær forma fra \hrs{figkule}{figur} frå toppen og rett ned gjennom sentrum av kula. Da får vi eit vertikalt. Ser vi på dette tverrsnittet rett horisontalt, vil sylinderen sjå ut som ein firkant, og kula som ein sirkel. 
\begin{figure}[H]
	\centering
	\includegraphics[]{\asym{volkule2}}
	\caption{}
\end{figure}
På dette tverrsnittet vandrer vi en lengde $ k $ rett opp fra sentrum til et punkt $ P $. Den halve bredden til kula i dette punktet kaller vi $ s $. Av Pytagoras' setning har vi at
\begin{equation} \label{volkule_s2}
	s^2 = r^2-k^2
\end{equation}
Videre forestiller vi oss at vi igjen skjærer formen i \hrs{figkule}{figur}, men denne gangen rett fra siden og gjennom punktet $ P $. Da får vi et horisontalt tverrsnitt. Studerer vi dette tverrsnittet rett ovenfra, får vi en figur som dette:
\begin{figure}[H]
	\centering
	\includegraphics[]{\asym{volkule3}}
	\caption{\centering Horisontalt tverrsnitt. Den svarte sirkelen er buen til sylinderen\newline og den blå er buen til kula.\label{figtverrar}}
\end{figure}
Vi definerer følgende:
\alg{
A_s &= \text{arealet til tverrsnittsoverflaten til sylinderen} \\
A_k &= \text{arealet til tverrsnittsoverflaten til kula} \\
A_i &= \text{arealet til tverrsnittsoverflaten mellom } \\
&\hspace{12pt} \text{ sylinderen og kula (grønn i figuren over)}
} 

Da er 
\begin{equation} \label{volkule_Ai}
	A_i=A_s-A_k
\end{equation}
Av \rref{arsirk} har vi at ${A_s= \pi r^2} $ og $ {A_k=\pi s^2} $ (se tilbake til\\ \hrs{figkule}{figur}). Av \eqref{volkule_s2} og \eqref{volkule_Ai} er
\begin{align*}
	A_i &= \pi r^2 - \pi s^2 \\
	&= \pi r^2 - \pi(r^2-k^2)\\
	&= \pi r^2 - \pi r^2+\pi k^2\\
	&= \pi k^2
\end{align*}
\newpage
$ A_i $ tilsvarer altså arealet til en sirkel med radius $ k $. 
\begin{figure}[H]
	\centering
	\includegraphics[]{\asym{volkule6}}
	\caption{Samme areal som det grøne arealet i \hrs{figtverrar}{figur}.}
\end{figure}
Vi tenker oss nå to kjegler, begge med høyde og radius lik $ r $, satt med spissene mot hverandre. Denne formen vil være like høy som formen fra \hrs{figkule}{figur}, og kan plasseres slik at punktet hvor spissene møtes sammenfaller med $ S $.
\begin{figure}[H]
	\centering
	\includegraphics[]{\asym{volkule4}}
	\caption{\label{kjegle}}
\end{figure}
Det vertikale tverrsnittet gjennom $ S $ av denne formen ser slik ut:
\begin{figure}[H]
	\centering
	\includegraphics[]{\asym{volkule5}}
	\caption{\label{figkjegler}}
\end{figure}
Om vi vandret $ k $ rett opp eller ned fra $ S $, så er den horisontale avstanden ut til siden også $ k $ (dette er overlatt til leseren å vise). Dette betyr at det horisontale tverrsnittsarealet til kjeglene er $ {\pi k^2=A_i} $. Altså har den inneklemte formen fra \hrs{figkule}{figur} og formen fra \hrs{figkjegler}{figur} samme tverrsnittsareal langs den samme høgden (begge har høgde $ 2r $). Av Cavalieris prinsipp og \rref{volforml} har vi da at 
\alg{
V_i &= \frac{2(\pi r^2\cdot r)}{3} \\
&= \frac{2\pi r^3}{3} \label{volkule_Vi}
}
Av \eqref{volkule_Vk}, \eqref{volkule_Vi} og \rref{volforml} har vi nå at
\begin{align*}
	V_k&=2\pi r^3- \frac{2\pi r^2}{3} \\
	&= \frac{6\pi r^3}{3}-\frac{2\pi r^3}{3} \\
	&= \frac{4\pi r^3}{3}
\end{align*}
}
\newpage
\fork{\ref{forform} \forform}{
	I figuren under er $ BB'||CC' $. Arealet til ein trekant $ \triangle ABC $ skriv vi her som $ ABC $.
	\fig{forml0}
	Med $ BB' $ som grunnlinje er $ HB' $ høgda i både $ \triangle CBB' $ og $ \triangle CBB' $. Derfor er
	\begin{equation}\label{a}
	CBB' = C'BB'
	\end{equation}
	Vidare har vi at
	\alg{
		ABB' &= AB\cdot HB' \vn
		CBB' &= BC\cdot HB'
	}
	Altså er
	\begin{equation}\label{b}
	\frac{ABB'}{CBB'}=\frac{AB}{BC}
	\end{equation}
	På liknande vis er
	\begin{equation}\label{c}
	\frac{ABB'}{C'BB'}=\frac{AB'}{B'C'}
	\end{equation}
	Av \eqref{a}, \eqref{b} og \eqref{c} følg det at
	\begin{equation}\label{form}
	\frac{AB}{BC}=\frac{ABB'}{CBB'}\frac{ABB'}{C'BB'}=\frac{AB'}{B'C'}
	\end{equation}
	For dei formlike trekantane $ \triangle ACC' $ og $\triangle ABB' $ er
	\alg{
		\frac{AC}{AB} &= \frac{AB+BC}{AB} \\[5pt]
		&= 1+\frac{BC}{AB} \\
		& \\
		\frac{AC'}{AB'}&=\frac{AB'+B'C'}{AB'}\\[5pt]
		&= 1+\frac{B'C'}{AB'}
	}
	Av \eqref{form} er dermed forholdet mellom dei samsvarande sidene likt.
}\vsk

\info{Merk}{
I dei komande forklaringane av vilkåra \textsl{ii} og \textsl{iii} fra \rref{konsttre} tek ein utgangspunkt i følgande:
	\begin{itemize}
		\item To sirklar skjær kvarandre i maksimalt to punkt.
		\item Gitt at eit koordinatsystem blir plassert med origo i senteret til den eine sirkelen, og slik at horisontalaksen går gjennom begge sirkelsentera. Viss $ (a, b) $ er det eine skjæringspunktet, er $ (a, -b) $ det andre skjæringspunktet.
	\end{itemize}
	\fig{sirk2} 
		Punkta over kan enkelt visast, men er såpass intuitivt sanne at vi tek dei for gitt. Punkta fortel oss at trekanten som består av dei to sentera og det eine skjæringspunktet er kongruent med trekanten som består av dei to sentera og det andre skjæringspunktet. Med dette kan vi studere eigenskapar til trekantar ved hjelp av halvsirklar.
} \vsk

\fork{\ref{konsttre} \konsttre}{
\textbf{Vilkår i}\os
Gitt ei lengde $ c $ og to vinklar $u $ og $ v $.
Vi lagar eit linjestykke $ AB $ med lengde $ c $. Så stiplar vi to vinkelbein slik at $ \angle {A= u} $ og $ {B=v} $. Så lenge desse vinkelbeina ikkje er parallelle, må dei naudsynleg skjære kvarandre i eitt, og berre eitt, punkt ($ C $ i figuren). I lag med $ A $ og $ B $ vil dette punktet danne ein trekant som er unikt gitt av $ c $, $ u $ og $ v $.
\fig{geo13c}	
	
\textbf{Vilkår ii}\os
Gitt tre lengder $ a $, $ b $ og $ c $. Vi lagar eit linjestykket $ AB $ med lengde $ c $. Så lagar vi to halvsirklar med høvesvis radius $ a $ og $ b $ og sentrum $ B $ og $ A $. Skal no ein trekant $ \triangle ABC $ ha sidelengder $ a $, $ b $ og $ c $, må $ C $ ligge på begge sirkelbogane. Da bogane berre kan møtast i eitt punkt, er forma og størrelsen til $ \triangle ABC $ unikt gitt av $a$, $ b $ og $ c $.
\fig{geo13a}	

\textbf{Vilkår iii} \os
Gitt to lengder $ b $ og $ c $ og ein vinkel $ u $. Vi startar med følgande:
\begin{enumerate}
	\item Vi lagar eit linjestykke $ AB $ med lengde $ c $.
	\item I $ A $ teiknar vi ein halvsirkel med radius $ b $. 
\end{enumerate}
Ved å la $ C $ vere plassert kor som helst på denne sirkelbua, har vi alle moglege variantar av ein trekant $ \triangle ABC $ med sidelengdene $ {AB=c} $ og $ {AC=b} $. Å plassere $ C $ langs bogen til halvsirkelen er det same som å gi $ \angle A $ ein bestemt verdi. Det gjenstår no å vise at kvar plassering av $ C $ gir ei unik lengde av $ BC $.
\fig{geo13b}
Vi let $ C_1 $ og $ C_2 $ vere to potensielle plasseringar av $ C $, der $ C_2 $ langs halvsirkelen ligg nærare $ E $  enn $ C_1 $. Vidare stiplar vi ein sirkelboge med radius $ BC_1 $ og sentrum i $ B $. Da den stipla sirkelbogen og halvsirkelen berre kan skjære kvarandre i $ C_1 $, vil alle andre punkt på halvsirkelen ligge enten innanfor eller utanfor den stipla sirkelbogen. Slik vi har definert $ C_2 $, må dette punktet ligge utanfor den stipla sirkelbogen, og dermed er $ BC_2 $ lengre enn $ BC_1 $. Av dette kan vi konkludere med at $ BC $ blir lengre dess nærare $ C $ beveger seg mot $ E $ langs halvsirkelen. Å sette $ {\angle A=u} $ vil altså gi ein unik verdi for $ BC $, og da ein unik trekant $ \triangle ABC $ der $ AC=b $, $ c=AB $ og $ \angle BAC=u $.
}\vsk

\fork{\ref{vilkform} \vilkform}{
\textbf{Vilkår i}\os
Gitt to trekantar $ \triangle ABC $ og $ \triangle DEF $. Av \rref{180} har vi at
\alg{
\angle A+ \angle B+\angle C=\angle D+\angle E+\angle F
}	
Viss $ \angle A=\angle D $ og $ \angle B=\angle E $, følger det at $ \angle C=\angle E $.\vsk

	
\textbf{Vilkår ii}\os
Vi tek utgangspunkt i trekantane $ \triangle ABC $ og $ \triangle DEF $ der
\begin{equation}
\dfrac{AC}{DF}=\dfrac{BC}{EF}\qquad,\qquad \angle C = \angle F\label{vilkforma}
\end{equation}
\fig{geo12}
Vi sett $ a=BC $, $ b=AC $, $ d=EF $ og $ e=DF $. Vi plasserer $ D' $ og $ E' $ på høvesvis $ AC $ og $ BC $, slik at $ D'C=e $ og $ AB\parallel D'E' $. Da er $ \triangle ABC \sim \triangle D'E'C$, altså har vi at
\alg{
\frac{E'C}{BC}&=\frac{D'C}{AC}\br
E'C&=\frac{ae}{b}
}

Av \eqref{vilkforma} har vi at
\[ EF=\frac{ae}{b} \]	
Altså er $ {E'C = EF} $. No har vi av vilkår \textsl{ii} fra \rref{kongtre} at $ \triangle D'E'C\cong\triangle DEF $. Dette betyr at $ \triangle ABC\sim \triangle DEF $.\vsk

\textbf{Vilkår iii}\os
Vi tek utgangspunkt i to trekantar $ \triangle ABC $ og $ \triangle DEF $ der
\begin{equation}
\frac{AB}{DE}=\frac{AC}{DF}=\frac{BC}{EF} \label{vilkformb}
\end{equation}
Vi plasserer $ D' $ og $ E' $ på høvesvis $ AC $ og $ BC $, slik at $ D'C=e $ og $ E'C=d $. Av vilkår \textsl{i} fra \rref{vilkform} har vi da at $ \triangle ABC\sim\triangle D'E'C $. Altså er
\alg{
\frac{D'E'}{AB}&=\frac{D'C}{AC} \br
D'E'&=\frac{ae}{c}
} 
Av \eqref{vilkformb} har vi at
\alg{
f&=\frac{ae}{c}
}
Altså har $ \triangle D'E'C $ og $ \triangle DEF $ parvis like sidelengder, og av vilkår \textsl{i} fra \rref{kongtre} er dei da kongruente. Dette betyr at $ {\triangle ABC \sim \triangle DEF}$.
\fig{geo12b}
}

\newpage


\end{document}


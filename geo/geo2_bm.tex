\documentclass[english,hidelinks,pdftex, 11 pt, class=report,crop=false]{standalone}
\usepackage[T1]{fontenc}
\usepackage[utf8]{luainputenc}
\usepackage{lmodern} % load a font with all the characters
\usepackage{geometry}
\geometry{verbose,paperwidth=16.1 cm, paperheight=24 cm, inner=2.3cm, outer=1.8 cm, bmargin=2cm, tmargin=1.8cm}
\setlength{\parindent}{0bp}
\usepackage{import}
\usepackage[subpreambles=false]{standalone}
\usepackage{amsmath}
\usepackage{amssymb}
\usepackage{esint}
\usepackage{babel}
\usepackage{tabu}
\makeatother
\makeatletter

\usepackage{titlesec}
\usepackage{ragged2e}
\RaggedRight
\raggedbottom
\frenchspacing

% Norwegian names of figures, chapters, parts and content
\addto\captionsenglish{\renewcommand{\figurename}{Figur}}
\makeatletter
\addto\captionsenglish{\renewcommand{\chaptername}{Kapittel}}
\addto\captionsenglish{\renewcommand{\partname}{Del}}

\addto\captionsenglish{\renewcommand{\contentsname}{Innhald}}

\usepackage{graphicx}
\usepackage{float}
\usepackage{subfig}
\usepackage{placeins}
\usepackage{cancel}
\usepackage{framed}
\usepackage{wrapfig}
\usepackage[subfigure]{tocloft}
\usepackage[font=footnotesize,labelfont=sl]{caption} % Figure caption
\usepackage{bm}
\usepackage[dvipsnames, table]{xcolor}
\definecolor{shadecolor}{rgb}{0.105469, 0.613281, 1}
\colorlet{shadecolor}{Emerald!15} 
\usepackage{icomma}
\makeatother
\usepackage[many]{tcolorbox}
\usepackage{multicol}
\usepackage{stackengine}

% For tabular
\usepackage{array}
\usepackage{multirow}
\usepackage{longtable} %breakable table

% Ligningsreferanser
\usepackage{mathtools}
\mathtoolsset{showonlyrefs}

% index
\usepackage{imakeidx}
\makeindex[title=Indeks]

%Footnote:
\usepackage[bottom, hang, flushmargin]{footmisc}
\usepackage{perpage} 
\MakePerPage{footnote}
\addtolength{\footnotesep}{2mm}
\renewcommand{\thefootnote}{\arabic{footnote}}
\renewcommand\footnoterule{\rule{\linewidth}{0.4pt}}
\renewcommand{\thempfootnote}{\arabic{mpfootnote}}

%colors
\definecolor{c1}{cmyk}{0,0.5,1,0}
\definecolor{c2}{cmyk}{1,0.25,1,0}
\definecolor{n3}{cmyk}{1,0.,1,0}
\definecolor{neg}{cmyk}{1,0.,0.,0}

% Lister med bokstavar
\usepackage[inline]{enumitem}

\newcounter{rg}
\numberwithin{rg}{chapter}
\newcommand{\reg}[2][]{\begin{tcolorbox}[boxrule=0.3 mm,arc=0mm,colback=blue!3] {\refstepcounter{rg}\phantomsection \large \textbf{\therg \;#1} \vspace{5 pt}}\newline #2  \end{tcolorbox}\vspace{-5pt}}

\newcommand\alg[1]{\begin{align} #1 \end{align}}

\newcommand\eks[2][]{\begin{tcolorbox}[boxrule=0.3 mm,arc=0mm,enhanced jigsaw,breakable,colback=green!3] {\large \textbf{Eksempel #1} \vspace{5 pt}\\} #2 \end{tcolorbox}\vspace{-5pt} }

\newcommand{\st}[1]{\begin{tcolorbox}[boxrule=0.0 mm,arc=0mm,enhanced jigsaw,breakable,colback=yellow!12]{ #1} \end{tcolorbox}}

\newcommand{\spr}[1]{\begin{tcolorbox}[boxrule=0.3 mm,arc=0mm,enhanced jigsaw,breakable,colback=yellow!7] {\large \textbf{Språkboksen} \vspace{5 pt}\\} #1 \end{tcolorbox}\vspace{-5pt} }

\newcommand{\sym}[1]{\colorbox{blue!15}{#1}}

\newcommand{\info}[2]{\begin{tcolorbox}[boxrule=0.3 mm,arc=0mm,enhanced jigsaw,breakable,colback=cyan!6] {\large \textbf{#1} \vspace{5 pt}\\} #2 \end{tcolorbox}\vspace{-5pt} }

\newcommand\algv[1]{\vspace{-11 pt}\begin{align*} #1 \end{align*}}

\newcommand{\regv}{\vspace{5pt}}
\newcommand{\mer}{\textsl{Merk}: }
\newcommand\vsk{\vspace{11pt}}
\newcommand\vs{\vspace{-11pt}}
\newcommand\vsb{\vspace{-16pt}}
\newcommand\sv{\vsk \textbf{Svar:} \vspace{4 pt}\\}
\newcommand\br{\\[5 pt]}
\newcommand{\asym}[1]{../fig/#1}
\newcommand\algvv[1]{\vs\vs\begin{align*} #1 \end{align*}}
\newcommand{\y}[1]{$ {#1} $}
\newcommand{\os}{\\[5 pt]}
\newcommand{\prbxl}[2]{
\parbox[l][][l]{#1\linewidth}{#2
	}}
\newcommand{\prbxr}[2]{\parbox[r][][l]{#1\linewidth}{
		\setlength{\abovedisplayskip}{5pt}
		\setlength{\belowdisplayskip}{5pt}	
		\setlength{\abovedisplayshortskip}{0pt}
		\setlength{\belowdisplayshortskip}{0pt} 
		\begin{shaded}
			\footnotesize	#2 \end{shaded}}}

\renewcommand{\cfttoctitlefont}{\Large\bfseries}
\setlength{\cftaftertoctitleskip}{0 pt}
\setlength{\cftbeforetoctitleskip}{0 pt}

\newcommand{\bs}{\\[3pt]}
\newcommand{\vn}{\\[6pt]}
\newcommand{\fig}[1]{\begin{figure}
		\centering
		\includegraphics[]{\asym{#1}}
\end{figure}}


\newcommand{\sectionbreak}{\clearpage} % New page on each section

% Equation comments
\newcommand{\cm}[1]{\llap{\color{blue} #1}}

\newcommand\fork[2]{\begin{tcolorbox}[boxrule=0.3 mm,arc=0mm,enhanced jigsaw,breakable,colback=yellow!7] {\large \textbf{#1 (forklaring)} \vspace{5 pt}\\} #2 \end{tcolorbox}\vspace{-5pt} }




%colors
\newcommand{\colr}[1]{{\color{red} #1}}
\newcommand{\colb}[1]{{\color{blue} #1}}
\newcommand{\colo}[1]{{\color{orange} #1}}
\newcommand{\colc}[1]{{\color{cyan} #1}}
\definecolor{projectgreen}{cmyk}{100,0,100,0}
\newcommand{\colg}[1]{{\color{projectgreen} #1}}

%%% SECTION HEADLINES %%%

% Our numbers
\newcommand{\likteikn}{Likskapsteiknet}
\newcommand{\talsifverd}{Tal, siffer og verdi}
\newcommand{\koordsys}{Koordinatsystem}

% Calculations
\newcommand{\adi}{Addisjon}
\newcommand{\sub}{Subtraksjon}
\newcommand{\gong}{Multiplikasjon (Gonging)}
\newcommand{\del}{Divisjon (deling)}

%Factorization and order of operations
\newcommand{\fak}{Faktorisering}
\newcommand{\rrek}{Reknerekkefølge}

%Fractions
\newcommand{\brgrpr}{Introduksjon}
\newcommand{\brvu}{Verdi, utviding og forkorting av brøk}
\newcommand{\bradsub}{Addisjon og subtraksjon}
\newcommand{\brgngheil}{Brøk gonga med heiltal}
\newcommand{\brdelheil}{Brøk delt med heiltal}
\newcommand{\brgngbr}{Brøk gonga med brøk}
\newcommand{\brkans}{Kansellering av faktorar}
\newcommand{\brdelmbr}{Deling med brøk}
\newcommand{\Rasjtal}{Rasjonale tal}

%Negative numbers
\newcommand{\negintro}{Introduksjon}
\newcommand{\negrekn}{Dei fire rekneartane med negative tal}
\newcommand{\negmeng}{Negative tal som mengde}

% Geometry 1
\newcommand{\omgr}{Omgrep}
\newcommand{\eignsk}{Eigenskapar for trekantar og firkantar}
\newcommand{\omkr}{Omkrins}
\newcommand{\area}{Areal}

%Algebra 
\newcommand{\algintro}{Introduksjon}
\newcommand{\pot}{Potensar}
\newcommand{\irrasj}{Irrasjonale tal}

%Equations
\newcommand{\ligintro}{Introduksjon}
\newcommand{\liglos}{Løysing ved dei fire rekneartane}
\newcommand{\ligloso}{Løysingsmetodane oppsummert}

%Functions
\newcommand{\fintro}{Introduksjon}
\newcommand{\lingraf}{Lineære funksjonar og grafar}

%Geometry 2
\newcommand{\geoform}{Formlar for areal og omkrins}
\newcommand{\kongogsim}{Kongruente og formlike trekantar}
\newcommand{\geofork}{Forklaringar}

% Names of rules
\newcommand{\gangdestihundre}{Å gange desimaltall med 10, 100 osv.}
\newcommand{\delmedtihundre}{Deling med 10, 100, 1\,000 osv.}
\newcommand{\ompref}{Omgjøring av prefikser}
\newcommand{\adkom}{Addisjon er kommutativ}
\newcommand{\gangkom}{Multiplikasjon er kommutativ}
\newcommand{\brdef}{Brøk som omskriving av delestykke}
\newcommand{\brtbr}{Brøk gonga med brøk}
\newcommand{\delmbr}{Brøk delt på brøk}
\newcommand{\gangpar}{Gonging med parentes (distributiv lov)}
\newcommand{\gangparsam}{Parantesar gonga saman}
\newcommand{\gangmnegto}{Gonging med negative tal I}
\newcommand{\gangmnegtre}{Gonging med negative tal II}
\newcommand{\konsttre}{Konstruksjon av trekantar}
\newcommand{\kongtre}{Kongruente trekantar}
\newcommand{\topv}{Toppvinklar}
\newcommand{\trisum}{Summen av vinklane i ein trekant}
\newcommand{\firsum}{Summen av vinklane i ein firkant}
\newcommand{\potgang}{Gonging med potensar}
\newcommand{\potdivpot}{Divisjon med potensar}
\newcommand{\potanull}{Spesialtilfellet \boldmath $a^0$}
\newcommand{\potneg}{Potens med negativ eksponent}
\newcommand{\potbr}{Brøk som grunntal}
\newcommand{\faktgr}{Faktorar som grunntal}
\newcommand{\potsomgrunn}{Potens som grunntal}
\newcommand{\arsirk}{Arealet til ein sirkel}
\newcommand{\artrap}{Arealet til eit trapes}
\newcommand{\arpar}{Arealet til eit parallellogram}
\newcommand{\pyt}{Pytagoras' setning}
\newcommand{\forform}{Forhold i formlike trekantar}
\newcommand{\vilkform}{Vilkår i formlike trekantar}
\newcommand{\omkrsirk}{Omkrinsen til ein sirkel (og $ \bm \pi $)}
\newcommand{\artri}{Arealet til ein trekant}
\newcommand{\arrekt}{Arealet til eit rektangel}
\newcommand{\liknflyt}{Flytting av ledd over likskapsteiknet}
\newcommand{\funklin}{Lineære funksjonar}

%Opg
\newcommand{\abc}[1]{
	\begin{enumerate}[label=\alph*),leftmargin=18pt]
		#1
	\end{enumerate}
}
\newcommand{\abcs}[2]{
	\begin{enumerate}[label=\alph*),start=#1,leftmargin=18pt]
		#2
	\end{enumerate}
}
\newcommand{\abcn}[1]{
	\begin{enumerate}[label=\arabic*),leftmargin=18pt]
		#1
	\end{enumerate}
}
\newcommand{\abch}[1]{
	\hspace{-2pt}	\begin{enumerate*}[label=\alph*), itemjoin=\hspace{1cm}]
		#1
	\end{enumerate*}
}
\newcommand{\abchs}[2]{
	\hspace{-2pt}	\begin{enumerate*}[label=\alph*), itemjoin=\hspace{1cm}, start=#1]
		#2
	\end{enumerate*}
}

\newcommand{\opgt}{\phantomsection \addcontentsline{toc}{section}{Oppgaver} \section*{Oppgaver for kapittel \thechapter}\vs \setcounter{section}{1}}
\newcounter{opg}
\numberwithin{opg}{section}
\newcommand{\op}[1]{\vspace{15pt} \refstepcounter{opg}\large \textbf{\color{blue}\theopg} \vspace{2 pt} \label{#1} \\}
\newcommand{\ekspop}[1]{\vsk\textbf{Gruble \thechapter.#1}\vspace{2 pt} \\}
\newcommand{\nes}{\stepcounter{section}
	\setcounter{opg}{0}}
\newcommand{\opr}[1]{\vspace{3pt}\textbf{\ref{#1}}}
\newcommand{\oeks}[1]{\begin{tcolorbox}[boxrule=0.3 mm,arc=0mm,colback=white]
		\textit{Eksempel: } #1	  
\end{tcolorbox}}
\newcommand\opgeks[2][]{\begin{tcolorbox}[boxrule=0.1 mm,arc=0mm,enhanced jigsaw,breakable,colback=white] {\footnotesize \textbf{Eksempel #1} \\} \footnotesize #2 \end{tcolorbox}\vspace{-5pt} }

%License
\newcommand{\lic}{\textit{Matematikken sine byggesteinar by Sindre Sogge Heggen is licensed under CC BY-NC-SA 4.0. To view a copy of this license, visit\\ 
		\net{http://creativecommons.org/licenses/by-nc-sa/4.0/}{http://creativecommons.org/licenses/by-nc-sa/4.0/}}}

%referances
\newcommand{\net}[2]{{\color{blue}\href{#1}{#2}}}
\newcommand{\hrs}[2]{\hyperref[#1]{\color{blue}\textsl{#2 \ref*{#1}}}}
\newcommand{\rref}[1]{\hrs{#1}{Regel}}
\newcommand{\refkap}[1]{\hrs{#1}{Kapittel}}
\newcommand{\refsec}[1]{\hrs{#1}{Seksjon}}

\usepackage{datetime2}

\usepackage[]{hyperref}


\begin{document}
\section{\geoform}

En \textit{formel}\index{formel} er en likning der en variabel (som oftest) står alene på én side av likhetstegnet. I \hrs{Areal}{seksjon} har vi allerede sett på formler for arealet til rektangel og trekantar, men der brukte vi ord i steden for symboler. Her skal vi gjengi desse to formlene i en mer algebraisk form, etterfulgt av andre klassiske formler for areal og omkrets.\regv

\reg[\arrekt\; (\ref{arrekt})]{ \index{areal!til rektangel}
Arealet $ A $ til et rektangel med grunnlinje $ g $ og høgde $ h $ er
	\[A = g h \]
	\fig{tri12_alg}
}
\eks[1]{
	Finn arealet til rektangelet.
	\fig{tri12ba} \vsb
	\sv
	Arealet $ A $ til rektangelet er 
	\[ A =b\cdot 2 =2b \]	
}
\eks[2]{
	Finn arealet til kvadratet.
	\fig{tri12ca} \vsb
	\sv
	Arealet $ A $ til kvadratet er 
	\[ A =a\cdot a =a^2 \]	
}

\reg[\artri\;(\ref{artri})]{ \index{til trekant}
	Arealet $ A $ til en trekant med grunnlinje $ g $ og høyde $ h $ er
	\[ A = \frac{g h}{2} \]
\fig{triar0}
}
\eks{
Hvilken av trekantane har størst areal? \vs
\begin{figure}
	\centering
	\subfloat{\includegraphics[]{\asym{triar0b}}}\qquad
	\subfloat{\includegraphics[]{\asym{triar0a}}}
	\qquad
	\subfloat{\includegraphics[]{\asym{triar0c}}}
\end{figure}
\sv

Vi lar $ A_1 $, $ A_2 $ og $ A_3 $ være arealene til henholdsvis trekanten til venstre, i midten og til høyre. Da har vi at
\alg{
A_1 &=\frac{4\cdot3}{2}=6 \vn
A_2 &=\frac{2\cdot 3}{2}=3\vn
A_3&=\frac{2\cdot5}{2}=5
}	
Altså er det trekanten til venstre som har størst areal.
}

\reg[\arpar \label{arpar}]{\index{til parallellogram}
	Arealet $ A $ til eit parallellogram med grunnlinje $ g $ og høgde $ h $ er
	\[ A = g h  \]
	\fig{tri21}
}
\eks{
Finn arealet til parallellogrammet
\fig{tri21b} \vsb
\sv
Arealet $ A $ til parallellogrammet er
\alg{
A&=5\cdot 2=10
}		
}
\fork{\ref{arpar} \arpar}{
	Av et parallellogram kan vi alltid lage oss to trekantar ved å tegne inn en av diagonalene:
	\fig{tri21c}
	De fargede trekantene på figuren over har begge grunnlinje $ g $ og høgde $ h $. Da vet vi at begge har areal lik $ \frac{g h}{2} $.
	Arealet $ A $ til parallelogrammet blir dermed
	\alg{
		A&=\frac{g h}{2}+\frac{g h}{2} \\
		&=g\cdot h	
	}	
}
\reg[\artrap \label{artrap}]{\index{areal!til trapes}
	Arealet $ A $ til et trapes med parallelle sider $ a $ og $ b $ og høgde $ h $ er
	\[ A = \frac{h(a+b)}{2}\]
	\fig{tri21a}
}
\eks{
Finn arealet til trapeset.
\fig{tri21e} \vsb
\sv 
Arealet $ A $ til trapeset er
\alg{
A&=\frac{3(6+4)}{2}\br
&=\frac{3\cdot10}{2}\br
&=15
}	
}
\info{Merk}{
Når man tar utgangspunkt i ei grunnlinje og ei høgde, er arealformlene for et parallellogram og et rektangel identiske. Å anvende \rref{artrap} på et parallellogram vil også resultere i et uttrykk tilsvarande $ gh $. Dette er fordi et parallellogram bare er et spesialtilfelle av et trapes (og et rektangel er bare et spesialtilfelle av et parallellogram).
}
\newpage
\fork{\ref{artrap} \artrap}{
	Også for et trapes får vi to trekanter viss vi tegner en av diagonalene:
	\fig{tri21d}
	I figuren over er
	\alg{
		\text{Arealet til den blå trekantet}&= \frac{a h}{2} \br
		\text{Arealet til den grønne trekanten}&= \frac{b  h}{2}
	}
	Arealet $ A $ til trapeset blir dermed
	\alg{
		A&=\frac{ah}{2}+\frac{bh}{2}\br &=\frac{h(a+b)}{2} 	
	}	
}
\newpage
\reg[\omkrsirk \label{omkrsirk}]{\index{omkrins!til sirkel}
	Omkretsen $ O $ til en sirkel med radius $ r $ er
	\[ O = 2\pi r \]
	\fig{tri22}
	$ \pi = 3.141592653589793...
	$.
}
\eks[1]{
	Finn omkretsen til sirkelen.
	\fig{tri22b}
	\sv 
	Omkretsen $ O $ er 
	\algv{
		O &= 2 \pi \cdot 3 \\
		&= 6\pi 
	}
} \vsk

\reg[\arsirk \label{arsirk}]{ \index{areal!til sirkel}
	Arealet $ A $ til en sirkel med radius $ r $ er
	\[ A = \pi r^2 \]
	\fig{tri22a}
}
\eks{
	Finn arealet til sirkelen.
	\fig{tri22c} \vsb
	\sv
	Arealet $ A $ til sirkelen er 
	\[ A=\pi\cdot 5^2 =25\pi \]
}
\newpage
\fork{\ref{arsirk} \arsirk}{
	I figuren under har vi delt opp en sirkel i 4, 10 og 50 (like store) biter, og lagt disse bitene etter hverandre.
	\fig{tri19}
	\fig{tri19a}
	\fig{tri19b}
	I hvert tilfelle må de små buelengdene til sammen utgjøre hele buelengden, altså omkretsen, til sirkelen. Hvis sirkelen har radius $ r $, betyr dette at summen av buelengdene er $ 2\pi r $. Og når vi har like mange biter med buen vendt opp som biter med buen vendt ned, må lengden være $ \pi r $ både oppe og nede. \vsk
	
	Men jo flere biter vi deler sirkelen inn i, jo mer ligner sammensetningen av bitene på et rektangel (i figuren under har vi 100 biter). Grunnlinja $ g $ til dette ''rektangelet'' vil være tilnærmet lik $ \pi r $, mens høgda vil være tilnærmet lik $ r $.
	\fig{tri19c}
	Arealet  $ A $ til ''rektangelet'', altså sirkelen, blir da
	\[ A\approx g h \approx \pi r\cdot r = \pi r^2 \]
}
\reg[\pyt \label{pyt}]{
	I en rettvinklet trekant er arealet til kvadratet dannet av \\hypotenusen lik summen av arealene til kvadratene dannet av\\ katetene.\regv
	
	\parbox[l][][l]{0.7\linewidth}{\[ a^2+b^2=c^2 \]
	}
	\parbox[r]{0.2\linewidth}{\includegraphics[]{\asym{tri26g}}} \vs
	\fig{tri26j}
}
\eks[1]{
	Finn lengden til $ c $.
	\fig{tri26h} \vs \vs
	\sv
	Vi vet at
	\[ c^2=a^2+b^2 \]
	der $ a $ og $ b $ er lengdene til de korteste sidene i trekanten. Dermed er
	\algv{
		c^2 &= 4^2 + 3^2 \\
		&= 16+9 \\
		&=25
	}
	Da $ \sqrt{25}=5 $, må lengden til $ c $ vere 5.
}
\newpage
\fork{\ref{pyt} \pyt}{ \label{pytforklaringintro}
	Under har vi tegnet to kvadrat som er like store, men som er inndelt i forskjellige former.
	\fig{tri26d}
	Vi observerer nå følgende:
	\begin{enumerate}
		\item Arealet til det røde kvadratet er $ a^2 $, arealet til det lilla kvadratet er $ b^2 $ og arealet til det blå kvadratet er $ c^2 $.
		\item Arealet til et oransje rektangel er $ ab $ og arealet til en grønn trekant er $ \frac{ab}{2} $.
		\item Om vi tar bort de to oransje rektanglene og de fire grønne trekantane, er det igjen (av pkt. 2) et like stort areal til venstre som til høyre.
		\fig{tri26e}
		Dette betyr at
		\begin{equation}\label{pytforkl}
		a^2+b^2=c^2
		\end{equation}
	\end{enumerate}
	Gitt en trekant med sidelengder $ a, b $ og $ c $, der $ c $ er den lengste sidelengden.
	Så lenge trekanten er rettvinklet, kan vi alltid lage to kvadrat med sidelengder $ {a+b} $, slik som i første figur. \eqref{pytforkl} gjelder dermed for alle rettvinklede trekanter.
}
\newpage
	
\section{\kongogsim}
\reg[\konsttre \label{konsttre}]{
En trekant $ \triangle ABC $, som vist i figuren under, kan bli unikt konstruert hvis en av følgende kriterium er oppfylt:
\prbxl{0.6}{\begin{enumerate}[label=\roman*)]
		\item $ c $, $ \angle A $ og $ \angle B $ er kjente.
		\item $ a $, $ b $ og $ c $ er kjente.
		\item $ b $, $ c $ og $ \angle A $ er kjente.
\end{enumerate}}
\parbox[r][][l]{0.3\linewidth}{
\fig{geo13}
}
}
\reg[\kongtre \label{kongtre}]{ \index{trekant!kongruet}
To trekanter som har samme form og størrelse er kongruente.
\fig{geo14}
At trekantane i figuren over er kongruente skrives
\[ \triangle ABC\cong\triangle DEF \]
}
\reg[Formlike trekantar \index{trekant!formlik}]{Formlike trekanter har tre vinkler som er parvis like store.
\fig{geo8}
At trekantane i figuren over er formlike skrives \[ \triangle ABC\sim \triangle DEF \]
} \vsk
\newpage
\textbf{Samsvarende sider}\os
Når vi studerer formlike trekantar er \textit{samsvarende sider}\index{side!samsvarende} et viktig begrep. Samsvarende sider er sider som i formlike trekantar står \textit{motstående} den samme vinkelen.
\fig{tri1}
For de formlike trekantane $ \triangle ABC $ og $ \triangle DEF $ har vi at
\begin{multicols}{2}
	\quad I $ \triangle ABC $ er
	\begin{itemize}
		\item $ BC $ motstående til $u$.
		\item $ AC $ motstående til $ v$
		\item $ AB$ motstående til $ w $.
	\end{itemize}
	\quad I $ \triangle DEF $ er
	
	\begin{itemize}
		\item $ FE $ motstående til $u$.
		\item $ FD $ motstående til $ v$
		\item $ ED$ motstående til $ w $.
	\end{itemize}
\end{multicols}
Dette betyr at disse er samsvarende sider:
\begin{itemize}
	\item $ BC $ og $ FE $\\
	\item $ AC $ og $ FD $ \\
	\item $ AB $ og $ ED $
\end{itemize}

\reg[\forform \label{forform}]{
	Når to trekantar er formlike, er forholdet mellom samsvarende\footnote{Vi tar det her for gitt at hvilke sider som er samsvarande kommer fram av figuren.} sider det samme.
	\[ \frac{AB}{DE}=\frac{AC}{DF}=\frac{BC}{EF} \]
\fig{geo8b}
}
\eks[]{
	Trekantene i figuren under er formlike. Finn lengden til $EF $.
	\begin{figure}
		\centering
		\includegraphics[scale=1]{\asym{tri3a}}\quad
		\includegraphics[scale=1]{\asym{tri3b}}
	\end{figure}
	\sv
	Vi observerer at $ AB $ samsvarer med $ DE $, $ BC $ med $ EF $ og $ AC $ med $ DF $. Det betyr at
	\alg{
		\frac{DE}{AB} &= \frac{EF}{BC} \br
		\frac{10}{5}&= \frac{EF}{3} \br
		2\cdot3&=\frac{EF}{\cancel{3}}\cdot\cancel{3}\\
		6 &= EF
	}
}
\info{Merk}{
Av \hrs{forform}{Regel} har vi at for to formlike trekanter $ \triangle ABC $ og $ \triangle DEF $ er
	\[ \frac{AB}{BC}=\frac{DE}{EF}\quad,\quad \frac{AB}{AC}=\frac{DE}{DF}\quad,\quad\frac{BC}{AC}=\frac{EF}{DF} \]
}
\reg[\vilkform \label{vilkform}]{
To trekanter $ \triangle ABC $ og $ \triangle DEF $ er formlike hvis en av disse vilkårene er oppfylt:
\begin{enumerate}[label=\roman*)]
	\item To vinkler i trekantane er parvis like store.
	\item $ \displaystyle \frac{AB}{DE}=\frac{AC}{DF}=\frac{BC}{EF} $
	\item $ \dfrac{AB}{DE}=\dfrac{AC}{DF} $ og $ \angle A=\angle D $.
\end{enumerate}

\fig{geo8b}
}
\eks[1]{
$ \angle ACB =90^\circ $. 
Vis at $ \triangle ABC \sim ACD $.
\fig{geo15} \vs
\sv
$ \triangle ABC $ og $ \triangle ACD $ er begge rettvinklede og de har $ \angle DAC $ felles. Dermed er vilkår \textsl{i} fra \rref{vilkform} oppfylt, og trekantene er da formlike.\vsk

\mer På en tilsvarende måte kan det vises at $ \triangle ABC \sim CBD$.
}
\newpage
\eks[2]{
Undersøk om trekantane er formlike.
\fig{geo15a} \vs
\sv
Vi har at
\alg{
\frac{AC}{FD}=\frac{18}{12}=\frac{3}{2}\quad,\quad\frac{BC}{FE}=\frac{9}{6}=\frac{3}{2}\quad,\quad\frac{AB}{DE}=\frac{12}{10}=\frac{6}{5}
}
\alg{
\frac{AC}{IG}=\frac{18}{12}=\frac{3}{2}\quad,\quad\frac{BC}{IH}=\frac{9}{6}=\frac{3}{2}\quad,\quad \frac{AC}{IG}=\frac{18}{12}=\frac{3}{2}
}
Dermed oppfyller $ \triangle ABC $ og $ \triangle GHI $ vilkår \textsl{ii} fra \rref{vilkform}, og trekantene er da formlike.	
}
\eks[3]{
Undersøk om trekantane er formlike.	\vs
\fig{geo15b}
\sv
Vi har at $ {\angle BAC=\angle EDF} $ og at
\[ \frac{ED}{AB}=\frac{8}{4}=2\quad,\quad \frac{FD}{AC}=\frac{14}{7}=2 \]
Altså er vilkår \textsl{iii} fra \rref{vilkform} oppfylt, og da er trekantene formlike.
}

\newpage
\section{\geofork}
\fork{\ref{omkrsirk} \omkrsirk}
{
\textit{Vi skal her bruke \textit{regulære} mangekanter langs veien til ønsket resultat. I regulære mangekanter har alle sidene lik lengde. Da det er utelukkande regulære mangekanter vi kommer til å bruke, vil de bli omtalt bare som mangekanter.}\vsk

Vi skal starte med se på tilnærminger for å finne omkretsen $ O_1 $ av en sirkel med radius 1. 
\fig{geo9l} \vsk

\textbf{Øvre og nedre grense}\os
En god vane når man skal prøve å finne en størrelse, er å spørre seg om man kan vite noe om hvor stor eller liten man \textsl{forventer} at den er. Vi starter derfor med å omslutte sirkelen med et kvadrat med sidelengder 2:
\fig{geo9c}
Omkretsen til sirkelen må vere mindre enn omkretsen til kvadratet, derfor vet vi at
\alg{
	O_1&<2\cdot4  \\
	&< 8
}
Videre innskriver vi en sekskant. Sekskanten kan deles inn i 6 likesidede trekantar som alle må ha sidelengder 1. Omkretsen til sirkelen må være større enn omkretsen til sekskanten, noe som gir at
\alg{
	O_1&>6\cdot1 \\
	&> 6
}
\begin{figure}
	\centering
	\subfloat{\includegraphics[]{\asym{geo9}}}\qquad
	\subfloat{\includegraphics[]{\asym{geo9d2}}}
\end{figure}
Når vi nå skal gå over til en mye mer nøyaktig jakt etter omkretsen, vet vi altså at vi søker en verdi mellom 6 og 8.\vsk

\textbf{Stadig bedre tilnærminger}\os
Vi fortsetter med tanken om å innskrive en mangekant. Av figurane under lar vi oss overbevise om at dess flere sider mangekanten har, dess bedre estimat vil omkretsen til mangekanten være for omkretsen til sirkelen.
\begin{figure}
	\centering
	\subfloat[6-kant]{\includegraphics[]{\asym{geo9}}}\qquad\qquad
	\subfloat[12-kant]{\includegraphics[]{\asym{geo9a}}}	
\end{figure}
Da vi vet at sidelengden til en 6-kant er 1, er det fristende å undersøke om vi kan bruke denne kunnskapen til å finne sidelengden til andre mangekanter. Om vi innskriver også en 12-kant i sirkelen vår (og i tillegg tegner en trekant) får vi en figur som denne:
\begin{figure}
	\centering
	\subfloat[En 6-kant og en 12-kant i lag med en trekant dannet av sentrum i sirkelen og en av sidene i 12-kanten.]{\includegraphics[]{\asym{geo9g}}}\qquad\qquad
	\subfloat[Utklipp av trekant fra figur \textsl{(a)}.]{\includegraphics[]{\asym{geo9h}}}	
\end{figure}
La oss kalle sidelengden til 12-kanten for $ s_{12} $ og sidelengden til 6-kanten for $ s_6 $. Videre legger vi merke til at punktene $ A $ og $ C $ ligger på sirkelbuen og at både $ \triangle ABC $ og $ \triangle BSC $ er rettvinklede trekantar (forklar for deg selv hvorfor!). Vi har at
\alg{
	SC &= 1 \\
	BC &= \frac{s_6}{2} \\
	SB &= \sqrt{SC^2-BC^2} \\
	BA &= 1-SB \\
	AC &= s_{12}\\
	s_{12}^2 &= BA^2+BC^2
}
For å finne $ s_{12} $ må vi finne $ BA $, og for å finne $ BA $ må vi finne $ SB $. Vi starter derfor med å finne $ SB $. Da ${ SC=1} $ og $ {BC=\frac{s_6}{2}} $, er
\alg{
	SB &=\sqrt{1-\left( \frac{s_6}{2}\right)^2} \\
	&= \sqrt{1-\frac{s_6^2}{4}}
}
Vi går så videre til å finne $ s_{12} $:
\alg{
	s_{12}^2 &= \left(1-SB\right)^2 + \left(\frac{s_6}{2}\right)^2 \\
	&= 1^2 - 2SB + SB^2 + \frac{s_6^2}{4}
}
Ved første øyekast ser det ut som vi ikke kan komme særlig lengre i å forenkle uttrykket på høyre side, men en liten operasjon vil endre på dette. Hadde vi bare hatt $ -1 $ som et ledd kunne vi slått saman $ -1 $ og $ \frac{s_6^2}{4} $ til å bli $ -SB^2 $. Derfor ''skaffer'' vi oss $ -1 $ ved å både addere og subtrahere 1 på høgresiden:
\alg{
	s_{12}^2&= 1 - 2SB + SB^2 + \frac{s_6^2}{4}-1+1\\
	&= 2-2SB+SB^2-\left(1-\frac{s_6^2}{4}\right) \\
	&= 2-2SB+SB^2-SB^2\\
	&= 2-2SB\\
	&= 2-2\sqrt{1-\frac{s_6^2}{4}} \\
	&= 2-\sqrt{4}\,\sqrt{1-\frac{s_6^2}{4}} \\
	&= 2- \sqrt{4-s_6^2}
}
Altså er
\[ s_{12} = \sqrt{2- \sqrt{4-s_6^2}} \]
Selv om vi her har utledet relasjonen mellom sidelengdene $ s_{12} $ og $ s_6 $, er dette en relasjon vi kunne vist for alle par av sidelengder der den ene er sidelengden til en mangekant med dobbelt så mange sider som den andre. La $ s_n $ og $ s_{2n} $ respektivt være sidelengden til en mangekant og en mangekant med dobbelt så mange sider. Da er
\begin{align}
s_{2n} = \sqrt{2- \sqrt{4-s_n^2}} \label{s2n}
\end{align}

Når vi kjenner sidelengden til en innskrevet mangekant, vil tilnærmingen til omkretsen til sirkelen være denne sidelengden ganget med antall sidelengder i mangekanten. Ved hjelp av \eqref{s2n} kan vi stadig finne sidelengden til en mangekant med dobbelt så mange sider som den forrige, og i tabellen under har vi funnet sidelengden og tilnærmingen til omkretsen til sirkelen opp til ein 96-kant:

\begin{center}
	\renewcommand{\arraystretch}{1.5}
	\begin{tabular}{l|l|l}
		\textit{Formel for sidelengde}&\textit{Sidelengde} & \textit{Tilnærming for omkrets} \\
		\hline
		& $s_6= 1 $ & $ \;\,6\cdot s_6\;\,=6 $ \\
		$ s_{12} = \sqrt{2- \sqrt{4-s_6^2}} $ & $ s_{12}=0.517... $ & 		 $ 12\cdot s_{12}=6.211... $ \\
		$ s_{24} = \sqrt{2- \sqrt{4-s_{12}^2}} $ & $ s_{24}=0.261... $ & 		 $ 24\cdot s_{24}=6.265... $ \\
		$ s_{48} = \sqrt{2- \sqrt{4-s_{24}^2}} $ & $ s_{48}=0.130... $ & 		 $ 48\cdot s_{48}=6.278... $ \\		 
		$ s_{96} = \sqrt{2- \sqrt{4-s_{48}^2}} $ & $ s_{96}=0.065... $ & 		 $ 96\cdot s_{96}=6.282... $ \\		 		 
	\end{tabular}
\end{center}
\begin{figure}
	\centering
	\subfloat[6-kant]{\includegraphics[scale=0.75]{\asym{geo9}}}\quad
	\subfloat[12-kant]{\includegraphics[scale=0.75]{\asym{geo9a}}}\quad	
	\subfloat[24-kant]{\includegraphics[scale=0.75]{\asym{geo9i}}}\quad	\\
	\subfloat[48-kant]{\includegraphics[scale=0.75]{\asym{geo9j}}}\quad	
	\subfloat[96-kant]{\includegraphics[scale=0.75]{\asym{geo9k}}}		
\end{figure}
Utregningene over er faktisk like langt som matematikeren \net{https://no.wikipedia.org/wiki/Arkimedes}{Arkimedes} kom allerede ca 250 f. kr!\vsk

For en datamaskin er det ingen problem å regne ut\footnote{For den datainteresserte skal det sies at iterasjonsalgoritmen må skrives om for å unngå instabiliteter i utregningene når antall sider blir mange.} dette for en mangekant med ekstremt mange sider. Regner vi oss frem til en 201\,326\,592-kant  finner vi at
\[ \text{Omkrins av sirkel med radius 1}=6.283185307179586... \]
(Ved hjelp av mer avansert matematikk kan det vises at omkretsen til en sirkel med radius 1 er et irrasjonalt tal, men at alle desimalane vist over er korrekte, derav likhetstegnet.) \vsk
\newpage
\textbf{Den endelige formelen og $\bm \pi $} \os
Vi skal nå komme fram til den kjente formelen for omkretsen til en sirkel. Også her skal vi ta for gitt at summen av sidelengdene til en innskrevet mangekant er en tilnærming til omkrinsen som blir bedre og bedre dess flere sidelengder det er.\vsk

For enkelhets skyld skal vi bruke innskrevne firkantar for å få fram poenget vårt. Vi tegner to sirkler som er vilkårleg store, men der den ene er større enn den andre, og innskriver en firkant (eit kvadrat) i begge. Vi lar $ R $ og $ r $ være radien til henholdsvis den største og den minste sirkelen, og $ K $ og $ k $ vere sidelengden til henholdsvis den største og den minste firkanten.
\fig{geo9e2}
Begge firkantene kan deles inn i fire likebeinte trekanter:
\begin{figure}
	\centering
	\subfloat{\includegraphics[scale=1]{\asym{geo9e}}}\qquad
	\subfloat{\includegraphics[scale=1]{\asym{geo9f}}}	
\end{figure}
Da trekantane er formlike, har vi at
\begin{align}
\frac{K}{R} &=\frac{k}{r} \label{arogar}
\end{align}
Vi lar $ {\tilde{O}=4K} $ og $ {\tilde{o}=4k} $ være tilnærmingen av omkretsen til henholdsvis den største og den minste sirkelen.
Ved å gange med 4 på begge sider av \eqref{arogar} får vi at
\begin{align}
\frac{4A}{R} &= \frac{4a}{r} \br
\frac{\tilde{O}}{R} &= \frac{\tilde{o}}{r} \label{arogarto}
\end{align}
Og nå merker vi oss dette:\vsk

\textsl{Selv om vi i hver av de to sirklene innskriver en mangekant med 4, 100 eller hvor mange sider det skulle vere, vil mangekantane alltid kunne deles inn i trekantar som oppfyller \eqref{arogar}. Og på samme måte som vi har gjort i eksempelet over kan vi omskrive \eqref{arogar} til \eqref{arogarto} i stedet.} \vsk

La oss derfor tenke oss mangekanter med så mange sider at vi godtar omkretsene deres som lik omkretsene til sirklene. Om vi da skriver omkretsen den største og den minste sirkelen henholdsvis $ O $ og $ o $, får vi at \vs
\[ \frac{O}{R}=\frac{o}{r} \]
Da de to sirklene våre er helt vilkårlig valgt, har vi nå kommet fram til at \textit{alle sirkler har det samme forholdet mellom omkretsen og radiusen}. En enda meir brukt formulering er at \textit{alle sirkler har det samme forholdet mellom omkretsen og diameteren}. Vi lar $ D $ og $ d $ være diameteren til henholdsvis sirkelen med radius $ R $ og $ r $. Da har vi at
\alg{
	\frac{O}{2R}=\frac{o}{2r} \br
	\frac{O}{D}=\frac{o}{d}
}
Forholdstalet mellom omkretsen og diameteren i en sirkel blir kalt $ \pi $ \index{$ \pi $}(uttales ''pi''):
\[ \frac{O}{D}=\pi \]
Likningen over fører oss til formelen for omkretsen til en sirkel:
\alg{
	O&= \pi D\\
	&=2\pi r
}
Tidligere fant vi at omkretsen til en sirkel med radius 1 (og diameter 2) er $ 6.283185307179586... $\,. Dette betyr at
\alg{
	\pi&= \frac{6.283185307179586...}{2} \br
	&= 3.141592653589793...
}
} \vsk

\fork{\ref{forform} \forform}{
	I figuren under er $ BB'||CC' $. Arealet til en trekant $ \triangle ABC $ skriver vi her som $ ABC $. \vspace{-1pt}
	\fig{forml0}
	Med $ BB' $ som grunnlinje har både $ \triangle CBB' $ og $ \triangle CBB' $ $ HB' $ som høyde, derfor er
	\begin{equation}\label{a}
	CBB' = C'BB'
	\end{equation}
	Videre har vi at
	\alg{
		ABB' &= AB\cdot HB' \vn
		CBB' &= BC\cdot HB'
	}
	Altså er
	\begin{equation}\label{b}
	\frac{ABB'}{CBB'}=\frac{AB}{BC}
	\end{equation}
	På lignende vis er
	\begin{equation}\label{c}
	\frac{ABB'}{C'BB'}=\frac{AB'}{B'C'}
	\end{equation}
	Av \eqref{a}, \eqref{b} og \eqref{c} følger det at
	\begin{equation}\label{form}
	\frac{AB}{BC}=\frac{ABB'}{CBB'}\frac{ABB'}{C'BB'}=\frac{AB'}{B'C'}
	\end{equation}
	For de formlike trekantene $ \triangle ACC' $ og $\triangle ABB' $ er
	\alg{
		\frac{AC}{AB} &= \frac{AB+BC}{AB} \\[5pt]
		&= 1+\frac{BC}{AB} \\
		& \\
		\frac{AC'}{AB'}&=\frac{AB'+B'C'}{AB'}\\[5pt]
		&= 1+\frac{B'C'}{AB'}
	} 
	Av \eqref{form} er dermed forholdet mellom de samsvarande sidene like.
}\vsk

\info{Merk}{
I de kommnde forklaringene av vilkårene \textsl{ii} og \textsl{iii} fra \rref{konsttre} tar man utgangspunkt i følgende:
	\begin{itemize}
		\item To sirkler skjærer kvarandre i maksimalt to punkt.
		\item Gitt at et koordinatsystem blir plassert med origo i senteret til den ene sirkelen, og slik at horisontalaksen går gjennom begge sirkelsentrene. Viss $ (a, b) $ er det ene skjæringspunktet, er $ (a, -b) $ det andre skjæringspunktet.
	\end{itemize}
	\fig{sirk2} 
		Punktene over kan enkelt vises, men er såpass intuitivt sanne at vi tar dem for gitt. Punktene forteller oss at trekanten som består av de to sentrene og det ene skjæringspunktet er kongruent med trekanten som består av de to sentrene og det andre skjæringspunktet. Med dette kan vi studere egenskaper til trekanter ved hjelp av halvsirkler.
} \vsk

\fork{\ref{konsttre} \konsttre}{
\textbf{Vilkår i}\os
Gitt en lengde $ c $ og to vinkler $u $ og $ v $.
Vi lager et linjestykke $ AB $ med lengde $ c $. Så stipler vi to vinkelbein slik at $ \angle {A= u} $ og $ {B=v} $. Så lenge disse vinkelbeina ikke er parallelle, må de nødvendigvis skjære hverandre i ett, og bare ett, punkt ($ C $ i figuren). I lag med $ A $ og $ B $ vil dette punktet danne en trekant som er unikt gitt av $ c $, $ u $ og $ v $.
\fig{geo13c}	
	
\textbf{Vilkår ii}\os
Gitt tre lengder $ a $, $ b $ og $ c $. Vi lager et linjestykket $ AB $ med lengde $ c $. Så lager vi to halvsirkler med henholdsvis radius $ a $ og $ b $ og sentrum $ B $ og $ A $. Skal nå en trekant $ \triangle ABC $ ha sidelengder $ a $, $ b $ og $ c $, må $ C $ ligge på begge sirkelbuene. Da buene bare kan møtes i ett punkt, er formen og størrelsen til $ \triangle ABC $ unikt gitt av $a$, $ b $ og $ c $.
\fig{geo13a}	

\textbf{Vilkår iii} \os
Gitt to lengder $ b $ og $ c $ og en vinkel $ u $. Vi starter med følgende:
\begin{enumerate}
	\item Vi lager et linjestykke $ AB $ med lengde $ c $.
	\item I $ A $ tegner vi en halvsirkel med radius $ b $. 
\end{enumerate}
Ved å la $ C $ vere plassert hvor som helst på denne sirkelbuen, har vi alle mulige varianter av en trekant $ \triangle ABC $ med sidelengdene $ {AB=c} $ og $ {AC=b} $. Å plassere $ C $ langs bogen til halvsirkelen er det samme som å gi $ \angle A $ en bestemt verdi. Det gjenstår nå å vise at hver plassering av $ C $ gir en unik lengde av $ BC $.
\fig{geo13b}
Vi lar $ C_1 $ og $ C_2 $ være to potensielle plasseringer av $ C $, der $ C_2 $, langs halvsirkelen, ligger nærmere $ E $ enn $ C_1 $. Videre stipler vi en sirkelbue med radius $ BC_1 $ og sentrum i $ B $. Da den stiplede sirkelbuen og halvsirkelen bare kan skjære hverandre i $ C_1 $, vil alle andre punkt på halvsirkelen ligge enten innenfor eller utenfor den stiplede sirkelbuen. Slik vi har definert $ C_2 $, må dette punktet ligge utenfor den stiplede sirkelbuen, og dermed er $ BC_2 $ lengre enn $ BC_1 $. Av dette kan vi konkludere med at $ BC $ blir lengre dess nærmere $ C $ beveger seg mot $ E $ langs halvsirkelen. Å sette $ {\angle A=u} $ vil altså gi en unik verdi for $ BC $, og da en unik trekant $ \triangle ABC $ der $ AC=b $, $ c=AB $ og $ \angle BAC=u $.
}\vsk

\fork{\ref{vilkform} \vilkform}{
\textbf{Vilkår i}\os
Gitt to trekanter $ \triangle ABC $ og $ \triangle DEF $. Av \rref{180} har vi at
\alg{
\angle A+ \angle B+\angle C=\angle D+\angle E+\angle F
}	
Hvis $ \angle A=\angle D $ og $ \angle B=\angle E $, følger det at $ \angle C=\angle E $.\vsk

\textbf{Vilkår ii}\os
Vi tar utgangspunkt i trekantene $ \triangle ABC $ og $ \triangle DEF $ der
\begin{equation}
\dfrac{AC}{DF}=\dfrac{BC}{EF}\qquad,\qquad \angle C = \angle F\label{vilkforma}
\end{equation}
\fig{geo12}
Vi setter $ a=BC $, $ b=AC $, $ d=EF $ og $ e=DF $. Vi plasserer $ D' $ og $ E' $ på henholdsvis $ AC $ og $ BC $, slik at $ D'C=e $ og $ AB\parallel D'E' $. Da er $ \triangle ABC \sim \triangle D'E'C$, altså har vi at
\alg{
\frac{E'C}{BC}&=\frac{D'C}{AC}\br
E'C&=\frac{ae}{b}
}
Av \eqref{vilkforma} har vi at
\[ EF=\frac{ae}{b} \]	
Altså er $ {E'C = EF} $. Nå har vi av vilkår \textsl{ii} fra \rref{kongtre} at $ \triangle D'E'C\cong\triangle DEF $. Dette betyr at $ \triangle ABC\sim \triangle DEF $.\vsk

\textbf{Vilkår iii}\os
Vi tar utgangspunkt i to trekanter $ \triangle ABC $ og $ \triangle DEF $ der
\begin{equation}
\frac{AB}{DE}=\frac{AC}{DF}=\frac{BC}{EF} \label{vilkformb}
\end{equation}
Vi plasserer $ D' $ og $ E' $ på henholdsvis $ AC $ og $ BC $, slik at $ D'C=e $ og $ E'C=d $. Av vilkår \textsl{i} fra \rref{vilkform} har vi da at $ \triangle ABC\sim\triangle D'E'C $. Altså er
\alg{
\frac{D'E'}{AB}&=\frac{D'C}{AC} \br
D'E'&=\frac{ae}{c}
} 
Av \eqref{vilkformb} har vi at
\alg{
f&=\frac{ae}{c}
}
Altså har $ \triangle D'E'C $ og $ \triangle DEF $ parvis like sidelengder, og av vilkår \textsl{i} fra \rref{kongtre} er de da kongruente. Dette betyr at $ {\triangle ABC \sim \triangle DEF}$.
\fig{geo12b}
}



\end{document}


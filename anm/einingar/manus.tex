\documentclass[english,hidelinks,pdftex, 11 pt, class=report,crop=false]{standalone}
\usepackage[T1]{fontenc}
\usepackage[utf8]{luainputenc}
\usepackage{geometry}
\geometry{verbose,paperwidth=16.4 cm, paperheight=29cm, inner=2.05cm, outer=2.05 cm, bmargin=2cm, tmargin=1.8cm}
\usepackage{amsmath}
\usepackage{amssymb}
\usepackage{esint}
\usepackage{babel}
\usepackage{tabu}
\usepackage{lmodern}


\begin{document}
\begin{itemize}
	\item Omgjering av eininga, også kalt einhete
	\item Når me ska gjere om mellom einhete, er det avgjerende å forstå prefiksan kilo, hekto, deka, desi, centi og milli.
	\item kilo betyr tusen, hekto betyr 100, deka betyr 10, desi betyr ein tidel, centi betyr ein hundredel og milli betyr ein tusendel.
	\item legg merke til at me ikkje he nåkkå ege ord for 1, men i praksis vil 1 vere det samme som blant anna gram, liter eller meter
\end{itemize}

\begin{itemize}
	\item\textbf{ Eksempel 1}
	\item La oss no finne ut ka 4,8 dm e, målt i mm
	\item Det første me legg merke te da, e at for å komme oss fra desi te milli, må me gå to plassa te høgre i tabellen vår.
	\item Så skriv me 4,8
	\item Deretter flytte me kommaet 1, 2 plassa mot høgre
	\item På dei blanke plassan me he passert, fylle me inn 0a. 
	\item Noe e svaret vårt at 4,8 dm tilsvare 480 mm
\end{itemize}

\begin{itemize}
	\item\textbf{ Eksempel 2}
	\item La oss no finne ut ka 52 km e, målt i m
	\item Det første me legg merke te da, e at for å komme oss fra kilo te meter, må me gå tre plassa te høgre i tabellen vår.
	\item Så skriv me 52
	\item På eit heiltall starte me alltid med kommaet heilt bakerst
	\item Deretter flytte me kommaet 1, 2, 3 plassa mot høgre
	\item På dei blanke plassan me he passert, fylle me inn 0a. 
	\item Noe e svaret vårt at 52 km tilsvare 52 000 m
\end{itemize}

\begin{itemize}
	\item\textbf{ Eksempel 3}
	\item La oss no finne ut ka 7 g e, målt i hg
	\item Det første me legg merke te da, e at for å komme oss fra gram te hekto, må me gå to plassa te venstre i tabellen vår.
	\item Så skriv me 7, og plassere eit komma heilt bakerst
	\item Deretter flytte me kommaet 1, 2 plassa mot venstre
	\item På dei blanke plassan me he passert, fylle me inn 0a. 
	\item E tillegg e det slik at viss eit komma står fremst te venstre, sett me også ein 0 foran foran kommet 
	\item Noe e svaret vårt at 7 g tilsvare 0,07 hg
\end{itemize}

\end{document}
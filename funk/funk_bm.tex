\documentclass[english,hidelinks,pdftex, 11 pt, class=report,crop=false]{standalone}
\usepackage[T1]{fontenc}
\usepackage[utf8]{luainputenc}
\usepackage{lmodern} % load a font with all the characters
\usepackage{geometry}
\geometry{verbose,paperwidth=16.1 cm, paperheight=24 cm, inner=2.3cm, outer=1.8 cm, bmargin=2cm, tmargin=1.8cm}
\setlength{\parindent}{0bp}
\usepackage{import}
\usepackage[subpreambles=false]{standalone}
\usepackage{amsmath}
\usepackage{amssymb}
\usepackage{esint}
\usepackage{babel}
\usepackage{tabu}
\makeatother
\makeatletter

\usepackage{titlesec}
\usepackage{ragged2e}
\RaggedRight
\raggedbottom
\frenchspacing

% Norwegian names of figures, chapters, parts and content
\addto\captionsenglish{\renewcommand{\figurename}{Figur}}
\makeatletter
\addto\captionsenglish{\renewcommand{\chaptername}{Kapittel}}
\addto\captionsenglish{\renewcommand{\partname}{Del}}

\addto\captionsenglish{\renewcommand{\contentsname}{Innhold}}

\usepackage{graphicx}
\usepackage{float}
\usepackage{subfig}
\usepackage{placeins}
\usepackage{cancel}
\usepackage{framed}
\usepackage{wrapfig}
\usepackage[subfigure]{tocloft}
\usepackage[font=footnotesize,labelfont=sl]{caption} % Figure caption
\usepackage{bm}
\usepackage[dvipsnames, table]{xcolor}
\definecolor{shadecolor}{rgb}{0.105469, 0.613281, 1}
\colorlet{shadecolor}{Emerald!15} 
\usepackage{icomma}
\makeatother
\usepackage[many]{tcolorbox}
\usepackage{multicol}
\usepackage{stackengine}

% For tabular
\usepackage{array}
\usepackage{multirow}
\usepackage{longtable} %breakable table

% Ligningsreferanser
\usepackage{mathtools}
\mathtoolsset{showonlyrefs}

% index
\usepackage{imakeidx}
\makeindex[title=Indeks]

%Footnote:
\usepackage[bottom, hang, flushmargin]{footmisc}
\usepackage{perpage} 
\MakePerPage{footnote}
\addtolength{\footnotesep}{2mm}
\renewcommand{\thefootnote}{\arabic{footnote}}
\renewcommand\footnoterule{\rule{\linewidth}{0.4pt}}
\renewcommand{\thempfootnote}{\arabic{mpfootnote}}

%colors
\definecolor{c1}{cmyk}{0,0.5,1,0}
\definecolor{c2}{cmyk}{1,0.25,1,0}
\definecolor{n3}{cmyk}{1,0.,1,0}
\definecolor{neg}{cmyk}{1,0.,0.,0}

% Lister med bokstavar
\usepackage{enumitem}

\newcounter{rg}
\numberwithin{rg}{chapter}
\newcommand{\reg}[2][]{\begin{tcolorbox}[boxrule=0.3 mm,arc=0mm,colback=blue!3] {\refstepcounter{rg}\phantomsection \large \textbf{\therg \;#1} \vspace{5 pt}}\newline #2  \end{tcolorbox}\vspace{-5pt}}

\newcommand\alg[1]{\begin{align} #1 \end{align}}

\newcommand\eks[2][]{\begin{tcolorbox}[boxrule=0.3 mm,arc=0mm,enhanced jigsaw,breakable,colback=green!3] {\large \textbf{Eksempel #1} \vspace{5 pt}\\} #2 \end{tcolorbox}\vspace{-5pt} }

\newcommand{\st}[1]{\begin{tcolorbox}[boxrule=0.0 mm,arc=0mm,enhanced jigsaw,breakable,colback=yellow!12]{ #1} \end{tcolorbox}}

\newcommand{\spr}[1]{\begin{tcolorbox}[boxrule=0.3 mm,arc=0mm,enhanced jigsaw,breakable,colback=yellow!7] {\large \textbf{Språkboksen} \vspace{5 pt}\\} #1 \end{tcolorbox}\vspace{-5pt} }

\newcommand{\sym}[1]{\colorbox{blue!15}{#1}}

\newcommand{\info}[2]{\begin{tcolorbox}[boxrule=0.3 mm,arc=0mm,enhanced jigsaw,breakable,colback=cyan!6] {\large \textbf{#1} \vspace{5 pt}\\} #2 \end{tcolorbox}\vspace{-5pt} }

\newcommand\algv[1]{\vspace{-11 pt}\begin{align*} #1 \end{align*}}

\newcommand{\regv}{\vspace{5pt}}
\newcommand{\mer}{\textsl{Merk}: }
\newcommand\vsk{\vspace{11pt}}
\newcommand\vs{\vspace{-11pt}}
\newcommand\vsb{\vspace{-16pt}}
\newcommand\sv{\vsk \textbf{Svar:} \vspace{4 pt}\\}
\newcommand\br{\\[5 pt]}
\newcommand{\asym}[1]{../fig/#1}
\newcommand\algvv[1]{\vs\vs\begin{align*} #1 \end{align*}}
\newcommand{\y}[1]{$ {#1} $}
\newcommand{\os}{\\[5 pt]}
\newcommand{\prbxl}[2]{
\parbox[l][][l]{#1\linewidth}{#2
	}}
\newcommand{\prbxr}[2]{\parbox[r][][l]{#1\linewidth}{
		\setlength{\abovedisplayskip}{5pt}
		\setlength{\belowdisplayskip}{5pt}	
		\setlength{\abovedisplayshortskip}{0pt}
		\setlength{\belowdisplayshortskip}{0pt} 
		\begin{shaded}
			\footnotesize	#2 \end{shaded}}}

\renewcommand{\cfttoctitlefont}{\Large\bfseries}
\setlength{\cftaftertoctitleskip}{0 pt}
\setlength{\cftbeforetoctitleskip}{0 pt}

\newcommand{\bs}{\\[3pt]}
\newcommand{\vn}{\\[6pt]}
\newcommand{\fig}[1]{\begin{figure}
		\centering
		\includegraphics[]{\asym{#1}}
\end{figure}}
\newcommand{\net}[2]{{\color{blue}\href{#1}{#2}}}

\newcommand{\hrs}[2]{\hyperref[#1]{\color{blue}\textsl{#2 \ref*{#1}}}}
\newcommand{\rref}[1]{\hyperref[#1]{\color{blue}\textsl{Regel \ref*{#1}}}}

\newcommand{\sectionbreak}{\clearpage} % New page on each section

% Equation comments
\newcommand{\cm}[1]{\llap{\color{blue} #1}}

\newcommand\fork[2]{\begin{tcolorbox}[boxrule=0.3 mm,arc=0mm,enhanced jigsaw,breakable,colback=yellow!7] {\large \textbf{#1 (forklaring)} \vspace{5 pt}\\} #2 \end{tcolorbox}\vspace{-5pt} }


%%% SECTION HEADLINES %%%

% Our numbers
\newcommand{\likteikn}{Likhetstegnet}
\newcommand{\talsifverd}{Tall, siffer og verdi}
\newcommand{\koordsys}{Koordinatsystem}

% Calculations
\newcommand{\adi}{Addisjon}
\newcommand{\sub}{Subtraksjon}
\newcommand{\gong}{Multiplikasjon (Gonging)}
\newcommand{\del}{Divisjon (deling)}

%Factorization and order of operations
\newcommand{\fak}{Faktorisering}
\newcommand{\rrek}{Regnerekkefølge}

%Fractions
\newcommand{\brgrpr}{Introduksjon}
\newcommand{\brvu}{Verdi, utviding og forkorting av brøk}
\newcommand{\bradsub}{Addisjon og subtraksjon}
\newcommand{\brgngheil}{Brøk ganget med heltall}
\newcommand{\brdelheil}{Brøk delt med heltall}
\newcommand{\brgngbr}{Brøk ganget med brøk}
\newcommand{\brkans}{Kansellering av faktorer}
\newcommand{\brdelmbr}{Deling med brøk}
\newcommand{\Rasjtal}{Rasjonale tall}

%Negative numbers
\newcommand{\negintro}{Introduksjon}
\newcommand{\negrekn}{De fire regneartane med negative tall}
\newcommand{\negmeng}{Negative tall som mengde}

% Geometry 1
\newcommand{\omgr}{Begrep}
\newcommand{\eignsk}{Egenskaper for trekanter og firkanter}
\newcommand{\omkr}{Omkrets}
\newcommand{\area}{Areal}

%Algebra 
\newcommand{\algintro}{Introduksjon}
\newcommand{\pot}{Potenser}
\newcommand{\irrasj}{Irrasjonale tall}

%Equations
\newcommand{\ligintro}{Introduksjon}
\newcommand{\liglos}{Løsing ved de fire regneartene}
\newcommand{\ligloso}{Løsingsmetodene oppsummert}

%Functions
\newcommand{\fintro}{Introduksjon}
\newcommand{\lingraf}{Lineære funksjoner og grafer}

%Geometry 2
\newcommand{\geoform}{Formler for areal og omkrets}
\newcommand{\kongogsim}{Kongruente og formlike trekanter}
\newcommand{\geofork}{Forklaringar}

% Names of rules
\newcommand{\adkom}{Addisjon er kommutativ}
\newcommand{\gangkom}{Multiplikasjon er kommutativ}
\newcommand{\brdef}{Brøk som omskriving av delestykke}
\newcommand{\brtbr}{Brøk ganget med brøk}
\newcommand{\delmbr}{Brøk delt på brøk}
\newcommand{\gangpar}{Ganging med parentes (distributiv lov)}
\newcommand{\gangparsam}{Paranteser ganget sammen}
\newcommand{\gangmnegto}{Ganging med negative tall I}
\newcommand{\gangmnegtre}{Ganging med negative tall II}
\newcommand{\konsttre}{Konstruksjon av trekanter}
\newcommand{\kongtre}{Kongruente trekanter}
\newcommand{\topv}{Toppvinkler}
\newcommand{\trisum}{Summen av vinklene i en trekant}
\newcommand{\firsum}{Summen av vinklene i en firkant}
\newcommand{\potgang}{Ganging med potenser}
\newcommand{\potdivpot}{Divisjon med potenser}
\newcommand{\potanull}{Spesialtilfellet \boldmath $a^0$}
\newcommand{\potneg}{Potens med negativ eksponent}
\newcommand{\potbr}{Brøk som grunntall}
\newcommand{\faktgr}{Faktorer som grunntall}
\newcommand{\potsomgrunn}{Potens som grunntall}
\newcommand{\arsirk}{Arealet til en sirkel}
\newcommand{\artrap}{Arealet til et trapes}
\newcommand{\arpar}{Arealet til et parallellogram}
\newcommand{\pyt}{Pytagoras' setning}
\newcommand{\forform}{Forhold i formlike trekanter}
\newcommand{\vilkform}{Vilkår i formlike trekanter}
\newcommand{\omkrsirk}{Omkretsen til en sirkel (og $ \bm \pi $)}
\newcommand{\artri}{Arealet til en trekant}
\newcommand{\arrekt}{Arealet til et rektangel}
\newcommand{\liknflyt}{Flytting av ledd over likhetstegnet}
\newcommand{\funklin}{Lineære funksjoner}

%Opg
% Opg
\newcommand{\abc}[1]{
	\begin{enumerate}[label=\alph*),leftmargin=18pt]
		#1
	\end{enumerate}
}
\newcommand{\opgt}{\phantomsection \addcontentsline{toc}{section}{Oppgaver} \section*{Oppgaver for kapittel \thechapter}\vs \setcounter{section}{1}}
\newcounter{opg}
\numberwithin{opg}{section}
\newcommand{\op}[1]{\vspace{15pt} \refstepcounter{opg}\large \textbf{\color{blue}\theopg} \vspace{2 pt} \label{#1} \\}
\newcommand{\ekspop}{\vsk\textbf{Gruble \thechapter}\vspace{2 pt} \\}
\newcommand{\nes}{\stepcounter{section}
	\setcounter{opg}{0}}
\newcommand{\opr}[1]{\vspace{3pt}\textbf{\ref{#1}}}

%License
\newcommand{\lic}{\textit{Matematikken sine byggesteiner by Sindre Sogge Heggen is licensed under CC BY-NC-SA 4.0. To view a copy of this license, visit\\ 
		\net{http://creativecommons.org/licenses/by-nc-sa/4.0/}{http://creativecommons.org/licenses/by-nc-sa/4.0/}}}

\usepackage{datetime2}
\usepackage[]{hyperref}



\begin{document}

\newpage
\section{\fintro}
Variabler er verdier som forandrar seg. En verdi som forandrar seg i takt med at en variabel forandrar seg, kaller vi en \textit{funksjon}\index{funksjon}.\vsk
\fig{funk1}
I figurene over forandrer antallet ruter seg etter et bestemt mønster. Matematisk kan vi skildre dette mønsteret slik:
\alg{
\text{Antal ruter i \textsl{Figur \color{blue}1}}= 2\cdot {\color{blue}1}+1=3 \\ 
\text{Antal ruter i \textsl{Figur \color{blue}2}}= 2\cdot {\color{blue}2}+1=5 \\ 
\text{Antal ruter i \textsl{Figur \color{blue}3}}= 2\cdot {\color{blue}3}+1=7 \\ 
\text{Antal ruter i \textsl{Figur \color{blue}4}}= 2\cdot {\color{blue}4}+1=9
}
For en figur med et vilkårlig nummer $ x $ har vi at
\[ \text{Antal ruter i \textsl{Figur }}{\color{blue}x}=2{\color{blue}x}+1 \]
Antall ruter forandrar seg altså i takt med at $ x $ forandrer seg, og da sier vi at\regv
\st{\text{''Antall ruter i \textsl{Figur }$ x $'' er en funksjon av $ x $}}\regv

\st{$ {2x+1} $\text{ er \textit{funksjonsuttrykket}\index{funksjonsuttrykk} til funksjonen ''Antall ruter i \textsl{Figur }$ x $''}}
\newpage
\textbf{Generelle uttrykk} \\
Skulle vi jobbet videre med funskjonen vi akkurat har sett på, ville det blitt tungvint å heile tiden måtte skrive ''Antall ruter i \textsl{Figur }$ x $''. Det er vanleg å kalle også funksjoner bare for en bokstav, og i tillegg skrive variabelen funskjonen er avhengig av i parentes. La oss nå omdøpe funksjonen ''Antal ruter i \textsl{Figur} $ x $'' til $ a(x) $. Da har vi at
\[ \text{Antall ruter i \textsl{Figur }}x=a(x)=2x+1 \]
Hvis vi skriver $ a(x) $, men erstatter $ x $ med et bestemt tall, betyr det at vi skal erstatte $ x $ med dette tallet i funksjonsuttrykket vårt:
\alg{
a({\color{blue}1})&=2\cdot{\color{blue}1} +1=3 \\
a({\color{blue}2})&=2\cdot{\color{blue}2}+1=5 \\
a({\color{blue}3})&=2\cdot{\color{blue}3}+1=7\\
a({\color{blue}4})&=2\cdot{\color{blue}4}+1=9
}
\fig{funk1a}
\newpage
\eks{
La antall ruter i mønsteret under være gitt av funksjonen $ a(x) $.
\fig{funk2}
\textbf{a)} Finn uttrykket for $ a(x) $.\bs
\textbf{b)} Hvor mange ruter er det når $ x=10 $? \bs
\textbf{c)} Hva er verdien til $ x $ når $ a(x)=628 $?

\sv
\textbf{a)} Vi legger merke til at
\begin{itemize}
	\item Når $ x=1 $, er det $ 1\cdot1+3=4 $ ruter.
	\item Når $ x=2 $, er det $ 2\cdot2+3=7 $ ruter.
	\item Når $ x=3 $, er det $ 3\cdot3+3=12 $ ruter.
	\item Når $ x=4 $, er det $ 4\cdot4+3=17 $ ruter.
\end{itemize} 
Altså er
\[ a(x)=x\cdot x +3 =x^2+3 \]
\textbf{b)}
\[ a(10)=10^2+3=100+3=103 \]
Når $ x=10 $, er det 103 ruter.\vsk

\textbf{c)} Vi har likningen
\algv{
x^2+3&=628 \\
x^2&=625
}	
Altså er
\[ x=15\qquad\vee\qquad x=-15 \]
Siden vi søker en positiv verdi for $ x $, er $ x=15 $.	
}
\section{\lingraf}
Når vi har en variabel $ x $ og en funksjon $ f(x) $, har vi to verdier; verdien til $ x $ og den tilhørende verdien til $ f(x) $. Disse parene av verdier kan vi sette inn i et koordinatsystem\footnote{Se \hrs{Koord}{seksjon}.}, og da får vi \textit{grafen}\index{funksjon!grafen til} til $ f(x) $. \vsk

La oss bruke funksjonen 
\[ f(x)=2x-1 \]
som eksempel. Vi har at
\alg{
f(0)&=2\cdot0-1=-1 \vn
f(1)&=2\cdot1-1=1 \vn
f(2)&=2\cdot2-1=3 \vn
f(3)&=2\cdot3-1=5
} 
Disse parene av verdiar kan vi sette opp i en tabell:
	\begin{center}
	\begin{tabular}{c | c |c |c|c}
		$ x $ & 0 & 1 & 2 & 3 \\ \hline
		$ f(x) $ &$  -1 $ & 1&3 &5
	\end{tabular}
\end{center}
Tabellen over gir punkta
\[ (0, -1)\quad\quad(1, 1)\quad\quad(2, 3)\quad\quad(3, 5) \]
Vi plasserer nå punktene i et koordinatsystem (se figur på side \pageref{funkfig}). I samband med funksjoner er det vanlig å kalle horisontalaksen og vertikalaksen for henholdsvis $ x $-aksen og $ y $-aksen. 
Grafen til $ f(x) $ er nå en tenkt strek som går gjennom alle de uendelig mange punktene vi kan lage av $ x$-verdier og de tilhørende $ f(x) $-verdiane. Vår funksjon er en \textit{lineær}\index{funksjon!lineær} funksjon, noe som betyr at grafen er ei rett linje. Altså kan grafen tegnes ved å tegne linja som går gjennom punktene vi har funnet.\vsk

Som vi har vært inne på før, kan vi aldri tegne ei hel linje, bare et utklipp av henne. Dette gjelder som regel også for grafer. I figuren på side \pageref{funkfig} har vi tegnet grafen til $ f(x) $ for $ x $-verdiar mellom $ -2 $ og $ 4 $. At $ x $ er i dette \textit{intervallet}\index{intervall} kan vi skrive som\footnote{Se symbolforklaringer på side \pageref{Symbol}.} $ -2\leq x\leq 4 $ eller $ x\in[-2, 4] $.
\fig{funk3} \label{funkfig}
\newpage
\reg[\funklin \label{funklin}]{
	En funksjon på formen \vs
	\[ f(x)=ax+b \]
	der $ a $ og $ b $ er konstanter, er en \textit{lineær} funksjon med \textit{stigningstall}\index{stigningstall} $ a $ og \textit{konstantledd}\index{konstantledd} $ b $. \vsk 
	
	Grafen til en lineær funksjon er ei rett linje som går gjennom punktet $ (0, b) $. \vsk
	
	For to forskjellige $ x $-verdier, $ x_1 $ og $ x_2 $, er
	\[ a=\frac{f(x_2)-f(x_1)}{x_2-x_1} \]
	\fig{funk6}
}
\eks[1]{Finn stigningstallet og konstantleddet til funksjonene.
	\alg{
		f(x)&=2x+1 \vn
		g(x)&=-3+\frac{7}{2}	\vn
		h(x)&=\frac{1}{4}x-\frac{5}{6}\vn
		j(x)&=4-\frac{1}{2}x
	}
	\sv \vs \vs
	\begin{itemize}
		\item $ f(x) $ har stigningstall 2 og konstantledd 1.
		\item $ g(x) $ har stigningstall $ -3 $ og konstantledd $ \frac{7}{2} $.
		\item $ h(x) $ har stigningstall $ \frac{1}{4} $ og konstantledd $ -\frac{5}{6} $.
		\item $ j(x) $ har stigningstall $ -\frac{1}{2} $ og konstantledd 4.		
	\end{itemize}	
}
\newpage
\eks[2]{
	Tegn grafen til
	\[ f(x)=\dfrac{3}{4}x-2 \]
	 for $ x\in[-5, 6] $.
	
	\sv
	For å tegne grafen til en lineær funksjon trenger vi bare å finne to punkt som ligger på grafen. Hvilke to punkt dette er, er det fritt å velge, så for enklest mulig utregning starter vi med å finne punktet der $ x=0 $:
	\[ f(0)=\frac{3}{4}\cdot0-2=-2 \] 
	Videre velger vi $ {x=4} $, siden dette også gir oss en enkel utregning:
	\[ f(4)=\frac{3}{4}\cdot4-2=1 \]
	Nå har vi informasjonen vi trenger, og for ordens skyld setter vi den inn i en tabell:
	\begin{center}
		\begin{tabular}{c | c |c }
			$ x $ & 0 & 4 \\ \hline
			$ f(x) $ &$  -2 $ & 1
		\end{tabular}
	\end{center}
Vi tegner punktene og trekker ei linje gjennom dem:
	\begin{figure}
		\centering
		\fig{funk4}
	\end{figure}
}
\eks[3]{
Finn funksjonsuttrykka til $ f(x) $ og $ g(x) $.	
\fig{funk5} \vs
\sv
Vi starter med å finne funksjonsuttrykket til $ f(x) $. Punktet $ (0, 3) $ ligger på grafen til $ f(x) $ (se også figur på neste side). Da vet vi at $ {f(0)=3} $, og dette må bety at $ 3 $ er konstantleddet til $ f(x) $. Videre ser vi at punktet $ (1, 2) $ også ligger på grafen til $ f(x) $. Stigningstallet til $ f(x) $ er da gitt ved brøken
\[ \frac{2-3}{1-0}=-1 \] 
Altså er 
\[  f(x)=-x+3 \]
\newpage
\fig{funk5a}
Vi går så over til å finne uttrykket til $ g(x) $. Punktet $ (0, -1) $ ligger på grafen til $ g(x) $. Da vet vi at $ {f(0)=-1} $, og dette må bety at $ -1 $ er konstantleddet til $ g(x) $. Videre ser vi at punktet $ (5, 2) $ også ligger på grafen til $ g(x) $. Stigningstallet til $ g(x) $ er da gitt ved brøken
\[ \frac{2-(-1)}{5-0}=\frac{3}{5} \]
Altså er
\[ g(x)=\frac{3}{5}x+1 \]
}

\newpage
\fork{\ref{funklin} \funklin}{
\textbf{Uttrykk for $ \bm a$}\os	
Gitt en lineær funksjon
\[ f(x)=ax+b \]
For to forskjellige $ x $-verdier, $ x_1 $ og $ x_2 $, har vi at
\begin{equation}
f(x_1)=ax_1+b \label{funkfork}
\end{equation}
\begin{equation}
f(x_2)=ax_2+b \label{funkfork1}
\end{equation}
Vi trekker \eqref{funkfork} fra \eqref{funkfork1}, og får at 
\alg{
f(x_2)-f(x_1)&=ax_2+b-(ax_1+b) \br
f(x_2)-f(x_1)&=ax_2-ax_1\\
f(x_2)-f(x_1)&=a(x_2-x_1) \\
\frac{f(x_2)-f(x_1)}{x_2-x_1}&=a \label{funka}
}\vsk

\textbf{Grafen til en lineær funksjon er ei rett linje}\os

Gitt en lineær funksjon $ f(x)=ax+b $ og to forskjellige $ x $-verdiar $ x_1 $ og $ x_2 $. Vi setter $ {A=(x_1, b)} $, $ B=(x_2, b) $, $ C=(b, f(x_1)) $,\\ $ D=(0, f(x_2)) $ og $ E=(0, b) $.
\fig{funk6a}
Av \eqref{funka} har vi at
\alg{
\frac{f(x_1)-f(0)}{x_1-0}&=a\br
\frac{ax_1+b-b}{x_1}&=a\br
\frac{ax_1}{x_1}=a \label{funkx1}
}
Tilsvarende er 
\begin{equation} \label{funkx2}
\frac{ax_2}{x_2}=a 
\end{equation}
Videre har vi at
\alg{
AC&=f(x_1)-b = ax_1 \vn
BD&=f(x_2)-b = ax_2 \vn
EA&= x_1 \vn
EB&= x_2
}
Av \eqref{funkx1} og \eqref{funkx2} har vi at
\[ \frac{ax_1}{x_1}=\frac{ax_2}{x_2} \]
Dette betyr at
\[ \frac{AC}{BD}=\frac{EA}{EB} \]
I tillegg er $ {\angle A=\angle B} $, altså oppfyller $ \triangle EAC $ og $ \triangle EBD $ vilkår iii fra \rref{vilkform}, og dermed er trekantane formlike. Dette betyr at $ C $ og $ D $ ligger på linje, og denne linja må vere grafen til $ f(x) $.
}


\end{document}
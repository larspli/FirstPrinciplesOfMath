\documentclass[english,hidelinks,pdftex, 11 pt, class=report,crop=false]{standalone}
\usepackage[T1]{fontenc}
\usepackage[utf8]{luainputenc}
\usepackage{lmodern} % load a font with all the characters
\usepackage{geometry}
\geometry{verbose,paperwidth=16.1 cm, paperheight=24 cm, inner=2.3cm, outer=1.8 cm, bmargin=2cm, tmargin=1.8cm}
\setlength{\parindent}{0bp}
\usepackage{import}
\usepackage[subpreambles=false]{standalone}
\usepackage{amsmath}
\usepackage{amssymb}
\usepackage{esint}
\usepackage{babel}
\usepackage{tabu}
\makeatother
\makeatletter

\usepackage{titlesec}
\usepackage{ragged2e}
\RaggedRight
\raggedbottom
\frenchspacing

% Norwegian names of figures, chapters, parts and content
\addto\captionsenglish{\renewcommand{\figurename}{Figur}}
\makeatletter
\addto\captionsenglish{\renewcommand{\chaptername}{Kapittel}}
\addto\captionsenglish{\renewcommand{\partname}{Del}}

\addto\captionsenglish{\renewcommand{\contentsname}{Innhald}}

\usepackage{graphicx}
\usepackage{float}
\usepackage{subfig}
\usepackage{placeins}
\usepackage{cancel}
\usepackage{framed}
\usepackage{wrapfig}
\usepackage[subfigure]{tocloft}
\usepackage[font=footnotesize,labelfont=sl]{caption} % Figure caption
\usepackage{bm}
\usepackage[dvipsnames, table]{xcolor}
\definecolor{shadecolor}{rgb}{0.105469, 0.613281, 1}
\colorlet{shadecolor}{Emerald!15} 
\usepackage{icomma}
\makeatother
\usepackage[many]{tcolorbox}
\usepackage{multicol}
\usepackage{stackengine}

% For tabular
\usepackage{array}
\usepackage{multirow}
\usepackage{longtable} %breakable table

% Ligningsreferanser
\usepackage{mathtools}
\mathtoolsset{showonlyrefs}

% index
\usepackage{imakeidx}
\makeindex[title=Indeks]

%Footnote:
\usepackage[bottom, hang, flushmargin]{footmisc}
\usepackage{perpage} 
\MakePerPage{footnote}
\addtolength{\footnotesep}{2mm}
\renewcommand{\thefootnote}{\arabic{footnote}}
\renewcommand\footnoterule{\rule{\linewidth}{0.4pt}}
\renewcommand{\thempfootnote}{\arabic{mpfootnote}}

%colors
\definecolor{c1}{cmyk}{0,0.5,1,0}
\definecolor{c2}{cmyk}{1,0.25,1,0}
\definecolor{n3}{cmyk}{1,0.,1,0}
\definecolor{neg}{cmyk}{1,0.,0.,0}

% Lister med bokstavar
\usepackage[inline]{enumitem}

\newcounter{rg}
\numberwithin{rg}{chapter}
\newcommand{\reg}[2][]{\begin{tcolorbox}[boxrule=0.3 mm,arc=0mm,colback=blue!3] {\refstepcounter{rg}\phantomsection \large \textbf{\therg \;#1} \vspace{5 pt}}\newline #2  \end{tcolorbox}\vspace{-5pt}}

\newcommand\alg[1]{\begin{align} #1 \end{align}}

\newcommand\eks[2][]{\begin{tcolorbox}[boxrule=0.3 mm,arc=0mm,enhanced jigsaw,breakable,colback=green!3] {\large \textbf{Eksempel #1} \vspace{5 pt}\\} #2 \end{tcolorbox}\vspace{-5pt} }

\newcommand{\st}[1]{\begin{tcolorbox}[boxrule=0.0 mm,arc=0mm,enhanced jigsaw,breakable,colback=yellow!12]{ #1} \end{tcolorbox}}

\newcommand{\spr}[1]{\begin{tcolorbox}[boxrule=0.3 mm,arc=0mm,enhanced jigsaw,breakable,colback=yellow!7] {\large \textbf{Språkboksen} \vspace{5 pt}\\} #1 \end{tcolorbox}\vspace{-5pt} }

\newcommand{\sym}[1]{\colorbox{blue!15}{#1}}

\newcommand{\info}[2]{\begin{tcolorbox}[boxrule=0.3 mm,arc=0mm,enhanced jigsaw,breakable,colback=cyan!6] {\large \textbf{#1} \vspace{5 pt}\\} #2 \end{tcolorbox}\vspace{-5pt} }

\newcommand\algv[1]{\vspace{-11 pt}\begin{align*} #1 \end{align*}}

\newcommand{\regv}{\vspace{5pt}}
\newcommand{\mer}{\textsl{Merk}: }
\newcommand\vsk{\vspace{11pt}}
\newcommand\vs{\vspace{-11pt}}
\newcommand\vsb{\vspace{-16pt}}
\newcommand\sv{\vsk \textbf{Svar:} \vspace{4 pt}\\}
\newcommand\br{\\[5 pt]}
\newcommand{\asym}[1]{../fig/#1}
\newcommand\algvv[1]{\vs\vs\begin{align*} #1 \end{align*}}
\newcommand{\y}[1]{$ {#1} $}
\newcommand{\os}{\\[5 pt]}
\newcommand{\prbxl}[2]{
\parbox[l][][l]{#1\linewidth}{#2
	}}
\newcommand{\prbxr}[2]{\parbox[r][][l]{#1\linewidth}{
		\setlength{\abovedisplayskip}{5pt}
		\setlength{\belowdisplayskip}{5pt}	
		\setlength{\abovedisplayshortskip}{0pt}
		\setlength{\belowdisplayshortskip}{0pt} 
		\begin{shaded}
			\footnotesize	#2 \end{shaded}}}

\renewcommand{\cfttoctitlefont}{\Large\bfseries}
\setlength{\cftaftertoctitleskip}{0 pt}
\setlength{\cftbeforetoctitleskip}{0 pt}

\newcommand{\bs}{\\[3pt]}
\newcommand{\vn}{\\[6pt]}
\newcommand{\fig}[1]{\begin{figure}
		\centering
		\includegraphics[]{\asym{#1}}
\end{figure}}


\newcommand{\sectionbreak}{\clearpage} % New page on each section

% Equation comments
\newcommand{\cm}[1]{\llap{\color{blue} #1}}

\newcommand\fork[2]{\begin{tcolorbox}[boxrule=0.3 mm,arc=0mm,enhanced jigsaw,breakable,colback=yellow!7] {\large \textbf{#1 (forklaring)} \vspace{5 pt}\\} #2 \end{tcolorbox}\vspace{-5pt} }




%colors
\newcommand{\colr}[1]{{\color{red} #1}}
\newcommand{\colb}[1]{{\color{blue} #1}}
\newcommand{\colo}[1]{{\color{orange} #1}}
\newcommand{\colc}[1]{{\color{cyan} #1}}
\definecolor{projectgreen}{cmyk}{100,0,100,0}
\newcommand{\colg}[1]{{\color{projectgreen} #1}}

%%% SECTION HEADLINES %%%

% Our numbers
\newcommand{\likteikn}{Likskapsteiknet}
\newcommand{\talsifverd}{Tal, siffer og verdi}
\newcommand{\koordsys}{Koordinatsystem}

% Calculations
\newcommand{\adi}{Addisjon}
\newcommand{\sub}{Subtraksjon}
\newcommand{\gong}{Multiplikasjon (Gonging)}
\newcommand{\del}{Divisjon (deling)}

%Factorization and order of operations
\newcommand{\fak}{Faktorisering}
\newcommand{\rrek}{Reknerekkefølge}

%Fractions
\newcommand{\brgrpr}{Introduksjon}
\newcommand{\brvu}{Verdi, utviding og forkorting av brøk}
\newcommand{\bradsub}{Addisjon og subtraksjon}
\newcommand{\brgngheil}{Brøk gonga med heiltal}
\newcommand{\brdelheil}{Brøk delt med heiltal}
\newcommand{\brgngbr}{Brøk gonga med brøk}
\newcommand{\brkans}{Kansellering av faktorar}
\newcommand{\brdelmbr}{Deling med brøk}
\newcommand{\Rasjtal}{Rasjonale tal}

%Negative numbers
\newcommand{\negintro}{Introduksjon}
\newcommand{\negrekn}{Dei fire rekneartane med negative tal}
\newcommand{\negmeng}{Negative tal som mengde}

% Geometry 1
\newcommand{\omgr}{Omgrep}
\newcommand{\eignsk}{Eigenskapar for trekantar og firkantar}
\newcommand{\omkr}{Omkrins}
\newcommand{\area}{Areal}

%Algebra 
\newcommand{\algintro}{Introduksjon}
\newcommand{\pot}{Potensar}
\newcommand{\irrasj}{Irrasjonale tal}

%Equations
\newcommand{\ligintro}{Introduksjon}
\newcommand{\liglos}{Løysing ved dei fire rekneartane}
\newcommand{\ligloso}{Løysingsmetodane oppsummert}

%Functions
\newcommand{\fintro}{Introduksjon}
\newcommand{\lingraf}{Lineære funksjonar og grafar}

%Geometry 2
\newcommand{\geoform}{Formlar for areal og omkrins}
\newcommand{\kongogsim}{Kongruente og formlike trekantar}
\newcommand{\geofork}{Forklaringar}

% Names of rules
\newcommand{\gangdestihundre}{Å gange desimaltall med 10, 100 osv.}
\newcommand{\delmedtihundre}{Deling med 10, 100, 1\,000 osv.}
\newcommand{\ompref}{Omgjøring av prefikser}
\newcommand{\adkom}{Addisjon er kommutativ}
\newcommand{\gangkom}{Multiplikasjon er kommutativ}
\newcommand{\brdef}{Brøk som omskriving av delestykke}
\newcommand{\brtbr}{Brøk gonga med brøk}
\newcommand{\delmbr}{Brøk delt på brøk}
\newcommand{\gangpar}{Gonging med parentes (distributiv lov)}
\newcommand{\gangparsam}{Parantesar gonga saman}
\newcommand{\gangmnegto}{Gonging med negative tal I}
\newcommand{\gangmnegtre}{Gonging med negative tal II}
\newcommand{\konsttre}{Konstruksjon av trekantar}
\newcommand{\kongtre}{Kongruente trekantar}
\newcommand{\topv}{Toppvinklar}
\newcommand{\trisum}{Summen av vinklane i ein trekant}
\newcommand{\firsum}{Summen av vinklane i ein firkant}
\newcommand{\potgang}{Gonging med potensar}
\newcommand{\potdivpot}{Divisjon med potensar}
\newcommand{\potanull}{Spesialtilfellet \boldmath $a^0$}
\newcommand{\potneg}{Potens med negativ eksponent}
\newcommand{\potbr}{Brøk som grunntal}
\newcommand{\faktgr}{Faktorar som grunntal}
\newcommand{\potsomgrunn}{Potens som grunntal}
\newcommand{\arsirk}{Arealet til ein sirkel}
\newcommand{\artrap}{Arealet til eit trapes}
\newcommand{\arpar}{Arealet til eit parallellogram}
\newcommand{\pyt}{Pytagoras' setning}
\newcommand{\forform}{Forhold i formlike trekantar}
\newcommand{\vilkform}{Vilkår i formlike trekantar}
\newcommand{\omkrsirk}{Omkrinsen til ein sirkel (og $ \bm \pi $)}
\newcommand{\artri}{Arealet til ein trekant}
\newcommand{\arrekt}{Arealet til eit rektangel}
\newcommand{\liknflyt}{Flytting av ledd over likskapsteiknet}
\newcommand{\funklin}{Lineære funksjonar}

%Opg
\newcommand{\abc}[1]{
	\begin{enumerate}[label=\alph*),leftmargin=18pt]
		#1
	\end{enumerate}
}
\newcommand{\abcs}[2]{
	\begin{enumerate}[label=\alph*),start=#1,leftmargin=18pt]
		#2
	\end{enumerate}
}
\newcommand{\abcn}[1]{
	\begin{enumerate}[label=\arabic*),leftmargin=18pt]
		#1
	\end{enumerate}
}
\newcommand{\abch}[1]{
	\hspace{-2pt}	\begin{enumerate*}[label=\alph*), itemjoin=\hspace{1cm}]
		#1
	\end{enumerate*}
}
\newcommand{\abchs}[2]{
	\hspace{-2pt}	\begin{enumerate*}[label=\alph*), itemjoin=\hspace{1cm}, start=#1]
		#2
	\end{enumerate*}
}

\newcommand{\opgt}{\phantomsection \addcontentsline{toc}{section}{Oppgaver} \section*{Oppgaver for kapittel \thechapter}\vs \setcounter{section}{1}}
\newcounter{opg}
\numberwithin{opg}{section}
\newcommand{\op}[1]{\vspace{15pt} \refstepcounter{opg}\large \textbf{\color{blue}\theopg} \vspace{2 pt} \label{#1} \\}
\newcommand{\ekspop}[1]{\vsk\textbf{Gruble \thechapter.#1}\vspace{2 pt} \\}
\newcommand{\nes}{\stepcounter{section}
	\setcounter{opg}{0}}
\newcommand{\opr}[1]{\vspace{3pt}\textbf{\ref{#1}}}
\newcommand{\oeks}[1]{\begin{tcolorbox}[boxrule=0.3 mm,arc=0mm,colback=white]
		\textit{Eksempel: } #1	  
\end{tcolorbox}}
\newcommand\opgeks[2][]{\begin{tcolorbox}[boxrule=0.1 mm,arc=0mm,enhanced jigsaw,breakable,colback=white] {\footnotesize \textbf{Eksempel #1} \\} \footnotesize #2 \end{tcolorbox}\vspace{-5pt} }

%License
\newcommand{\lic}{\textit{Matematikken sine byggesteinar by Sindre Sogge Heggen is licensed under CC BY-NC-SA 4.0. To view a copy of this license, visit\\ 
		\net{http://creativecommons.org/licenses/by-nc-sa/4.0/}{http://creativecommons.org/licenses/by-nc-sa/4.0/}}}

%referances
\newcommand{\net}[2]{{\color{blue}\href{#1}{#2}}}
\newcommand{\hrs}[2]{\hyperref[#1]{\color{blue}\textsl{#2 \ref*{#1}}}}
\newcommand{\rref}[1]{\hrs{#1}{Regel}}
\newcommand{\refkap}[1]{\hrs{#1}{Kapittel}}
\newcommand{\refsec}[1]{\hrs{#1}{Seksjon}}

\usepackage{datetime2}

\usepackage[]{hyperref}



\begin{document}

\newpage
\section{\fintro}
Variablar er verdiar som forandrar seg. Ein verdi som forandrar seg i takt med at ein variabel forandrar seg, kallar vi ein \textit{funksjon}\index{funksjon}.\vsk
\fig{funk1}
I figurane over forandrar antalet ruter seg etter eit bestemt mønster. Matematisk kan vi skildre dette mønsteret slik:
\alg{
\text{Antal ruter i \textsl{Figur \color{blue}1}}= 2\cdot {\color{blue}1}+1=3 \\ 
\text{Antal ruter i \textsl{Figur \color{blue}2}}= 2\cdot {\color{blue}2}+1=5 \\ 
\text{Antal ruter i \textsl{Figur \color{blue}3}}= 2\cdot {\color{blue}3}+1=7 \\ 
\text{Antal ruter i \textsl{Figur \color{blue}4}}= 2\cdot {\color{blue}4}+1=9
}
For ein figur med eit vilkårleg nummer $ x $ har vi at
\[ \text{Antal ruter i \textsl{Figur }}{\color{blue}x}=2{\color{blue}x}+1 \]
Antal ruter forandrar seg altså i takt med at $ x $ forandrar seg, og da seier vi at\regv
\st{\text{''Antal ruter i \textsl{Figur }$ x $'' er ein funksjon av $ x $}}\regv

\st{$ {2x+1} $\text{ er \textit{funksjonsuttrykket}\index{funksjonsuttrykk} til funksjonen ''Antal ruter i \textsl{Figur }$ x $''}}
\newpage
\textbf{Generelle uttrykk} \\
Skulle vi jobba vidare med funskjonen vi akkurat har sett på, ville det blitt tungvint å heile tida måtte skrive ''Antal ruter i \textsl{Figur }$ x $''. Det er vanleg å kalle også funksjonar berre for ein bokstav, og i tillegg skrive variabelen funskjonen er avhengig av i parentes. La oss no omdøpe funksjonen ''Antal ruter i \textsl{Figur} $ x $'' til $ a(x) $. Da har vi at
\[ \text{Antal ruter i \textsl{Figur }}x=a(x)=2x+1 \]
Viss vi skriv $ a(x) $, men erstattar $ x $ med eit bestemt tal, betyr det at vi skal erstatte $ x $ med dette talet i funksjonsuttrykket vårt:
\alg{
a({\color{blue}1})&=2\cdot{\color{blue}1} +1=3 \\
a({\color{blue}2})&=2\cdot{\color{blue}2}+1=5 \\
a({\color{blue}3})&=2\cdot{\color{blue}3}+1=7\\
a({\color{blue}4})&=2\cdot{\color{blue}4}+1=9
}
\fig{funk1a}
\newpage
\eks{
La antal ruter i mønsteret under vere gitt av funksjonen $ a(x) $.
\fig{funk2}
\textbf{a)} Finn uttrykket for $ a(x) $.\bs
\textbf{b)} Kor mange ruter er det når $ x=10 $? \bs
\textbf{c)} Kva er verdien til $ x $ når $ a(x)=628 $? 

\sv
\textbf{a)} Vi legg merke til at
\begin{itemize}
	\item Når $ x=1 $, er det $ 1\cdot1+3=4 $ ruter.
	\item Når $ x=2 $, er det $ 2\cdot2+3=7 $ ruter.
	\item Når $ x=3 $, er det $ 3\cdot3+3=12 $ ruter.
	\item Når $ x=4 $, er det $ 4\cdot4+3=17 $ ruter.
\end{itemize} 
Altså er
\[ a(x)=x\cdot x +3 =x^2+3 \]
\textbf{b)}
\[ a(10)=10^2+3=100+3=103 \]
Når $ x=10 $, er det 103 ruter.\vsk

\textbf{c)} Vi har likninga
\algv{
x^2+3&=628 \\
x^2&=625
}	
Altså er
\[ x=15\qquad\vee\qquad x=-15 \]
Sidan vi søker ein positiv verdi for $ x $, er $ x=15 $.	
}
\section{\lingraf}
Når vi har ein variabel $ x $ og ein funksjon $ f(x)  $, har vi to verdiar; verdien til $ x $ og den tilhøyrande verdien til $ f(x) $. Desse para av verdiar kan vi sette inn i eit koordinatsystem\footnote{Sjå \hrs{Koord}{seksjon}.}, og da får vi \textit{grafen}\index{funksjon!grafen til} til $ f(x) $. \vsk

La oss bruke funksjonen 
\[ f(x)=2x-1 \]
som eksempel. Vi har at
\alg{
f(0)&=2\cdot0-1=-1 \vn
f(1)&=2\cdot1-1=1 \vn
f(2)&=2\cdot2-1=3 \vn
f(3)&=2\cdot3-1=5
} 
Desse para av verdiar kan vi sette opp i ein tabell:
	\begin{center}
	\begin{tabular}{c | c |c |c|c}
		$ x $ & 0 & 1 & 2 & 3 \\ \hline
		$ f(x) $ &$  -1 $ & 1&3 &5
	\end{tabular}
\end{center}
Tabellen over gir punkta
\[ (0, -1)\quad\quad(1, 1)\quad\quad(2, 3)\quad\quad(3, 5) \]
Vi plasserer no punkta i eit koordinatsystem (sjå figur på side \pageref{funkfig}). I samband med funksjonar er det vanleg å kalle horisontalaksen og vertikalaksen for høvesvis $ x $-aksen og $ y $-aksen. 
Grafen til $ f(x) $ er no ein tenkt strek som går gjennom alle dei uendeleg mange punkta vi kan lage av $ x$-verdiar og dei tilhøyrande $ f(x) $-verdiane. Vår funksjon er ein \textit{lineær}\index{funksjon!lineær} funksjon, noko som betyr at grafen er ei rett linje. Altså kan grafen teiknast ved å teikne linja som går gjennom punkta vi har funne.\vsk

Som vi har vore inne på før, kan vi aldri teikne ei heil linje, berre eit utklipp av ho. Dette gjeld som regel også for grafar. I figuren på side \pageref{funkfig} har vi teikna grafen til $ f(x) $ for $ x $-verdiar mellom $ -2 $ og $ 4 $. At $ x $ er i dette \textit{intervallet}\index{intervall} kan vi skrive som\footnote{Sjå symbolforklaringar på side \pageref{Symbol}.} $ -2\leq x\leq 4 $ eller $ x\in[-2, 4] $.
\fig{funk3} \label{funkfig}

\newpage
\reg[\funklin \label{funklin}]{
	Ein funksjon på forma \vs
	\[ f(x)=ax+b \]
	der $ a $ og $ b $ er konstantar,
	er ein \textit{lineær} funksjon med \textit{stigingstal}\index{stigingstal} $ a $ og \textit{konstantledd}\index{konstantledd} $ b $. \vsk 
	
	Grafen til ein lineær funksjon er ei rett linje som går gjennom punktet $ (0, b) $. \vsk
	
	For to forskjellige $ x $-verdiar, $ x_1 $ og $ x_2 $, er
	\[ a=\frac{f(x_2)-f(x_1)}{x_2-x_1} \]
	\fig{funk6}
}
\eks[1]{Finn stigingstalet og konstantleddet til funksjonane.
	\alg{
		f(x)&=2x+1 \vn
		g(x)&=-3+\frac{7}{2}	\vn
		h(x)&=\frac{1}{4}x-\frac{5}{6}\vn
		j(x)&=4-\frac{1}{2}x
	}
	\sv \vs \vs
	\begin{itemize}
		\item $ f(x) $ har stigingstal 2 og konstantledd 1.
		\item $ g(x) $ har stigingstal $ -3 $ og konstantledd $ \frac{7}{2} $.
		\item $ h(x) $ har stigingstal $ \frac{1}{4} $ og konstantledd $ -\frac{5}{6} $.
		\item $ j(x) $ har stigingstal $ -\frac{1}{2} $ og konstantledd 4.		
	\end{itemize}	
}
\newpage
\eks[2]{
	Teikn grafen til
	\[ f(x)=\dfrac{3}{4}x-2 \]
	 for $ x\in[-5, 6] $.
	
	\sv
	For å teikne grafen til ein linjeær funksjon treng vi berre å finne to punkt som ligg på grafen. Kva to punkt dette er, er det fritt å velge, så for enklast mogleg utrekning startar vi med å finne punktet der $ x=0 $:
	\[ f(0)=\frac{3}{4}\cdot0-2=-2 \] 
	Vidare velg vi $ {x=4} $, sidan dette også gir oss ei enkel utrekning:
	\[ f(4)=\frac{3}{4}\cdot4-2=1 \]
	No har vi informasjonen vi treng, og for ordens skuld set vi han inn i ein tabell:
	\begin{center}
		\begin{tabular}{c | c |c }
			$ x $ & 0 & 4 \\ \hline
			$ f(x) $ &$  -2 $ & 1
		\end{tabular}
	\end{center}
Vi teiknar punkta og trekk ei linje gjennom dei:
	\begin{figure}
		\centering
		\fig{funk4}
	\end{figure}
}
\eks[3]{
Finn funksjonsuttrykka til $ f(x) $ og $ g(x) $.	
\fig{funk5} \vs
\sv
Vi startar med å finne funksjonsuttrykket til $ f(x) $. Punktet $ (0, 3) $ ligg på grafen til $ f(x) $ (sjå også figur på neste side). Da veit vi at $ {f(0)=3} $, og dette må bety at $ 3 $ er konstantleddet til $ f(x) $. Vidare ser vi at punktet $ (1, 2) $ også ligg på grafen til $ f(x) $. Stigingstalet til $ f(x) $ er da gitt ved brøken
\[ \frac{2-3}{1-0}=-1 \] 
Altså er 
\[  f(x)=-x+3 \]
\newpage
\fig{funk5a}
Vi går så over til å finne uttrykket til $ g(x) $. Punktet $ (0, -1) $ ligg på grafen til $ g(x) $. Da veit vi at $ {f(0)=-1} $, og dette må bety at $ -1 $ er konstantleddet til $ g(x) $. Vidare ser vi at punktet $ (5, 2) $ også ligg på grafen til $ g(x) $. Stigingstalet til $ g(x) $ er da gitt ved brøken
\[ \frac{2-(-1)}{5-0}=\frac{3}{5} \]
Altså er
\[ g(x)=\frac{3}{5}x+1 \]
}
\newpage
\eks[4]{
Finn stigningstalet til $ f(x) $.
\fig{funk7}
\sv
\fig{funk7b}
Vi legg merke til at punkta $ (1, 2) $ og $ (4, 4) $ ligg på grafen til $ f(x) $. Altså er stigningstalet til $ f(x) $ gitt ved brøken
\alg{
\frac{4-2}{4-1} =\frac{2}{3}
}
}
\newpage
\fork{\ref{funklin} \funklin}{
\textbf{Uttrykk for $ \bm a$}\os	
Gitt ein lineær funksjon
\[ f(x)=ax+b \]
For to forskjellige $ x $-verdiar, $ x_1 $ og $ x_2 $, har vi at
\begin{equation}
f(x_1)=ax_1+b \label{funkfork}
\end{equation}
\begin{equation}
f(x_2)=ax_2+b \label{funkfork1}
\end{equation}
Vi trekk \eqref{funkfork} fra \eqref{funkfork1}, og får at 
\alg{
f(x_2)-f(x_1)&=ax_2+b-(ax_1+b) \br
f(x_2)-f(x_1)&=ax_2-ax_1\\
f(x_2)-f(x_1)&=a(x_2-x_1) \\
\frac{f(x_2)-f(x_1)}{x_2-x_1}&=a \label{funka}
}\vsk

\textbf{Grafen til ein lineær funksjon er ei rett linje}\os
Gitt ein lineær funksjon $ f(x)=ax+b $ og to forskjellige $ x $-verdiar $ x_1 $ og $ x_2 $. Vi set $ {A=(x_1, b)} $, $ {B=(x_2, b)} $, $ {C=(b, f(x_1))} $, $ D=(0, f(x_2)) $ og $ E=(0, b) $.
\fig{funk6a}
Av \eqref{funka} har vi at
\alg{
\frac{f(x_1)-f(0)}{x_1-0}&=a\br
\frac{ax_1+b-b}{x_1}&=a\br
\frac{ax_1}{x_1}=a \label{funkx1}
}
Tilsvarande er 
\begin{equation} \label{funkx2}
\frac{ax_2}{x_2}=a 
\end{equation}
Vidare har vi at
\alg{
AC&=f(x_1)-b = ax_1 \vn
BD&=f(x_2)-b = ax_2 \vn
EA&= x_1 \vn
EB&= x_2
}
Av \eqref{funkx1} og \eqref{funkx2} har vi at
\[ \frac{ax_1}{x_1}=\frac{ax_2}{x_2} \]
Dette betyr at
\[ \frac{AC}{BD}=\frac{EA}{EB} \]
I tillegg er $ {\angle A=\angle B} $, altså oppfyller $ \triangle EAC $ og $ \triangle EBD $ vilkår iii fra \rref{vilkform}, og dermed er trekantane formlike. Dette betyr at $ C $ og $ D $ ligg på linje, og denne linja må vere grafen til $ f(x) $.
}


\end{document}